\emph{
	Pruebe que $D$ es una martingala bajo $\p$. Pruebe que si 
	$D$ es uniformemente integrable entonces $\q\ll\p$.
}

\afterstatement\pn

Primero veamos que se trata de una martingala bajo $\P$.

\begin{itemize}
	\item   (\textbf{Adaptada a la filtración})
        Por definición de derivada de Radon-Nykodim, $D_n$ es $\F_n$-medible.\pn
    
    \item   (\textbf{Cada término es integrable})
        Por ser $\Q$ una medida de probabilidad, es no negativa. Por lo tanto cualquier versión que se
        escoja para la derivada de Radon-Nykodim, es no negativa casi seguramente y por lo tanto
        $\E(\abs{D_n}) = \E(D_n)$. Ahora
        \begin{align}
            \E(D_n)     &=  \int D_n d\P                                    \\
                        &=  \int D_n d\P|_{\F_n}                            \\
                        &\comment{gracias a que $D_n$ es $\F_n$-medible}    \\
                        &=  \Q|_{\F_n}(\Omega)                              \\
                        &=  \Q(\Omega)                                      \\
                        &=  1                                               \\
                        &<  \infty.
        \end{align}\pn
        
    \item   (\textbf{Condición de martingala})
        Sea $A \in \F_n$. Entonces
        \begin{align}
            \E(D_{n+1} \indic_A)    &=  \int D_{n+1} \indic_A d\P                                       \\
                                    &=  \int D_{n+1} \indic_A d\P|_{\F_{n+1}}                           \\
                                    &\comment{porque $D_{n+1}, \indic_A$ son $\F_{n+1}$-medibles}       \\
                                    &=  \Q|_{\F_{n+1}}(A)                                               \\
                                    &=  \Q|_{\F_n}(A)                                                   \\
                                    &\comment{porque $D_n, \indic_A$ son $\F_n$-medibles}               \\
                                    &=  \int D_n \indic_A d\P|_{\F_n}                                   \\
                                    &=  \int D_n \indic_A d\P                                           
                                    &=  \E(D_n \indic_A).   
        \end{align}
        
        Lo que demuestra, que bajo $\F_n$, tanto $D_{n+1}$ como $D_n$ integran lo mismo. Es decir
        $\E(D_{n+1} | \F_{n}) = \F_n$, como queríamos demostrar.\pn
\end{itemize}

Ahora supongamos que $D$ es uniformemente integrable. Tenemos un teorema que dice que si un martingala es 
uniformemente integrable entonces existe una variable $D_\infty$ tal que $\E(D_\infty | \F_n)$.
(Este teorema es el 1.9 de la versión de las notas de la clase que se incluye junto con esta tarea, 
ver [\ref{notas}]).\pn

Definimos $\F_\infty = \sigma\paren(\bigcup \F_i)$. Y  definimos

\begin{align}
    P   =   \{ A \in \F_\infty : \exists i\in\N \text{  con  } A \in \F_i\}
\end{align}\pn

Claramente $P$ es no vacío, todos los $\F_i$ son subconjuntos de él. Además, si $A_1, A_2 \in P$ entonces
existen $i, j \in \N$ tales que $A_1 \in \F_i$ y $A_2 \in \F_j$ y por lo tanto $A_1 \cap A_2 \in \F_{\max(i,j)}$.
Entonces $P$ es un $\pi$-sistema.\pn

Ahora sea 
\begin{align}
    D   =   \bigg\{ A \in \F_infty   :   \Q(A) = \int_A D_\infty d\P \bigg\}
\end{align}