\emph{
    Si $\rho_2$ denota la primera vez que ambos generadores est\'an trabajando al mismo tiempo, 
    encuentre la distribuci\'on de $\rho_2$ cuando s\'olo un generador est\'a trabajando al tiempo cero.\pn
}

\afterstatement\pn

Este problema es similar al anterior. En el inciso anterior $P_t(2,3)$ representa la probabilidad de pasar del estado $2$ al $3$
en $t$, pero en la cadena cualquier estado es alcanzable desde cualquier otro, es decir que $P_t(2,3)$ concidera la posibilidad
de que se haya pasado varias veces por $3$ antes de que terminara $t$. Lo único que hay que suponer entonces, es que el estado 
``dos generadores están funcionando'' es absorbente y entonces lo que buscamos es la $\P(\rho_2 \leq t)$.\pn

Entonces, sustituimos el tercer renglón de $Q$ por puros ceros para convertir al estado ``dos generadores están funcionando''
en un estado absorvente y dejar a todas las distribuciones que no implican salir del estado ``dos generadores están 
funcionando'' sin cambios.\pn

Definimos entonces $Q'$ como
\begin{align}
    Q'  =
        \begin{pmatrix}
            -6  &6  &   0  \\
            1   &-7 &   6  \\
            0   &0  &   0
        \end{pmatrix}.
\end{align}

De ahora en adelante, la el problema es idéntico al del inciso anterior. Buscamos el polinomio característico de $Q'$.
\begin{align}
    q'(\lambda)   &=  \lambda(4+\lambda)(9+\lambda)
\end{align}

Vemos que sus valores propios son $0, -4 y -9$. Después de cuentas, encontramos los vectores propios asociados
$(1, 1, 1),(3, 1, 0),(-2, 1, 0)$ y entonces escribimos
\begin{align}
    Q'  = \frac{1}{5}
        \begin{pmatrix}
            1   &3  &   -2 \\
            1   &1  &   1  \\
            1   &0  &   0
        \end{pmatrix}
        \begin{pmatrix}
            0   &0  &   0  \\
            0   &-4 &   0  \\
            0   &0  &   -9
        \end{pmatrix}        
        \begin{pmatrix}
            0   &0  &   5 \\
            1   &2  &   -3  \\
            -1  &3  &   -2
        \end{pmatrix}.
\end{align}

De donde
\begin{align}
    P'_t  = \frac{1}{5}
            \begin{pmatrix}
                1   & 3  &   -2 \\
                1   & 1  &   1  \\
                1   & 0  &   0
            \end{pmatrix}
            \begin{pmatrix}
                1   & 0          &   0           \\
                0   & e^{-4t}    &   0           \\
                0   & 0          &   e^{-9t} 
            \end{pmatrix}        
            \begin{pmatrix}
                0   & 0  &   5   \\
                1   & 2  &   -3  \\
                -1  & 3  &   -2
            \end{pmatrix}.
\end{align}

Lo que buscamos es $\P(\rho_2 \leq t) = P'_t(2, 3)$, es decir
\begin{align}
    P'_t(2, 3)   &=  \frac{1}{5} (1,1,1)\cdot(5, -3e^{-4t}, -2e^{-9t})           \\
                &=  1 - \frac{3}{5}e^{-4t} - \frac{2}{5} e^{-9t}.
\end{align}
