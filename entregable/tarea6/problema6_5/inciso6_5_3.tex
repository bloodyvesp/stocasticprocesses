\emph{
    Al utilizar Borel-Cantelli, pruebe que, para cualquier $C>0$ fija
    \begin{esn}
        \limsup_{n\to\infty}\sup_{t\in [0,1]}\abs{B_{n+t}-B_n}/n\leq C
    \end{esn} 
    casi seguramente.\pn
}
\afterstatement\pn

Sea $C > 0$ y sean $A_n = \left\{ \sup_{t\in [0,1]}\abs{B_{n+t}-B_n}\leq nC \right\}$. Entonces tenemos
$A_n^c = \left\{  \sup_{t\in [0,1]} \abs{B_{n+t}-B_n} > nC \right\}$ y por desigualdad de Markov.

\begin{align}
        \P(A_n^c)   &=  \P(\sup_{t\in [0,1]}\abs{B_{n+t}-B_n} > nC)         \\
                    &=  \frac{\E(\sup_{t\in [0,1]}\abs{B_{n+t}-B_n})}{nC}
\end{align}\pn

De [\ref{problema6_5:inciso2}] tenemos que para todo $n\in \N$, 
$\E(\sup_{t\in [0,1]}\abs{B_{n+t}-B_n}) = r < \infty$ y por lo tanto
$\lim_{n \rightarrow \infty} \P(A_n^c) = 0$. Es decir que $\lim_{n \rightarrow \infty} \P(A_n) = 1$.\pn

Una condición necesaria para la convergencia de una serie es que el límite de la sucesión
sea $0$. Por lo tanto la serie $\sum \P(A_n)$ diverge.\pn

De  [\ref{problema6_5:inciso2}] tenemos que los $A_n$ son independientes. Y por lo tanto, Borel-Cantelli
nos asegura que
\begin{align}
    \P(\limsup A_n) = 1.
\end{align}\pn

Con lo que queda demostrado que
\begin{esn}
    \limsup_{n\to\infty}\sup_{t\in [0,1]}\abs{B_{n+t}-B_n}/n\leq C
\end{esn}\pn
 
casi seguramente. Es decir
\begin{esn}
    \limsup_{n\to\infty}\sup_{t\in [0,1]}\abs{B_{n+t}-B_n}/n = 0.
\end{esn} 