\emph
{
	Sea $T_1=\min\set{n\geq 0:S_n=1}$. Explique por qu\'e $T_1$ es un 
	tiempo de paro y calcule su esperanza.
}

\afterstatement\par\null

	Consideremos a la filtración $(\F_n)_{n\in\N}$ como la filtración 
	generada por $X_1, X_2, \dots$.\par\null

	Es decir, $F_0 = \{\emptyset, \Omega\}$, $\F_n = \sigma(X_1, X_2, \dots, X_n)$\par\null

	Nótese que $S_0$ es medible bajo cualquier sigma álgebra por ser constante, en particular bajo
	$\F_0$.\par\null

	Basta demostrar que $(T_1 = n) \in \F_n$ para ver que $T_1$ es tiempo de paro. $T_1$ 
	representa el primer tiempo en que la suma es igual a $1$. Es decir, para cualquier 
	momento anterior, la suma no es $1$.\par\null

	Eso escrito en símbolos significa:

    \begin{align}
        (T_1 = n) = \bigcap_{i=0}^{n-1}(S_i \not= 1) \cup (S_n = 1).
    \end{align}\par\null

	Para $n=0$, $(S_0 = 1) = \omega \in \F_0$.\par\null

	Como $(S_i \not= 1) \in \F_j$ siempre que $i \leq j$. Para $n>0$, $(T_1 = n)$ es el resultado de 
	unir e intersectar conjuntos $\F_n$-medibles, lo cual resulta $\F_n$-medible.\par\null

	Para $m \in \N$. Definamos $T_m = min\{n \geq 0 : S_n = m\}$ 
	(Nótese que para el caso $m=1$, esta definición	coincide con la definición previa de $T_1$).\par\null
	
	Para $a,b \in \N$, podemos definir el tiempo de paro $T_{a,b} = T_{-a} \wedge T_b$, y 
	corresponde al 	tiempo de paro del problema de la ruina. Para este tiempo de paro ya conocemos 
	la esperanza y es
	
    \begin{align}
        \E(T_{a,b}) = ab.
    \end{align}\par\null
	
	Ahora, tomemos la sucesion de variables aleatorias
    
    \begin{align}
        T_{1,1}, T_{2,1}, T_{3,1}, \dots, T_{n,1},\dots
    \end{align}\par\null 
    
    Notemos que si $a>a' \in N$ entonces $T_{-a} > T_{-a'}$, pues $T_{-a}$ es la primera vez
	que se llega a $-a$, y para poder alcanzar $-a$ era necesario haber pasado por $-a'$.
	De aqui tenemos que si $a>a'$, entonces $T_{a,1} \geq T_{a',1}$. De donde nuestra suceción es 
	no decreciente.\par\null
	
	Por otro lado, que si $a>a' \in N$ entonces $T_{-a} > T_{-a'}$ implica que $T_{-n} n \in \N$ es 
	una suceción extrictamente creciente y por lo tanto 
	$\lim\limits_{n \rightarrow \infty} T_{-n} = \infty$, con esto tenemos que el límite de nuestra 
	suceción es 
	
    \begin{align}		
        \lim_{n\rightarrow\infty} T_{n,1}   &=  \lim_{n\rightarrow\infty} T_{-n} \wedge T_1 \\
                                            &=  \infty \wedge T_1                           \\
                                            &=  T_1
	\end{align}\par\null

	Tenemos todos los ingredientes para usar Teorema de convergencia monótona sobre nuestra suceción
	y la variable $T_1$. Nuestra sucecion es monótona y converge puntualmente a $T_1$. Utilizando
	dicho teorema obtenemos:
	
	\begin{align}
        \E(T_1)     &=  \E(\lim_{n\rightarrow\infty} T_{n,1})       \\
                    &=  \lim_{n\rightarrow\infty} \E(T_{n,1})       \\
                    &=  \lim_{n\rightarrow\infty} n\cdot 1          \\
                    &=  \infty.
	\end{align}\par\null
	
	Lo cual era intuitivo. Si $\E(T_1)$ fuese finito, diría que existe un número de volados donde
	uno puede apostar con mucha certeza que ganará un peso después de jugar "cerca" de esa cantidad
	de volados. Intuitivamente, esto vuelve injusto un juego de volados donde la moneda es
	justa.	