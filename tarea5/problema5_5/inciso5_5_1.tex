\emph{
    Encuentre la matriz de transici\'on de la cadena de Markov de los estados distintos que toma $X$, 
    clasifique los estados, diga si existe una \'unica distribuci\'on invariante y en caso afirmativo, 
    encu\'entrela. Calcule expl\'icitamente las potencias de la matriz de transici\'on. 
    (Recuerde que de ser posible diagonalizar, esta es una buena estrategia.)\pn
}

\afterstatement\pn

Como no hay estados absorbentes, podemos utilizar la fórmula
\begin{align}
        P(i,j)  &=  \frac{Q(i, j)}{c(i)} (i-\delta_{i,j})
\end{align}

donde $c(i)$ es la suma de las entradas positivas de la $i$-ésima fila de $Q$ y $\delta_{i,j} = 1$
si $i = j$ y $\delta_{i,j} = 0$ en cualquier otro caso.\pn

Después de algunas cuentas se obtiene:
\begin{align}
     P=
        \begin{pmatrix}
            0           &   1   &   0               \\
            \frac{1}{7} &   0   &   \frac{6}{7}     \\
            0           &   1   &   0
        \end{pmatrix}.   
\end{align}\pn

De $P$ se puede todos los estados son alcanzables desde cualquier otro (de los estados 0 y 2 se puede llegar al $1$, y del 1 a cualquiera
de los otros).