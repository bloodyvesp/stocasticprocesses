\emph{
	Pruebe que si $S\leq T\leq n$ entonces $\F_S\subset\F_T$. Si adem\'as $T$ es acotado entonces $M_S,M_T \in L_1$ y 
	\begin{align}
		\espc{M_T}{\F_S} = M_S &\mbox{\;\;\;(En el caso de que M sea martingala)} \label{problema1_4:caso_martingala}\\
		\espc{M_T}{\F_S}\leq M_S &\mbox{\;\;\;(En el caso de que M sea supermartingala)} \label{problema1_4:caso_supermartingala}
	\end{align}	
}

	Primero probaremos que $\F_S \subset \F_T$. Sea $n \in \N$. Notemos que $\{ T \leq n \} \subset \{ S \leq n \}$ pues
	si $\omega \in \{ T \leq n \}$, entonces $S(\omega) \leq T(\omega) \leq n $ y por lo tanto $\omega \in \{ S \leq n \}$.\\

	Ahora sea $A \in \F_S$. Entonces

	\begin{align}
		A \cap \{ T \leq n \} 	&=		A  \cap \{ T \leq n \} \cap \{ S \leq n \} \\
								&=		(A  \cap \{ S \leq n \}) \cap \{ T \leq n \} \in \F_n
	\end{align}

	Como escogimos $n$ arbitrariamente, esto es cierto para toda $n \in \N$. Y por lo tanto $A \in \F_T$ y $\F_S \subset \F_T$.\\

	Ahora supongamos que $T$ es acotado y sea $N$ una cota para $T$. Demostraremos que $M_S,M_T \in L_1$.
	
	\begin{align}
		\E(|M_T|) 	&=		\E\bigg(	\bigg| \sum_{i \in \N} M_i \cdot \indic_{T = i} \bigg|	\bigg) \\ 
					&\leq	\E\bigg(	\sum_{i \in \N} |M_i \cdot \indic_{T = i} |\bigg) \\
					&\leq	\E\bigg(	\sum_{i \in \N} |M_i|\bigg) \\
					&=		\E\bigg(	\sum_{i \leq N} |M_i|\bigg) \\
					&< 		\infty
	\end{align}
	
	Y con esto queda demostrado que $M_T \in L_1$. La demostración de que $M_S \in L_1$ es completamente análoga.\\
	
	Nos queda por demostrar (\ref{problema1_4:caso_martingala}) y (\ref{problema1_4:caso_supermartingala}).