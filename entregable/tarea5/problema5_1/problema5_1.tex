\begin{problema}
	Sea $N$ un proceso Poisson de par\'ametro $\lambda$ y sea $T_n$ el tiempo de su en\'esimo salto. 
	
	\begin{enumerate}
		\item[(i)]		[\ref{problema5_1:inciso1}] 
			Pruebe que condicionalmente a $T_2$, $T_1$ es uniforme en $[0,T_2]$.\pn
			
		\item[(ii)]		[\ref{problema5_1:inciso2}] 
			Pruebe que si $W_1$ y $W_2$  son  exponenciales de par\'ametro 
			$\lambda$  independientes entre si y de una variable uniforme $U$, 
			entonces $U\paren{W_1+W_2}$ es una variable aleatoria exponencial 
			de par\'ametro $\lambda$.\pn

		\item[(iii)]	[\ref{problema5_1:inciso3}] 
			Conjeture c\'omo se  generaliza lo anterior con $T_n$ y $T_1$.\pn

		\item[(iv)]		[\ref{problema5_1:inciso4}] 
			Escriba dos programas en Octave que simulen al proceso de Poisson 
			de par\'ametro $\lambda$ en el intervalo $[0,1]$. En uno utilizar\'a 
			s\'olo variables exponenciales y en el otro puede utilizar una 
			variable Poisson.\pn
	\end{enumerate}
\end{problema}

\begin{proof}
    \subsection{Inciso (i)} \label{problema5_1:inciso1}
    \emph{
	Pruebe que condicionalmente a $T_2$, $T_1$ es uniforme en $[0,T_2]$.
}

\afterstatement\pn

Por definición de proceso Poisson de parámetro $\lambda$, la distribución de los saltos $S_n$ es
exponencial de parámetro $\lambda$. Como $T_1 = S_1$ y $T_2 = S_1 + S_2$, entonces
$T_1  = S_1 \sim exp(\lambda) $ y  $T_2 - T_1 = S_2 \sim exp(\lambda)$.\pn

También por definición de proceso de Poisson, los saltos son independientes y por lo tanto
$T_1$ y $T_2 - T_1$ son independientes.\pn

Podemos dar su distribución conjunta como sigue:
\begin{align}
    f_{T_1, T_2 - T_1}(u, v)    &=  (\lambda e^{-\lambda u}) (\lambda e^{-\lambda v}) \indic_{\{ u > 0, v > 0 \}}   \\
                                &=  \lambda^2 e^{-\lambda (u + v)} \indic_{\{ u > 0, v > 0 \}}.
\end{align}

Hacemos el cambio de variable dado por la transformación lineal $l : (u, v-u) \longrightarrow (u, v)$.
La matriz asociada a esta transformación lineal tiene la forma

\[
    \left(
            \begin{array}{cc}
                    1   &   1   \\
                    0   &   1   \end{array}
    \right)
\]

cuyo determinante es $(1 \cdot 1) - (1 \cdot 0) = 1$ y por lo tanto tiene inversa (es fácil comprobar que $l^{-1}(u,v) = (u, v - u)$). 
Entonces, podemos escribir

\begin{align}
    f_{T_1, T_2}(u, v)      &=    f_{T_1, T_2 - T_1}(l^{-1}(u, v))                                      \\
                            &=    f_{T_1, T_2 - T_1}(u, v - u)                                          \\
                            &=    \lambda^2 e^{-\lambda (u + v - u)} \indic_{\{ u > 0, v - u > 0 \}}.     \\
                            &=    \lambda^2 e^{-\lambda (v)} \indic_{\{ v > u > 0 \}}.                    
\end{align}\pn

Ahora, $T_2 = (T_1) + (T_2 - T_1)$, de donde $T_2 \sim gamma(2, \lambda)$ y por lo tanto su función de densidad esta dada por

\begin{align}
    f_{T_2}(v) = \lambda^2 v e^{-\lambda v} \indic_{v \geq 0}.
\end{align}

Con todo esto, ahora podemos calcular la densidad condicional $f_{T_1 | T_2}$.

\begin{align}
    f_{T_1 | T_2}(u | v)    &=     \frac{f_{T_1 , T_2}(u, v)}{ f_{T_2}(v)}                                                                      \\
                            &=     \frac{\lambda^2 e^{-\lambda (v)} \indic_{\{ v > u > 0 \}}}{ \lambda^2 v e^{-\lambda v} \indic_{v \geq 0}}    \\
                            &=     \frac{ \indic_{\{ v > u > 0 \}}}{ v  \indic_{v \geq 0}}                                                      \\
                            &=     \frac{ \indic_{\{ v > u > 0 \}}}{v}.                                                      
\end{align}\pn

Y esto no es otra cosa que la distribución uniforme sobre $[0, v]$, es decir, la distribución uniforme sobre $[0, T_2]$.
    \newpage

    \subsection{Inciso (ii)} \label{problema5_1:inciso2}
    \emph{
	Pruebe que si $W_1$ y $W_2$  son  exponenciales de par\'ametro 
	$\lambda$  independientes entre si y de una variable uniforme $U$, 
	entonces $U\paren{W_1+W_2}$ es una variable aleatoria exponencial 
	de par\'ametro $\lambda$.
}
    \newpage

    \subsection{Inciso (iii)} \label{problema5_1:inciso3}
    \emph{
	Conjeture c\'omo se  generaliza lo anterior con $T_n$ y $T_1$.
}

Por definición de proceso de Poisson de parámetro $\lambda$, los saltos $S_n$ se distribuyen
de manera exponencial de parámetro $\lambda$ y son independientes entre sí.\pn

Es decir que $T_1 = S_1, T_2 - T_1 = S_2, \dots, T_n - T_{n-1} = S_n$ son independientes y 
todos tienen distribución exponencial de parámetro $\lambda$. Por lo tanto, su función
de densidad conjunta está dada por

\begin{align}
    f_{T_1, T_2 - T_1, \dots, T_n - T_{n-1}}(u_1, u_2, \dots, u_n)  &=  \prod_{1 \leq i \leq n} \lambda e^{-\lambda u_i}.
\end{align}

De manera similar a como se procedió en [\ref{problema5_1:inciso1}] definimos la transformación lineal 
$l: (t_1, t_2 - t_1, \dots, t_n - t_{n-1}) \longrightarrow (t_1, t_2, \dots, t_n)$ cuya matriz asociada es la 
matriz triangular superior

\[
    \left(
            \begin{array}{ccccc}
                    1       &   1       & 1      &\dots   &  1      \\
                    0       &   1       & 1      &\dots   &  1      \\
                    0       &   0       & 1      &\dots   &  1      \\
                    0       &   0       & 0      &\dots   &  1      \\
                    \vdots  &   \vdots  & \vdots &\vdots  & \vdots  \\
                    0       &   0       & 0      &\dots   &  1
            \end{array}
    \right)
\]\pn

cuyo determinante es $1$ (esto úlitmo no es difícil de verificar). Es decir, tiene inversa y también es fácil de verificar que 
su inversa está dada por\par $l^{-1}: (x_1, x_2, \dots, x_n) \longrightarrow (x_1, x_2-x_1, \dots, x_n - x_{n-1})$.


	\newpage
	
    \subsection{Inciso (iv)} \label{problema5_1:inciso4}
    \emph{
	Escriba dos programas en Octave que simulen al proceso de Poisson 
	de par\'ametro $\lambda$ en el intervalo $[0,1]$. En uno utilizar\'a 
	s\'olo variables exponenciales y en el otro puede utilizar una 
	variable Poisson.
}
\end{proof}