\begin{problema}
El objetivo de este ejercicio es ver ejemplos de cadenas de Markov $X$ y de funciones $f$ tales que $\imf{f}{X}=\paren{\imf{f}{X_n},n\in\na}$ sean o no cadenas de Markov.
\begin{enumerate}
\item Considere el hipercubo $n$-dimensional $E=\set{0,1}^n$. A $E$ lo pensaremos como la composici\'on de la primera de dos urnas que tienen en total $n$ bolas etiquetadas del $1$ al $n$. Si $x=\paren{x_1,\ldots, x_n}\in E$, interpretaremos $x_i=1$ como que la bola $i$ est\'a en la urna $1$. Considere el siguiente experimento aleatorio: inicialmente la composici\'on de las urnas est\'a dada por $x$ y a cada instante de tiempo escogemos una bola al azar y la cambiamos de urna. Modele esta situaci\'on por medio de una cadena de Markov $X$ en $E$. Sea $\fun{f}{E}{\set{0,\ldots, n}}$ dada por $\imf{f}{x}=\sum_i x_i$. Pruebe que $\imf{f}{X}=\paren{\imf{f}{X_n},n\in\na}$ es una cadena de Markov cuya matriz de transici\'on determinar\'a.

\item Sea $\paren{S_n}_{n\in\na}$ una cadena de Markov con espacio de estados $\z$ y matriz de transici\'on\begin{esn}
P_{i,i+1}=p\quad P_{i,i-1}=1-p
\end{esn}donde $p\in [0,1]$. D\'e una condici\'on necesaria y suficiente para que $\paren{\abs{S_n},n\in\na}$ sea una cadena de Markov.
\end{enumerate}

\defin{Categor\'ias:} proyecciones de cadenas de Markov
\end{problema}