\begin{problema}
Sean $\p$ y $\q$ dos medidas de probabilidad en el espacio can\'onico $E^\na$ para sucesiones con valores en un conjunto a lo m\'as numerable $E$. Decimos que $\q$ es \defin{localmente absolutamente continua} respecto de $\p$ si para cada $n\in\na$, $\q|_{\F_n}\ll\p|_{\F_n}$. Sea\begin{esn}
D_n=\frac{d \q|_{\F_n}}{d \p|_{\F_n}}.
\end{esn}
\begin{enumerate}
\item Pruebe que $D$ es una martingala bajo $\p$. Pruebe que si $D$ es uniformemente integrable entonces $\q\ll\p$. 
\item Pruebe que si $T$ es un tiempo de paro finito entonces $\q|_{\F_T}\ll\p|_{\F_T}$. 
\item Sea $\p^p$ la distribuci\'on de una caminata aleatoria simple que comienza en $0$ y va de $k$ a $k+1$ con probabilidad $p$, donde $p\in (0,1)$. Pruebe que $\p^p$ es localmente absolutamente continua respecto de $\p^{1/2}$ y encuentre la martingala $D_n$ asociada.
\item Para $a,b>0$, sea $T=\min\set{n\in\na: X_n\in \set{-a,b}}$. Pruebe que $T$ y $X_T$ son independientes bajo $\p^{1/2}$. Al utilizar la continuidad absoluta local, pruebe que $T$ y $X_T$ tambi\'en son independientes bajo $\p^p$. Utilice alguna martingala de ejercicios anteriores para calcular $\esp{T^2}$. 
\end{enumerate}

\defin{Categor\'ias: }Cambio de medida, Caminata aleatoria simple.
\end{problema}