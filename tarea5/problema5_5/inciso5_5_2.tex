\emph{
    ?`Cu\'al es la probabilidad de que ambos generadores est\'en trabajando al tiempo $t$ 
    si s\'olo uno trabaja al tiempo cero?\pn
}

\afterstatement\pn

Ahora la estrategia consiste en diagonalizar $Q$. 
Su polinomio característico es
\begin{align}
        q(x) =  \lambda (\lambda + 5) (\lambda + 10).
\end{align}

Por lo tanto sus valores propios son $0, -5, -10$. Después de resolver los
sistemas de ecuaciones correspondientes se obtiene que los vectores propios correspondientes son
$(1,1,1), (18, 3, -2), (-6, 4, -1)$.

Con lo que definimos a la matriz
\begin{align}
     A      =
                \begin{pmatrix}
                    1   &   18  &   -6      \\
                    1   &   3   &   4       \\
                    1   &   -2  &   -1
                \end{pmatrix}   
\end{align}
la cual tiene inversa
\begin{align}
     A^{-1}     =   \frac{1}{25}
                    \begin{pmatrix}
                        1       &   6   &   18      \\
                        1       &   1   &   -2      \\
                        -1      &   4   &   -3
                    \end{pmatrix}.   
\end{align}

Ahora definimos
\begin{align}
    D = A^{-1} Q A  =
                        \begin{pmatrix}
                            0   &   0   &   0      \\
                            0   &   -5  &   0      \\
                            0   &   0   &   -10
                        \end{pmatrix}.
\end{align}

Para calcular la probablidad de que $2$ generadores estén funcionando al tiempo $t$ dado que se empezó con $1$, necesitamos 
encontrar la entrada $(2,3)$ de $P_t$ (recordemos que el estado representado en el renglón $1$ significa ningún generador 
está funcionando, el representado por el renglón $2$ significa que un generador está funcionando y el representado por el 
$3$ significa que ambos generadores están funcionando).\pn
  
Para ello utilizamos que $P_t = e^tQ = e^{t(ADA^{-1})} = Ae^{tD}A^{-1}$ y que
\begin{align}
    e^{tD}  =   
                \begin{pmatrix}
                    e^{0 t} &   0           &   0      \\
                    0       &   e^{-5 t}    &   0      \\
                    0       &   0           &   e^{-10 t}
                \end{pmatrix}
            =   
                \begin{pmatrix}
                    1       &   0           &   0      \\
                    0       &   e^{-5 t}    &   0      \\
                    0       &   0           &   e^{-10 t}
                \end{pmatrix}.
\end{align}

Entonces, 
\begin{align}
        P_t(2,3) &= \frac{1}{25} (1,3,4) \cdot (18, -2e^{-5t}, -3e^{10t})                                                                                 \\
                 &= \frac{1}{25} (18 - 6e^{-5t} - 12e^{10t}).                                                                                                \\
\end{align}