\emph{
	?`Qu\'e quiere decir la independencia condicional cuando $\G=\set{\oo,\emptyset}$?
}
\afterstatement\par\null

Sabemos que cualquier $\sigma$-álgebra es
independiente de la $\sigma$-álgebra trivial $\G$. Y entonces $\E(X | \G) = \E(X)$, para toda $X$.\par\null

Si $\F_1, \dots, \F_n$ son condicionalmente independietnes dada $\G$ y $H_i$ es $\F_i$-medible para toda $i$, entonces tenemos

\begin{align}
    \E(H_1 \cdots H_n) &=  \E(H_1 \cdots H_n | \G)            \\
                        &=  \E(H_1 | \G) \cdots \E(H_n | \G)    \\
                        &=  \E(H_1) \cdots  \E(H_n | \G).
\end{align}

Entonces, podemos concluir que cualesquiera funciones $H_i$ así definidas son siempre
independientes.\par\null

Esto ocurre si y sólamente si las $\sigma$-álgebras $\F_1, \F_2,\dots$ son independientes.\par\null

Entonces podemos concluir que cuando una suceción de $\sigma$-álgebras $\F_1, \F_2,\dots$ son condicionalmente independientes
dada la $\sigma$-álgebra trivial, entonces son independientes.