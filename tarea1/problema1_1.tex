\begin{problema}
	Sea ${\left({X_n}_{n\in\mathbb{N}}\right)}$ un proceso estocástico con valores reales y ${A\subset \mathbb{R}}$ un boreliano. 
	Pruebe que si
	\begin{align}
		T_0=0\quad\text{y}\quad T_{n+1}=\min\{k>T_n: X_k\in A \subset \mathbb{R}\}
	\end{align}
	entonces ${T_n}$ es un tiempo de paro para toda ${n}$ y ${T_n\to \infty}$ puntualmente conforme ${n\to\infty}$. 

	\begin{categoria} 
		Tiempos de paro.
	\end{categoria}
\end{problema}
		
\begin{proof}
\\

	Sea ${(\mathscr{F}_n = \sigma(  X_0, X_1, \dots, X_n ))_{ n \in \mathbb{N}}}$ una filtración.
\\

	Veamos que ${T_0}$ es tiempo de paro. ${\{T_0 = 0\} = \Omega \in \mathscr{F}_n}$ y 
	${\{T_0 = c\} = \emptyset \in \mathscr{F}_n}$ para ${c \not= 0 \in \mathbb{N}}$. Recordemos que:		
	\begin{align}\label{problema_1_1:equivalencia_varable_aleatoria}
		\{T_0 = n\} \in \mathscr{F}_n \; \forall n \in \mathbb{N} \iff 
		\{T_0 \leq n\} \in \mathscr{F}_n \; \forall n \in \mathbb{N}.
	\end{align}
	Y por lo tanto ${T_0}$ es variable aleatoria sobre la sigma álgebra ${\sigma(\cup_{n \in \mathbb{N}} 
	\mathscr{F}_n)}$.

	Para ${n=1}$, veamos que:
 
	\begin{align}
			\{T_1 = 1\}     &= \{ X_1 \in A \}	\in \mathscr{F}_1 
	\end{align}
	
	\begin{align}
			\{T_1 = 2\} 	&= 	\{ X_1 \not\in A, X_2 \in A \} \\
							&= 	\{ X_1 \not\in A \} \cap \{X_2 \in A \} \in \mathscr{F}_2	
	\end{align}
	
	\begin{align}
			\{T_1 = 3\} 	&=	\{ X_1 \not\in A, X_2 \not\in A,  X_3 \in A \} \\ 
							&= 	\{ X_1 \not\in A \} \cap \{ X_2 \not\in A \} \cap \{X_3 \in A \} \in \mathscr{F}_3
	\end{align}
	
	\begin{align*}
		\vdots
	\end{align*}
	
	\begin{align}
			\{T_1 = n\} 	&=	\{ X_1 \not\in A, X_2 \not\in A, \dots, X_{n-1} \not\in A, X_n \in A \} \\
							&= 	\bigcap_{i=1}^{i=n-1} \{ X_i \not\in A \} \cap \{X_n \in A \} \in \mathscr{F}_n
	\end{align} 
 
	Por hipótesis de inducción supongamos que ${\{T_k = n\} \in \mathscr{F}_n}$ para cierta ${k>1}$ 
	y para toda ${n \in \mathbb{N}}$ . \\

	Para demostrar la primera parte del ejercicio, basta probar que ${\{T_{k+1} = n\} \in \mathscr{F}_n}$ para 
	toda ${n \in \mathbb{N}}$. Pues por lo dicho en (\ref{problema_1_1:equivalencia_varable_aleatoria}) resultarían ser 
	variables aleatorias en ${\sigma(\cup_{n \in \mathbb{N}} \mathscr{F}_n)}$.
	\\
	
	Dado que 
	\begin{align}\label{problema_1_1:T_n->infty}
		0=T_0<T_1<T_2<\dots<T_k
	\end{align}  
	tenemos que ${\{ T_k = n\} = \emptyset \in \mathscr{F}_l}$ para ${n < k}$. 
	
	Ahora, sea ${n \geq k}$
	
	\begin{align}\label{problema_1_1:T_k+1}
		\{ T_{k+1} = n\} = 
		\left( \bigcup_{i < n } \{T_k = i \} \right) 
		\cap 
		\left( \bigcap_{i < n} \{ X_i \not\in A \} \right)
		\cap
		\{ X_n \in A\}
	\end{align}
	
	
	Donde 
	\begin{itemize}
		\item	 ${\{ T_k = i\} \in \mathscr{F}_n \; \forall i < n \in \mathbb{N} \; }$ (Por hipótesis de inducción).	
		\item	 ${\{ X_i \not \in A\} \in \mathscr{F}_n \; \forall i < n \in \mathbb{N} \;}$ (Por ser ${\mathscr{F}}$ filtración).
		\item ${\{ X_n \in A\} \in \mathscr{F}_n}$ (Por ser ${X_n}$ variable aleatoria en ${\mathscr{F}_n}$)	
	\end{itemize}
	
	Y por lo tanto, (\ref{problema_1_1:T_k+1}) pertenece a ${\mathscr{F}_n}$. Con lo que termina la demostración de la primera parte del ejercicio.\\
	
	Ahora, de (\ref{problema_1_1:T_n->infty}) se sigue inmediatamente que que ${T_n \rightarrow \infty}$.
\end{proof}