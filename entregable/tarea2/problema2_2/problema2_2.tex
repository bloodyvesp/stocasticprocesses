\begin{problema}
	\begin{enumerate}
		\item[(i)] 
			Instale \href{www.octave.org}{Octave} en su computadora
		\item[(ii)] 
			\'Echele un ojo a la documentaci\'on
		\item[(iii)] 
			Ejecute el siguiente c\'odigo linea por linea: \lstinputlisting[caption=]{tarea2/problema2_2/polya1.R}
		\item[(iv)] 
			Lea las secciones sobre 
			\href{http://www.gnu.org/software/octave/doc/interpreter/Simple-Examples.html#Simple-Examples}{simple examples}, 
			\href{http://www.gnu.org/software/octave/doc/interpreter/Ranges.html#Ranges}{ranges}, 
			\href{http://www.gnu.org/software/octave/doc/interpreter/Random-Number-Generation.html#Random-Number-Generation}{random number generation} 
			y 
			\href{http://www.gnu.org/software/octave/doc/interpreter/Comparison-Ops.html#Comparison-Ops}{comparison operators} 
			y escriba su interpretaci\'on de lo que hace el c\'odigo anterior. Nota: est\'a relacionado con uno de los ejemplos del curso.
		\item[(v)] 
			Vuelva a correr el c\'odigo varias veces y escriba sus impresiones sobre lo que est\'a sucediendo.
	\end{enumerate}
\end{problema}

\begin{center}
	\includegraphics[width=10cm]{tarea2/problema2_2/poyla.PNG}
\end{center}
\begin{center}
	Gráfica de una ejecución de urnas de Poyla \\
	con 1 bola verde inicial, 1 bola roja inicial y constante 1.
\end{center}