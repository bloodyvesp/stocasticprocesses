\emph{
    Pruebe que existe un espacio de probabilidad $\ofp$ en el que existe 
    una sucesi\'on $B^1,B^2,\ldots$ de movimientos brownianos en $[0,1]$ 
    independientes. (Sugerencia: utilice la construcci\'on del movimiento 
    browniano de L\'evy  para que la soluci\'on sea corta.)
}
\afterstatement\pn

La idea principal es utilizar que $\aleph_0 \times \aleph_0 = \aleph_0$ y aprovechar
la construcción que se dió en las notas [véase \ref{notas}].\pn

Se sabe que existe un espacio de probabilidad $(\Omega, \F, \P)$ donde están
definidas $\xi_{i,n}$ ($0 \leq i \leq 2^n$) con distribuciónes normales de media 0 y 
varianza 1 e independientes [ver \ref{notas}].\pn

Entonces, podemos para cada $m \in \N$ dar una sucesión de variables $\xi_{i,n}^m$
($0 \leq i \ 2^n$) con distribuciones normales de media 0 y varianza 1 tales que
son independientes entre sí y de las otras $\xi_{i,n}^{m'}$.\pn

Entonces, para cada $m$ podemos construir un movimiento browniano $B^m$ en $[0, 1]$ utilizando
las $\xi_{i,n}^m$. Dichos movimientos serán independientes entre sí por la selección
de las $(\xi_{i,n}^m)_{m \in \N}$.