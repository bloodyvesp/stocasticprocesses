\begin{problema}[Extensiones del teorema de paro opcional]
	Sea ${M=\paren{M_n,n\in\na}}$ una (super)martingala respecto de una filtraci\'on ${\paren{\F_n,n\in\na}}$ y sean ${S}$ y ${T}$ tiempos de paro.
	
\begin{enumerate}
                \item[(i)] 
                	Pruebe que ${S\wedge T}$, ${S+T}$ y ${S\vee T}$ son tiempos de paro.
                
                \item[(ii)] 
                	\begin{esn}
                		\F_T=\set{A\in\F:A\cap\set{T\leq n}\in\F_n\text{ para toda } n}
                	\end{esn}
                	es una ${\sigma}$-\'algebra, a la que nos referimos como la ${\sigma}$-\'algebra 
                	detenida en ${\tau}$. Comente qu\'e puede fallar si ${T}$ no es tiempo de paro. 
                	Pruebe que ${T}$ es ${F_T}$-medible. 
                
                \item[(iii)] 
                	Pruebe que si ${T}$ es finito, entonces ${M_T}$ es ${\F_T}$-medible.
                
                \item[(iv)] 
                	Pruebe que si ${S\leq T\leq n}$ entonces ${\F_S\subset\F_T}$. Si adem\'as ${T}$ es acotado entonces ${X_S,X_T\in L_1}$ y 
                	\begin{esn}
	                	\espc{M_T}{\F_S}\leq M_S.
                	\end{esn}

                \item[(v)] 
                	Si ${X=\paren{X_n,n\in\na}}$ es un proceso estoc\'astico ${\paren{\F_n}}$-adaptado y tal que ${X_n\in L_1}$ y tal que 
                	para cualesquiera tiempos de paro acotados ${S}$ y ${T}$ se tiene que ${\esp{X_S}=\esp{X_T}}$ entonces ${X}$ es una 
                	martingala. Sugerencia: considere tiempos de paro de la forma ${n\indi{A}+(n+1)\indi{A^c}}$ con ${A\in\F_n}$.
\end{enumerate}

\defin{Categor\'ias: }Tiempos de paro, Muestreo opcional
\end{problema}

\begin{proof}
	\subsubsection{Inciso (i)} 
	\input{tarea1/problema1_4/inciso1.tex}
	
	\subsubsection{Inciso (ii)}
	\emph{
	\begin{align}
		\F_T=\set{A\in\F:A\cap\set{T\leq n}\in\F_n\text{ para toda } n}
	\end{align}
	es una $\sigma$-\'algebra, 
	a la que nos referimos como la $\sigma$-\'algebra detenida en $\tau$. Comente qu\'e puede fallar si $T$ no es tiempo de paro. 
	Pruebe que $T$ es $\F_T$-medible.\\		
}			
		
	Primero hay que demostrar que $\F_T$ es $\sigma$-algebra.\\
	
	\begin{itemize}
		\item $\Omega \in \F_T$. \\
		
			Notemos que 
			\begin{align}
				\Omega \cap \{T \leq n\} = \{T \leq n\} \in \F_n.
			\end{align}
			Donde la pertenencia a $\F_n$ es gracias a que $T$ es tiempo de paro. (Esta parte podria fallar si 
			$T$ no fuera tiempo de paro).\\
		
		\item $\F_T$ es cerrado bajo complementación.\\
		
			Sea $A \in \F_T$. Eso significa que para todo $n \in  \N$ ocurre que $B = A \cap \{ T \leq n \} \in \F_n$. 
			Por ser $\F_n$ una $\sigma$-algebra tenemos que el complemento de $B$ también debe estar en $\F_n$. 
			Escrito en símbolos:
			
			\begin{align}
				B^c 	&= (A   \cap \{ T \leq n \})^c \\
						&=  A^c \cup \{ T > n \} \in \F_n
			\end{align}
			
			Dado que $B^c$ y $\{ T \leq n \}$ se encuentran en $\F_n$, también \\
			$B^c \cap \{ T \leq n \} \cap \{ T \leq n \}$.\\
			
			Escribiendo esto último de otra manera:
			\begin{align}
				B^c \cap \{ T \leq n \} \cap \{ T \leq n \} 	&=		A^c \cup \{ T > n \} \cap \{ T \leq n \} \cap \{ T \leq n \} \\
																&= 		A^c \cap \{ T \leq n \} \cup \{ T > n \} \cap \{ T \leq n \} \\
																&= 		\bigg(A^c \cap \{ T \leq n \}\bigg) 
																			\bigcup 
																		\bigg(\{ T > n \} \cap \{ T \leq n \}\bigg) \\
																&=		\bigg(A^c \cap \{ T \leq n \}\bigg)	\bigcup \emptyset \\
																&=		A^c \cap \{ T \leq n \} \in \F_n
			\end{align}
			
			Y por lo tanto $A^c \in \F_T$.\\
			
		\item $\F_T$ es cerrado bajo uniones numerables.
		
			Sean $A_m \in \F_T$ para $m \in \N$.
			\begin{align}
				\bigg( \bigcup_{k=1}^\infty A_k \bigg) \cap \{ T \leq n \} 	&=	\bigcup_{k=1}^\infty \bigg( A_k \cap \{ T \leq n \} \bigg)
			\end{align}								
			
			La última parte, por definición de $\F_T$ es una unión numerable de elementos de $\F_n$ y por lo tanto, dicha unión
			tambien pertenece a $\F_n$. Y por último $\cup A_k \in \F_T$.
	\end{itemize}
	\null
	
	Ahora veamos que $T$ es $\F_T$-medible. Dado un $k \in \N$, ocurre que:
	
	\begin{align}
		\{T \leq k\} \cap \{T \leq n\} = \{T \leq min(k,n)\} \in \F_{min(k, n)} \subset \F_n \forall n \in \N.
	\end{align}

	Y por lo tanto $\{T \leq k\} \in \F_T$.
	
		
	\subsubsection{Inciso (iii)}
	\input{tarea1/problema1_4/inciso3.tex}
	
	\subsubsection{Inciso (iv)} 
	\emph{
	Pruebe que si ${S\leq T\leq n}$ entonces ${\F_S\subset\F_T}$. Si adem\'as ${T}$ es acotado entonces ${M_S,M_T \in L_1}$ y 
	\begin{align}
		\espc{M_T}{\F_S}\leq M_S.
	\end{align}	
}

	Primero probaremos que ${\F_S \subset \F_T}$. Sea $n \in \N$. Notemos que $\{ T \leq n \} \subset \{ S \leq n \}$ pues
	si ${\omega \in \{ T \leq n \}}$, entonces ${S(\omega) \leq T(\omega) \leq n }$ y por lo tanto ${\omega \in \{ S \leq n \}}$.\\

	Ahora sea ${A \in \F_S}$. Entonces

	\begin{align}
		A \cap \{ T \leq n \} 	&=		A  \cap \{ T \leq n \} \cap \{ S \leq n \} \\
								&=		(A  \cap \{ S \leq n \}) \cap \{ T \leq n \} \in \F_n
	\end{align}

	Como escogimos $n$ arbitrariamente, esto es cierto para toda $n \in \N$. Y por lo tanto $A \in \F_T$ y $\F_S \subset \F_T$.\\

	
	\subsubsection{Inciso (v)}
	\emph{
	Si ${X=\paren{X_n,n\in\na}}$ es un proceso estoc\'astico ${\paren{\F_n}}$-adaptado y tal que ${X_n\in L_1}$ y tal que 
	para cualesquiera tiempos de paro acotados ${S}$ y ${T}$ se tiene que ${\esp{X_S}=\esp{X_T}}$ entonces ${X}$ es una 
	martingala. Sugerencia: considere tiempos de paro de la forma ${n\indi{A}+(n+1)\indi{A^c}}$ con ${A\in\F_n}$.
}
\end{proof}