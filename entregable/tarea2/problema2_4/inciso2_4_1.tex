\emph{
	Pruebe que si existen $\lambda_1<0<\lambda_2$ tales que $\imf{\phi}{\lambda_i}<\infty$ entonces $\imf{\phi}{\lambda}<\infty$ 
	para toda $\lambda\in [\lambda_1,\lambda_2]$. Sugerencia: escriba $\lambda=a\lambda_1+(1-a)\lambda_2$ para alg\'un $a\in [0,1]$ 
	y aplique la desigualdad de H\"older. A partir de ahora se asume la premisa de este inciso.
}

\afterstatement\pn

Sea $\lambda = \lambda_1 (1-t) + \lambda_2 t$ con $t \in [0, 1]$. Por lo tanto 

\begin{align}
    \phi(\lambda)   &= \E(e^{\lambda S_n})                                              \\
                    &= \E\paren{e^{(\lambda_1 (1-t) + \lambda_2 t) S_n}}                \\
                    &= \E\paren{e^{(\lambda_1 (1-t) S_n + \lambda_2 tS_n) }}            \\  
                    &= \E\paren{e^{\lambda_1 (1-t) S_n} e^{\lambda_2 tS_n) }}           \\  
                    &= \E\paren{{e^{(\lambda_1 S_n)}}^{1-t} {e^{(\lambda_2 S_n)}}^t}      
\end{align}

Si definimos $p = 1/(1-t)$ y $q = 1/t$, tenemos que
${e^{(\lambda_1 S_n)}}^{1-t} \in L_p$ pues $\paren{{e^{(\lambda_1 S_n)}}^{1-t}}^p = e^{(\lambda_1 S_n)}$ quien 
pertenece a $L_1$ por hipótesis. Análgomanente tenemos que  ${e^{(\lambda_2 S_n)}}^t \in L_q$.\pn

Ahora, recordando que $|e^x| = e^x$ y utilizando la desigualdad de Hölder \eqref{Desigualdad_de_Holder}

\begin{align}
    \phi(\lambda)   &=      \E\paren{{e^{(\lambda_1 S_n)}}^{1-t} {e^{(\lambda_2 S_n)}}^t}           \\      
                    &\leq   \E\paren{e^{(\lambda_1 S_n)}}^{1-t} \E\paren{e^{(\lambda_2 S_n)}}^t     
\end{align}\pn

Donde cada uno de los factores es finito por hipótesis y por lo tanto $\phi(\lambda) < \infty$. Como queríamos demostrar.