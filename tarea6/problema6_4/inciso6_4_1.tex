\emph{
    Explique y pruebe formalmente por qu\'e, para toda 
    $n\geq 1$, $\partial^n M^\lambda_t/\partial \lambda^n$ es una martingala.\pn
}

\afterstatement\pn

Analicemos las primeras derivadas.

\begin{align}
    \frac{d}{d\lambda}  M_t^\lambda &=  \frac{d}{d\lambda} \left( e^{\lambda B_t - \lambda^2 t/2} \right)                       \\
                                    &=  \frac{d}{d\lambda} \left( e^{\lambda B_t} e^{- \lambda^2 t/2} \right)                   \\
                                    &=  B_t e^{\lambda B_t} e^{- \lambda^2/2}  - \lambda t e^{\lambda B_t} e^{- \lambda^2 t/2}  \\
                                    &=  B_t M_t^\lambda - \lambda t M_t^\lambda                                                 \\
                                    &=  M_t^\lambda (B_t - \lambda t).                                                          \\
\end{align}

La de segundo orden
\begin{align}
    \frac{d^2}{d\lambda^2}  M_t^\lambda     &=  \frac{d}{d\lambda} \left( M_t^\lambda (B_t - \lambda t) \right)                     \\
                                            &=  M_t^\lambda (B_t - \lambda t) (B_t - \lambda t) -  t M_t^\lambda                    \\ 
                                            &=  M_t^\lambda \left[ (B_t - \lambda t)^2 -  t \right]                                 \\    
                                            &=  t M_t^\lambda \left[ (B_t - \lambda t)^2 (t^{-1/2})^2 -  1 \right]                       
\end{align}\pn

La de tercer orden
\begin{align}
    &   \;\;\;\;\frac{d^3}{d\lambda^3}  M_t^\lambda                                                                                                                                                         \\
    &=  \frac{d}{d\lambda} \left(t M_t^\lambda  \left[ (B_t - \lambda t)^2 (t^{-1/2})^2 -  1 \right] \right)                                                                                                \\ 
    &=  t \frac{d}{d\lambda} \left( M_t^\lambda  \left[ (B_t - \lambda t)^2 (t^{-1/2})^2 -  1 \right] \right)                                                                                               \\
    &=  t   M_t^\lambda \left[ (B_t - \lambda t)^3 (t^{-1/2})^2 -  (B_t - \lambda t) \right]    +   M_t^\lambda \frac{d}{d\lambda} \left[ (B_t - \lambda t)^2 (t^{-1/2})^2 -  1 \right]                  \\
    &=  t   M_t^\lambda \left(\left[ (B_t - \lambda t)^3 (t^{-1/2})^2 -  (B_t - \lambda t) \right]  +  \frac{d}{d\lambda} \left[ (B_t - \lambda t)^2 (t^{-1/2})^2 -  1 \right]\right)                  \\
    &=  t   M_t^\lambda \left(\left[ (B_t - \lambda t)^3 (t^{-1/2})^2 -  (B_t - \lambda t) \right]  +  \left[ -2t(B_t - \lambda t) (t^{-1/2})^2 \right]\right)                  \\
    &=  t   M_t^\lambda \left(\left[ (B_t - \lambda t)^3 (t^{-1/2})^2 -  (B_t - \lambda t) \right]  +  \left[ -2(B_t - \lambda t) \right]\right)                  \\
    &=  t   M_t^\lambda \left[ (B_t - \lambda t)^3 (t^{-1/2})^2 -  3(B_t - \lambda t) \right]                    \\
    &=  t \, t^{1/2} t^{-1/2}  M_t^\lambda \left[ (B_t - \lambda t)^3 (t^{-1/2})^2 -  3(B_t - \lambda t) \right]                   \\
    &=  t^{3/2}   M_t^\lambda \left[ (B_t - \lambda t)^3 (t^{-1/2})^3 -  3 (t^{-1/2})(B_t - \lambda t) \right]                    \\
\end{align}\pn

(Estamos haciendo un poquito de trampa para facilitar el inciso siguiente). Piénsese en el factor 
que se encuentra entre corchetes como un polinomio en $(B_t - \lambda t) (t^{-1/2})$ (en la parte 
anterior, se trata de un polinomio de grado $3$, con sólo los términos de grado $3$ y $1$, donde 
los coeficientes son $1$ y $3$ respectivamente).\pn

Para facilitar la notación, definimos $B_t^\lambda = (B_t - \lambda t) (t^{-1/2})$ y sólo hay 
que tener presente que $\frac{d}{d\lambda} B_t^\lambda = -t^{1/2}$ y en general, que 
$\frac{d}{d\lambda} (B_t^\lambda)^n = -t^{1/2} n (B_t^\lambda)^{n-1}$.\pn

Con esta nueva notación, tenemos
\begin{align}
    \frac{d}{d\lambda}  M_t^\lambda     &= t^{1/2} M_t^\lambda \left[ B_t^\lambda \right]                       \\
    \frac{d^2}{d\lambda^2}  M_t^\lambda &= t M_t^\lambda \left[ (B_t^\lambda)^2 - 1  \right]                    \\
    \frac{d^3}{d\lambda^3}  M_t^\lambda &= t^{3/2} M_t^\lambda \left[ (B_t^\lambda)^3 - 3 B_t^\lambda  \right]  \\
\end{align}\pn

Hagamos la siguiente hipótesis de inducción. Para cierta $n \geq 1$, se tiene que 
\begin{align}
    \frac{d^n}{d\lambda^n} M_t^\lambda = t^{n/2} M_t^\lambda \left[ r_n (B_t^\lambda)^n + r_{n-1} (B_t^\lambda)^{n-1} + \dots + r_1 B_t^\lambda + r_0 \right]
\end{align}
Donde $r_0, r_1, \dots, r_n \in \R$ y $r_n \neq 0$.\pn

Derivemos una vez más para ver qué ocurre con la derivada de orden $n+1$
\tiny
\begin{align}
    & \;\;\;\;\;\frac{d^{n+1}}{d\lambda^{n+1}} M_t^\lambda                                                                                                                                                                                      \\
    &= t^{n/2} \frac{d}{d \lambda} \left( M_t^\lambda \left[ r_n (B_t^\lambda)^n + \dots + r_1 B_t^\lambda + r_0 \right] \right)                                                                                                                \\
    &= t^{n/2}  M_t^\lambda \left( t^{1/2} \left[ r_n (B_t^\lambda)^{n+1}  + \dots + r_1 (B_t^\lambda)^2 + r_0 B_t^\lambda \right] -t^{1/2} \left[ r_n n (B_t^\lambda)^{n-1} + \dots + r_1 \right]\right)                                       \\
    &= t^{(n+1)/2}  M_t^\lambda \left( \left[ r_n (B_t^\lambda)^{n+1}  + \dots + r_1 (B_t^\lambda)^2 + r_0 B_t^\lambda \right] - \left[ r_n n (B_t^\lambda)^{n-1} + \dots + r_1 \right]\right)                                                  \\
    &= t^{(n+1)/2}  M_t^\lambda \left( \left[ r_n (B_t^\lambda)^{n+1}  + \dots + (r_{m-2} - r_{m})(B_t^\lambda)^{m-1} + \dots + r_1 (B_t^\lambda)^2 + r_0 B_t^\lambda \right]\right) 
\end{align}\pn
\normalsize

En resumen, tenemos la siguiente relación recursiva:
\tiny
\begin{align}\label{problema6_4:equivalencia_recursiva}
    &\;\;\;\;\;\frac{d^{n+1}}{d\lambda^{n+1}} M_t^\lambda     \\
    &= t^{(n+1)/2}  M_t^\lambda \left( \left[ r_n (B_t^\lambda)^{n+1}  + \dots + (r_{m-2} - r_{m})(B_t^\lambda)^{m-1} + \dots + r_1 (B_t^\lambda)^2 + r_0 B_t^\lambda \right]\right) 
\end{align}\pn
\normalsize

Una variable aleatoria con distribución normal es acotada casi seguramente y 
por lo tanto tiene momentos de todos los órdenes. Por el mismo argumento, el valor 
absoluto de una variable aleatoria con distribución normal tiene momentos de todos los
órdenes.

