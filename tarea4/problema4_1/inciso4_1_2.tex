\emph{
		Pruebe que $\F_1$ y $\F_2$ son condicionalmente independientes dada $\G$ 
		(denotado $\condind{\F_1}{\F_2}{\G}$) si y s\'olo si para 
		cualquier $H$ que sea $\F_1$-medible y acotada se tiene que
		\begin{esn}
			\espc{H}{\F_2,\G}=\espc{H}{\G}.
		\end{esn}
}

\afterstatement\par\null

Supongamos que $\condind{\F_1}{\F_2}{\G}$ y, sea $H$ una función $\F_1$-medible.\par\null

Sea $P = \{ G \cap F : G \in \G, F \in \F_2 \}$. Es claro ver que este conjunto no es vacío (el total pertenece a él) y
es fácil notar que es cerrado bajo intersecciones finitas. En otras palabras $P$ es un $\pi$-sistema.\par\null

Sea ahora $D = \{ A \in \sigma(\G, \F_2) : \E(H \indic_A) = \E(\E(H | \G) \indic_A)  \}$. Veamos que $D$ así definido
es un $\lambda$-sistema (o sistema de Dynkin).

\begin{itemize}
	\item 
        Es claro que $\Omega \in D$.

    \item
        Veamos que $D$ es cerrado bajo suceciones crecientes.\par\null

        Sean $A_1, A_2, A_3, \dots \in D$ tales que $A_i \subset A_{i+1}$. Entonces

        \begin{align}
         \E\paren{H \indic_{\bigcup A_i}}  &=  \lim_{n \rightarrow \infty}    \E(H \indic_{A_n})             \\
                                        &=  \lim_{n \rightarrow \infty}    \E(\E( H | \G) \indic_{A_n})   \\
                                        &=  \E\paren{\E( H | \G) \indic_{\bigcup A_i}}.
        \end{align}

        Es decir, $\bigcup A_i \in D$.

    \item
        Sean $A, B \in D$ con $B \subset A$. Veamos ahora que $D$ es cerrado bajo diferencias de este tipo de conjuntos. 

        \begin{align}
                \E(H \indic_{(A \setminus B)}   &=  \E(H (\indic_A - \indic_B))                                                 \\
                                                &\lcomment{Esta descomposición es válida gracias}                               \\
                                                &\rcomment{ a la contención de $B$ en $A$}                                      \\
                                                &=  \E(H \indic_A) - \E(H \indic_B)                                             \\
                                                &=  \E(\E(H | \G) \indic_A) - \E(\E(H | \G) \indic_B)                           \\
                                                &\comment{Hipótesis de que $A, B \in D$}                                        \\
                                                &=  \E(E(H | \G) \indic_{A \setminus B})
        \end{align}

        Es decir, $A \setminus B \in D$.
\end{itemize}

Con esto queda demostrado que $D$ es un sistema de Dynkin. Si $P \subset D$ entonces el lema de clases de Dynkin asegura que
$\sigma(P) \subset D$ y entonces habríamos terminado la demostración de la suficiencia.

