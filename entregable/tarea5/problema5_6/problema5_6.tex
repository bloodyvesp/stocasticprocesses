\begin{problema}[Procesos de ramificaci\'on a tiempo continuo]
    Sea $\mu$ una distribuci\'on en $\na$. A $\mu_k$ lo interpretamos como la probabilidad de 
    que un individuo tenga $k$ hijos. Nos imaginamos la din\'amica de la poblaci\'on como sigue: 
    a tasa $\lambda$, los individuos de una poblaci\'on se reproducen. Entonces tienen $k$ hijos 
    con probabilidad $\mu_k$. Se pueden introducir dos modelos: uno en que el individuo que se 
    reproduce es retirado de la poblaci\'on (nos imaginamos que muere) y otro en que no es retirado 
    de la poblaci\'on (por ejemplo cuando se interpreta a la poblaci\'on como especies y a sus 
    descendientes como mutaciones). En el caso particular del segundo modelo en que $\mu_1=1$, 
    se conoce como proceso de Yule. 

    \begin{enumerate}
        \item[(i)]      [\ref{problema5_6:inciso1}]
            Especifique un modelo de cadenas de Markov a tiempo continuo para cada uno de los
            modelos anteriores. A estos procesos se les conoce como procesos de ramificaci\'on
            a tiempo continuo.\pn
    \end{enumerate}

        Nuestro primer objetivo ser\'a encontrar una relaci\'on entre procesos de ramificaci\'on a
        tiempo continuo y procesos de Poisson compuestos. Sea $N$ un proceso de Poisson  y $S$ una
        caminata aleatoria independiente de $N$ tal que $\proba{S_1=j}=\mu_{j-1}$ \'o $\mu_{j}$
        dependiendo de si estamos en el primer caso o en el segundo. Sea $k\geq 0$ y definamos a
        $X_t=k+S_{N_t}$.
        
    \begin{enumerate}[resume]
        \item[(ii)]     [\ref{problema5_6:inciso2}] 
            Diga brevemente por qu\'e $X$ es una cadena de Markov a tiempo continuo e identifique 
            su matriz infinitesimal para ambos modelos.\pn
    \end{enumerate}

        Sea ahora $\tau=\min\set{t\geq 0: X_t=0}$ y $Y_t=X_{t\wedge \tau}$. 
        
    \begin{enumerate}[resume]
        \item[(iii)]    [\ref{problema5_6:inciso3}] 
            Argumente por qu\'e $Y$ es una cadena de Markov a tiempo continuo e identifique su 
            matriz infinitesimal.\pn
            
        \item[(iv)]     [\ref{problema5_6:inciso4}] 
            Argumente por qu\'e existe un \'unico proceso $Z$ que satisface
            \begin{esn}
                Z_t=Y_{\int_0^t Z_s\, ds}
            \end{esn}
            y que dicho proceso es un proceso de ramificaci\'on a tiempo continuo. Sugerencia: Recuerde que las 
            trayectorias de $Y$ son constantes por pedazos.\pn
    \end{enumerate}

    Ahora nos enfocaremos en el proceso de Yule. 

    \begin{enumerate}[resume]
        \item[(v)]      [\ref{problema5_6:inciso5}]
            Escriba las ecuaciones backward de Kolmogorov para las probabilidades de transici\'on 
            $\imf{P_t}{x,y}$. Al argumentar por qu\'e $\imf{P_{t}}{x,x}=e^{-\lambda x}$, resuelva 
            las ecuaciones backward por medio de la t\'ecnica de factor integrante (comenzando con 
            $\imf{P_t}{x,x+1}$) y pruebe que
            \begin{esn}
                \imf{P_t}{x,y}=\binom{y-1}{y-x} e^{-\lambda x t}\paren{1-e^{-\lambda t}}^{y-x}.
            \end{esn}\pn
        
        \item[(vi)]     [\ref{problema5_6:inciso6}]
            Al utilizar la f\'ormula para la esperanza de una variable binomial negativa, 
            pruebe que
            \begin{esn}
                \imf{\se_x}{Z_t}= xe^{\lambda t}.
            \end{esn}\pn
        
        \item[(vii)]    [\ref{problema5_6:inciso7}]
            Pruebe que $e^{-\lambda t}Z_t$ es una martingala no-negativa y que por lo tanto 
            converge casi seguramente a una variable aleatoria $W$.\pn
        
        \item[(viii)]   [\ref{problema5_6:inciso8}]
            Al calcular la transformada de Laplace de $e^{-\lambda t}Z_t$, pruebe que $W$ tiene 
            distribuci\'on exponencial. Por lo tanto, argumente que casi seguramente $Z$ crece exponencialmente.
            %La distribuci�n l�mite est� tomada de Beroin-Goldschmidt, ellos citan y corrigen un error de Athreya.
            \pn
    \end{enumerate}
\end{problema}

\begin{proof}
    \subsection{Inciso (i)} \label{problema5_6:inciso1}
    \emph{
    Especifique un modelo de cadenas de Markov a tiempo continuo para cada uno de los
    modelos anteriores. A estos procesos se les conoce como procesos de ramificaci\'on
    a tiempo continuo.\pn
}
\afterstatement\pn

Para el primer caso, en el que los padres mueren, notemos que cualquier población es posible de un paso al 
siguiente. Si $i$ es la cantidad de individuos en un momento determinado, cada individuo puede tener como
descendencia cualquier cantidad de individuos (mayor o igual que cero). Sin embargo, si $i$ es cero, el proceso
se estaciona porque ya no existen individuos que dejen descendencia. Entonces la matriz de tasas de cambio $\alpha$
la podemos expresar como:

\begin{align}
        \alpha(i,j)  &=
                \begin{cases}
                    0                   \;\; \text{si $i$ = 0 ó si $j - i < -1$}          \\
                    \lambda i \mu_{k+1} \;\; \text{si $j - i = k$}
                \end{cases}
\end{align}\pn

Donde $\mu_1 = 0$, puesto que tener un individuo como descendencia y morir no afecta la población.
Notemos que esta definición hace que $\sum_j \alpha(0,j) = 0$, es decir, el estado ``población 0'' es
absorbente. Justo como necesitamos.\pn

Con esto ahora podemos definir $P$ como:
\begin{align}
        P(i,j)  = 
            \begin{cases}
                \frac{\alpha(i,j)}{\sum_{j'}\alpha(i, j')} = \mu_{j-i+1}  \;\; \text{si $\sum_{j'}\alpha(i, j') \not= 0$} \\
                1                                                         \;\; \text{en cualquier otro caso}
            \end{cases}
\end{align}

Donde por comodidad de notación hacemos $\mu_{j-i+1} = 0$ si $j - i + 1< 0$.\pn

Para el segundo caso, donde nadie muere y puede tener cualquier cantidad de descendientes, la población nunca tiende a decrecer. 
Entonces la matriz de tasas de cambio $\alpha$ se puede expresar como:

\begin{align}
        \alpha(i,j)  &=
                \begin{cases}
                    0                   \;\; \text{si $i$ = 0 ó si $j < i$}          \\
                    \lambda i \mu_{k}   \;\; \text{si $j - i = k > 0$}
                \end{cases}
\end{align}\pn

Donde hacemos $\mu_0 = 0$, puesto que no tener descendientes no afecta a la población. Entonces podemos calcular $P$ de la misma manera que antes
\begin{align}
        P(i,j)  = 
            \begin{cases}
                \frac{\alpha(i,j)}{\sum_{j'}\alpha(i, j')} = \mu_{j-i+1}  \;\; \text{si $\sum_{j'}\alpha(i, j') \not= 0$} \\
                1                                                         \;\; \text{en cualquier otro caso}
            \end{cases}
\end{align}

En las notas, en la sección 3 ``Matrices infinitesimales y construcción de procesos de Markov'' del capítulo 5 
``Procesos de Markov constantes por pedazos'' (véase \ref{notas}), se explica como a partir de las tasas de cambio se puede
construir una cadena de Markov a tiempo continuo con dichas tasas. Aplicándo dicho método, conseguimos las cadenas de Markov que buscamos.

    \newpage

    \subsection{Inciso (ii)} \label{problema5_6:inciso2}
    \emph{
    Diga brevemente por qu\'e $X$ es una cadena de Markov a tiempo continuo e identifique 
    su matriz infinitesimal para ambos modelos.\pn
}
\afterstatement\pn
    \newpage

    \subsection{Inciso (iii)} \label{problema5_6:inciso3}
    \emph{
    Argumente por qu\'e $Y$ es una cadena de Markov a tiempo continuo e identifique su 
    matriz infinitesimal.\pn
}
\afterstatement\pn
	\newpage
	
    \subsection{Inciso (iv)} \label{problema5_6:inciso4}
    \emph{
    Argumente por qu\'e existe un \'unico proceso $Z$ que satisface
    \begin{esn}
        Z_t=Y_{\int_0^t Z_s\, ds}
    \end{esn}
    y que dicho proceso es un proceso de ramificaci\'on a tiempo continuo. Sugerencia: Recuerde que las 
    trayectorias de $Y$ son constantes por pedazos.\pn
}

\afterstatement\pn

Definamos $T_n$ como en [\ref{problema5_6:inciso1}]. Nos interesa ver cómo se comporta $Y_t$ en los intervalos
$[T_{n-1}, T_{n})$, pues en estos intervalos $Y_t$ es constante.\pn

Analizemos el primer segmento, sea $t \in [0, T_{1})$, donde la población no cambia, tenemos que 
$int_{0}^{t} Y_t = tk$. Supongamos que $Z_t = k$ tenemos que la integral de $\int_{0}^{t} Z_s ds = kt$ y entonces
si $t \in [0, \frac{T_1}{k})$, tenemos que $\int_{0}^{t} Z_s ds < T_1$ y entonces $Y_{\int_{0}^{t} Z_s ds} = k = Z_t$. 
Entonces, al menos para $t \in [0, \frac{T_1}{k})$, $Z_t = k$. Por continuidad de la integral, para
$t \in [0, \frac{T_1}{k})$, cualquier otro porceso $Z$ que cumpla con la misma condición, es casi seguramente igual a 
$k$.\pn

Analizemos el siguiente segmento, sea $t \in [T_{1}, T_{2})$. Aquí $Y_t = k + S_1$. Queremos que $\int_{T_1}^t Z_s \,ds$
recorra todos los valores entre $T_1$ y $T_2$. Es decir 

\begin{align}
        T_1 \leq \int_{0}^t k + \int_{\frac{T_1}{k}}^t S_1 < T_2
\end{align}\pn

Que resulta equivalente a pedir:
\begin{align}
        \frac{T_1}{k} \leq t < \frac{T_1}{k} + \frac{T_2 - T_1}{k + S_1}
\end{align}\pn

Es decir que basta definir a $Z_t = k + S_1$ cuando $t \in [\frac{T_1}{k}, \frac{T_1}{k} + \frac{T_2 - T_1}{k + S_1})$ para
que $Z$ se comporte como queremos. De nuvo, la continuidad de la integral nos dice que cualquier otra función que cumpla
la condición, es casi seguramente igual a $k + S_1$ dentro del intervalo mencionado.\pn

Generalizando esta idea, si $t \in [T_{n-1}, T_n)$, definimos $I_n = \sum_{0 \leq i}^{n-1} \frac{T_{i+1} - T_{i}}{k + S_i}$, 
donde $T_0 = S_0 = I_0 = 0$ por comodidad de notación. Para que $Z_t = Y_{\int_0^t Z_s \, ds}$ necesitamos
\begin{align}
    T_{n-1} \leq  \sum_{0 \leq i}^{n-2} \int_{I_i}^{I_{i+1}} k + S_i + \int_{I_{n-1}}^{t} k + S_{n-1} < T_n
\end{align}\pn

Después de muchas cuentas, esto significa que
\begin{align}
         I_{n-1} \leq t < I_{n}
\end{align}\pn

E igual que antes, si $t \in [I_{n-1}, I_{n})$ se cumple con la condición $Z_t = \int_{0}^t Z_s \,ds$ se satisface. Otra vez
argumentando la continuidad de la integral, cualquier otro proceso que cumpla lo que $Z$, es casi seguramente igual a $Z$.\pn

Ahora, por ser $N$ un proceso de Poisson, $I_{n} - I_{n-1}$ resulta tener distribución exponencial y en estos intervalos
$Z$ es constante. Así que $Z$ es una cadena de Markov a tiempo continuo. Y entonces es un proceso de ramificación.
    \newpage
    
    \subsection{Inciso (v)} \label{problema5_6:inciso5}
    \emph{
    Escriba las ecuaciones backward de Kolmogorov para las probabilidades de transici\'on 
    $\imf{P_t}{x,y}$. Al argumentar por qu\'e $\imf{P_{t}}{x,x}=e^{-\lambda x}$, resuelva 
    las ecuaciones backward por medio de la t\'ecnica de factor integrante (comenzando con 
    $\imf{P_t}{x,x+1}$) y pruebe que
    \begin{esn}
        \imf{P_t}{x,y}=\binom{y-1}{y-x} e^{-\lambda x t}\paren{1-e^{-\lambda t}}^{y-x}.
    \end{esn}\pn
}
\afterstatement\pn
    \newpage

    \subsection{Inciso (vi)} \label{problema5_6:inciso6}
    \emph{
    Al utilizar la f\'ormula para la esperanza de una variable binomial negativa, 
    pruebe que
    \begin{esn}
        \imf{\se_x}{Z_t}= xe^{\lambda t}.
    \end{esn}\pn
}
\afterstatement\pn
    \newpage

    \subsection{Inciso (vii)} \label{problema5_6:inciso7}
    \emph{
    Pruebe que $e^{-\lambda t}Z_t$ es una martingala no-negativa y que por lo tanto 
    converge casi seguramente a una variable aleatoria $W$.\pn
}
\afterstatement\pn
	\newpage
	
    \subsection{Inciso (viii)} \label{problema5_6:inciso8}
    \emph{
    Al calcular la transformada de Laplace de $e^{-\lambda t}Z_t$, pruebe que $W$ tiene 
    distribuci\'on exponencial. Por lo tanto, argumente que casi seguramente $Z$ crece exponencialmente.
    %La distribuciÑn lÕmite està tomada de Beroin-Goldschmidt, ellos citan y corrigen un error de Athreya.
    \pn
}
\afterstatement\pn

La transformada de Laplace de una distribución binomial negativa de parámetros $p$, $r$ está dada por
\begin{align}
        \E(e^{u Z_t})   &=  \left( \frac{p e^u}{1-(1-p)e^u}\right)^r
\end{align}
[véase \href{http://mathworld.wolfram.com/NegativeBinomialDistribution.html}{propiedades de la distribución binomial negativa}]\pn

En nuestro caso, $r = x$ y $p = e^{-\lambda t}$, suponiendo $x=1$ como población inicial.\pn

Tenemos entonces
\begin{align}
        \E(e^{u Z_t})   &=  \left( \frac{e^{-\lambda t} e^u}{1-(1-e^{-\lambda t})e^u}\right)^1       \\
                        &=  \frac{e^{-\lambda t} e^u}{1-(1-e^{-\lambda t})e^u}                       \\ 
                        &=  \frac{e^{-\lambda t}}{e^{-u}-(1-e^{-\lambda t})}                         
\end{align}\pn

Hacemos un cambio de variable $u = v e^{-\lambda t}$ y entonces obtenemos
\begin{align}
    \E(e^{v e^{-\lambda t} Z_t})    &=  \frac{e^{-\lambda t}}{e^{-v e^{-\lambda t}}-(1-e^{-\lambda t})}
\end{align}\pn

De donde
\begin{align}
    \lim{t \rightarrow \infty}   \E(e^{v e^{-\lambda t} Z_t})   &= \lim{t \rightarrow \infty}  \frac{e^{-\lambda t}}{e^{-v e^{-\lambda t}}-(1-e^{-\lambda t})}  \\
                                                                &= \lim{s \rightarrow 0}  \frac{s}{e^{-v s}-(1-s)}                                              \\                                                                         
                                                                &\lcomment{se hizo un cambio de variable $s = e^{-\lambda t}$, el}                              \\
                                                                &\rcomment{ cual tiende a 0 conforme $t$ a infinito}                                            \\
                                                                &= \lim{s \rightarrow 0}  \frac{1}{-v e^{-v s}+1}                                               \\                                                                          
                                                                &\comment{se aplicó L'Hopital, pues quedaba un límite del estilo $\frac{0}{0}$}                 \\
                                                                &= \frac{1}{1-v}                                                                                \\                                                                          
\end{align}

Lo cual corresponde a la función característica de una exponencial de parámetro $1$ y por lo tanto $W$ tiene distribución exponencial.
\end{proof}