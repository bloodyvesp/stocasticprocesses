\begin{problema}(Ejercicios sueltos sobre martingalas)\par\null
    \begin{enumerate}
        \item[(i)]        [\ref{problema2_3:inciso1}]
            Sea $\paren{X_n,n\geq 0}$ una sucesi\'on $\paren{\F_n}$-adaptada.\par
            Pruebe que        
            \begin{align}
                \sum_{k=1}^n X_k-\espc{X_k}{\F_{k-1}}, \quad n\geq 0
            \end{align}
            es una $\paren{\F_n}$-martingala.\par\null
        
        \item[(ii)]        [\ref{problema2_3:inciso2}]
            Descomposici\'on de Doob para submartingalas:\par
            Sea $X=\paren{X_n}_{n\in\na}$ una submartingala. 
            Pruebe que $X$ se puede descomponer de manera \'unica como $X=M+A$ donde $M$ es una martingala y $A$ 
            es un proceso previsible con $A_0=0$. Sugerencia: Asuma que ya tiene la descomposici\'on y calcule 
            esperanza condicional de $X_{n+1}$ dada $X_n$.\par\null 
        
        \item[(iii)]    [\ref{problema2_3:inciso3}]
            Sea $S_n=\xi_1+\cdots+\xi_n$ donde las variables $\xi$ son independientes y $\xi_i$ tiene 
            media cero y varianza finita $\sigma_i^2$. Pruebe que si $\sum_i \sigma_i^2<\infty$ entonces 
            $S_n$ converge casi seguramente y en $L_2$ conforme $n\to\infty$. Construya un ejemplo de 
            variables aleatorias $\xi_i$ tales que la serie $\sum_i \xi_i$ sea casi seguramente absolutamente 
            divergente y casi seguramente condicionalmente convergente (considere ejemplos simples!). 
            Explique heur\'isticamente por qu\'e cree que suceda esto.\par\null
            %Ser\'a que \sum_i\abs{x_i}=\infty casi seguramente si \sum_i\abs\esp{\xi_i}=\infty? 
        
        \item[(iv)]        [\ref{problema2_3:inciso4}]
            Sean $X$ y $Y$ dos martingalas (respecto de la misma filtraci\'on) y tales que $\esp{X_i},\esp{Y_i}<\infty$ 
            para toda $i$. Pruebe la siguiente f\'ormula de integraci\'on por partes: 
            \begin{align}
                \esp{X_nY_n}-\esp{X_0Y_0}=\sum_{i=1}^n \esp{\paren{X_i-X_{i-1}}\paren{Y_i-Y_{i-1}}}. 
            \end{align}\par\null
        
        \item[(v)]        [\ref{problema2_3:inciso4}]
            Desigualdad de Azema-Hoeffding, tomado de \cite[E14.2, p.237]{MR1155402}
            \begin{enumerate}
                \item[(v.i)] [\ref{problema2_3:subinciso5_1}]
                        Muestre que si $Y$ es una variable aleatoria con valores en $[-c,c]$ y 
                        media cero entonces, para $\theta\in\re$
                        \begin{align}
                            \esp{e^{\theta Y}}\leq\imf{\cosh}{\theta c}\leq \imf{\exp}{\frac{1}{2}\theta^2c^2}. 
                        \end{align}\par\null
                \item[(v.ii)] [\ref{problema2_3:subinciso5_2}]
                        Pruebe que si $M$ es una martingala nula en cero tal que para algunas 
                        constantes $\paren{c_n,n\in\na}$ se tiene que
                        \begin{align}
                            \abs{M_n-M_{n-1}}\leq c_n\quad\forall n
                        \end{align}
                        entonces, para $x>0$
                        \begin{align}
                            \proba{\max_{k\leq n} M_k\geq x}\leq \imf{\exp}{\frac{x^2}{2\sum_{k=1}^n c_k^2}}.
                        \end{align}\par\null
            \end{enumerate}
    \end{enumerate}
\end{problema}

\begin{proof}
    \subsection{Inciso (i)}        \label{problema2_3:inciso1}
    \emph{
    Sea $\paren{X_n,n\geq 0}$ una sucesi\'on $\paren{\F_n}$-adaptada. Pruebe que
    \begin{align}
        \sum_{k=1}^n X_k-\espc{X_k}{\F_{k-1}}, \quad n\geq 0
    \end{align}
    es una $\paren{\F_n}$-martingala.
}

\afterstatement\par\null

Nombremos $M_n = \sum_{k=1}^n X_k-\espc{X_k}{\F_{k-1}}, n \in \N$.\par\null

\begin{itemize}
	\item 
        Veamos que $(M_n)_{n \in \N}$ es $(\F_n)_{n \in \N}$-adaptada.\par\null
    
        $\sum_{k = 1}^n X_k$ es $\F_n$-medible por definición. $\E(X_k | \F_{k-1})$ es $\F_{k-1}$-medible por definición
        de esperanza condicional. \par\null
        
        Por lo tanto $\sum_{k = 1}^n \E(X_k | \F_{k-1})$ es $\F_{n-1}$-medible y por lo tanto
        $\F_n$-medible. Entonces
        
        \begin{align}
            \sum_{k = 1}^n X_k - \sum_{k = 1}^n \E(X_k | \F_{k-1}) &=  \sum_{k = 1}^n X_k - \E(X_k | \F_{k-1})    \\
                                                                    &=  M_n   
        \end{align}\par\null
    
        Es $\F_n$-medible. Como queríamos ver.\par\null
        
    \item
        Veamos que $M_n \in L_1$.\par\null
        
        Como no fue mencionado. Supondremos que $X_n \in L_1$. Si logramos demostrar que $\E(X_k | F_{k-1}) \in L_1$ habremos terminado.
        Puesto que $M_n$ sería suma finita de variables aleatorias en $L_1$ y por lo tanto pertenecería a $L_1$.\par\null
        
        \begin{align}
            \E(\abs{\E(X_k | F_{k-1})})     &\leq   \E(\E(\abs{X_k} | F_{k-1})) \\
                                            &=      \E(\abs{X_k}). 
        \end{align}
        
        Donde $\E(\abs{X_k})$ por supocisión es finita.\par\null
        
        Por lo tanto $\E(\abs{\E(X_k | F_{k-1})})$ también lo es, como queríamos demostrar.\par\null

\end{itemize}
    \newpage
    
    \subsection{Inciso (ii)}    \label{problema2_3:inciso2}
    \emph{
    Descomposici\'on de Doob para submartingalas: Sea $X=\paren{X_n}_{n\in\na}$ una submartingala. 
    Pruebe que $X$ se puede descomponer de manera \'unica como $X=M+A$ donde $M$ es una martingala y $A$ 
    es un proceso previsible con $A_0=0$. Sugerencia: Asuma que ya tiene la descomposici\'on y calcule 
    esperanza condicional de $X_{n+1}$ dada $X_n$. 
}

\afterstatement\par\null

$\paren{X_n}_{n\in\na}$ es $(\F_n)_{n \in \N}$-adaptada por definición de submartingala. Utilizando el inciso anterior tenemos
que $M_n = \sum_{k=1}^n X_k-\espc{X_k}{\F_{k-1}}, n \in \N$ es martingala.\par\null

%Ahora sea $M_n = N_n + X_0$. Veamos que $(M_n)_{n \in \N}$ sigue siendo martingala.
%
%\begin{itemize}
	%\item 
        %Veamos que $(M_n)_{n \in \N}$ es $(\F_n)_{n \in \N}$-adaptada.\par\null
        %
        %Por ser $(N_n)_{n \in \N}$ una martingala, $N_n$ es $\F_n$-medible y $X_0$ es 
        %$\F_0 \subset \F_n$-medible. Entonces $M_n$ es suma finita de variables $\F_n$-medibles y por lo tanto $\F_n$-medible.\par\null
    %
    %\item 
        %Veamos que $M_n \in L_1$ para toda $n \in \N$.\par\null
        %
        %Por ser $(N_n)_{n \in \N}$ una martingala, $N_n \in L_1$ para todo $n \in \N$. 
        %Por ser $(X_n)_{n \in \N}$ una subpartingala, $X_0 \in  L_1$.\par\null
        %
        %Por lo tanto $M_n$ es suma finita de variables en $L_1$ y por lo tanto pertenece a $L_1$.\par\null
    %
    %\item
        %Veamos que $\E(M_n | F_{n-1}) = M_{n-1}$.\par\null
        %
        %\begin{align}
            %\E(M_n | F_{n-1})   &=  \E(N_n + X_0 | F_{n-1})                     \\
                                %&=  \E(N_n | F_{n-1}) + \E( X_0 | F_{n-1})      \\
                                %&=  N_{n-1} + X_0                               \\
                                %&=  M_{n-1}.
        %\end{align}
        %
        %Y con esto terminamos de demostrar que $(M_n)_{n \in \N}$ es martingala.
%\end{itemize}

Si definimos a $A_n = X_n - M_n$ tendremos que $M_n + A_n = M_n + X_n - M_n = X_n$. Probemos ahora que
$(A_n)_{n \in \N}$ es previsible.\par\null

\begin{align}
    A_n     &=      X_n - M_n                                                                   \\
            &=      X_n - \sum_{k=1}^n X_k-\espc{X_k}{\F_{k-1}}                                 \\
            &=      X_n - X_n + \E(X_n | \F_{n-1}) + \sum_{k=1}^{n-1} X_k-\E(X_k|\F_{k-1})      \\     
            &=      \E(X_n | \F_{n-1}) + \sum_{k=1}^{n-1} X_k-\E(X_k|\F_{k-1}).     
\end{align}\par\null

Donde por ser $(X_n){n \in \N}$ submartingala tenemos que $\E(X_k|\F_{k-1})$ es $F_{n-1}$ siempre que $0 \leq k \leq n-1$.
Y por lo tanto $A_n$ es suma finita de variables $F_{n-1}$-medibles y por lo tanto $F_{n-1}$-medible.\par\null

Hemos demostrado hasta ahora que una submartingala se puede descomponer como la suma de una martingala y un proceso previsible.
Demostremos ahora la unicidad.\par\null

Sean entonces $M'_n$ y $A'_n$ una martingala y un proceso previsible respectivamente tal que $X_n = M'_n + A'_n$.\par\null

Entonces para toda $n \in \N$ tenemos que $M'_n + A'_n = M_n + A_n$. Por lo tanto 

\begin{align}
    (M'_n + A'_n) - (M'_{n+1} + A'_{n+1}) = (M_n + A_n) - (M_{n+1} + A_{n+1})
\end{align}\par\null

Tomemos esperanzas condicionales sobre el lado derecho.

\begin{align}
                                                    & \E((M_n + A_n) -( M_{n+1} + A_{n+1}) | \F_n)                                              \\ 
                                                    &=  \E((M_n + A_n) -( M_{n+1} + A_{n+1}) | \F_n)                                            \\
                                                    &=  \E(M_n - M_{n+1}| \F_n) + \E(A_n - A_{n+1} | \F_n)                                      \\
                                                    &=  M_n - M_{n} + \E(A_n - A_{n+1} | \F_n) \label{problema2_3:hipotesis_de_martingala}      \\
                                                    &=  \E(A_n - A_{n+1} | \F_n)                                                                \\
                                                    &=  \E(A_n| \F_n) - \E(A_{n+1} | \F_n)                                                      \\
                                                    &=  A_n - A_{n+1} \label{problema2_3:hipotesis_de_proceso_previsible}                                                                               
\end{align}\par\null

Donde \eqref{problema2_3:hipotesis_de_martingala} es gracias a que $(M_n)_{n\in\N}$ es martingala y \eqref{problema2_3:hipotesis_de_proceso_previsible}
es gracias a que $(A_n)_{n\in\N}$ es un proceso previsible.\par\null

De manera análoga obtenemos que $\E((M'_n + A'_n) - (M'_{n+1} + A'_{n+1}) | \F_n) = A'_n - A'_{n+1}$ y por lo tanto $A_n - A_{n+1} = A'_n - A'_{n+1}$.\par\null

Lo que esto último dice, es que $(A_n)_{n\in\N}$ y $(A'_n)_{n\in\N}$ crecen igual. Recordemos que por hipótesis $A'_0, A_0 = 0$.\par\null

Entonces
\begin{align}
    A_0 - A_1 &= A'_0 - A'_1 &\Rightarrow A_1 &= A'_1
\end{align}\par\null

Inductivamente, si $A_n = A'_n$, entonces
           
\begin{align}
    A_n - A_{n+1} &= A'_n - A'_{n+1} &\Rightarrow A_{n+1} &= A'_{n+1}
\end{align}                                                                                                         

Y con esto queda demostrada la unicidad de $(A_n)_{n\in\N}$. Para la demostración de la unicidad de $M_n$, sólo recordemos que 
$M_n = X_n - A_n$ y que $M'_n = X_n - A'_n$, por ser $A_n$ y $A'_n$ iguales, entonces $M_n$ y $M'_n$ también lo son.
    \newpage
        
    \subsection{Inciso (iii)}    \label{problema2_3:inciso3}
    \emph{
    Sea $S_n=\xi_1+\cdots+\xi_n$ donde las variables $\xi$ son independientes y $\xi_i$ tiene 
    media cero y varianza finita $\sigma_i^2$. Pruebe que si $\sum_i \sigma_i^2<\infty$ entonces 
    $S_n$ converge casi seguramente y en $L_2$ conforme $n\to\infty$. Construya un ejemplo de 
    variables aleatorias $\xi_i$ tales que la serie $\sum_i \xi_i$ sea casi seguramente absolutamente 
    divergente y casi seguramente condicionalmente convergente (considere ejemplos simples!). 
    Explique heur\'isticamente por qu\'e cree que suceda esto.
    %Ser\'a que \sum_i\abs{x_i}=\infty casi seguramente si \sum_i\abs\esp{\xi_i}=\infty? 
}

\afterstatement\pn

Sea $\F_n = \sigma( \xi_0, \xi_1, \dots, \xi_n)$.\pn


Probemos ahora que $(S_n)_{n \in \N}$ es martingala con respecto de la \newline 
filtración $(\F_n)_{n \in \N}$.\pn

Por como se definió $\F_n$, $S_n$ es $\F_n$-medible. Por otra parte, $S_n$ es suma finita de variables en $L_1$ 
(pues las $\xi_i$'s tienen esperanza finita).\pn

Ahora
\begin{align}
    \E(S_{n+1} | \F_n)  &=  \E(\xi_{n+1} | \F_n) + \E(S_n | \F_n)                       \\
                        &=  \E(\xi_{n+1} | \F_n) + S_n                                  \\
                        &\;\;\;\;\mbox{Porque $S_n$ es $F_n$-medible}                   \\
                        &=  \E(\xi_{n+1}) + S_n                                         \\
                        &\comment{Porque $\xi_{n+1}$ es independiente de$F_n$}        	\\
                        &=  0 + S_n.                                                    \\
\end{align}\pn

Y con esto hemos demostrado que $S_n$ es martingala.\pn

Ahora, como $\E(\xi_n) = 0$ para todo $n \in \N$, $\E(S_n) = 0$ para todo $n \in \N$.

\begin{align}
        \var{S_n}       &=  \E(S_n^2) - \E(S_n)^2                                               \\
                        &=  \E(S_n^2) - 0                                                       \\
                        &=  \sum_{i \leq n} \sigma_i^2                                          \\
                        &\comment{Esto último gracias a que los $\xi_i$'s son independientes}.
\end{align}

Ahora $\| S_n \|_2 = \sqrt{\var{S_n}} = \sqrt{\sum_{i \leq n} \sigma_i^2}$. Recordemos que por ser  $\sqrt{\cdot}$
función continua, manda sucesiones convergentes en sucesiones convergentes. Como 
$\lim\limits_{n \rightarrow \infty} \sum\limits_{i < n} \sigma_i^2 < \infty$, 
entonces $\lim\limits_{n \rightarrow \infty}\| S_n \|_2 < \infty$.\pn

En particular, $\sup\limits_n \| S_n \|_2 < \infty$.\pn

El teorema de convergencia en $L_p$ de Doob nos garantiza que $S_n$ converge casi seguramente y en $L_2$. Como queríamos
hacer ver.\pn

Definamos ahora $\xi_i$ con $\mw(\xi_i = \pm \frac{1}{i+1}) = \frac{1}{2}$.

Dado que la serie $\sum_{n \geq 1} \frac{1}{n}$ es divergente. $S_n$ es absolutamente divergente casi seguramente.\pn

Ahora, $\E(\xi_i) = \frac{1}{2(i+1)}-\frac{1}{2(i+1)} = 0$ y $\var{\xi_i} = \frac{1}{(i+1)^2}$. Entonces, por lo dicho en la primera
parte de este ejercicio, $S_n$ converge casi seguramente y en $L_2$. Es decir que la serie $S_n$ converge condicinalmente casi seguramente.\pn

Heurísticamente, para que la serie diverja es necesario mantener el mismo signo por periodos ``muy largos''. La probabilidad de mantener el 
mismo signo por mucho tiemp decrece exponencialmente. Por eso es razonable que si elegimos en base a volados los signos de los términos,
la serie terminará convergiendo.
    \newpage
    
    \subsection{Inciso (iv)}    \label{problema2_3:inciso4}
    \emph{
    Sean $X$ y $Y$ dos martingalas (respecto de la misma filtraci\'on) y tales que $\esp{X_i},\esp{Y_i}<\infty$ 
    para toda $i$. Pruebe la siguiente f\'ormula de integraci\'on por partes: 
    \begin{align}
        \E{X_nY_n}-\E{X_0 Y_0}=\sum_{i=1}^n \E{\paren{X_i-X_{i-1}}\paren{Y_i-Y_{i-1}}}. 
    \end{align}
}

\afterstatement\pn

Descompongamos $\sum_{i \leq n} \E((X_i - X_{i-1}) (Y_i - Y_{i-1}))$

\begin{align}
    &\sum_{1 \leq i \leq n} \E((X_i - X_{i-1}) (Y_i - Y_{i-1}))                                                                         \\
    &=  \sum_{1 \leq i \leq n} \E(X_i Y_i - X_i Y_{i-1} - X_{i-1} Y_i + X_{i-1} Y_{i-1})                                                \\
    &=  \sum_{1 \leq i \leq n} \E(X_i Y_i) - \E(X_i Y_{i-1}) - \E(X_{i-1} Y_i) + \E(X_{i-1} Y_{i-1})                                    \\
    &=  \sum_{1 \leq i \leq n} \E(X_i Y_i) - \E(\E(X_i Y_{i-1}) | \F_{i-1}) - \E(\E(X_{i-1} Y_i) | \F_{i-1}) + \E(X_{i-1} Y_{i-1})      \\
    &=  \sum_{1 \leq i \leq n} \E(X_i Y_i) - \E(Y_{i-1} \E(X_i | \F_{i-1})) - \E(X_{i-1}\E(Y_i | \F_{i-1})) + \E(X_{i-1} Y_{i-1})       \\
    &\comment{Esto último gracias a que $X_{i-1}, Y_{i-1}$ son $F_{i-1}$ medibles}                                                      \\
    &=  \sum_{1 \leq i \leq n} \E(X_i Y_i) - \E(Y_{i-1} X_{i-1}) - \E(X_{i-1} Y_{i-1}) + \E(X_{i-1} Y_{i-1})                            \\
    &\comment{Esto último gracias a que $X,Y$ son martingalas}                                                                          \\
    &=  \sum_{1 \leq i \leq n} \E(X_i Y_i) - \E(Y_{i-1} X_{i-1})                                                                        
\end{align}\pn

En esta última suma podemos notar que se trata de una telescópica, y por lo tanto

\begin{align}
    \sum_{1 \leq i \leq n} \E((X_i - X_{i-1}) (Y_i - Y_{i-1})) &=  \E(X_n Y_n) - \E(Y_{0} X_{0}).
\end{align}\pn

Que es precisamente lo que buscábamos demostrar.
    \newpage
    
    \subsection{Inciso (v)}        \label{problema2_3:inciso5}
    Desigualdad de Azema-Hoeffding, tomado de \par
\cite[E14.2, p.237]{MR1155402}\par\null

\begin{enumerate}
    \item[(v.i)]    [\ref{problema2_3:subinciso5_1}]
         Muestre que si $Y$ es una variable aleatoria con valores en $[-c,c]$ y media cero entonces, para $\theta\in\re$
        
        \begin{align}
            \esp{e^{\theta Y}}\leq\imf{\cosh}{\theta c}\leq \imf{\exp}{\frac{1}{2}\theta^2c^2}. 
        \end{align}\par\null

    \item[(v.ii)]    [\ref{problema2_3:subinciso5_2}]
        Pruebe que si $M$ es una martingala nula en cero tal que para algunas constantes $\paren{c_n,n\in\na}$ se tiene que
        
        \begin{align}
            \abs{M_n-M_{n-1}}\leq c_n\quad\forall n
        \end{align}
        
        entonces, para $x>0$
        
        \begin{align}
            \proba{\max_{k\leq n} M_k\geq x}\leq \imf{\exp}{\frac{x^2}{2\sum_{k=1}^n c_k^2}}.
        \end{align}
\end{enumerate}
    
\subsubsection{Subinciso (v.1)}     \label{problema2_3:subinciso5_1}
    \emph{
     Muestre que si $Y$ es una variable aleatoria con valores en $[-c,c]$ y media cero entonces, para $\theta\in\re$
    \begin{align}
        \esp{e^{\theta Y}} \leq \cosh(\theta c)\leq e^{(\frac{1}{2}\theta^2c^2)}. 
    \end{align}
}

\afterstatement\pn

Notemos primero que la función $e^{\theta y}$ es convexa. Esto es fácil viendo que su segunda derivada es
$\theta^2 e^{\theta y} \geq 0$.\pn

Ahora
\begin{align}
    Y   &=  \frac{2cY}{2c}                              \\
        &=  \frac{cY + cY}{2c}                          \\
        &=  \frac{c^2 + cY - c^2+ cY}{2c}               \\
        &=  \frac{c(c + Y) - c (c - Y)}{2c}             \\  
        &=  c\frac{(c + Y)}{2c} - c \frac{(c - Y)}{2c}  
\end{align}\pn

De donde $\theta Y = c\theta \frac{(c + Y)}{2c} - c\theta \frac{(c - Y)}{2c}$.\pn

Entonces
\begin{align}
    e^{\theta Y} = e^{\paren{c\theta \frac{(c + Y)}{2c} - c\theta \frac{(c - Y)}{2c}}}
\end{align}\pn

Ya que $\abs{Y} \leq c$ tenemos que $0 \leq c + Y \leq 2c$ y $0 \leq c - Y \leq 2c$. Y por lo tanto 
$0 \leq \frac{c + Y}{2c} \leq 1$ y $0 \leq \frac{c - Y}{2c} \leq 1$. 
Además $\frac{c + Y}{2c} + \frac{c - Y}{2c} = \frac{c + Y + c - Y }{2c} = 1$. Entonces podemos utilizar 
la convexidad de $e^{\theta y}$ sobre los puntos $-c$ y $c$ para obtener lo siguiente

\begin{align}
    e^{\theta Y} \leq e^{c\theta} \frac{(c + Y)}{2c} + e^{-c\theta} \frac{(c - Y)}{2c}.
\end{align}\pn

Tomando esperanza en ambos lados obtenemos y recordando que $\cosh{x} = \frac{e^x + e^{-x}}{2}$

\begin{align}
    \E(e^{\theta Y})    &\leq   \E(e^{c\theta} \frac{(c + Y)}{2c} + e^{-c\theta} \frac{(c - Y)}{2c})            \\
                        &=      e^{c\theta} \frac{(c + \E(Y))}{2c} + e^{-c\theta} \frac{(c - \E(Y))}{2c}        \\
                        &=      e^{c\theta} \frac{c}{2c} + e^{-c\theta} \frac{c}{2c}                            \\
                        &=      e^{c\theta} \frac{1}{2} + e^{-c\theta} \frac{1}{2}                              \\                       
                        &=      \frac{e^{c\theta} + e^{-c\theta}}{2}                                            \\
                        &=      \cosh(c \theta).
\end{align}\pn

Con lo que terminamos de demostrar la primera desigualdad.\pn

Para la segunda desigualdad recordemos que la expansión en serie de Taylor de $e^x$ es

\begin{align}
    e^x     &=  \sum_{n \in \N} \frac{x^{n}}{n!}.   \label{Expansion_de_taylor_de_e}
\end{align}\pn

De donde 

\begin{align}
    e^{-x}  &=  \sum_{n \in \N} \frac{-x^{n}}{n!}           \\
            &=  \sum_{n \in \N} (-1)^{n}\frac{x^{n}}{n!}.   
\end{align}\pn

Sumando estos dos y dividendo entre 2

\begin{align}
    \cosh(x)    &=  \frac{e^x + e^{-x}}{2}                                                  \\
                &=  \sum_{n \in \N} \frac{\frac{x^{n}}{n!} + (-1)^{n}\frac{x^{n}}{n!}}{2}   \\
                &=  \sum_{n \in \N} \frac{x^{2n}}{(2n)!}. \label{problema2_3:expansion_en_serie_de_taylor}
\end{align}\pn

Ahora notemos lo siguiente.

\begin{align}
    2^0(0!) = 1     &\leq  1     = (2 \cdot 0)!      \\
    2^1(1!) = 2     &\leq  2     = (2 \cdot 1)!      \\
    2^2(2!) = 8     &\leq  24    = (2 \cdot 2)!      \\
    2^3(3!) = 48    &\leq 720    = (2 \cdot 3)!      
\end{align}\pn

Apliquemos inducción sobre esto. Supongamos que para cierto $n$ ocurre que $2^n(n!) \leq (2n)!$. Como $n \geq 0$, 
entonces $2 \leq 2(n + 1)$ y $n + 1\leq 2n+1$ , por lo tanto $2 (n+1) \leq (2n+1) (2n + 2)$. 
De donde 

\begin{align}
    2^{n+1}(n+1)! = 2(n+1) 2^n(n!) \leq (2n)! (2n+1)(2n+2) = (2(n+1))!.
\end{align}\pn

Con esto, podemos acotar \eqref{problema2_3:expansion_en_serie_de_taylor} de la siguiente manera

\begin{align}
    \cosh(x)    &=      \sum_{n \in \N} \frac{x^{2n}}{(2n)!}            \\
                &\leq   \sum_{n \in \N} \frac{x^{2n}}{2^n(n)!}          \\
                &=   \sum_{n \in \N} \frac{(x^2)^{n}}{2^n(n)!}          \\
                &=   \sum_{n \in \N} \frac{(\frac{x^2}{2})^{n}}{(n)!}   \\   
                &=   e^{\frac{x^2}{2}}   
\end{align}

Y utilizando esto tenemos que $\cosh(c \theta) \leq e^{\frac{c^2 \theta^2}{2}}$.



    \newpage
    
\subsubsection{Subinciso (v.ii)}    \label{problema2_3:subinciso5_2} 
    \emph{
	Pruebe que si $M$ es una martingala nula en cero tal que para algunas constantes $\paren{c_n,n\in\na}$ se tiene que
	\begin{align}
		\abs{M_n-M_{n-1}}\leq c_n\quad\forall n
	\end{align}
	entonces, para $x>0$
	\begin{align}
		\proba{\max_{k\leq n} M_k\geq x}\leq \imf{\exp}{\frac{x^2}{2\sum_{k=1}^n c_k^2}}.
	\end{align}
}


\end{proof}