\begin{problema}[Extensiones del teorema de paro opcional]
	Sea $M=\paren{M_n,n\in\na}$ una (super)martingala respecto de una filtraci\'on $\paren{\F_n,n\in\na}$ y sean $S$ y $T$ tiempos de paro.
	
\begin{enumerate}
                \item[(i)] 
                	Pruebe que $S\wedge T$, $S+T$ y $S\vee T$ son tiempos de paro.
                
                \item[(ii)] 
                	\begin{esn}
                		\F_T=\set{A\in\F:A\cap\set{T\leq n}\in\F_n\text{ para toda } n}
                	\end{esn}
                	es una $\sigma$-\'algebra, a la que nos referimos como la $\sigma$-\'algebra 
                	detenida en $\tau$. Comente qu\'e puede fallar si $T$ no es tiempo de paro. 
                	Pruebe que $T$ es $F_T$-medible. 
                
                \item[(iii)] 
                	Pruebe que si $T$ es finito, entonces $M_T$ es $\F_T$-medible.
                
                \item[(iv)] 
                	Pruebe que si $S\leq T\leq n$ entonces $\F_S\subset\F_T$. Si adem\'as $T$ es acotado entonces $X_S,X_T\in L_1$ y 
                	\begin{esn}
	                	\espc{M_T}{\F_S}\leq M_S.
                	\end{esn}

                \item[(v)] 
                	Si $X=\paren{X_n,n\in\na}$ es un proceso estoc\'astico $\paren{\F_n}$-adaptado y tal que $X_n\in L_1$ y tal que 
                	para cualesquiera tiempos de paro acotados $S$ y $T$ se tiene que $\esp{X_S}=\esp{X_T}$ entonces $X$ es una 
                	martingala. Sugerencia: considere tiempos de paro de la forma $n\indi{A}+(n+1)\indi{A^c}$ con $A\in\F_n$.
\end{enumerate}

\defin{Categor\'ias: }Tiempos de paro, Muestreo opcional
\end{problema}

\begin{proof}
                \subsubsection{(i)} 
                	Pruebe que $ S \wedge T $, $ S + T $ y $ S \vee T$ son tiempos de paro.\\
                
                	\begin{enumerate}
                		\item 
                			Comprobemos que:
                			\begin{align}
                				\{ S \wedge T \leq n \} = \{ T \leq n \} \cup \{ S \leq n\}.
                			\end{align}
                			
							Sea $\omega \in \{ S \wedge T \leq n \}$. Entonces $T(\omega) \leq n$ ó $S(\omega) \leq n$.
							Por lo tanto:
							\begin{align}
		                			\{ S \wedge T \leq n \} \subset \{ T \leq n \} \cup \{ S \leq n\}.							
							\end{align}
							
							Por otro lado, si $\omega \in \{ T \leq n \} \cup \{ S \leq n\}$ entonces $T(\omega) \leq n$ ó $S(\omega) \leq n$.
							En particular, el mínimo tendrá que ser menor que $n$ y por lo tanto:
							\begin{align}
		                			\{ T \leq n \} \cup \{ S \leq n\} \subset \{ S \wedge T \leq n \}.						
							\end{align}
							
							Por último, $\{ T \leq n \} \in \F_n$ y $\{ S \leq n \} \in \F_n$. Por lo tanto	\\	
							$\{ T \leq n \} \cup \{ S \leq n\} = \{ S \wedge T \leq n \} \in \F_n$ y con esto demostramos que 
							$ S \wedge T$ es tiempo de paro.\\
							
						\item
							Comprobemos que:
                			\begin{align}
                				\{ S \vee T \leq n \} = \{ T \leq n \} \cap \{ S \leq n\}.
                			\end{align}
                			
							Sea $\omega \in \{ S \vee T \leq n \}$. Entonces $T(\omega) \leq n$ y $S(\omega) \leq n$.
							Por lo tanto:
							\begin{align}
		                			\{ S \vee T \leq n \} \subset \{ T \leq n \} \cap \{ S \leq n\}.							
							\end{align}
							
							Por otro lado, si $\omega \in \{ T \leq n \} \cap \{ S \leq n\}$ entonces $T(\omega) \leq n$ y $S(\omega) \leq n$.
							En particular, el máximo tendrá que ser menor que $n$ y por lo tanto:
							\begin{align}
		                			\{ T \leq n \} \cap \{ S \leq n\} \subset \{ S \vee T \leq n \}.						
							\end{align}
							
							Por último, $\{ T \leq n \} \in \F_n$ y $\{ S \leq n \} \in \F_n$. Por lo tanto	\\	
							$\{ T \leq n \} \cap \{ S \leq n\} = \{ S \vee T \leq n \} \in \F_n$ y con esto demostramos que 
							$ S \vee T$ es tiempo de paro.\\
							
						\item
							Comprobemos que:
							\begin{align}
								\{ S + T = n \} = \bigcup_{i = 1}^{n-1} \bigg( \{S = n - i\} \cap \{T = i\} \bigg).
							\end{align}
							
							Si $\omega \in \{ S + T = n \}$, entonces $S(\omega) + T(\omega) = n$, como $S$ y $T$ son positivas,
							entonces $1 \leq S(\omega) \leq n-1$. Entonces basta elegir $i = S(\omega)$ para afirmar que 
							$\omega \in \{S = n - i\} \cap \{T = i\}$ y por lo tanto
							\begin{align}
							\{ S + T = n \} \subset \bigcup_{i = 1}^{n-1} \bigg( \{S = n - i\} \cap \{T = i\} \bigg).
							\end{align}
							
							Por otro lado, si $\omega \in \bigcup_{i = 1}^{n-1} \bigg( \{S = n - i\} \cap \{T = i\} \bigg)$
							significa que existe un $i$ tal que $1\leq i \leq n-1$ y $\omega \in \{S = n - i\} \cap \{T = i\}$.
							Y por lo tanto: $S(\omega) = n-i$ y $T(\omega) = i$. De donde $(T + S)(\omega) = n$.
							
							De aquí que
							\begin{align}
								\bigcup_{i = 1}^{n-1} \bigg( \{S = n - i\} \cap \{T = i\} \bigg) \subset \{ S + T = n \}. 
							\end{align}
							
							Ahra, para cada $i$ tal que $1 \leq i \leq n-1$, $\{S = n - i\} \cap \{T = i\} \in \F_n$.
							Por lo tanto $\{ S + T = n\} \in \F_n$ y con esto queda demostrado que $ S + T $ es tiempo de paro.
                	\end{enumerate}
                	
                \subsubsection{(ii)}
                	\begin{align}
                		\F_T=\set{A\in\F:A\cap\set{T\leq n}\in\F_n\text{ para toda } n}
                	\end{align}
                	es una $\sigma$-\'algebra, 
                	a la que nos referimos como la $\sigma$-\'algebra detenida en $\tau$. Comente qu\'e puede fallar si $T$ no es tiempo de paro. 
                	Pruebe que $T$ es $F_T$-medible. 
                
                \subsubsection{(iii)}
                	Pruebe que si $T$ es finito, entonces $M_T$ es $\F_T$-medible.
                
                \subsubsection{(iv)} 
                	Pruebe que si $S\leq T\leq n$ entonces $\F_S\subset\F_T$. Si adem\'as $T$ es acotado entonces $X_S,X_T\in L_1$ y 
                	\begin{esn}
	                	\espc{M_T}{\F_S}\leq M_S.
                	\end{esn}

                \subsubsection{(v)}
                	Si $X=\paren{X_n,n\in\na}$ es un proceso estoc\'astico $\paren{\F_n}$-adaptado y tal que $X_n\in L_1$ y tal que 
                	para cualesquiera tiempos de paro acotados $S$ y $T$ se tiene que $\esp{X_S}=\esp{X_T}$ entonces $X$ es una 
                	martingala. Sugerencia: considere tiempos de paro de la forma $n\indi{A}+(n+1)\indi{A^c}$ con $A\in\F_n$.
\end{proof}