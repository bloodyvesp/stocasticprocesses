
\begin{problema}[Ejercicios sueltos sobre martingalas]
	\begin{enumerate}
		\item 
			Sea $\paren{X_n,n\geq 0}$ una sucesi\'on $\paren{\F_n}$-adaptada. Pruebe que
			
			\begin{esn}
				\sum_{k=1}^n X_k-\espc{X_k}{\F_{k-1}}, \quad n\geq 0
			\end{esn}
			
			es una $\paren{\F_n}$-martingala.
		
		\item
			Descomposici\'on de Doob para submartingalas: Sea \(X=\paren{X_n}_{n\in\na}\) una submartingala. 
			Pruebe que $X$ se puede descomponer de manera \'unica como $X=M+A$ donde $M$ es una martingala y $A$ 
			es un proceso previsible con $A_0=0$. Sugerencia: Asuma que ya tiene la descomposici\'on y calcule 
			esperanza condicional de $X_{n+1}$ dada $X_n$. 
		
		\item 
			Sea \(S_n=\xi_1+\cdots+\xi_n\) donde las variables \(\xi\) son independientes y \(\xi_i\) tiene 
			media cero y varianza finita \(\sigma_i^2\). Pruebe que si \(\sum_i \sigma_i^2<\infty\) entonces 
			\(S_n\) converge casi seguramente y en \(L_2\) conforme \(n\to\infty\). Construya un ejemplo de 
			variables aleatorias \(\xi_i\) tales que la serie \(\sum_i \xi_i\) sea casi seguramente absolutamente 
			divergente y casi seguramente condicionalmente convergente (considere ejemplos simples!). 
			Explique heur\'isticamente por qu\'e cree que suceda esto.
			%Ser\'a que \sum_i\abs{x_i}=\infty casi seguramente si \sum_i\abs\esp{\xi_i}=\infty? 
		
		\item 
			Sean \(X\) y \(Y\) dos martingalas (respecto de la misma filtraci\'on) y tales que \(\esp{X_i},\esp{Y_i}<\infty\) 
			para toda \(i\). Pruebe la siguiente f\'ormula de integraci\'on por partes: 
			
			\begin{esn}
				\esp{X_nY_n}-\esp{X_0Y_0}=\sum_{i=1}^n \esp{\paren{X_i-X_{i-1}}\paren{Y_i-Y_{i-1}}}. 
			\end{esn}
		
		\item Desigualdad de Azema-Hoeffding, tomado de \cite[E14.2, p.237]{MR1155402}
			\begin{enumerate}
				\item Muestre que si \(Y\) es una variable aleatoria con valores en \([-c,c]\) y media cero entonces, para \(\theta\in\re\)
						$$\esp{e^{\theta Y}}\leq\imf{\cosh}{\theta c}\leq \imf{\exp}{\frac{1}{2}\theta^2c^2}. $$
				\item Pruebe que si \(M\) es una martingala nula en cero tal que para algunas constantes \(\paren{c_n,n\in\na}\) se tiene que
						$$\abs{M_n-M_{n-1}}\leq c_n\quad\forall n $$
						entonces, para \(x>0\)
						$$
						\proba{\max_{k\leq n} M_k\geq x}\leq \imf{\exp}{\frac{x^2}{2\sum_{k=1}^n c_k^2}}.
						$$
			\end{enumerate}
	\end{enumerate}
\end{problema}