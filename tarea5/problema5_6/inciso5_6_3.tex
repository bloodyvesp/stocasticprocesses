\emph{
    Argumente por qu\'e $Y$ es una cadena de Markov a tiempo continuo e identifique su 
    matriz infinitesimal.\pn
}

\afterstatement\pn

Para el caso en el que los padres mueren. Detener a la martingala en $\tau$, no hace
ninguna diferencia. El proceso ya era absorbente en $0$. Es decir, que si $X_t = 0$,
$X_{t+s} = 0$ y por lo tanto $X_{(t+s) \wedge \tau} = X_t$. Entonces, la matriz infinitesimal
de este caso es

\begin{align}
        Q(i,j)  &=
                \begin{cases}
                    \lambda i \mu_{j-i+1}   &    \text{si $i \leq j+1$, $j \neq i$ e $i \neq 0$}    \\
                    -\lambda i              &    \text{si $i = j$}                                  \\
                    0                       &    \text{en cualquier otro caso}
                \end{cases}
\end{align}

Para el caso en el que los padres no mueren. Notemos que $\tau = \infty$. Puesto que la población
nunca dismiuye. Entonces $t \wedge \tau = t$ y $Y_t = X_t$. Es decir, que la matriz infinitesimal
de $Y$ sería:
\begin{align}
        Q(i,j)  &=
                \begin{cases}
                    \lambda i \mu_{j-i}     &    \text{si $j > i$ e $i > 0$}        \\
                    -\lambda i              &    \text{si $i = j$}                  \\
                    0                       &    \text{en cualquier otro caso}
                \end{cases}
\end{align}