\RequirePackage{pgfornament,tkzexample,tikzrput}  

\documentclass[a5paper,oneside]{amsart}
\usepackage{color}
\definecolor{azul}{RGB}{20,40,140}

\usepackage{courier}

\usepackage{dsfont}
\usepackage{enumitem}
\usepackage[scale={.8,.85}]{geometry}

\usepackage[utf8]{inputenc}
\usepackage[colorlinks,linkcolor=azul, citecolor=azul,urlcolor=azul]{hyperref}

\usepackage{listings}

\usepackage{mathrsfs}
\usepackage{mathtools}







\usepackage{embedfile}




\allowdisplaybreaks
\setlength\parindent{0pt}

\mathtoolsset{showonlyrefs}
\newtheoremstyle{dotless}{}{}{\itshape}{}{\bfseries}{}{ }{}
\theoremstyle{definition}
\renewenvironment{proof}{{\bfseries Demostración:}}

\newtheorem{teorema}{\;\;Teorema}[section]
\newtheorem{problema}[teorema]{\;\;Problema}
\newtheorem{ejercicio}[teorema]{\;\;Ejercicio}
\newtheorem{proposicion}[teorema]{\;\;Proposicion}
\newtheorem{definicion}[teorema]{\;\;Definición}
\newtheorem*{categoria}{Categorías:}
\newtheorem*{remark}{Remark}
\setcounter{secnumdepth}{4}
\setcounter{tocdepth}{4}

\input{definiciones/definitions.tex}
\newcommand{\N}{\mathbb{N}}
\newcommand{\R}{\mathbb{R}}
\newcommand{\E}{\mathbb{E}}
\newcommand{\indic}{\mathds{1}}
\newcommand{\F}{\mathscr{F}}
\newcommand{\textme}[1]{\;\text{#1}\;}
\newcommand{\nqed}{\center{\textbf{---------------------------------------------------------------------------------------}}}



\numberwithin{section}{part}
\numberwithin{equation}{subsection}




\renewcommand{\nqed}
{		
		\begin{tikzpicture}
			\node (A) at (0,0) {};
			\node (B) at (3,0) {};
			\node (C) at (6,0) {};
			\path (A.center) to [ornament=86,at=0] (B.center);
			\path (A.center) to [ornament=85,at=0.50] (B.center);
			\path (A.center) to [ornament=88,at=0.75] (C.center);
    		\path (B.center) to [ornament=85,at=1.50] (C.center);
			\path (B.center) to [ornament=86,at=2] (C.center);
		\end{tikzpicture}\\
}


\title[Tarea de procesos estocásticos 2013-2]{
    Problemas de Procesos Estocásticos\\ 
    Semestre 2013-II\\ 
    Posgrado en Ciencias Matemáticas\\ 
    Universidad Nacional Autónoma de México\\
    Héctor Manuel Téllez Gómez
}

\normalfont
\begin{document}
    \maketitle
    \part{Tarea 1}
        \section{Problema 1.1}
\input{tarea1/problema1_1/problema1_1.tex}
\newpage

\section{Problema 1.2}
\begin{problema}
		Suponga que \(T\) es un tiempo de paro tal que para algún 
		\(N\in\mathbb{N}\) y \(\varepsilon>0\) se tiene que para toda \(n\in\mathbb{N}\):
		
        \begin{equation}\label{problema1_2:hipotesis_del_problema}
		\mathbb{P} (T \leq N + n | F_n) > \varepsilon \text{ casi seguramente}
		\end{equation}
		
        Al verificar la desomposici\'on
		
        \begin{equation}\label{problema1_2:sugerencia_del_problema}
			\mathbb{P} (T>kN)= \mathbb{P} (T>kN,T>(k-1)N)
		\end{equation}
		
        pruebe por inducci\'on que para cada \(k=1,2,\ldots\):
		
        \begin{align}
			\mw(T>kN)\leq (1-\varepsilon)^k.
		\end{align}
		
        Pruebe que \( \mathbb{E}(T)<\infty \). 
	\begin{categoria} Tiempos de paro.\end{categoria}
\end{problema}
\afterstatement
\begin{proof}
	Tenemos que: 
	
	\begin{align}
		\left(T>kN \Rightarrow T>(k-1)N\right) 	&\Rightarrow (T>kN) \subset (T>(k-1)N)          \\ 
                                                &\Rightarrow (T>kN) \cap (T>(k-1)N) = (T>kN)    \\ 
                                                &\Rightarrow \mw(T>kN, T>(k-1)N) = \mw(T>kN)	
	\end{align}\pn
		
	\textbf{Base de inducción.} $k=1$. Usando \eqref{problema1_2:hipotesis_del_problema}, con $n=0$ 
	
    \begin{align}
		\mw(T\leq 1N | \F_0) > \varepsilon\Rightarrow   \\
		\mw(T>N| \F_0) < 1 - \varepsilon
	\end{align}\pn
    
	Sustituyendo por la definición de probabilidad condicional tenemos:

    \begin{align}
        \E(\indic_{T>N} | \F_0)	&= \mw(T>N| \F_0) \\
                                &< 1 - \varepsilon
    \end{align}\pn

	Aplicando esperanza en ambos lados tenemos:
    
    \begin{align} 
        \mw(T>N) 	&= 	\E(\E(\indic_{T>N} | \F_0)) \\
                    &< 	\E(1 - \varepsilon)         \\
                    &= 1 - \varepsilon.
    \end{align}\pn
	
	\textbf{Hipótesis de induccion.} 
    Supongamos que $\mw(T>k_0N)\leq(1 - \varepsilon)^{k_0}$ para algún $k_0 \geq 1$.\pn
	
	\textbf{Paso inductivo.} 
	Utilizando \eqref{problema1_2:sugerencia_del_problema} tenemos que
    
    \begin{align}
        \mw(T>(k_0+1)N) = \mw(T>(k_0+1)N, T>k_0N) = \E(\indic_{T>(k_0+1)N} \cdot \indic_{T>k_0N}).
    \end{align}\pn
        
	Ahora, dado que $T>k_0N$ es un conjunto $\F_{k_0N}$-medible tenemos:
    
	\begin{align} 
		\E(\indic_{T>(k_0+1)N} \cdot \indic_{T>k_0N}) 	&=		\E\left(\E\left(\indic_{T>(k_0+1)N} \cdot \indic_{T>k_0N} | \F_{k_0N}\right)\right)                                     \\ 
														&=		\E\left(\indic_{T>k_0N} \E\left(\indic_{T>(k_0+1)N}|\F_{k_0N}\right)\right) \label{problema1_2:resultado_preliminar}
	\end{align}\pn
    
	Utilizando  $n=k_0N$ en \eqref{problema1_2:hipotesis_del_problema} tenemos
    
	\begin{align}
		\mw\left(T>k_0N+N|\F_{k_0N}\right) = \E\left(\indic_{T>(k_0+1)N}|\F_{k_0N}\right) < 1-\varepsilon
	\end{align}\pn
    
	Sustituyendo esto último en \eqref{problema1_2:resultado_preliminar} obtenemos:
    
    \begin{align}
             E\left(\indic_{T>k_0N} \E\left(\indic_{T>(k_0+1)N}|\F_{k_0N}\right)\right) 	&< 		E\left(\indic_{T>k_0N} (1-\varepsilon)\right)   \\
                                                                                            &=	 	(1-\varepsilon) E\left(\indic_{T>k_0N} \right)  \\
                                                                                            &=		(1-\varepsilon) \mw(T>k_0N)                     \\
                                                                                            &\leq   (1-\varepsilon)(1-\varepsilon)^{k_0}            \\
                                                                                            &= (1-\varepsilon)^{k_0 + 1}.
    \end{align}\pn
        
	Con lo que concluimos
    
    \begin{align}
        \mw(T>(k_0+1)N) &\leq (1-\varepsilon)^{k_0 + 1}.
    \end{align}\pn
        
	Terminando así la demostración por inducción.\pn
	
	
	Para la siguiente prueba notemos que si $X$ es una variable aleatoria y $r > s \in \mathbb{R}$ entonces
	$(X > s) \subset (X > r)$ y por lo tanto $\mw(X > s) \leq \mw(X > r)$.\pn
	
	En particular para nuestro tiempo de paro $T$, si dado $n \in \mathbb{N}$, $k \in \mathbb{N}$ es tal que 
	$kN \geq n < (k+1)N$, entonces $\mw(T > n) \leq \mw(T > kN)$.\pn

	Trabajando por bloques de tamaño $N$ tenemos que 
    
	\begin{align}
		\sum_{n = kN}^{(k+1)N -1} \mw(T > n) \leq N \mw(T > kN).
	\end{align}\pn		
	
	También recordemos que si $T$ es una variable aleatoria positiva con valores en los enteros entonces 
	
    \begin{align}
		\E(T) = \sum_{n=0}^{\infty} \mw(T > n).
	\end{align}\pn
	
	Sustituyendo nuestro penúltimo razonamiento en esta ultima fórmula tenemos:
	
	\begin{align}
		\E(T) 	&= 		\sum_{n=0}^{\infty} \mw(T > n)                              \\
				&= 		\sum_{k=0}^{\infty} \sum_{n = kN}^{(k+1)N -1} \mw(T > n)    \\
				&\leq 	N \sum_{k=0}^{\infty} \mw(T > kN)
	\end{align}\pn
		
	Sustituyendo nuetstro resultado de que $\mw(T>kN)<(1-\varepsilon)^{k}$ resulta que:
    
	\begin{align}
		\E(T)\leq N \sum_{k=0}^{\infty} \mw(T > kN) \leq N \sum_{k=0}^{\infty} (1-\varepsilon)^k.
	\end{align}\pn
	
	De lado derecho de esta última desigualdad tenemos una serie geométrica. Dado que $\varepsilon$ es mayor 
	que $0$ y entonces que $1-\varepsilon$ es menor que $1$, tenemos que es una serie geométrica que converge 
	en los reales y por lo tanto $\E(T)$ está acotada por un número real.
\end{proof}	
\newpage

\section{Problema 1.3}
\begin{problema}
	\emph{Tomado de Mathematical Tripos, Part III, Paper 33, 2012, \url{http://www.maths.cam.ac.uk/postgrad/mathiii/pastpapers/}}

	Sean $\paren{X_i,i\in\na}$ variables aleatorias 
	independientes con $\proba{X_i=\pm 1}=1/2$. Sean $S_0=0$ y $S_n=\sum_{i=1}^n X_i$. 

	\begin{enumerate}
		\item[(i)] Sea $T_1=\min\set{n\geq 0:S_n=1}$. Explique por qu\'e $T_1$ es un 
		tiempo de paro y calcule su esperanza.
		
		\item[(ii)] Mediante el inciso anterior, construya una martingala que converge 
		casi seguramente pero no lo hace en $L_1$.
		
		\item[(iii)] Sea $M_n$ la martingala obtenida al detener a $-S$ en $T_1$. Utilice la solución al
		Problema de la Ruina para probar que $\mw(max_n M_n \geq M) = 1/(M+1)$ para todo $M \geq 1$. Concluya que
		$\E(max_m M_n) = \infty$ y que por lo tanto $\E(max_{m\leq n} M_m) \rightarrow \infty$ conforme 
		$n \rightarrow \infty$. Finalmente, deduzca que no puede haber una desigualdad de tipo Doob cuando $p=1$.
		
		\item[(iv)] Sea $T=\min\set{n\geq 2:S_n=S_{n-2} + 2}$ y $U=T-2$. ?`Son $T$ y $U$ 
		tiempos de paro? Justifique su respuesta.
		
		\item[inciso (v)] Para la variable $T$ que hemos definido, calcule $\esp{T}$. 
	\end{enumerate}

	\defin{Categor\'ias: } Tiempos de paro, problema de la ruina
\end{problema}
\begin{proof}
	\subsubsection{Inciso (i)}
	\emph
{
	Sea $T_1=\min\set{n\geq 0:S_n=1}$. Explique por qu\'e $T_1$ es un 
	tiempo de paro y calcule su esperanza.\\
}
\afterstatement
	Consideremos a la filtración $(\F_n)_{n\in\N}$ como la filtracion 
	generada por $X_1, X_2, \dots$.\par\null

	Es decir, $F_0 = \{\emptyset, \Omega\}$, $\F_n = \sigma(X_1, X_2, \dots, X_n)$\\

	Nótese que $S_0$ es medible bajo cualquier sigma álgebra por ser constante, en particular bajo
	$\F_0$.\par\null

	Basta demostrar que $(T_1 = n) \in \F_n$ para ver que $T_1$ es tiempo de paro. $T_1$ 
	representa el primer tiempo en que la suma es igual a $1$. Es decir, para cualquier 
	momento anterior, la suma no es $1$.\par\null

	Eso escrito en símbolos significa:

    \begin{align}
        (T_1 = n) = \bigcap_{i=0}^{n-1}(S_i \not= 1) \cup (S_n = 1).
    \end{align}\par\null

	Para $n=0$, $(S_0 = 1) = \omega \in \F_0$.\par\null

	Como $(S_i \not= 1) \in \F_j$ siempre que $i \leq j$. Para $n>0$, $(T_1 = n)$ es el resultado de 
	unir e intersectar conjuntos $\F_n$-medibles, lo cual resulta $\F_n$-medible.\par\null

	Para $m \in \N$. Definamos $T_m = min\{n \geq 0 : S_n = m\}$ 
	(Nótese que para el caso $m=1$, esta definición	coincide con la definición previa de $T_1$).\par\null
	
	Para $a,b \in \N$, podemos definir el tiempo de paro $T_{a,b} = T_{-a} \wedge T_b$, y 
	corresponde al 	tiempo de paro del problema de la ruina. Para este tiempo de paro ya conocemos 
	la esperanza y es
	
    \begin{align}
        \E(T_{a,b}) = ab.
    \end{align}\par\null
	
	Ahora, definamos la sucesion de variables aleatorias $T_{1,1}, T_{2,1}, T_{3,1}, \dots, T_{n,1}, 
	\dots$. Notemos que si $a>a' \in N$ entonces $T_{-a} > T_{-a'}$, pues $T_{-a}$ es la primera vez
	que se llega a $-a$, y para poder alcanzar $-a$ era necesario haber pasado por $-a'$.
	De aqui tenemos que si $a>a'$, entonces $T_{a,1} \geq T_{a',1}$. De donde nuestra suceción es 
	no decreciente.\par\null
	
	Por otro lado, que si $a>a' \in N$ entonces $T_{-a} > T_{-a'}$ implica que $T_{-n} n \in \N$ es 
	una suceción extrictamente creciente y por lo tanto 
	$\lim\limits_{n \rightarrow \infty} T_{-n} = \infty$, con esto tenemos que el límite de nuestra 
	suceción es 
	
    \begin{align}		
        \lim_{n\rightarrow\infty} T_{n,1}   &=  \lim_{n\rightarrow\infty} T_{-n} \wedge T_1 \\
                                            &=  \infty \wedge T_1                           \\
                                            &=  T_1
	\end{align}

	Tenemos todos los ingredientes para usar Teorema de convergencia monótona sobre nuestra suceción
	y la variable $T_1$. Nuestra sucecion es monótona y converge puntualmente a $T_1$. Utilizando
	dicho teorema obtenemos:
	
	\begin{align}
        \E(T_1)     &=  \E(\lim_{n\rightarrow\infty} T_{n,1})       \\
                    &=  \lim_{n\rightarrow\infty} \E(T_{n,1})       \\
                    &=  \lim_{n\rightarrow\infty} n\cdot 1          \\
                    &=  \infty.
	\end{align}
	
	Lo cual era intuitivo. Si $\E(T_1)$ fuese finito, diría que existe un número de volados donde
	uno puede apostar con mucha certeza que ganará un peso después de jugar "cerca" de esa cantidad
	de volados. Intuitivamente, esto vuelve injusto un juego de volados donde la moneda es
	justa.	
	\newpage
	
	\subsubsection{Inciso (ii)}
	\emph
{	
	Mediante el inciso anterior, construya una martingala que converge 
	casi seguramente pero no lo hace en $L_1$.\\
}
\afterstatement	
	En [\ref{problema1_4}] se probará que si $T$ y $S$ son tiempos de paro, entonces $T\wedge S$ también 
	es tiempo de paro. Con esto tenemos que si $T_1$ es tiempo de paro, entonces $T_1 \wedge n$ con 
	$n \in \N$ también es tiempo de paro. Definimos entonces 
	$$M_n = S_{T_1 \wedge n}.$$
	
	Veamos que los $M_n$ forman una martingala.
	
	\begin{itemize}
		\item 
			$M_n$ es adaptada a la filtración.
			\begin{align}
				M_n(w) = S_{T_1 \wedge n}(w) = 
				S_{T_1 \wedge n (w)}(w) = 
				\sum_{k=1}^{T_1 \wedge n (w)} X_k = 
				\sum_{k=1}^{n} (X_k \cdot \indic_{T_1 \geq k})(w).
			\end{align}
			
			De donde, podemos escribir:
			\begin{align}\label{problema1_3:descomposicion_de_M_n}
				M_n = \sum_{k=1}^{n} (X_k \cdot \indic_{T_1 \geq k}).
			\end{align}\pn							 		
			
			Recordemos que $X_k$ es $\F_n$-medible para toda $k \leq n $. Por ser
			$T_1$ tiempo de paro, los conjuntos $A_k = \{T_1 = k\}$ y 
			$B_k = \{T_1 \leq k\}$	son $F_k$ medibles y por lo tanto 
			$A_k \cup B_k^c = \{ T_1 \geq k\}$ también lo es. De aquí que 
			$\indic_{T_1 \geq k}$ es $\F_k$-medible y por lo tanto también $\F_n$-medible
			para toda $n$ tal que $n \geq k$.\pn
			  
			Entonces $M_n$ es suma y productos de funciones $\F_n$-medibles y por lo tanto
			$F_n$-medible. Que es lo que queríamos demostrar.\pn
			
		\item
			$M_n \in L_1$\pn
			
			De \eqref{problema1_3:descomposicion_de_M_n} podemos ver que $M_n$ es 
			suma finita de variables acotadas. Por lo tanto $M_n \in L_1$.\pn
			
		\item Ahora probaremos que	$\E(M_{n+1} | \F_{n}) = M_{n}$\pn
			
			Primero:
			\begin{align}
				\E(M_{n+1} | \F_{n})    &=  \E( S_{T_1 \wedge (n+1)} | \F_{n})                                                          \\
                                        &=  \E\bigg( \sum_{k=1}^{n+1} (X_k \cdot \indic_{T_1 \geq k})\bigg| \F_{n}\bigg)                \\	 			
                                        &=  \E\bigg( \sum_{k=1}^{n} (X_k \cdot \indic_{T_1 \geq k}) \bigg| \F_{n}\bigg) +
                                            \E\bigg((X_{n+1} \cdot \indic_{T_1 \geq n+1}) \bigg| \F_{n}\bigg)                           \\
                                        &\comment{Este paso es gracias a que $X_k$ y $\indic_{T_1 \geq k}$ son $\F_n$-medibles}  		\\
                                        &=  \sum_{k=1}^{n} (X_k \cdot \indic_{T_1 \geq k}) + 
                                            \E((X_{n+1} \cdot \indic_{T_1 \geq n+1}) | \F_{n})                                          \\
                                        &=  S_{T_1 \wedge n} + \E((X_{n+1} \cdot \indic_{T_1 \geq n+1}) | \F_{n})                       \\
                                        &=  M_n + \E((X_{n+1} \cdot \indic_{T_1 \geq n+1}) | \F_{n})
			\end{align}
			
			Entonces, nos basta probar que $\E(X_{n+1} \cdot \indic_{T_1 \geq n+1} |
			 \F_{n}) = 0$ para terminar nuestra demostración.\pn
			 
			Sean $A = \{T_1 = n\}$ y $B = \{ T_1 \leq n\}$. Por ser $T_1$ tiempo de paro,
			$A$ y $B$ son $\F_n$-medibles. Por lo tanto $B \setminus A$ también es $\F_n$-medible. 
			Notemos que $\{T_1 \geq n+1\} = (B \setminus A)^c$. Por lo tanto $\{T_1 \geq n+1\}$ es
			$\F_n$-medible. De donde  $\indic_{T_1 \geq n+1})$ es $\F_n$-medible.\pn
			
			Con esto, ahora tenemos que:
			\begin{align}
				\E((X_{n+1} \cdot \indic_{T_1 \geq n+1}) | \F_{n})  &=  \indic_{T_1 \geq n+1} \cdot \E(X_{n+1} | \F_{n})                        \\
                                                                    &\;\;\;\;\mbox{(Este paso es gracias a que los }                            \\
                                                                    &\;\;\;\;\;\mbox{$X_n$ son independientes)}                                 \\
                                                                    &=  \indic_{T_1 \geq n+1} \cdot \E(X_{n+1} )                                \\
                                                                    &=  \indic_{T_1 \geq n+1} \cdot 0                                           \\
                                                                    &=  0
			\end{align}\pn
			
			Como queríamos demostrar.
	\end{itemize}
	
	Ahora que tenemos que $(M_n)_{n \in \N}$ es martingala, confirmemos que converge casi seguramente.\pn
	
	Notemos que $(T_1 \wedge n)_{n \rightarrow \infty} \rightarrow T_1$ c.s.\pn
	
	De aquí que $(M_n)_{n \rightarrow \infty} = (S_{T_1 \wedge n})_{n \rightarrow \infty} = S_{T_1}$ c.s.\pn				
	
	Veamos que la convergencia no ocurre en $L_1$.\pn
				
	Dado que $T_1 \wedge n$ es un tiempo de paro acotado para toda $n \in \N$,
	podemos aplicar el Teorema de Muestreo Opcional de 	Doob. 
	El cual nos dice que $\E(M_n) = \E(S_{T_1 \wedge n}) = \E(S_0) = 0$.\pn
	
	Por otro lado, por definición de $T_1$, $S_{T_1} = 1$ c.s.	De donde $\E(S_{T_1}) = 1$.
	
	\begin{align}
		\E(M_n) = 0 \not\rightarrow 1 = \E(S_{T_1}).
	\end{align}\pn			
	
	Y con esto, queda demostrado que la convergencia no se da en $L_1$.	
	\newpage
	
	\subsubsection{Inciso (iii)}
	\emph{
	Sea $M_n$ la martingala obtenida al detener a $-S$ en $T_1$. Utilice la solución al
	Problema de la Ruina para probar que \\
	$\mw(max_n M_n \geq M) = 1/(M+1)$ para todo $M \geq 1$. Concluya que\\
	$\E(max_m M_n) = \infty$ y que por lo tanto $\E(max_{m\leq n} M_m) \rightarrow \infty$ conforme 
	$n \rightarrow \infty$. Finalmente, deduzca que no puede haber una desigualdad de tipo Doob cuando $p=1$.
}\pn
	
\afterstatement

	Definimos $M_n = -S_{T_1 \wedge n}$. Notemos que $M_n$ únicamente toma valores en $[-1, \infty]$.
	Para calcular $\mw(max_n M_n \geq M)$ notemos primero que:

	\begin{align}
		\mw(max_n M_n \geq M) = 1 - \mw(max_n M_n < M).
	\end{align}\pn
	
	$\max_n M_n < M$ significa que $M_n$ nunca alcanza el valor $M$.\pn
	 
	Intentando hacer analogía con el problema de la ruina, pensemos en dos concursantes,
	uno con $1$ peso y otro con $M$ pesos. Nunca alcanzar $M$ significa que nunca gana el que tiene $1$ peso.\pn
	
	Esta probabilidad ya la conocemos y es 
	
	\begin{align}
		\mw(max_n M_n < M) = \frac{M}{M + 1}
	\end{align}\pn
		
	Por lo tanto
	
	\begin{align}
		\mw(max_n M_n \geq M) 	&= 1 - \mw(max_n M_n < M) \\
								&= 1 - \frac{M}{M + 1}\\
								&= \frac{M+1}{M+1} - \frac{M}{M + 1}\\
								&= \frac{1}{M+1}
	\end{align}\pn
	
	Utilizando este resultado:

	\begin{align} \label{problema1_3:esperanza_del_maximo_de_M_n}
		\E(max_n M_n) 	&= - \mw(max_n M_n = -1) + \sum_{M=1}^{\infty} \mw(max_n M_n \geq M) \\
						&= - \mw(max_n M_n = -1) + \sum_{M=1}^{\infty} \frac{1}{M+1} \\ 
						&= - \mw(max_n M_n = -1) + \infty \\
						&= \infty
	\end{align}\pn						
	
	Ahora, tenemos que:
    
	\begin{align}
		\|\overline{M_{n}^{+}}\|_1  &=    \E{\overline{M_{n}^{+}}} \\
									&=    \E{\max_{m \leq n}M_m^+} \\
									&\geq \E{\max_{m \leq n}M_m}										
	\end{align}\pn
		
	Donde, el último término, tiende a infinito en base al resultado 
	\eqref{problema1_3:esperanza_del_maximo_de_M_n}.\pn

	Por otro lado:
    
	\begin{align}
		\|M_n^+\|_1=\|-S_{T_{1\wedge n}}^{+}\|_1  \longrightarrow  \|-S_{T_1}^+\|_1 = 0 < \infty
	\end{align}\pn
	
	Por lo tanto, no existe número $K$, tal que
	
    \begin{align}
		 \|\overline{M_n^+}\|_1 \leq  K \|M_n^+\|_1
	\end{align}\pn
	
	En otras palabras, no tenemos una desigualdad de tipo Doob para $p=1$.		
	\newpage
	
	\subsubsection{Inciso (iv)}
	\emph
{
	Sea $T=\min\set{n\geq 2:S_n=S_{n-2} + 2}$ y $U=T-2$. ?`Son $T$ y $U$ 
	tiempos de paro? Justifique su respuesta.
}\pn

\afterstatement

	Intuitivamente, $T$ significa, el primer tiempo tal que ganamos en dos volados consecutivos.
	También intuitivamente, esto debería ser un tiempo de paro.\pn
	
	Veamos que efectivamente así ocurre. Utilizando la siguiente prueba por inducción:\pn
	
	\textbf{Base de inducción:}\pn		
    
		\begin{align}
			\{T = 0\} 		&= \emptyset  				& 	\in \F_0 \\
			\{T = 1\} 		&= \emptyset  				& 	\in \F_1 \\
			\{T = 2\} 		&= \{ X_1 = 1, X_2 = 1\} 	&	\in \F_2
		\end{align}\pn					
	
	\textbf{Hipótesis de inducción:}\pn
	
		Supongamos que $\{T = n\} \in \F_n$ para cierto $n \geq 2$.\pn
		
	\textbf{Paso inductivo:}\pn
		
		\begin{align}
			\{T = n + 1 \} = \{ X_n = 1, X_{n+1} = 1\} \setminus \bigcup_{i=0}^{n} \{T = i\}.
		\end{align}\pn				
	
		Es claro que $\{ X_n = 1, X_{n+1} = 1\} \in \F_{n + 1}$ y que por hipótesis de inducción
		$\bigcup_{i=0}^{n} {T = i} \in \F_n \subset \F_{n + 1}$. Por lo tanto
		$\{T = n + 1 \} \in \F_{n+1}$ para toda $n \geq 2$ y con esto termina la demostración.\pn
		
	Ahora, intuitivamente $U$ significa el momento justo antes de ganar dos volados consecutivos.
	Esto, quedría decir que tenemos información sobre eventos que aún no ocurren. Así que intuitivamente
	esto no debería ser un tiempo de paro.\pn
	
	Efectivamente, si tomamos como ejemplo el conjunto: 

    \begin{align}
        \{ U = 1 \} = \{ T - 2 = 1\} = \{ T = 3\} = \{X_1 = -1, X_2 = 1, X_3 = 1\}
    \end{align}\pn
			
	Es fácil notar que es un conjunto que pertenece a $\F_3$, pero no a $\F_1$. Pues $\F_1$
	no contiene información alguna sobre $X_2$ y $X_3$. Así que el conjunto más pequeño de $\F_1$ 
	que contiene a $\{ U = 1 \}$ es $\{ X_1 = -1 \}$.
	\newpage
	
	\subsubsection{Inciso (v)}
	\emph
{
	Para la variable $T$ que hemos definido, calcule $\esp{T}$.\\
}
	Primero, platicaré de manera intuitiva cómo vamos a proceder para solucionar este problema.\\

	Imaginemos un juego de casino a base de un juego de volados con las siguientes reglas:\\
	\begin{itemize}
			\item En cada turno, cada jugador tiene que apostar todo el dinero que tiene.
			\item Si un jugador se queda sin dinero, tiene que abandonar el juego.
			\item Si cae ``sol" cada jugador recibe el doble de lo que había apostado en ese turno.  
	\end{itemize}
   
	\;El casino tiene dinero infinito y cada habitante cuenta con exactamente $1$ peso antes de
	iniciar el juego.\\
   
	Además, en cada nuevo turno entra exactamente un nuevo jugador al juego.\\
   
	Para nuestro problema, supongamos que $X_n = 1$ significa que en el $n$-ésimo turno, salió sol.
	Entonces, $T$ nos indica cuando es la primera vez que caen dos soles consecutivos.\\
   
	Sea $D_n$ la variable que indica cuánto dinero ha ganado el casino para el tiempo $n$.\\
   
	Observemos que en el momento que cae ``águila", todo jugador pierde todo su dinero y abandona el juego.
	Y que por cada jugador que pierde, el casino gana exáctamente $1$ peso (pues cada jugador en cada
	turno apuesta todo el dinero que posee, es decir el peso con el que empezó y todo lo que le habia
	ganado al casino).\\
   
	Entonces, al tiempo $T-2$, todo mundo había perdido. Es decir que al tiempo $T-2$ el casino ha ganado
	$T-2$ pesos.\\
   
	Luego, al tiempo $T-1$, ha caido un sol y hay exactamente un jugador al que el casino tuvo 
	que pagar $1$ peso.\\
   
	Al tiempo $T$, al jugador del turno pasado el casino tuvo que darle $2$ pesos y al jugador del nuevo
	turno tuvo que darle $1$ peso.\\
   
	Entonces, ya podemos decir cuanto dinero ha ganado el casino al tiempo $T$.
	\begin{align}\label{problema1_3:Dinero_al_tiempo_T}
		D_T = T-2 - 1 - 3 = T - 6. 
	\end{align}					   
   
	Notemos que además el juego es justo, en cada turno cada jugador tiene $1/2$ de probabilidad de
	ganar $2^t$ y $1/2$ de probabilidad de perder $2^t$. Es decir, la esperanza es $0$.\\
	
	$D_n$ es suma de este tipo de variables y por lo tanto su esperanza también será $0$.\\
   
	Esto, nos da la intuición de que $D_n$ es martingala, pero esa es la parte que demostraremos más adelante.\\
   
	Si logramos demostrar que $\E(D_T) = 0$. De (\ref{problema1_3:Dinero_al_tiempo_T}) concluimos
	\begin{align}
		0 = \E(T - 6) = \E(T) - 6
	\end{align}
	
	De donde $\E(T) = 6$.\\
	
	Ahora, para terminar con las formalidades, definamos bien a $D$ y comprobemos que es martingala y 
	que podemos utilizar el Teoremoa del muestreo opcional de Doob como lo hemos hecho.\\
	
	Sean entonces $(Y_n)_{n \in \N}$ variables aleatorias Bernulli de parámetro $1/2$ independientes.
	Y sea $Z_n^m$ la cantidad de dinero que el jugador $m$ ha dado al casino definidas como:
	
	\begin{itemize}
		\item 
			Si $n < m$ entonces $Z_n^m = 0$ . (El jugador $m$ no participa en el juego sino hasta el turno $m$).
		\item
			$Z_{n+1}^{m} = (Z_n^{m} - 1) \cdot 2(Y_{n+1}) + 1$. (Si $Y_{n + 1} = 1$ [El jugador gana el volado], entonces el casino
			pierde la cantidad apostada, que para el turno $n+1$ es $Z_n^n + 1$). Nótese que en cuanto un $Y_{n_0}$ se hace cero,
			$Z_{n_0}^{n}$ y todos los que le sigan son todos iguales a $1$ (Como el jugador deja el juego después de haber perdido
			un volado, deja su peso en el casino y entonces de ahí en adelante la cantidad que ha dado al casino es exactamente 1).
	\end{itemize}
	
	Veamos que para cada $m \in \N$, $(Z_n^m)_{n \in \N}$ forma una martingala con respecto a la filtración $(\G_n)_{n \in \N}$ 
	definida por los $Y_n$.\\
	
	Cada $Z_n^m$ es $\G_m$-medible por definición.\\
	
	Cada $Z_n^m$ es suma finita de variables acotadas. Por lo tanto cada una pertenece a $L_1$.\\
	
	Sólo nos falta verificar la propiedad de martingala.
	
	\begin{align}\label{problema1_3:Propiedad_de_martingala_para_el_dinero_perdido_por_un_apostador}
		\E(Z_{n+1}^{m} | \G_n) &= \E((Z_n^{m} - 1) \cdot 2(Y_{n+1}) + 1 | \G_n)\\
							   &= \E((Z_n^{m} -1) \cdot 2(Y_{n+1}) | \G_n) + \E( 1 | \G_n)\\
							   &\;\;\;\;\mbox{[Por ser $(Z_n^{m} - 1)$ una variable $\G_n$-medible.]}\\
							   &= (Z_n^{m} -1) \cdot \E(  2(Y_{n+1}) | \G_n) + \E( 1 | \G_n)\\
							   &= (Z_n^{m} -1) \cdot 2\E( Y_{n+1} | \G_n) + \E( 1 | \G_n)\\
							   &\;\;\;\;\mbox{[Por ser $Y_{n + 1}$ independiente $\G_n$.]}\\
							   &= (Z_n^{m} -1) \cdot 2\E( Y_{n+1}) + \E( 1 | \G_n)\\
							   &= (Z_n^{m} -1) \cdot 2 \frac{1}{2} + \E( 1 | \G_n)\\
							   &= (Z_n^{m} -1) \cdot + 1 \\
							   &= Z_n^m.
	\end{align}
	
	Ahora definamos a $D_n = \sum_{i=1}^n (Z_n^i)$. Que significa, La cantidad de dinero que el casino ha ganado al tiempo $n$. 
	Justo como lo habíamos dicho en la ``demostración intuitiva".\\
	
	Ver que $D$ es martingala es fácil. Cada $D_n$ es suma finita de variables finitas y por lo tanto pertenece a $L_1$.
	$D_n$ es suma de variables $\G_n$-medibles y por lo tanto también lo es. Y la propiedad de martingala se sigue directamente de
	(\ref{problema1_3:Propiedad_de_martingala_para_el_dinero_perdido_por_un_apostador}).\\
	
	Es cierto que podemos aplicar Doob sobre el tiempo $T \wedge n$ por ser acotado y de aquí que:
	$\E(D_{T \wedge n}) = \E(D_1) = 0$.\\
	
	Notemos que $D_{T \wedge n} \longrightarrow D_T \; c.s.$\\
	
	Nos gustaría poder decir lo mismo de sus esperanzas y para eso utilizaremos teroema de convergencia dominada.\\
	
	Es claro que al tiempo $T \wedge n$ el casino a lo más pudo haber ganado $T \wedge n$ pesos.
	De aquí que $D_{T \wedge n} \geq T \wedge n \geq T$.\\
	
	Además, por definición de $T$, para el tiempo $T$ el casino a lo más ha perdido $4$ pesos y es la primera vez que 
	pierde tanto. Así que $-4 \leq D_{T \wedge n}$.\\ 
	
	Entonces, nuestra martingala $D$ esta dominada por $\max(T, 4) < T + 4$. Bastaría demostrar que $\E(T+4) < \infty$ para poder
	utilizar el teorema de convergencia dominada.\\
	
	Notemos que 
	\begin{align}\label{problema1_3:Acotando_T}
		\mw(T > n) &= \frac{1}{2} \mw(T > n-1) + \frac{1}{4} \mw(T > n-2). 
	\end{align}
	
	Pues queremos garantizar que en los primeros $n$ turnos, no pierde dos veces consecutivas el casino.\\
	
	Si en el turno 1, gana el casino (de aquí el $\frac{1}{2}$), en el resto de los $n-1$ turnos tenemos que garantizar que 
	el casino no pierde dos veces consecutivas, como las variables están idénticamente distribuidas, esto es equivalente a que
	$T>n-1$. (De aqui el $\mw(T > n-1)$.\\
	
	Si el casino pierde en el turno 1, necesariamente tiene que ganar en el turno 2 (de aquí el $\frac{1}{4})$. Y la probabilidad
	de que no pierda dos veces consecutivas en los siguientes $n-2$ turnos es $\mw(T > n-2)$.\\					    
	
	De (\ref{problema1_3:Acotando_T}) podemos notar fácilmente que $\mw(T > 2) \leq \frac{3}{4}$\\
	
	También notemos que $\mw(T > n ) \geq \mw(T > n + 1)$ (El primer evento contiene al segundo).
	
	Ahora
	\begin{align}
		\mw(T > 2 (2)) &=     \frac{1}{2} \mw(T > 2+1) + \frac{1}{4} \mw(T > 2) \\
					   &\leq  \frac{1}{2} \mw(T > 2) + \frac{1}{4} \mw(T > 2) \\
					   &\leq  \frac{1}{2} \cdot \frac{3}{4} + \frac{1}{4} \cdot \frac{3}{4} = \bigg(\frac{3}{4}\bigg)^2.
	\end{align}
	
	De manera recursiva tenemos que
	\begin{align}
		\mw(T > 2(n+1)) &=     \frac{1}{2} \mw(T > 2(n+1) - 1) + \frac{1}{4} \mw(T > 2(n+1) - 2) \\
						&=     \frac{1}{2} \mw(T > 2(n) + 1) + \frac{1}{4} \mw(T > 2(n)) \\
					   &\leq  \frac{1}{2} \mw(T > 2n) + \frac{1}{4} \mw(T > 2n) \\
					   &\leq  \frac{1}{2} \cdot \bigg(\frac{3}{4}\bigg)^{n} + \frac{1}{4} \cdot \bigg(\frac{3}{4}\bigg)^{n} = 		
					   \bigg(\frac{3}{4}\bigg)^{n+1}.
	\end{align}
	
	Entonces,
	\begin{align}
		\E(T) 	&= 		\sum_{n=0}^\infty \mw(T > n) \\
				&\leq 	\sum_{n=0}^\infty 2\mw(T > 2n) \\
				&= 		2 \sum_{n=0}^\infty \mw(T > 2n) \\
				&\leq 	2 \sum_{n=0}^\infty \bigg(\frac{3}{4}\bigg)^{n} < \infty.
	\end{align}
	
	Por lo tanto $\E(T + 4) \leq \infty$. Y entonces nuestra martingala $D$ está dominada por una variable integrable y finalmente
	podemos aplicar teorema de convergencia dominada para concluir que:
	\begin{align}
		0 = \lim_{n \longrightarrow \infty} \E(D_{T \wedge n}) = \E(D_T).
	\end{align}
	
	Y con esto terminamos de demostrar todas las formalidades que nos hacían falta.	

\end{proof}
\newpage

\section{Problema 1.4}\label{problema1_4}
\begin{problema}[Extensiones del teorema de paro opcional]
	Sea ${M=\paren{M_n,n\in\na}}$ una (super)martingala respecto de una filtraci\'on ${\paren{\F_n,n\in\na}}$ y sean ${S}$ y ${T}$ tiempos de paro.
	
\begin{enumerate}
                \item[(i)] 
                	Pruebe que ${S\wedge T}$, ${S+T}$ y ${S\vee T}$ son tiempos de paro.
                
                \item[(ii)] 
                	\begin{esn}
                		\F_T=\set{A\in\F:A\cap\set{T\leq n}\in\F_n\text{ para toda } n}
                	\end{esn}
                	es una ${\sigma}$-\'algebra, a la que nos referimos como la ${\sigma}$-\'algebra 
                	detenida en ${\tau}$. Comente qu\'e puede fallar si ${T}$ no es tiempo de paro. 
                	Pruebe que ${T}$ es ${F_T}$-medible. 
                
                \item[(iii)] 
                	Pruebe que si ${T}$ es finito, entonces ${M_T}$ es ${\F_T}$-medible.
                
                \item[(iv)] 
                	Pruebe que si ${S\leq T\leq n}$ entonces ${\F_S\subset\F_T}$. Si adem\'as ${T}$ es acotado entonces ${X_S,X_T\in L_1}$ y 
                	\begin{esn}
	                	\espc{M_T}{\F_S}\leq M_S.
                	\end{esn}

                \item[(v)] 
                	Si ${X=\paren{X_n,n\in\na}}$ es un proceso estoc\'astico ${\paren{\F_n}}$-adaptado y tal que ${X_n\in L_1}$ y tal que 
                	para cualesquiera tiempos de paro acotados ${S}$ y ${T}$ se tiene que ${\esp{X_S}=\esp{X_T}}$ entonces ${X}$ es una 
                	martingala. Sugerencia: considere tiempos de paro de la forma ${n\indi{A}+(n+1)\indi{A^c}}$ con ${A\in\F_n}$.
\end{enumerate}

\defin{Categor\'ias: }Tiempos de paro, Muestreo opcional
\end{problema}

\begin{proof}
	\subsubsection{Inciso (i)} 
	\input{tarea1/problema1_4/inciso1.tex}
	
	\subsubsection{Inciso (ii)}
	\emph{
	\begin{align}
		\F_T=\set{A\in\F:A\cap\set{T\leq n}\in\F_n\text{ para toda } n}
	\end{align}
	es una $\sigma$-\'algebra, 
	a la que nos referimos como la $\sigma$-\'algebra detenida en $\tau$. Comente qu\'e puede fallar si $T$ no es tiempo de paro. 
	Pruebe que $T$ es $\F_T$-medible.\\		
}			
		
	Primero hay que demostrar que $\F_T$ es $\sigma$-algebra.\\
	
	\begin{itemize}
		\item $\Omega \in \F_T$. \\
		
			Notemos que 
			\begin{align}
				\Omega \cap \{T \leq n\} = \{T \leq n\} \in \F_n.
			\end{align}
			Donde la pertenencia a $\F_n$ es gracias a que $T$ es tiempo de paro. (Esta parte podria fallar si 
			$T$ no fuera tiempo de paro).\\
		
		\item $\F_T$ es cerrado bajo complementación.\\
		
			Sea $A \in \F_T$. Eso significa que para todo $n \in  \N$ ocurre que $B = A \cap \{ T \leq n \} \in \F_n$. 
			Por ser $\F_n$ una $\sigma$-algebra tenemos que el complemento de $B$ también debe estar en $\F_n$. 
			Escrito en símbolos:
			
			\begin{align}
				B^c 	&= (A   \cap \{ T \leq n \})^c \\
						&=  A^c \cup \{ T > n \} \in \F_n
			\end{align}
			
			Dado que $B^c$ y $\{ T \leq n \}$ se encuentran en $\F_n$, también \\
			$B^c \cap \{ T \leq n \} \cap \{ T \leq n \}$.\\
			
			Escribiendo esto último de otra manera:
			\begin{align}
				B^c \cap \{ T \leq n \} \cap \{ T \leq n \} 	&=		A^c \cup \{ T > n \} \cap \{ T \leq n \} \cap \{ T \leq n \} \\
																&= 		A^c \cap \{ T \leq n \} \cup \{ T > n \} \cap \{ T \leq n \} \\
																&= 		\bigg(A^c \cap \{ T \leq n \}\bigg) 
																			\bigcup 
																		\bigg(\{ T > n \} \cap \{ T \leq n \}\bigg) \\
																&=		\bigg(A^c \cap \{ T \leq n \}\bigg)	\bigcup \emptyset \\
																&=		A^c \cap \{ T \leq n \} \in \F_n
			\end{align}
			
			Y por lo tanto $A^c \in \F_T$.\\
			
		\item $\F_T$ es cerrado bajo uniones numerables.
		
			Sean $A_m \in \F_T$ para $m \in \N$.
			\begin{align}
				\bigg( \bigcup_{k=1}^\infty A_k \bigg) \cap \{ T \leq n \} 	&=	\bigcup_{k=1}^\infty \bigg( A_k \cap \{ T \leq n \} \bigg)
			\end{align}								
			
			La última parte, por definición de $\F_T$ es una unión numerable de elementos de $\F_n$ y por lo tanto, dicha unión
			tambien pertenece a $\F_n$. Y por último $\cup A_k \in \F_T$.
	\end{itemize}
	\null
	
	Ahora veamos que $T$ es $\F_T$-medible. Dado un $k \in \N$, ocurre que:
	
	\begin{align}
		\{T \leq k\} \cap \{T \leq n\} = \{T \leq min(k,n)\} \in \F_{min(k, n)} \subset \F_n \forall n \in \N.
	\end{align}

	Y por lo tanto $\{T \leq k\} \in \F_T$.
	
		
	\subsubsection{Inciso (iii)}
	\input{tarea1/problema1_4/inciso3.tex}
	
	\subsubsection{Inciso (iv)} 
	\emph{
	Pruebe que si ${S\leq T\leq n}$ entonces ${\F_S\subset\F_T}$. Si adem\'as ${T}$ es acotado entonces ${M_S,M_T \in L_1}$ y 
	\begin{align}
		\espc{M_T}{\F_S}\leq M_S.
	\end{align}	
}

	Primero probaremos que ${\F_S \subset \F_T}$. Sea $n \in \N$. Notemos que $\{ T \leq n \} \subset \{ S \leq n \}$ pues
	si ${\omega \in \{ T \leq n \}}$, entonces ${S(\omega) \leq T(\omega) \leq n }$ y por lo tanto ${\omega \in \{ S \leq n \}}$.\\

	Ahora sea ${A \in \F_S}$. Entonces

	\begin{align}
		A \cap \{ T \leq n \} 	&=		A  \cap \{ T \leq n \} \cap \{ S \leq n \} \\
								&=		(A  \cap \{ S \leq n \}) \cap \{ T \leq n \} \in \F_n
	\end{align}

	Como escogimos $n$ arbitrariamente, esto es cierto para toda $n \in \N$. Y por lo tanto $A \in \F_T$ y $\F_S \subset \F_T$.\\

	
	\subsubsection{Inciso (v)}
	\emph{
	Si ${X=\paren{X_n,n\in\na}}$ es un proceso estoc\'astico ${\paren{\F_n}}$-adaptado y tal que ${X_n\in L_1}$ y tal que 
	para cualesquiera tiempos de paro acotados ${S}$ y ${T}$ se tiene que ${\esp{X_S}=\esp{X_T}}$ entonces ${X}$ es una 
	martingala. Sugerencia: considere tiempos de paro de la forma ${n\indi{A}+(n+1)\indi{A^c}}$ con ${A\in\F_n}$.
}
\end{proof}        
        \nqed
        
    \part{Tarea 2}
        \section{Problema 2.1} \label{problema2_1}
\begin{problema}
        Sea $S_n=X_1+\cdots+X_n$ una caminata aleatoria con saltos $X_i\in \{-1,0,1,\ldots\}$. 
        Sea $C_p$ una variable aleatoria geom\'etrica de par\'ametro $p$ independiente de $S$ y definimos 
        
        \begin{align}
            M_p=-\min_{n\leq C_p} S_n. 
        \end{align}\par\null
        
        El objetivo del ejercicio es determinar la distribuci\'on de $M_p$.\par\null

        (A las caminatas aleatorias como $S$ se les ha denominado Skip-free random walks Para aplicaciones de este tipo 
        de procesos, ver \cite{MR1978607}. Tambi\'en aparecen en el estudio de Procesos Galton-Watson. 
        Este ejercicio es el resultado b\'asico del estudio de sus extremos, denominado teor\'ia de fluctuaciones.)

    \begin{enumerate}
        \item[(i)]  [\ref{problema2_1:inciso1}]
                    Sea
                    
                    \begin{align}
                        g(\lambda)=E(e^{- \lambda X_1}).
                    \end{align} 
                    
                    Pruebe que $g(\lambda)\in (0,\infty)$ y que
                    
                    \begin{align}
                        M_n=e^{-\lambda S_n}g(\lambda)^{-n},n\geq 0
                    \end{align}
                    
                    es una martingala.
        
        \item[(ii)] [\ref{problema2_1:inciso2}]
                    Pruebe que $g$ es log-convexa al aplicar la desigualdad de H\"older. Pruebe que si $P(X_1=-1)>0$ 
                    (hip\'otesis que se utilizar\'a desde ahora) 
                    entonces $g(\lambda)\to\infty$ conforme $\lambda\to\infty$. Utilice esta informaci\'on para esbozar la gr\'afica de $g$. 
                    Defina $ f(s)=\inf \{ \lambda>0:g(\lambda)^{-1} < s\} $. Note que $1/g\circ f=Id$ en $(0,1)$. Pruebe que si $g(\lambda)>1$, 
                    la martingala $M$ es acotada hasta el tiempo de arribo de $S$ a $-k$ dado por 
                    
                    \begin{align}
                        T_k =\min \{n\in\na:S_n=-k\} 
                    \end{align}
                    
                    (donde se utiliza la convenci\'on $\inf\emptyset=\infty$ ). Aplique el teorema de muestreo opcional de Doob para mostrar que 
                    
                    \begin{align}
                        E(s^{T_k})=e^{-k f(s)}.
                    \end{align}
                    
                    Justifique MUY bien por qu\'e la f\'ormula es válida aún cuando $T_k$ puede tomar el valor $\infty$ 
                    y deduzca que de hecho $\p (T_k=\infty)=0$.
                        
        \item[(iii)] [\ref{problema2_1:inciso3}]
                    Argumente que
        
                    \begin{align}
                        P(M_p\geq n)=P(T_n\leq C_p)=E((1-p)^{T_n})
                    \end{align}
                    
                    para demostrar que $M_p$ tiene distribuci\'on geom\'etrica de par\'ametro $1-e^{-f(1-p)}$.
                    
        \item[(iv)] [\ref{problema2_1:inciso4}]
                    Tome el límite conforme $p\to 0$ para mostrar que la variable aleatoria 
                    \begin{align}
                        M=-\min_{n\geq 0}S_n
                    \end{align}
                    tiene una distribuci\'on geom\'etrica de par\'ametro $1-e^{-f(1)}$. Interprete esto cuando $f(1)=0$.
    \end{enumerate}

        \defin{Categor\'ias:} Caminatas aleatorias, muestreo opcional, fluctuaciones.
\end{problema}


\begin{proof}
    \subsection{Inciso (i)}     \label{problema2_1:inciso1}
    \emph{
	Sea
	\begin{align}
		g(\lambda)=E(e^{- \lambda X_1}).
	\end{align} 
	Pruebe que $g(\lambda)\in (0,\infty)$ y que
	\begin{align}
		M_n=e^{-\lambda S_n}g(\lambda)^{-n},n\geq 0
	\end{align}
	es una martingala.
}
    \newpage
    
    \subsection{Inciso (ii)}    \label{problema2_1:inciso2} 
    \emph{
    Pruebe que $g$ es $\log$-convexa al aplicar la desigualdad de H\"older. Pruebe que si $P(X_1=-1)>0$ (hip\'otesis que se utilizar\'a desde ahora) 
    entonces $g(\lambda)\to\infty$ conforme $\lambda\to\infty$. Utilice esta informaci\'on para esbozar la gr\'afica de $g$. 
    Defina $ f(s)=\inf \{ \lambda>0:g(\lambda)^{-1} < s\} $. Note que $1/g\circ f=Id$ en $(0,1)$. Pruebe que si $g(\lambda)>1$, 
    la martingala $M$ es acotada hasta el tiempo de arribo de $S$ a $-k$ dado por 
    \null
    \begin{align}
        T_k =\min \{n\in\na:S_n=-k\} 
    \end{align}
    \null
    (donde se utiliza la convenci\'on $\inf\emptyset=\infty$ ). Aplique el teorema de muestreo opcional de Doob para mostrar que 
    \null
    \begin{align}
        E(s^{T_k})=e^{-k f(s)}.
    \end{align}
    \null
    Justifique MUY bien por qu\'e la f\'ormula es válida aún cuando $T_k$ puede tomar el valor $\infty$ y deduzca que de hecho 
    $\p (T_k=\infty)=0$.
}
\afterstatement
    Primero probemos que $g$ es $\log$-convexa, es decir, que $\log \circ g$ es convexa.Sean $a < b$ en el dominio de $g$ 
    y sea $t \in [0, 1]$.\\
    
    Queremos demostrar que:
    
    \begin{align}
            \log\circ g ((1-t)a + (t)b) \leq (1-t)(\log\circ g (a)) + (t)(\log\circ g (b)).
    \end{align}
    
    Si $t=0$ tenemos:
    
    \begin{align}
        \log\circ g ((1-0)a + (0)b)     &= \log\circ g (a) \\
                                        &= (1-0)(\log\circ g (a)) + (0)(\log\circ g (b))
    \end{align}
    
    y por lo tanto no hay nada que demostrar. Análogamente ocurre con $t=1$.\\
    
    Entonces concentrémonos en el caso donde $t\in (0, 1)$.\\
    
    Recordemos que la desigualdad de Hölder dice que si $f \in L_p$ y $g \in L_q$ con 
    $p,q \in (1,\infty)$ y $\frac{1}{p} + \frac{1}{q} = 1$. Entonces
                
    \begin{align}
                \E(| fg |) \leq (\E(\abs{f}^p))^{\frac{1}{p}} (\E(\abs{g}^q))^\frac{1}{q}.
    \end{align}
    
    Sean entonces $p = \frac{1}{1-t}$ y $q = \frac{1}{t}$. Tenemos que\\
    
    \begin{align}
        \frac{1}{p} + \frac{1}{q}   &= \frac{1}{\frac{1}{1-t}} + \frac{1}{\frac{1}{t}}  \\
                                    &= t + 1 - t                                        \\ 
                                    &= 1.
    \end{align}
    
    Veamos que $e^{-a(1-t) X_1}$ pertenece a $L_p$. \\
    
    Como $e^x > 0$ para todo $x \in \R$. $\abs{e^x} = e^x$. Entonces
    
    \begin{align}
         \bigg(\E\bigg(\abs{e^{-a(1-t) X_1}}^p\bigg)\bigg)^{\frac{1}{p}} 
                &=  \bigg(\E\bigg(\abs{e^{-a(1-t) X_1}}^{\frac{1}{1-t}}\bigg)\bigg)^{{1-t}} \\
                &=  \bigg(\E\bigg(\abs{e^{-aX_1}}\bigg)\bigg)^{{1-t}}                       \\
                &=  \bigg(\E\bigg(e^{-aX_1}\bigg)\bigg)^{{1-t}}                             \\
                &=  ( g(a))^{{1-t}}                                                         \\
                &<   \infty
    \end{align}
    
    Donde la última desigualdad es gracias a que $a$ fue tomado en el dominio de $g$ y a lo demostrado en 
    [\ref{problema2_1:inciso1}]. Con esto hemos demostrado que $e^{-a(1-t) X_1}$ pertenece a $L_p$.\\
    
    De manera análoga se puede demostrar que $e^{-b(t) X_1}$ pertenece a $L_q$.\\
    
    Ahora que tenemos todas las hipótesis para la desigualdad de Hölder, basta aplicarla.\\
        
    \begin{align}
        \log\circ g ((1-t)a + (t)b)     &=       \log\bigg( \E\bigg(e^{-a(1-t) + b(t) X_1}\bigg) \bigg)                              \\
                                        &=       \log\bigg( \E\bigg(e^{-a(1-t) X_1} \cdot e^{-b(t) X_1}\bigg) \bigg)                 \\
                                        &=       \log\bigg( \E\bigg(\abs{e^{-a(1-t) X_1} \cdot e^{-b(t) X_1}}\bigg) \bigg)           \\
                                        &\leq    \log\bigg(
                                                            \E
                                                                \bigg(
                                                                    \abs{e^{-a(1-t) X_1}}^p
                                                                \bigg)^{\frac{1}{p}} 
                                                        \cdot 
                                                            \E
                                                                \bigg(
                                                                    \abs{e^{-b(t) X_1}}^q
                                                                \bigg)^\frac{1}{q}
                                                    \bigg)                                                                          \\
                                        &\;\;\;\;\mbox{(Esta desigualdad es gracias a la desigualdad de }                           \\
                                        &\;\;\;\;\mbox{ Hölder y a que $\log$ es una función creciente)}                            \\
                                        &=      \log\bigg( 
                                                            \E
                                                                \bigg(
                                                                    e^{-a(1-t) X_1 p}
                                                                \bigg)^{\frac{1}{p}} 
                                                        \cdot     
                                                            \E
                                                                \bigg(
                                                                    e^{-b(t) X_1 q}
                                                                \bigg)^\frac{1}{q}
                                                     \bigg)                                                                         \\
                                        &=      \log\bigg( 
                                                            \E
                                                                \bigg(
                                                                    e^{-a(1-t) X_1 \frac{1}{1-t}}
                                                                \bigg)^{\frac{1}{\frac{1}{1-t}}} 
                                                        \cdot    
                                                            \E
                                                                \bigg(
                                                                    e^{-b(t) X_1 \frac{1}{t}}
                                                                \bigg)^\frac{1}{\frac{1}{t}}
                                                     \bigg)                                                                         \\
                                        &=      \log\bigg( 
                                                        \E
                                                            \bigg(
                                                                e^{-a X_1}
                                                            \bigg)^{1-t} 
                                                        \cdot                                                             
                                                        \E
                                                            \bigg(
                                                                e^{-b X_1}
                                                            \bigg)^{t}
                                                     \bigg)                                                                         \\
                                        &=      \log\bigg( 
                                                        \E
                                                            \bigg(
                                                                e^{-a X_1}
                                                            \bigg)^{1-t} 
                                                \bigg)
                                                        + 
                                                \log\bigg(
                                                        \E
                                                            \bigg(
                                                                e^{-b X_1}
                                                            \bigg)^{t}
                                                \bigg)                                                                              \\
                                         &=      \log\bigg( 
                                                        g(a)^{1-t} 
                                                \bigg)
                                                        + 
                                                \log\bigg(
                                                        g(b)^{t}
                                                \bigg)                                                                              \\
                                         &=      (1-t)\log(g(a))+(t)\log(g(b))                                                      \\
                                         &=      (1-t)(\log \circ g (a))+(t)(\log \circ g (b))                                        
    \end{align}
    
    Que es lo que necesitabamos mostrar para probar que $g$ es $\log$-convexa.\\
    
    Ahora supongamos que $\mw(X_1 = -1) > 0$.\\
    
    Para ver que $g$ tiende a infinito conforme $\lambda$ crece, descompongamos a $g(\lambda)$ de la siguiente manera.
    
    \begin{align}
        g(\lambda)      &=      \E(e^{-\lambda X_1})                \\
                        &=      \sum_{-1 \leq i} e^{-\lambda i} \mw( X_1 = i)
    \end{align}
    
    De donde obtenemos que $g(\lambda) \geq e^{\lambda} \mw( X_1 = -1)$. Dado que $e^{\lambda}$ tiende a infinito comforme
    $\lambda$ crece, tenemos que $g(\lambda)$ también lo hace.\\
    
    Ahora notemos que $g$ es convexa también. $\log \circ g$ es convexa por la primera parte de este inciso y 
    $e^x$ es convexa y creciente porque su primera y segunda derivada siempre son mayor que cero. 
    Dado esto, notemos que $g = e^{\log(g)}$ y entonces por ser $g$ una composición de una función convexa con una
    convexa creciente tenemos que $g$ es convexa.\\
    
    Uno de los resultados de los cursos de cálculo de la licencuatura es que una función convexa con dominio abierto, es continua.
    $g$ está definida sobre todo $\R$, que es abierto, y por lo tanto $g$ es continua.\\
    
    Ahora sea $s \in (0,1)$. Y sea $\lambda_0 = f(s)$. Por definición, $\lambda_0 = \inf\{ \lambda > 0 : g(\lambda)^{-1} < s\}$. Por definición de ínfimo,
    para cualquier $n \in \N$, existe $\lambda_n > 0$ tal que 
    
    \begin{align}
        \lambda_0 \leq \lambda_n \leq \lambda_0 + \frac{1}{n}. \label{problema2_1:sucesion_convergente_a_lambda_0}
    \end{align}
     
    y que
    
    \begin{align}
        g(\lambda_n)^{-1} < s. \label{problema2_1:sucesion_dominada_por_s}
    \end{align}
    
    Por \eqref{problema2_1:sucesion_convergente_a_lambda_0} sabemos que $\lambda_n \rightarrow \lambda_0$.
    De \eqref{problema2_1:sucesion_dominada_por_s} obtenemos $\frac{1}{s} < g(\lambda_n)$. Por continuidad de $g$
    tenemos entonces que $\frac{1}{s} \leq g(\lambda_0)$.
    \newpage
        
    \subsection{Inciso (iii)}    \label{problema2_1:inciso3}
    \emph
{
    Argumente que
}

\begin{align}
    \mw(M_p\geq n)  &=  \mw(T_n\leq C_p)    \\
                    &=\E((1-p)^{T_n})
\end{align}\pn

\emph
{
    para demostrar que $M_p$ tiene distribuci\'on geom\'etrica de par\'ametro $1-e^{-f(1-p)}$.
}

\afterstatement\pn

    Recordemos que
    
    \begin{align}
        M_p     &=  -\min_{n\leq C_p} S_n. 
    \end{align}\pn
    
    Donde $C_p$ es una variable aleatoria con distribución geométrica de parámetro $p$.\pn
    
    Entonces 
    
    \begin{align}
        \{ M_p \geq n\}     &=  \{ -\min_{k\leq C_p} S_k \geq n \}          \\
                            &=  \{ \min_{k \leq C_p} S_k \leq -n \}         \\
                            &=  \{ \min\{ k \in \N : S_k = -n \}\leq C_p \}   \\
                            &=  \{ T_n \leq C_p \}.
    \end{align}\pn
    
    De donde tenemos $\mw(M_p\geq n)=\mw(T_n\leq C_p)$.\pn
    
    Ahora descompongamos $\mw(T_n \leq C_p)$.
    
    \begin{align}
        \mw(T_n \leq C_p)   &=  \sum_{i \in \N} \mw(T_n \leq C_p | T_n = i) \mw(T_n = i)        \\
                            &=  \sum_{i \in \N} \mw(i \leq C_p) \mw(T_n = i)                    \\
                            &=  \sum_{i \in \N} \sum_{j = i}^{\infty} \mw(C_p = j) \mw(T_n = i) \\
                            &=  \sum_{i \in \N} \sum_{j = i}^{\infty} (1-p)^{k-1} p\mw(T_n = i) \\
                            &=  \sum_{i \in \N} p\mw(T_n = i) \sum_{j = i}^{\infty} (1-p)^{k-1} \\                            
                            &=  \sum_{i \in \N} p\mw(T_n = i) \frac{(1-p)^{i-1}}{p}             \\                            
                            &=  \sum_{i \in \N} \mw(T_n = i) (1-p)^{i-1}                        \\                     
                            &=  \E((1-p)^{T_n}).                                                            
    \end{align}\pn
    
    Ahora calculemos $\mw(M_p = n)$.\pn
    
    \begin{align}
        \mw(M_p = n)    &=  \mw(M_p \geq n) - \mw(M_p \geq n + 1)       \\
                        &=  \mw(T_n \leq Cp)- \mw(T_{n+1} \leq n + 1)   \\
                        &=  \E((1-p)^{T_n}) - \E((1-p)^{T_{n+1}})       \\
                        &=  e^{-nf(1-p)}    - e^{-(n+1)f(1-p)}          \\
                        &=  e^{-nf(1-p)} (1- e^{-f(1-p)})               \\
                        &=  (e^{-f(1-p)})^n (1- e^{-f(1-p)}).           
    \end{align}\pn
    
    De donde tenemos que $M_n$ tiene la distribución geométrica de parámetro $(1- e^{-f(1-p)})$.
    \newpage
    
    \subsection{Inciso (iv)}    \label{problema2_1:inciso4}
    \emph{
    Tome el límite conforme $p\to 0$ para mostrar que la variable aleatoria 
    \begin{align}
        M=-\min_{n\geq 0}S_n
    \end{align}
    tiene una distribuci\'on geom\'etrica de par\'ametro $1-e^{-f(1)}$. Interprete esto cuando $f(1)=0$.
}

Cuando hablamos de $C_p$, sólo nos interesa su distribución.\pn

Así que para ejemplificar cómo se comporta una sucesión de geométricas con sus parámetros tendien a $0$,
mientras conservemos a $C_p$ con distribución geométrica, somos libres de elegir de que manera se comportan.\pn

Definamos $C_p : [0,1] \rightarrow \N$ (donde en $[0,1]$ usamos la medida de Lebesgue) de la siguiente manera:

\begin{align}
    C_{p}(x) = 
        \begin{cases}
            1       & \mbox{if } x \in [0, p)                                                                               \\
            2       & \mbox{if } x \in [p, p + (1-p)p)                                                                      \\
            \vdots  &                                                                                                       \\
            n       & \mbox{if } x \in \bigg[\sum_{i = 1}^{n - 1} (1 - p)^{i - 1}p , \sum_{i = 1}^n (1-p)^{i-1}p \bigg)     \\
            \vdots  &                                               
        \end{cases}
\end{align}

Claramente, $C_p$ definida de esta manera tiene distribución de una geométrica de parámetro $p$ (Porque no le dejamos de otra).\pn

Ahora podemos ver con más claridad que 

\begin{align}
    \mw(C_p > n) = 1 - \sum_{i = 1}^n (1-p)^{i-1}p
\end{align}

Analicemos la derivada del término $(1-p)^{i}p$ ($i$ fija) con respecto a $p$. Su derivada es $(1-p)^i - p i (1-p)^{i-1}$.
Evaluando en $0$ tenemos \\
$(1-0)^i - 0 i (1-0)^{i-1} = 1 > 0$. Es decir que cerca de $0$, cada uno de los términos de este tipo, es 
estrictamente creciente.\pn

Es decir, que si $p$ se va acercando hacia 0, cada uno de estos términos, decrece.\pn

A continuación una gráfica de cómo se comporta $1 - \sum_{i = 1}^{100} (1-p)^{i-1}p$ conforme variamos $p$.
(Se incluye el script de R que se implementó para realizar esta gráfica, junto con algunas otras gráficas para
distintos valores de $n$).\pn

\begin{center}
    \includegraphics[width=6cm]{tarea2/problema2_1/graficas_inciso2_1_4/probabilidadDeQueC_pSupere100.png}
\end{center}\pn

Con todo esto dicho, podemos garantizar que $C_p$ diverge a $\infty$ en distribución. Pues para todo rango $[m, n]$ con $m<n,\in \N$
$\mw(C_p \in [m, n]) \rightarrow 0$. \pn

Entonces 
\begin{align}
    - \min_{n \leq C_p} \rightarrow - \min_{n \leq \infty} S_n = \min_{n \geq 0} S_n = M
\end{align}\pn

Ahora recordemos que $g$ era una función continua y que $g>0$. Por lo tanto $1/g$ es una función continua,
con inverso derecho $f$ en $(0, 1)$. Es decir

Por lo tanto
\begin{align}
    \mw(M = n)  &=  \lim_{p \rightarrow 0} \mw(M_p = n)                         \\
                &=  \lim_{p \rightarrow 0} e^{-n f(1-p)}(1-e^{- f(1-p)})        \\
                &=   e^{-n f(1)}(1-e^{- f(1)}) \label{problema2_1:distribucion_de_M}
\end{align}

Lo cual corresponde a la distribución de una geométrica de parámetro $1-e^{- f(1)}$.\pn

El caso donde $f(1^-) = 0$, implica que $\lim_{p\rightarrow0} (1-e^{- f(1-p)}) = 0$. El cual es complétamente
análogo al caso anterior donde considerábamos $p \rightarrow 0$ en geométricas de parámetro $p$.\pn

Donde, habíamos dejado claro que conforme el parámetro disminuía hacia 0, las distribuciones de las geométricas divergía 
a la de $\infty$ y que por lo tanto la probabilidad de que nuestras geométricas tomaran cierto 
valor $n$ disminuía hacia $0$ conforme el parámetro tendía a $0$.\pn

Incluso para este caso, la ecuación \eqref{problema2_1:distribucion_de_M} es válida según nuestro análisis, pues

\begin{align}
\mw(M = n)  &=  \lim_{p \rightarrow 0} \mw(M_p = n)                         \\
                &=  \lim_{p \rightarrow 0} e^{-n f(1-p)}(1-e^{- f(1-p)})    \\
                &=  e^{-n 0}(1-e^{- 0})                                     \\
                &=  1(1-1)                                                  \\
                &=  0.
\end{align}

Y esto, para toda $n \in \N$, justo como nuestro análisis de las distribuciones geométricas
nos dijo.
\end{proof}
\newpage

\section{Problema 2.2}  \label{problema2_2}
\begin{problema}
    \begin{enumerate}
        \item[(i)] 
            Instale \href{www.octave.org}{Octave} en su computadora
        \item[(ii)] 
            \'Echele un ojo a la documentaci\'on
        \item[(iii)] 
   
            Ejecute el siguiente c\'odigo linea por linea: 
            \texttt{
                    \lstinputlisting[caption=]{tarea2/problema2_2/polya1.R}
                    }
        \item[(iv)] 
            Lea las secciones sobre 
            \href{http://www.gnu.org/software/octave/doc/interpreter/Simple-Examples.html#Simple-Examples}{simple examples}, 
            \href{http://www.gnu.org/software/octave/doc/interpreter/Ranges.html#Ranges}{ranges}, 
            \href{http://www.gnu.org/software/octave/doc/interpreter/Random-Number-Generation.html#Random-Number-Generation}{random number generation} 
            y 
            \href{http://www.gnu.org/software/octave/doc/interpreter/Comparison-Ops.html#Comparison-Ops}{comparison operators} 
            y escriba su interpretaci\'on de lo que hace el c\'odigo anterior. Nota: est\'a relacionado con uno de los ejemplos del curso.
        \item[(v)] 
            Vuelva a correr el c\'odigo varias veces y escriba sus impresiones sobre lo que est\'a sucediendo.
    \end{enumerate}
\end{problema}

\afterstatement\par\null

En el código de arriba, se implementó una ejecución del proceso de las urnas de Poyla.\par\null

El vector \texttt{x} representa la proporción de las bolas ``verdes'' en la urna conforme el tiempo avanza. Que
al tiempo inicial se tenga \texttt{x=[1/2]} quiere decir que al inicio había tantas bolas ``verdes'' como ``rojas''.\par\null

Para octave los booleanos pueden operarse numéricamente. \texttt{true} es equivalente a \texttt{1} y
\texttt{false} es equivalente a \texttt{0}. Entonces, la parte que dice \\
\texttt{+(u(i)$<$x(i))} significa que se está sumando
uno o cero dependiendo de si la condición se satisface. 
Lo cual quiere decir que la constante de bolas que se agregan a la urna es $1$.\par\null

El vector \texttt{u} representa el resultado de sacar una bola. Si \texttt{u(i)$<$x(i)} significa que en el turno \texttt{i}, 
se obtuvo una bola ``verde''.\par\null

Notemos que en la implementación de arriba, \texttt{x(2) = 3/8} ó \texttt{x(2) = 5/8}. Lo que quiere decir que para 
el turno \texttt{2} existen al menos \texttt{8} bolas. Aún más, podemos asegurar que para el turno \texttt{2} existe 
un número par de bolas. Es decir que en el turno 1 el número de bolas era impar, pero la proporción de bolas ``verdes'' 
con respecto al total era \texttt{1/2}. Esto revela que existe un error de desfase en la implementación.\par\null 

Imagino, que lo que se intentó hacer fue una urna con una bola ``verde'' y una bola ``roja'' inicialmente. En este caso, lo que 
debería decir en el ciclo \texttt{for} es lo siguiente:\par\null

\begin{verbatim}
for i = 1:600
    x(i+1) = (2+i-1)/(2+i) x(i) + (u(i)<x(i))/(2+i);
endfor
\end{verbatim}\par\null

Con una chequeo rápido uno puede comprobar que en esta versión modificada de la implementación, \texttt{x(2) = 1/3} ó \texttt{x(2) = 2/3}. 
Como al inicio la proporción era \texttt{1/2}, esto significa que al inicio existían una bola ``verde'' y una ``roja'' exactamente.\par\null

A continuación una gráfica obtenida de ejcutar el código.

\begin{center}
    \includegraphics[width=8cm]{tarea2/problema2_2/poyla.PNG}
\end{center}
\begin{center}
    Gráfica de una ejecución de urnas de Poyla \par
    $1$ bola verde inicial, $1$ bola roja inicial y constante $1$. $1500$ iteraciones.
\end{center}\par\null

Se puede apreciar que conforme avanza el tiempo, la proporción de bolas ``verdes'' se estabiliza. Justo como se demostró en clase para
martingalas positivas.
\newpage

\section{Problema 2.3}  \label{problema2_3}

\begin{problema}[Ejercicios sueltos sobre martingalas]
	\begin{enumerate}
		\item 
			Sea $\paren{X_n,n\geq 0}$ una sucesi\'on $\paren{\F_n}$-adaptada. Pruebe que
			
			\begin{esn}
				\sum_{k=1}^n X_k-\espc{X_k}{\F_{k-1}}, \quad n\geq 0
			\end{esn}
			
			es una $\paren{\F_n}$-martingala.
		
		\item
			Descomposici\'on de Doob para submartingalas: Sea \(X=\paren{X_n}_{n\in\na}\) una submartingala. 
			Pruebe que $X$ se puede descomponer de manera \'unica como $X=M+A$ donde $M$ es una martingala y $A$ 
			es un proceso previsible con $A_0=0$. Sugerencia: Asuma que ya tiene la descomposici\'on y calcule 
			esperanza condicional de $X_{n+1}$ dada $X_n$. 
		
		\item 
			Sea \(S_n=\xi_1+\cdots+\xi_n\) donde las variables \(\xi\) son independientes y \(\xi_i\) tiene 
			media cero y varianza finita \(\sigma_i^2\). Pruebe que si \(\sum_i \sigma_i^2<\infty\) entonces 
			\(S_n\) converge casi seguramente y en \(L_2\) conforme \(n\to\infty\). Construya un ejemplo de 
			variables aleatorias \(\xi_i\) tales que la serie \(\sum_i \xi_i\) sea casi seguramente absolutamente 
			divergente y casi seguramente condicionalmente convergente (considere ejemplos simples!). 
			Explique heur\'isticamente por qu\'e cree que suceda esto.
			%Ser\'a que \sum_i\abs{x_i}=\infty casi seguramente si \sum_i\abs\esp{\xi_i}=\infty? 
		
		\item 
			Sean \(X\) y \(Y\) dos martingalas (respecto de la misma filtraci\'on) y tales que \(\esp{X_i},\esp{Y_i}<\infty\) 
			para toda \(i\). Pruebe la siguiente f\'ormula de integraci\'on por partes: 
			
			\begin{esn}
				\esp{X_nY_n}-\esp{X_0Y_0}=\sum_{i=1}^n \esp{\paren{X_i-X_{i-1}}\paren{Y_i-Y_{i-1}}}. 
			\end{esn}
		
		\item Desigualdad de Azema-Hoeffding, tomado de \cite[E14.2, p.237]{MR1155402}
			\begin{enumerate}
				\item Muestre que si \(Y\) es una variable aleatoria con valores en \([-c,c]\) y media cero entonces, para \(\theta\in\re\)
						$$\esp{e^{\theta Y}}\leq\imf{\cosh}{\theta c}\leq \imf{\exp}{\frac{1}{2}\theta^2c^2}. $$
				\item Pruebe que si \(M\) es una martingala nula en cero tal que para algunas constantes \(\paren{c_n,n\in\na}\) se tiene que
						$$\abs{M_n-M_{n-1}}\leq c_n\quad\forall n $$
						entonces, para \(x>0\)
						$$
						\proba{\max_{k\leq n} M_k\geq x}\leq \imf{\exp}{\frac{x^2}{2\sum_{k=1}^n c_k^2}}.
						$$
			\end{enumerate}
	\end{enumerate}
\end{problema}
\newpage

\section{Problema 2.4}  \label{problema2_4}
\begin{problema}
	Sea $S_n=\sum_{i=1}^n X_i$ donde $X_1,X_2,\ldots$ son iid. Sea
	
	\begin{esn}
		\imf{\phi}{\lambda}=\esp{e^{\lambda S_n}}\in (0,\infty].
	\end{esn}

	\begin{enumerate}
		\item[(i)]
			Pruebe que si existen $\lambda_1<0<\lambda_2$ tales que $\imf{\phi}{\lambda_i}<\infty$ entonces $\imf{\phi}{\lambda}<\infty$ para toda $\lambda\in [\lambda_1,\lambda_2]$. Sugerencia: escriba $\lambda=a\lambda_1+(1-a)\lambda_2$ para alg\'un $a\in [0,1]$ y aplique la desigualdad de H\"older. A partir de ahora se asume la premisa de este inciso.
		\item[(ii)] 
			Pruebe que $\esp{\abs{S_n}^k}<\infty$ para toda $k\geq 0$. 
		\item[(iii)] 
			Sea $M^\lambda_t=e^{\lambda S_t}/\imf{\phi}{\lambda}$. Argumente que si $M^n$ es el proceso dado por
			
			\begin{esn}
				M^n_t=\left.\frac{\partial^n}{\partial \lambda^n}\right|_{\lambda=0}M^\lambda_t,
			\end{esn}
			
			entonces $M^n$ es una martingala para toda $n$. 
		\item[(iv)] 
			Calcule las primeras $4$ martingalas resultantes si $\proba{X_i=\pm 1}=1/2$. Util\'icelas para calcular el valor de $\esp{T^2}$ donde

			\begin{esn}
				T=\min\set{n\geq 0: S_n\in\set{-a,b}}
			\end{esn}y $a,b>0$. 
	\end{enumerate}

	\defin{Categor\'ias:} Caminatas aleatorias, muestreo opcional, ejemplos de martingalas. 
\end{problema}        
        \nqed
        
    \part{Tarea 3}
        \section{Problema 3.1}  \label{problema3_1}
\begin{problema}
Sea $M$ una $\paren{\F_n}$-martingala. Pruebe que si $T$ es un tiempo de paro finito entonces $\esp{M_T}=\esp{M_0}$ bajo cada una de las siguientes condiciones:
\begin{enumerate}
    \item \ref{problema3_1:inciso1}
        $M$ es acotada.
    \item \ref{problema3_1:inciso2}
        $T$ es integrable y la sucesi\'on $\paren{M_n-M_{n-1}}$ es acotada.
    \item \ref{problema3_1:inciso3}
        $\paren{M_{n\wedge T}}$ es uniformemente integrable. 
\end{enumerate}

\defin{Categor\'ias: } Muestreo opcional.
\end{problema}

\begin{proof}
    \subsection{Inciso (i)}     \label{problema3_1:inciso1}
    \emph{
    $M$ es acotada.
}

\afterstatement\pn

Que $M$ sea acotada significa que existen $a \in  \R$ tal que $\abs{M_n} < r$ para todo $n \in \N$. En particular 
tenemos que $\abs{M_{T \wedge n}} < r$.\pn

Por otro lado, que $T$ sea finito implica que $T \wedge n  \underset{c.s.}\longrightarrow T$ conforme $n \rightarrow \infty$.
(Para todo $\omega \in \Omega$, $T(\omega) = n_0 < \infty$ y por lo tanto $T \wedge n (\omega) = T(\omega) = n_0$ para todo
$n \geq n_0$). De donde afirmamos que $M_{T \wedge n} \underset{c.s.}\longrightarrow M_T$. \pn

Ahora, $T \wedge n$ es tiepo de paro acotado, y por Teorema del Muestreo Opcional de Doob tenemos que 
$\E(M_{T \wedge n}) = \E(M_0)$.\pn

Tenemos todas las hipótesis para aplicar el Teorema de Convergencia Acotada y por lo tanto:

\begin{align}
    \E(M_T)     &=  \E\left(\lim_{n \rightarrow \infty} M_{T \wedge n}\right)       \\
                &=  \lim_{n \rightarrow \infty} \E(M_{T \wedge n})                  \\
                &=  \E(M_0).
\end{align}\pn

Con lo que termina la demostración.
 
    \newpage
    
    \subsection{Inciso (ii)}    \label{problema3_1:inciso2}
    \emph{
    $T$ es integrable y la sucesi\'on $\paren{M_n-M_{n-1}}$ es acotada.
}

Que la sucesión $\paren{M_n-M_{n-1}}$ sea acotada, significa que existe $r \in R$ tal que
$\abs{M_n - M_{n-1}} < r$ para todo $n \geq 1$.\par\null

Que $T$ sea integrable significa que $\E(T) < \infty$ y por lo tanto que $\{ T = \infty \}$ tiene
probabilidad 0. De esto, podemos afirmar que $T \wedge n \underset{c.s.}\longrightarrow T$, pues los
valores de $T$ que $T \wedge n$ nunca va a alcanzar, son un conjunto de probabilidad cero.\par\null

Otra vez, $T \wedge n$ es tiepo de paro acotado, y por teorema de paro opcional de Doob tenemos que 
$\E(M_{T \wedge n}) = \E(M_0)$. En este caso nos interesa escribirlo de la siguiente manera, 
$\E(M_{T \wedge n} - M_0) = 0$.\par\null

Ahora, 
\begin{align}
    \abs{M_{T \wedge n} - M_0}  &=      \abs{\sum_{1 \leq i \leq (T \wedge n)} (M_i - M_{i-1})}     \\
                                &\leq   \sum_{1 \leq i \leq (T \wedge n)} \abs{(M_i - M_{i-1})}     \\
                                &\leq   (T \wedge n) r                                              \\      
                                &\leq   T r                                                    
\end{align}

De donde 

\begin{align}
        \E(\abs{M_{T \wedge n} - M_0})  &\leq   \E(T r) \\
                                        &= r \E(T)      \\
                                        &< \infty.
\end{align} \par\null

Con esto tenemos que la suceción $(M_{T \wedge n} - M_0)_{n \in \N}$ es dominada por $Tr$ y que cada 
elemento de la suceción es integrable. Entonces tenemos todas las hipótesis del Teorema de convergencia dominada.
Y por lo tanto

\begin{align}
        \E(M_T - M_0)   &=  \E(\lim_{n \rightarrow \infty} M_{T \wedge n} - M_0)        \\
                        &=  \lim_{n \rightarrow \infty} \E(M_{T \wedge n} - M_0)        \\
                        &=  \lim_{n \rightarrow \infty} 0                               \\
                        &=  0.                        
\end{align}

Lo cual implica $\E(M_T) = \E(M_0)$, como queríamos demostrar.
    \newpage
        
    \subsection{Inciso (iii)}    \label{problema3_1:inciso3}
    \emph{
    $\paren{M_{n\wedge T}}$ es uniformemente integrable. 
}
    \newpage
\end{proof}
\newpage

\section{Problema 3.2}  \label{problema3_2}
\begin{problema}
    Sea $M$ una $\paren{\F_n}$-martingala con saltos acotados. Sean

    \begin{align}
    C   &=  \set{\limsup M_n=\liminf M_n\in\re}                 \\
        &y                                                      \\
    D   &=  \set{\limsup M_n=-\infty\text{ y }\limsup M_n=\infty}.
    \end{align}\par\null

    Pruebe que $\proba{C\cup D}=1$. Deduzca que las caminatas aleatorias centradas 
    con saltos acotados oscilan. Sugerencia: Para cada $K>0$ defina

    \begin{align}
    T=\min\set{n\geq 0: \abs{M_n}\geq K}
    \end{align}

    y aplique el teorema de convergencia de martingalas a $M^T$.\par\null

    Sea $M$ una caminata aleatoria no trivial con saltos integrables en 
    $-1,0,1,\ldots$ y  media cero.\par\null 

    Pruebe que $\proba{M\text{ converge en }\na}=0$ y  concluya que $\liminf M_n=-\infty$ 
    casi seguramente. (Este resultado permitir\'a dar una prueba adicional de que un Galton-Watson cr\'itico se extingue).  
    Sugerencia: proceda como en el párrafo anterior y pruebe la integrabilidad uniforme de $(M_{T\wedge n})_{n \in \N}$.

    \defin{Categor\'ias: } Teoremas de convergencia de martingalas
\end{problema}

\afterstatement\par\null

Definamos 

\begin{align}
    T_k = \min\{ n \geq 0 : \abs{M_n} \geq k \}.
\end{align}

Que significa ``la primera vez que la martingala se aleja de cero al menos $k$ unidades''. 
Veamos que se trata de un tiempo de paro.


\begin{align}
    \{ T_k = n \}   &=  \paren{\bigcup_{i \leq n-1} \{ \abs{M_i} < k \}} \cap \paren{\{ \abs{M_n} \geq k \}}                                 \\
                    &=  \paren{\bigcap_{i \leq n-1} \{ M_i < k \} \cap \{ M_i > -k \}} \cap \paren{\{ M_n \geq k \} \cup \{ M_n \leq -k \}}.
\end{align}\par\null

Donde, $\{ M_i < k \}, \{ M_i > -k \} \in \F_i \subset \F_n$ y $\{ M_n \geq k \}, \{ M_n \leq -k \} \in \F_n$. 
Y por lo tanto $\{ T_k = n\} \in \F_n$, de donde concluimos que $T_K$ es efectivamente tiempo de paro.\par\null


Sea $C$ una cota para los saltos de $M$. Por definición de $T_k$, es claro que para el tiempo $T_k$ (cuando $T_k$ es finito), 
la martingala se aleja en al menos $k$ unidades de 0, pero forzosamente menos que $k + C$, pues su último salto no 
puede superar $C$. Es decir $\abs{M_{T_k}} < k + C$. Pero esto únicamente será cierto en los casos en que $T_k$ resulte 
ser finito. Tomemos en cuenta entonces al tiempo $T_k \wedge n$, qué satisface ser finito y que $T_k \wedge n \leq T_k$. 
Para este tiempo, nuestro razonamiento anterior es válido y entonces tenemos $\abs{M_{T_k \wedge n}} < k + C$.\par\null

Entonces $\E(\abs{M_{T_k \wedge n}}) < \E(k + C) = k + C$ para toda $n \in N$. Sabemos de [\ref{problema1_4}] que 
$(M_{T_k \wedge n})_{n \in \N}$ es martingala. Por el Teorema de Convergencia de Martingalas, tenemos que 
$M_{T_k \wedge n} \longrightarrow M_{T_k}$ y es finita c.s.\par\null

En particular $M_{T_k}(\{ T_k = \infty \}) < \infty$ c.s.\par\null



\newpage

\section{Problema 3.3}  \label{problema3_3}
\begin{problema}
Sean $X_1,X_2,\ldots$ variables aleatorias intercambiables:\begin{esn}
\paren{X_1,\ldots, X_n}\stackrel{d}{=}\paren{X_{\pi_1},\ldots, X_{\pi_n}}
\end{esn}para cada permutaci\'on $\sigma$ de $\set{1,\ldots,n}$. 
\begin{enumerate}
\item Para $\G,\h$ sub$\sigma$-\'algebras de $\F$ definimos a $\G\vee\h=\sag{\G\cup\h}$. Sea \begin{esn}
\G^n=\sag{\imf{f}{X_1,\ldots, X_n}: \fun{f}{\re^n}{\re}\text{ es sim\'etrica}}\vee\sag{X_{n+1},X_{n+2},\ldots}. 
\end{esn}Pruebe que $\G^n,n\geq 1$ es una filtraci\'on al rev\'es. Sea $\G$ su intersecci\'on.
\item Para cada $A\in\mc{B}_{\re}$, defina a\begin{esn}
\imf{\Xi_n}{A}=\frac{1}{n}\sum_{i=1}^n \indi{X_i\in A}.
\end{esn}Pruebe que\begin{esn}
\probac{X_1\in A}{\G^n}=\imf{\Xi_n}{A}. 
\end{esn}?`Por qu\'e puede definir a  $\imf{\Xi}{A}=\lim_{n\to\infty}\imf{\Xi_n}{A}$?
\item Al considerar a la martingala\begin{esn}
\frac{1}{n\paren{n-1}}\sum_{1\leq i<j\leq n}\indi{X_i\in A}\indi{X_j\in A},
\end{esn}pruebe que $\probac{X_1\in A,X_2\in A}{\G}=\probac{X_1\in A}{\G}\probac{X_2\in A}{\G}$. Extienda la afirmaci\'on de independencia condicional anterior a $X_1,\ldots, X_n$. 
\end{enumerate}

\defin{Cagegor\'ias: }Teorema de convergencia de martingalas, variables intercambiables, teorema de de Finetti.
\end{problema}
\newpage

\section{Problema 3.4}  \label{problema3_4}
\begin{problema}
\mbox{}
    \begin{enumerate}
        \item 
            Ejecute y explique la funci\'on del siguiente c\'odigo en Octave. Comente qu\'e 
            teoremas del curso (y del curso de probabilidad) son importantes para interpretar 
            la figura.
            \tiny
            \texttt
            {
                \lstinputlisting[caption=]{tarea3/problema3_4/polya2.R}
            }
            \normalsize
            (Conmigo se negó a brincar de linea. Tuve que hacerlo diminuto para que apareciera el código completo.)\par\null
        \item 
            Ejecute y explique la funci\'on del siguiente c\'odigo en Octave. 
            Incluya una gr\'afica en la que la longitud de la variable k sea mayor a 1000. 
            (Puede modificar el programa...) En la gr\'afica observara un esbozo de la 
            trayectoria de un proceso de ramificaci\'on continuo (en una escala distinta...).
            \texttt{
                \lstinputlisting[caption=]{tarea3/problema3_4/binaryGW.R}
            }
    \end{enumerate}
\end{problema}
        
        \nqed
            
    \part{Tarea 4}
        \section{Problema 4.1}
\begin{problema}
	Sean $\F_1,\F_2,\ldots $ y $\G$ sub\sa s de $\F$. 
	Decimos que $\F_1,\F_2,\ldots$ son condicionalmente independientes dada $\G$ si para 
	cualquier $H_i$ que sea $\F_i$ medible y acotada se tiene que
	
	\begin{esn}
		\espc{H_1\cdots H_n}{\G}=\espc{H_1}{\G}\cdots \espc{H_n}{\G}.
	\end{esn}
	
	\begin{enumerate}
		\item[(i)]	[\ref{problema4_1:inciso1}]
			?`Qu\'e quiere decir la independencia condicional cuando $\G=\set{\oo,\emptyset}$?
			\pn

		\item[(ii)] 	[\ref{problema4_1:inciso2}]
			Pruebe que $F_1$ y $\F_2$ son condicionalmente independientes dada $\G$ 
			(denotado $\condind{\F_1}{\F_2}{\G}$) si y s\'olo si para 
			cualquier $H$ que sea $\F_1$-medible y acotada se tiene que
			\begin{esn}
				\espc{H}{\F_2,\G}=\espc{H}{\G}.
			\end{esn}

		\item[(iii)]	[\ref{problema4_1:inciso3}]
			Pruebe que $\F_1,\F_2,\ldots, $ son condicionalmente independientes dada 
			$\G$ si y s\'olo si para cada $n\geq 1$, $\F_{n+1}$ es condicionalmente 
			independiente de $\F_1,\ldots, \F_n$ dada $\G$. 
	\end{enumerate}

	\defin{Categor\'ias: } Esperanza condicional, Independencia condicional.
\end{problema}

\begin{proof}
    \subsection{Inciso (i)} \label{problema4_1:inciso1}
    \emph{
	?`Qu\'e quiere decir la independencia condicional cuando $\G=\set{\oo,\emptyset}$?
}
\afterstatement\pn

Sabemos que cualquier $\sigma$-álgebra es
independiente de la $\sigma$-álgebra trivial $\G$. Y entonces $\E(X | \G) = \E(X)$, para toda $X$.\pn

Si $\F_1, \dots, \F_n$ son condicionalmente independietnes dada $\G$ y $H_i$ es $\F_i$-medible para toda $i$, entonces tenemos

\begin{align}
    \E(H_1 \cdots H_n) &=  \E(H_1 \cdots H_n | \G)            \\
                        &=  \E(H_1 | \G) \cdots \E(H_n | \G)    \\
                        &=  \E(H_1) \cdots  \E(H_n | \G).
\end{align}

Entonces, podemos concluir que cualesquiera funciones $H_i$ así definidas son siempre
independientes.\pn

Esto ocurre si y sólamente si las $\sigma$-álgebras $\F_1, \F_2,\dots$ son independientes.\pn

Entonces podemos concluir que cuando una sucesión de $\sigma$-álgebras $\F_1, \F_2,\dots$ son condicionalmente independientes
dada la $\sigma$-álgebra trivial, entonces son independientes.
    \newpage

    \subsection{Inciso (ii)} \label{problema4_1:inciso2}
    \emph{
		Pruebe que $F_1$ y $\F_2$ son condicionalmente independientes dada $\G$ 
		(denotado $\condind{\F_1}{\F_2}{\G}$) si y s\'olo si para 
		cualquier $H$ que sea $\F_1$-medible y acotada se tiene que
		\begin{esn}
			\espc{H}{\F_2,\G}=\espc{H}{\G}.
		\end{esn}
}
    \newpage

    \subsection{Inciso (iii)} \label{problema4_1:inciso3}
    \emph{
	Pruebe que $\F_1,\F_2,\ldots, $ son condicionalmente independientes dada 
	$\G$ si y s\'olo si para cada $n\geq 1$, $\F_{n+1}$ es condicionalmente 
	independiente de $\F_1,\ldots, \F_n$ dada $\G$. 
}

\afterstatement\pn

Primero demostraremos la suficiencia. Para ello emplearemos el ejercicio anterior.
Sean $H_1, \dots, H_{n+1}$ variables aleatorias $\F_1, \dots, \F_{n+1}$-medibles (respectivamente) 
y acotadas.\pn

Sea $P = \{ A : A = F_1 \cap \dots \cap F_n \cap G \text{ donde } F_i \in \F_i \text{ y } G \in \G \}$,
Es claro que $P \subset \sigma(\F_1, \dots, \F_n, \G)$. También es fácil ver que no es vacía ($\Omega$ es uno de sus elementos), 
y que es cerrada bajo intersecciones finitas. En otras palabras $C$ es un $\pi$-sistema. Además, justo
igual que antes, tenemos que $\sigma(P) = \sigma(\F_1, \dots, \F_n, \G)$.\pn

Ahora sea $D = \{ A \in \sigma(\F_1, \dots, \F_n, \G) : \E(H_{n+1} \indic_A) = \E(\E(H_{n+1} | \G) \indic_A) \}$

$D$ es un sistema de Dynkin. Verificar que $\Omega \in D$ es trivial. Y las propiedades de ser cerrado bajo sucesiones crecientes y 
bajo diferencias (diferencias donde un conjunto está contenido en el otro) se demuestran exactamente igual que en el inciso
anterior de este mismo problema [\ref{problema4_1:inciso2}].\pn

Entonces, como en el ejercicio anterior, basta demostrar que $P \subset D$, y entonces el lema de clases
de Dynkin nos asegurará que 
\begin{align}
 \E(H_{n+1} | \sigma(\F_1, \dots, \F_n, \G)) = \E(H_{n+1} | \G)   \label{problema4_1:hipotesis_suficiencia_4_1_3}
\end{align}\pn

Sea entonces $A \in P$. Donde $A = F_1 \cap \dots \cap F_n \cap G$ con $F_i \in \F_i$ y $G \in \G$.

\begin{align}
    \E(H_{n+1} \indic_A)    &=  \E(H_{n+1} \indic_{F_1} \cdots \indic_{F_n} \indic_{G})                                                     \\ 
                            &=  \E(\E(H_{n+1} \indic_{F_1} \cdots \indic_{F_n} \indic_{G} | \G))                                            \\ 
                            &=  \E(\indic_{G} \E(H_{n+1} \indic_{F_1} \cdots \indic_{F_n} | \G))                                            \\
                            &\comment{gracias a que $G \in \G$}                                                                             \\
                            &=  \E\bigg(\indic_{G} \E(H_{n+1} | \G) \E\big(\indic_{F_1} \cdots \indic_{F_n} | \G\big)\bigg)                 \\
                            &\lcomment{gracias a que $\F_1, \F_2, \dots$ son}                                                               \\
                            &\rcomment{condicionalmente independientes dada $\G$}                                                           \\
                            &=  \E\bigg(\E\bigg( \indic_{G} \E(H_{n+1} | \G) \indic_{F_1} \cdots \indic_{F_n} \bigg| \G\bigg)\bigg)         \\
                            &\lcomment{nótese que las variables que entraron son todas}                                                     \\
                            &\rcomment{$\G$-medibles}                                                                                       \\
                            &=  \E\bigg(\E\bigg( \E(H_{n+1} | \G) \indic_{G} \indic_{F_1} \cdots \indic_{F_n} \bigg| \G\bigg)\bigg)         \\
                            &=  \E\bigg(\E\bigg( \E(H_{n+1} | \G) \indic_{A} \bigg| \G\bigg)\bigg)                                          \\
                            &=  \E\big( \E(H_{n+1} | \G) \indic_{A} \big).
\end{align}\pn

Con esto queda demostrado que $A \in D$ y por lo tanto $P \subset D$. Utilizando el lema de clases de Dynkin obtenemos
\eqref{problema4_1:hipotesis_suficiencia_4_1_3}, y aplicando el resultado del inciso anterior [\ref{problema4_1:inciso2}],
obtenemos que $\F_{n+1}$ es condicionalmente independiente de $\F_1, \dots, \F_n$ dada $\G$.\pn


Ahora demostraremos la necesidad. Nuestra hipótesis es que para cada $n \geq 1$, $\F_{n+1}$ 
es condicionalemente independiente de $\F_1, \dots, \F_n$ dada $\G$. La demostración se hará por 
inducción sobre $n$.\pn

\begin{itemize}
	\item 
        \textbf{Base De Inducción.}
        
        Si $n = 1$, no hay nada que demostrar, pues por hipótesis $\F_2$ es condicionalmente independiente de $\F_1$ dada $\G$.\pn
        
    \item
        \textbf{Hipótesis De Inducción.}
        
        Sea $n \geq 2$ tal que $\F_1, \F_2, \dots, \F_n$ son condicionalmente independientes (nótese que la base de inducción 
        asegura que existe tal $n \geq 2$).\pn
        
    \item
        \textbf{Paso Inductivo.}
        
        Sean $H_i$ variables $\F_{i}$-medibles y acotadas con $1 \leq i \leq n+1$.\pn 
        
        Dado que $\F_{n + 1}$ es condicionalmente independiente de $\F_1, \dots, \F_n$ dada $\G$, la 
        equivalencia de [\ref{problema4_1:inciso2}] nos dice
        
        \begin{align}
            \E(H | \sigma(\F_1, \dots, \F_n, \G)) = \E(H | \G). \label{problema4_1:equivalencia_necesidad_4_1_3}
        \end{align}\pn

        Sea entonces $A \in \G$.
        
        \begin{align}
            \E(H_1 \cdots H_{n+1} \indic_A)     &=  \E(\E(H_1 \cdots H_{n+1} \indic_A | \sigma(\F_1, \F_2, \dots, \F_n, \G)))               \\
                                                    &=  \E(H_1 \cdots H_n \indic_A \E(H_{n+1}  | \sigma(\F_1, \F_2, \dots, \F_n, \G)))      \\
                                                    &=  \E(H_1 \cdots H_n \indic_A \E(H_{n+1}  | \G))                                       \\
                                                    &\comment{ésto último gracias a \eqref{problema4_1:equivalencia_necesidad_4_1_3}}       \\
                                                    &=  \E(\E(H_1 \cdots H_n \indic_A \E(H_{n+1}  | \G) | \G))                              \\
                                                    &=  \E(\E(H_1 \cdots H_n | \G) \indic_A \E(H_{n+1}  | \G))                              \\
                                                    &\comment{se sacaron las variables $\G$-medibles}                                       \\
                                                    &=  \E(\E(H_1 \cdots H_n | \G) \E(H_{n+1} | \G) \indic_A)                               \\
                                                    &=  \E(\E(H_1 | \G) \cdots \E(H_n | \G) \E(H_{n+1} | \G) \indic_A)                      \\
                                                    &\comment{por hipótesis de inducción}                                              
        \end{align}
            
        Por lo tanto, sobre $\G$, cualquier versión de $\E(H_1 | \G) \cdots \E(H_n | \G) \E(H_{n+1} | \G)$ siempre integra lo mismo que 
        $H_1 \cdots H_{n+1}$.\pn
\end{itemize}

Con esto queda demostrado que $F_1, F_2, \cdots$ son condicionalmente independientes dada $\G$.
\end{proof}
\newpage

\section{Problema 4.2}
\begin{problema}
Sea $\mu$ una distribuci\'on de progenie y defina $\tilde \mu_j=\mu_{j+1}$. Sea $S=\paren{S_n}$ una caminata aleatoria con distribuci\'on de salto $\tilde\mu$. Sea $k$ un entero no-negativo y defina recursivamente\begin{esn}
Z_0=k=C_0,\quad Z_{n+1}=k+S_{C_n}\quad\text{y} C_{n+1}=C_n+Z_{n+1}.
\end{esn}
\begin{enumerate}
\item Pruebe que $Z_n\geq 0$ para toda $n$ y que si $Z_n=0$ entonces $Z_{n+1}=0$.
\item Pruebe que $C_n$ es un tiempo de paro para la filtraci\'on can\'onica asociada a $S$.
\item Pruebe que $Z$ es un proceso de Galton-Watson con ley de progenie $\mu$. 
\item Pruebe que si $S$ alcanza $-1$ entonces existe $n$ tal que $Z_n=0$. Deduzca que si la media de $\mu$ es $1$ entonces $Z$ se extingue. (Sugerencia: utilice un ejercicio anterior sobre martingalas con saltos acotados hacia abajo.) 
\end{enumerate}

\defin{Categor\'ias: } Caminatas aleatorias, Procesos de Galton-Watson%, Propiedad de Markov fuerte.
\end{problema}
\newpage

\section{Problema 4.3}
\begin{problema}
	El objetivo de este ejercicio es ver ejemplos de cadenas de Markov $X$ y de funciones $f$ tales que 
	$\imf{f}{X}=\paren{\imf{f}{X_n},n\in\na}$ sean o no cadenas de Markov.
	
	\begin{enumerate}
	\item[(i)]	[\ref{problema4_3:inciso1}]
		Considere el hipercubo $n$-dimensional $E=\set{0,1}^n$. A $E$ lo pensaremos como la composici\'on 
		de la primera de dos urnas que tienen en total $n$ bolas etiquetadas del $1$ al $n$. 
		Si $x=\paren{x_1,\ldots, x_n}\in E$, interpretaremos $x_i=1$ como que la bola $i$ est\'a en la urna $1$. 
		Considere el siguiente experimento aleatorio: inicialmente la composici\'on de las urnas est\'a dada por 
		$x$ y a cada instante de tiempo escogemos una bola al azar y la cambiamos de urna. 
		Modele esta situaci\'on por medio de una cadena de Markov $X$ en $E$. Sea $\fun{f}{E}{\set{0,\ldots, n}}$ 
		dada por $\imf{f}{x}=\sum_i x_i$. Pruebe que $\imf{f}{X}=\paren{\imf{f}{X_n},n\in\na}$ es una cadena de 
		Markov cuya matriz de transici\'on determinar\'a.

	\item[(ii)]	[\ref{problema4_3:inciso2}]
		Sea $\paren{S_n}_{n\in\na}$ una cadena de Markov con espacio de estados $\z$ y matriz de transici\'on
		\begin{esn}
		P_{i,i+1}=p\quad P_{i,i-1}=1-p
		\end{esn}	
		donde $p\in [0,1]$. D\'e una condici\'on necesaria y suficiente para que $\paren{\abs{S_n},n\in\na}$ 
		sea una cadena de Markov.
	\end{enumerate}

\defin{Categor\'ias:} proyecciones de cadenas de Markov
\end{problema}

\begin{proof}
    \subsection{Inciso (i)} \label{problema4_3:inciso1}
    \emph{
	Considere el hipercubo $n$-dimensional $E=\set{0,1}^n$. A $E$ lo pensaremos como la composici\'on 
	de la primera de dos urnas que tienen en total $n$ bolas etiquetadas del $1$ al $n$. 
	Si $x=\paren{x_1,\ldots, x_n}\in E$, interpretaremos $x_i=1$ como que la bola $i$ est\'a en la urna $1$. 
	Considere el siguiente experimento aleatorio: inicialmente la composici\'on de las urnas est\'a dada por 
	$x$ y a cada instante de tiempo escogemos una bola al azar y la cambiamos de urna. 
	Modele esta situaci\'on por medio de una cadena de Markov $X$ en $E$. Sea $\fun{f}{E}{\set{0,\ldots, n}}$ 
	dada por $\imf{f}{x}=\sum_i x_i$. Pruebe que $\imf{f}{X}=\paren{\imf{f}{X_n},n\in\na}$ es una cadena de 
	Markov cuya matriz de transici\'on determinar\'a.
}
\afterstatement\pn

Sean $i,j \in E$. Definamos las siguientes entradas de una matriz de transición:

\begin{align}
        p_{i,j} &=   
                    \begin{cases}
                        \frac{1}{n},    &       \|i-j\| = 1 \comment{donde $\|\cdot\|$ es la norma euclidiana}  \\
                        0,              &       \text{en cualquier otro caso}
                    \end{cases}
\end{align}

Dado $i \in E$, sólo existen $n$ vectores que distan exáctamente $1$ de $i$. Entonces 
$\sum_{j \in E} \frac{1}{n} = n \frac{1}{n} = 1$ y nuestra matriz sí es de transición.\pn

Ya que nuestro proceso está restringido a comenzar en $x$, nuestra distribución inicial
será $\lambda = (\lambda_i : i \in E)$ tal que $\lambda_x = 1$ y $\lambda_i = 0$ si $i \not= x$.\pn

Es fácil notar que la evolución del proceso depende únicamente del presente. Para llegar a una configuración
de bolas en particular, únicamente es importante en qué configuración se encuentra actualmente.\pn

Dado que tenemos una matriz de transición y una distribución inicial bien definidas, existe una cadena de
Markov $X$ con dichas matriz de transcición y distribución inicial que describen la evolución del proceso.\pn

El proceso $(f(X_n))_{n \in N}$ es el proceso que cuenta la cantidad de bolas en la urna $1$ que equivale al proceso
de Ehrenfest, el cual es una cadena de Markov y las entradas de su matriz de transición están dadas por

\begin{align}
        p_{i,j}     &=  
                        \begin{cases}
                            \frac{n-i}{n},  &   j=i+1   \\
                            \frac{i}{n},    &   j=i-1   \\
                            0,              &   \text{en cualquier otro caso.}
                        \end{cases}
\end{align}
    \newpage

    \subsection{Inciso (ii)} \label{problema4_3:inciso2}
    \emph{
	Sea $\paren{S_n}_{n\in\na}$ una cadena de Markov con espacio de estados $\z$ y matriz de transici\'on
	\begin{esn}
	P_{i,i+1}=p\quad P_{i,i-1}=1-p
	\end{esn}	
	donde $p\in [0,1]$. D\'e una condici\'on necesaria y suficiente para que $\paren{\abs{S_n},n\in\na}$ 
	sea una cadena de Markov.
}
\end{proof}
\newpage

\section{Problema 4.4}
\begin{problema}
Sean $\p$ y $\q$ dos medidas de probabilidad en el espacio can\'onico $E^\na$ para sucesiones con valores en un conjunto a lo m\'as numerable $E$. Decimos que $\q$ es \defin{localmente absolutamente continua} respecto de $\p$ si para cada $n\in\na$, $\q|_{\F_n}\ll\p|_{\F_n}$. Sea\begin{esn}
D_n=\frac{d \q|_{\F_n}}{d \p|_{\F_n}}.
\end{esn}
\begin{enumerate}
\item Pruebe que $D$ es una martingala bajo $\p$. Pruebe que si $D$ es uniformemente integrable entonces $\q\ll\p$. 
\item Pruebe que si $T$ es un tiempo de paro finito entonces $\q|_{\F_T}\ll\p|_{\F_T}$. 
\item Sea $\p^p$ la distribuci\'on de una caminata aleatoria simple que comienza en $0$ y va de $k$ a $k+1$ con probabilidad $p$, donde $p\in (0,1)$. Pruebe que $\p^p$ es localmente absolutamente continua respecto de $\p^{1/2}$ y encuentre la martingala $D_n$ asociada.
\item Para $a,b>0$, sea $T=\min\set{n\in\na: X_n\in \set{-a,b}}$. Pruebe que $T$ y $X_T$ son independientes bajo $\p^{1/2}$. Al utilizar la continuidad absoluta local, pruebe que $T$ y $X_T$ tambi\'en son independientes bajo $\p^p$. Utilice alguna martingala de ejercicios anteriores para calcular $\esp{T^2}$. 
\end{enumerate}

\defin{Categor\'ias: }Cambio de medida, Caminata aleatoria simple.
\end{problema}        
        \nqed    
    
    \part{Tarea 5}
        \section{Problema 5.1}
\begin{problema}
	Sea $N$ un proceso Poisson de par\'ametro $\lambda$ y sea $T_n$ el tiempo de su en\'esimo salto. 
	
	\begin{enumerate}
		\item 
			Pruebe que condicionalmente a $T_2$, $T_1$ es uniforme en $[0,T_2]$. 
			
		\item 
			Pruebe que si $W_1$ y $W_2$  son  exponenciales de par\'ametro 
			$\lambda$  independientes entre si y de una variable uniforme $U$, 
			entonces $U\paren{W_1+W_2}$ es una variable aleatoria exponencial 
			de par\'ametro $\lambda$. 

		\item 
			Conjeture c\'omo se  generaliza lo anterior con $T_n$ y $T_1$.

		\item 
			Escriba dos programas en Octave que simulen al proceso de Poisson 
			de par\'ametro $\lambda$ en el intervalo $[0,1]$. En uno utilizar\'a 
			s\'olo variables exponenciales y en el otro puede utilizar una 
			variable Poisson.
	\end{enumerate}
\end{problema}

\newpage

\section{Problema 5.2}
\begin{problema}
%Simulaci\'on de un proceso Poisson puntual... subordinador...
    Sea $\Xi$ una medida de Poisson aleatoria en $(0,\infty)\times (0,\infty)$ cuya 
    medida de intensidad $\nu$ est\'a dada por $\imf{\nu}{ds,dx}=\indi{0<x} \indi{s<t}C/x^{1+\alpha}\, ds\,dx$. 
    
    \begin{enumerate}
        \item[(i)]		[\ref{problema5_2:inciso1}] 
			Determine los valores de $\alpha$ para los cuales $\int 1 \wedge x\,\imf{\nu}{dx}<\infty$. 
    \end{enumerate}
    
    Nos restringimos ahora a valores de $\alpha$ para los cuales la integral anterior sea finita. 
	Sean $\imf{f_t}{s,x}=\indi{s\leq t}x$ y $X_t=\Xi f_t$. 
    
    \begin{enumerate}[resume]
        \item[(ii)]		[\ref{problema5_2:inciso2}] 
			Determine los valores  de $\alpha$ para los cuales $X_t < \infty$ para toda $t \geq 0$ casi seguramente.
    \end{enumerate}

    Nos restringiremos a dichos valores de $\alpha$. 
    
    \begin{enumerate}[resume]
        \item[(iii)]	[\ref{problema5_2:inciso3}] 
			Calcule $\esp{e^{-\lambda X_t}}$ y pruebe que $X_{t}$ tiene la misma distribuci\'on que $t^{1/\alpha}X_1$. 
        \item[(iv)]		[\ref{problema5_2:inciso4}] 
			Diga por qu\'e el siguiente c\'odigo en Octave simula la trayectoria aproximada del proceso $X$ en el intervalo $[0,1]$.
			\texttt{
				\lstinputlisting{tarea5/problema5_2/SuborEst.m}
			}
    \end{enumerate}
\end{problema}

\begin{proof}
	\subsection{Inciso (i)}	\label{problema5_2:inciso1}
	\emph{
	Determine los valores de $\alpha$ para los cuales \pn
	$\int 1\wedge x\,\imf{\nu}{dx}<\infty$.
}

\afterstatement\pn

Separemos la integral de manera conveniente:

\begin{align}
   \int 1\wedge x\, \nu(dx) &= \int (1 \wedge x) \indi{0<x} C/x^{1+\alpha} \,dx                                                 \\
                            &= \int_0^\infty (1 \wedge x) C/x^{1+\alpha} \,dx                                                   \\
                            &= \int_0^1 (1 \wedge x) C/x^{1+\alpha} \,dx + \int_1^\infty (1 \wedge x) C/x^{1+\alpha} \,dx       \\
                            &= \int_0^1 x C/x^{1+\alpha} \,dx + \int_1^\infty C/x^{1+\alpha} \,dx                               \label{problema5_2:descomposicion_de_integral}
\end{align}

Para que la integral que nos interesa sea finita es necesario y suficiente que los dos sumandos de la última 
parte de la ecuación lo sean. \pn

Veamos qué ocurre en el primer termino si $\alpha = 1$.

\begin{align}
    \int_0^1 x C/x^{1+\alpha} \,dx  &=  \int_0^1 x C/x^{2} \,dx                                                 \\
                                    &=  \int_0^1 C/x \,dx                                                       \\
                                    &=  \lim_{t \rightarrow 0^+}\int_t^1 C/x \,dx                               \\
                                    &=  C \lim_{t \rightarrow 0^+} \int_t^1 1/x \,dx                            \\
                                    &=  C \lim_{t \rightarrow 0^+} [\log(x)]_t^1                                \\
                                    &=  C \lim_{t \rightarrow 0^+} \log(1) - \log(t)                            \\
                                    &=  C \lim_{t \rightarrow 0^+} - \log(t)                                    \\
                                    &=  C \cdot \infty.                                                                
\end{align}\pn

Por lo tanto $\alpha \not= 1$. Veamos ahora qué ocurre con el primer sumando de \eqref{problema5_2:descomposicion_de_integral}
suponiendo que $\alpha \not= 1$.

\begin{align}
    \int_0^1 x C/x^{1+\alpha} \,dx  &=  \int_0^1 C/x^{\alpha}   \,dx                                                                                    \\
                                    &=  C \int_0^1 x^{-\alpha} \,dx                                                                                     \\
                                    &=  C \left[ \frac{x^{(1-\alpha)}}{(1-\alpha)}\right]_0^1                                                           \\
                                    &=  C \left( \frac{1^{(1-\alpha)}}{(1-\alpha)} - \lim_{t \rightarrow 0}\frac{t^{(1-\alpha)}}{(1-\alpha)}\right)     \\
\end{align}\pn

El límte $\lim_{t \rightarrow 0}\frac{t^{(1-\alpha)}}{(1-\alpha)}$ exíste únicamente si $0 < 1-\alpha$, es decir, si $\alpha < 1$. Entonces basta con 
que $\alpha < 1$ para que el primer sumando de \eqref{problema5_2:descomposicion_de_integral} sea finito. Ahora, si $\alpha = 0$ y analizamos el 
segundo sumando de \eqref{problema5_2:descomposicion_de_integral} vemos que

\begin{align}
    \int_1^\infty C/x^{1+\alpha} \,dx   &=  C \int_1^\infty 1/x \,dx                                            \\ 
                                        &=  C \lim_{t \rightarrow \infty} \int_1^t 1/x \,dx                     \\ 
                                        &=  C \lim_{t \rightarrow \infty} \left[ \log(x) \right]_1^t            \\ 
                                        &=  C \lim_{t \rightarrow \infty} \log(t) - \log(1)                     \\ 
                                        &=  C \lim_{t \rightarrow \infty} \log(t)                               \\ 
                                        &=  C \cdot \infty.                                                       
\end{align}

Por lo cual necesitamos $\alpha \not= 0$. Supongamos entonces $\alpha \not= 0$ y analicemos el segundo sumando de \eqref{problema5_2:descomposicion_de_integral}:

\begin{align}
    \int_1^\infty C/x^{1+\alpha} \,dx   &=  C \int_1^\infty 1/x^{1+\alpha} \,dx                                                                     \\
                                        &=  C \int_1^\infty x^{-1-\alpha} \,dx                                                                      \\
                                        &=  C \lim_{t \rightarrow \infty} \int_1^t x^{-1-\alpha} \,dx                                               \\
                                        &=  C \lim_{t \rightarrow \infty} \left[ \frac{x^{-\alpha}}{-\alpha} \right]_1^t                            \\
                                        &=  C \left( \lim_{t \rightarrow \infty} \frac{t^{-\alpha}}{-\alpha} - \frac{1}{-\alpha} \right)            \\
\end{align}

Donde el límite $\lim_{t \rightarrow \infty} \frac{t^{-\alpha}}{-\alpha}$ existe únicamente si $-\alpha < 0$, es decir, si $0 < \alpha$.\pn

Entonces podemos concluir que los valores de $\alpha$ para los que \eqref{problema5_2:descomposicion_de_integral} es finito están en $(0, 1)$. 
	\newpage
	
	\subsection{Inciso (ii)}	\label{problema5_2:inciso2}
	\emph{
	Determine los valores  de $\alpha$ para los cuales $X_t < \infty$ para toda $t \geq 0$ casi seguramente.
}

\afterstatement\pn
	\newpage
	
	\subsection{Inciso (iii)}	\label{problema5_2:inciso3}
	\emph{
	Calcule $\esp{e^{-\lambda X_t}}$ y pruebe que $X_{t}$ tiene la misma distribuci\'on que $t^{1/\alpha}X_1$.
}

\afterstatement\pn

La primera parte de la proposición $4.6$ de la versión de las notas que se adjuntó en este documento (ver [\ref{notas}]) dice que
si $f$ es medible y no negativa entonces la integral de $f$ con respecto de $\Xi$ es una variable aleatoria y 

\begin{align}
    \E\left( e^{-\Xi f} \right)   &=  e^{-\int (1-e^{-f}) \,d\nu}.
\end{align}\pn

Usando esto mismo pero con $\Xi \lambda f_t = \lambda X_t$, tenemos

\begin{align}
    \E\left( e^{- \lambda X_t} \right)   &=  e^{-\int (1-e^{-\lambda f_t}) \,d\nu}. \label{problema5_2:resultado4_6}
\end{align}\pn.

Analicemos qué ocurre con $\int (1-e^{-\lambda f_t}) \,d\nu$.

\begin{align}
    \int (1-e^{-\lambda f_t}) \,d\nu    &=  \int \int (1-e^{-\lambda f_t})\frac{C}{x^{1+\alpha}}\,ds\,dx                                        \\
                                        &=  \int \int (1-e^{-\lambda \indic_{\{s \leq t\}} x})\frac{C}{x^{1+\alpha}}\,ds\,dx                    \\
                                        &=  \int_0^\infty \int_0^t (1-e^{-\lambda  x})\frac{C}{x^{1+\alpha}}\,ds\,dx                            \\
                                        &=  \int_0^\infty t (1-e^{-\lambda  x})\frac{C}{x^{1+\alpha}}\,dx                                       \\
                                        &=  tC \int_0^\infty  (1-e^{-\lambda  x})\frac{1}{x^{1+\alpha}}\,dx                                     \\
    \intertext{
    Ahora reescribimos a $(1-e^{-\lambda  x})$ como $\int_0^x \lambda e^{-\lambda y} \,dy$ y entonces la ecuación 
    anterior se transforma en}
                                        &=  tC \int_0^\infty  \left(\int_0^x  \lambda e^{-\lambda y} \,dy \right)\frac{1}{x^{1+\alpha}}\,dx     \\
                                        &=  tC \int_0^\infty  \frac{\int_0^x  \lambda e^{-\lambda y}}{x^{1+\alpha}}\,dy\,dx                     \\
                                        &=  tC \int_0^\infty  \int_0^x\frac{ \lambda e^{-\lambda y}}{x^{1+\alpha}} \lambda \,dy\,dx                      \\
    \intertext{
    La función que se encuentra en la integral es positiva, por lo tanto podemos aplicar el Teorema de Tonelli 
    para intercambiar las integrales de la siguiente manera:}
                                        &=  tC \int_0^\infty  \int_y^\infty\frac{ \lambda e^{-\lambda y}}{x^{1+\alpha}}\,dx\,dy                 \\
    \intertext{
    Recordando que $\alpha \in (0,1)$, tenemos que 
    $\int_y^\infty\frac{ 1 }{x^{1+\alpha}}\,dx = \lim_{x \rightarrow \infty} \frac{x^{-\alpha}}{-\alpha} - \frac{y^{-\alpha}}{-\alpha} = 
    - \frac{y^{-\alpha}}{-\alpha} = \frac{y^{-\alpha}}{\alpha}$. Sustituyendo esto tenemos}
                                        &=  tC \int_0^\infty  \frac{y^{-\alpha}}{\alpha} \lambda e^{-\lambda y}\,dy                             \\
                                        &=  \frac{tC}{\alpha} \int_0^\infty  y^{-\alpha}  e^{-\lambda y} \lambda\,dy                            \\
                                        &=  \frac{tC \lambda^{\alpha}}{\alpha} \int_0^\infty  (\lambda y)^{-\alpha}  e^{-\lambda y}\,dy         \\
    \intertext{
    Recordemos que $\int_0^\infty  (\lambda y)^{-\alpha}  e^{-\lambda y}\,dy$ se puede escribir 
    como $\Gamma(1-\alpha)$ y entonces nuestra ecuación se puede escribir como}
                                        &=  \frac{tC \lambda^{\alpha}}{\alpha} \Gamma(1-\alpha).                                                 \\
\end{align}

Ahora, regresando a \eqref{problema5_2:resultado4_6}, tenemos que 

\begin{align}
    \E\left( e^{- \lambda X_t} \right)  &=  e^{-\int (1-e^{-\lambda f_t}) \,d\nu}                               \\
                                        &=  e^{-\frac{tC \lambda^{\alpha}}{\alpha} \Gamma(1-\alpha)}            \\
\end{align}\pn

Ahora de la ecuación anterior, remplacemos $t$ por $1$ y entonces obtenemos

\begin{align}
    \E\left( e^{- \lambda X_1} \right)   &= e^{-\frac{C \lambda^{\alpha}}{\alpha} \Gamma(1-\alpha)}
\end{align}

De esto último, remplacemos ahora $\lambda$ por $\lambda t^{1/\alpha}$

\begin{align}
    \E\left( e^{- \lambda t^{1/\alpha} X_1} \right)   &= e^{-\frac{C (\lambda t^{1/\alpha})^{\alpha}}{\alpha} \Gamma(1-\alpha)}    \\
                                                      &= e^{-\frac{C \lambda^\alpha t}{\alpha} \Gamma(1-\alpha)}    \\
\end{align}

De donde podemos observar fácilmente que
\begin{align}
    \E\left( e^{- \lambda X_t} \right)  &=  \E\left( e^{- \lambda t^{1/\alpha} X_1} \right).
\end{align}

Lo que acabamos de escribir no son mas que las funciones generadoras de momentos de $X_t$ y $t^{1/\alpha} X_1$ evaluadas en $-\lambda$. Lo que 
acabamos de demostrar es que sus funciones generadoras de momentos son idénticas para cada $\lambda$ y por lo tanto, sus distribuciones son iguales.
	\newpage
	
	\subsection{Inciso (iv)}	\label{problema5_2:inciso4}
	\emph{
	Diga por qu\'e el siguiente c\'odigo en Octave simula la trayectoria aproximada del proceso $X$ en el intervalo $[0,1]$.
	\texttt{
		\lstinputlisting{tarea5/problema5_2/SuborEst.m}
	}
}

\afterstatement\pn
\end{proof}
\newpage

\section{Problema 5.3}
\begin{problema}
Pruebe que si $X$ tiene incrementos independientes entonces el proceso $X^t$ dado por $X^t_s=X_{t+s}-X_t$ es independiente de $\F^X_t=\sag{X_s:s\geq 0}$.

Calcular la esperanza y varianza del proceso de Poisson y de Poisson compuesto (en t\'erminos de la intensidad y la distribuci\'on de salto). Probar que si $X$ es\begin{esn}
\esp{e^{iu Z_t}}=e^{-\lambda t\paren{1-\imf{\psi}{u}}}\quad\text{donde}\quad \imf{\psi}{u}=\esp{e^{iu \xi_1}}. 
\end{esn}

Sea $N$ un proceso de L\'evy tal que $N_t$ tiene distribuci\'on de par\'ametro $\lambda t$. 
\begin{enumerate}
\item Pruebe que casi seguramente las trayectorias de $N$ son no-decrecientes.
\item Sea $\Xi$ la \'unica medida en $\mc{B}_{\re_+}$ tal que $\imf{\Xi}{[0,t]}=N_t$. Pruebe que $\Xi$ es una medida de Poisson aleatoria de intensidad $\lambda \cdot\leb$.
\item Concluya que $N$ es un proceso de Poisson de intensidad $\lambda$. 
\end{enumerate}
\end{problema}
\newpage

\section{Problema 5.4}
\begin{problema}
Sea $P_t$ la probabilidad de transici\'on en $t$ unidades de tiempo para el proceso de Poisson de par\'ametro $\lambda$. 

Al utilizar el teorema del biniomio, pruebe directamente que las probabilidades de transici\'on del proceso de Poisson satisfacen las ecuaciones de Chapman-Kolmogorov $P_{t+s}=P_tP_s$. D\'e adem\'as un argumento probabil\'istico, basado en condicionar con lo que sucede al tiempo $s$, para probar dicha ecuaci\'on. 

Sea\begin{esn}
\imf{Q}{i,j}=\begin{cases}
-\lambda&j=i\\
\lambda&j=i+1\\
0&j\neq i,i+1
\end{cases}.
\end{esn}Pruebe directamente que se satisfacen las ecuaciones de Kolmogorov\begin{equation*}
%\label{CKEquationsForPoisson}
\frac{d}{dt}\imf{P_t}{i,j}=\imf{QP_t}{i,j}=\imf{P_tQ}{i,j},
\end{equation*}donde $QP_t$ es el producto de las matrices $Q$ y $P_t$.
\end{problema}

\newpage

\section{Problema 5.5}
\begin{problema}[Tomado del examen general de probabilidad del Posgrado en Ciencias Matem\'aticas, UNAM, \href{http://www.posgradomatematicas.unam.mx/contenidoEstatico/archivo/files/pdf/Examenes_Generales/Probabilidad/Probabilidad2011-1.pdf}{Febrero 2011}]
Una planta de producci\'on toma su energ\'ia de dos generadores. La cantidad de generadores al tiempo $t$ est\'a representado por una cadena de Markov a tiempo continuo $\set{X_t,t\geq 0}$ con espacio de estados $E=\set{0,1,2}$ y matriz infinit\'esimal $Q$ dada por\begin{esn}
Q=\begin{pmatrix}
-6&6&0\\
1&-7&6\\
0&2&-2
\end{pmatrix}.
\end{esn}
\begin{enumerate}
\item Encuentre la matriz de transici\'on de la cadena de Markov de los estados distintos que toma $X$, clasifique los estados, diga si existe una \'unica distribuci\'on invariante y en caso afirmativo, encu\'entrela. Calcule expl\'icitamente las potencias de la matriz de transici\'on. (Recuerde que de ser posible diagonalizar, esta es una buena estrategia.)
\item ?`Cu\'al es la probabilidad de que ambos generadores est\'en trabajando al tiempo $t$ si s\'olo uno trabaja al tiempo cero? 
\item Si $\rho_2$ denota la primera vez que ambos generadores est\'an trabajando al mismo tiempo, encuentre la distribuci\'on de $\rho_2$ cuando s\'olo un generador est\'a trabajando al tiempo cero. 
\item Encuentre la proporci\'on de tiempo asint\'otica en que los dos generadores est\'an trabajando. Si cada generador produce 2.5 MW de energ\'ia por unidad de tiempo, ?`Cu\'al es la cantidad promedio de energ\'ia producida a largo plazo por unidad de tiempo?
\end{enumerate}
\end{problema}

\newpage

\section{Problema 5.6}
\begin{problema}[Procesos de ramificaci\'on a tiempo continuo]
    Sea $\mu$ una distribuci\'on en $\na$. A $\mu_k$ lo interpretamos como la probabilidad de 
    que un individuo tenga $k$ hijos. Nos imaginamos la din\'amica de la poblaci\'on como sigue: 
    a tasa $\lambda$, los individuos de una poblaci\'on se reproducen. Entonces tienen $k$ hijos 
    con probabilidad $\mu_k$. Se pueden introducir dos modelos: uno en que el individuo que se 
    reproduce es retirado de la poblaci\'on (nos imaginamos que muere) y otro en que no es retirado 
    de la poblaci\'on (por ejemplo cuando se interpreta a la poblaci\'on como especies y a sus 
    descendientes como mutaciones). En el caso particular del segundo modelo en que $\mu_1=1$, 
    se conoce como proceso de Yule. 

    \begin{enumerate}
        \item[(i)]      [\ref{problema5_6:inciso1}]
            Especifique un modelo de cadenas de Markov a tiempo continuo para cada uno de los
            modelos anteriores. A estos procesos se les conoce como procesos de ramificaci\'on
            a tiempo continuo.\pn
    \end{enumerate}

        Nuestro primer objetivo ser\'a encontrar una relaci\'on entre procesos de ramificaci\'on a
        tiempo continuo y procesos de Poisson compuestos. Sea $N$ un proceso de Poisson  y $S$ una
        caminata aleatoria independiente de $N$ tal que $\proba{S_1=j}=\mu_{j-1}$ \'o $\mu_{j}$
        dependiendo de si estamos en el primer caso o en el segundo. Sea $k\geq 0$ y definamos a
        $X_t=k+S_{N_t}$.
        
    \begin{enumerate}[resume]
        \item[(ii)]     [\ref{problema5_6:inciso2}] 
            Diga brevemente por qu\'e $X$ es una cadena de Markov a tiempo continuo e identifique 
            su matriz infinitesimal para ambos modelos.\pn
    \end{enumerate}

        Sea ahora $\tau=\min\set{t\geq 0: X_t=0}$ y $Y_t=X_{t\wedge \tau}$. 
        
    \begin{enumerate}[resume]
        \item[(iii)]    [\ref{problema5_6:inciso3}] 
            Argumente por qu\'e $Y$ es una cadena de Markov a tiempo continuo e identifique su 
            matriz infinitesimal.\pn
            
        \item[(iv)]     [\ref{problema5_6:inciso4}] 
            Argumente por qu\'e existe un \'unico proceso $Z$ que satisface
            \begin{esn}
                Z_t=Y_{\int_0^t Z_s\, ds}
            \end{esn}
            y que dicho proceso es un proceso de ramificaci\'on a tiempo continuo. Sugerencia: Recuerde que las 
            trayectorias de $Y$ son constantes por pedazos.\pn
    \end{enumerate}

    Ahora nos enfocaremos en el proceso de Yule. 

    \begin{enumerate}[resume]
        \item[(v)]      [\ref{problema5_6:inciso5}]
            Escriba las ecuaciones backward de Kolmogorov para las probabilidades de transici\'on 
            $\imf{P_t}{x,y}$. Al argumentar por qu\'e $\imf{P_{t}}{x,x}=e^{-\lambda x}$, resuelva 
            las ecuaciones backward por medio de la t\'ecnica de factor integrante (comenzando con 
            $\imf{P_t}{x,x+1}$) y pruebe que
            \begin{esn}
                \imf{P_t}{x,y}=\binom{y-1}{y-x} e^{-\lambda x t}\paren{1-e^{-\lambda t}}^{y-x}.
            \end{esn}\pn
        
        \item[(vi)]     [\ref{problema5_6:inciso6}]
            Al utilizar la f\'ormula para la esperanza de una variable binomial negativa, 
            pruebe que
            \begin{esn}
                \imf{\se_x}{Z_t}= xe^{\lambda t}.
            \end{esn}\pn
        
        \item[(vii)]    [\ref{problema5_6:inciso7}]
            Pruebe que $e^{-\lambda t}Z_t$ es una martingala no-negativa y que por lo tanto 
            converge casi seguramente a una variable aleatoria $W$.\pn
        
        \item[(viii)]   [\ref{problema5_6:inciso8}]
            Al calcular la transformada de Laplace de $e^{-\lambda t}Z_t$, pruebe que $W$ tiene 
            distribuci\'on exponencial. Por lo tanto, argumente que casi seguramente $Z$ crece exponencialmente.
            %La distribuci�n l�mite est� tomada de Beroin-Goldschmidt, ellos citan y corrigen un error de Athreya.
            \pn
    \end{enumerate}
\end{problema}

\begin{proof}
    \subsection{Inciso (i)} \label{problema5_6:inciso1}
    \emph{
    Especifique un modelo de cadenas de Markov a tiempo continuo para cada uno de los
    modelos anteriores. A estos procesos se les conoce como procesos de ramificaci\'on
    a tiempo continuo.\pn
}
\afterstatement\pn
    \newpage

    \subsection{Inciso (ii)} \label{problema5_6:inciso2}
    \emph{
    Diga brevemente por qu\'e $X$ es una cadena de Markov a tiempo continuo e identifique 
    su matriz infinitesimal para ambos modelos.\pn
}
\afterstatement\pn
    \newpage

    \subsection{Inciso (iii)} \label{problema5_6:inciso3}
    \emph{
    Argumente por qu\'e $Y$ es una cadena de Markov a tiempo continuo e identifique su 
    matriz infinitesimal.\pn
}
\afterstatement\pn
	\newpage
	
    \subsection{Inciso (iv)} \label{problema5_6:inciso4}
    \emph{
    Argumente por qu\'e existe un \'unico proceso $Z$ que satisface
    \begin{esn}
        Z_t=Y_{\int_0^t Z_s\, ds}
    \end{esn}
    y que dicho proceso es un proceso de ramificaci\'on a tiempo continuo. Sugerencia: Recuerde que las 
    trayectorias de $Y$ son constantes por pedazos.\pn
}

\afterstatement\pn

Definamos $T_n$ como en [\ref{problema5_6:inciso1}]. Nos interesa ver cómo se comporta $Y_t$ en los intervalos
$[T_{n-1}, T_{n})$, pues en estos intervalos $Y_t$ es constante.\pn

Analizando el primer segumento, sea $t \in [T_{0}, T_{1})$, donde la población no cambia, tenemos que 
$int_{0}^{t} Y_t = tk$
    \newpage
    
    \subsection{Inciso (v)} \label{problema5_6:inciso5}
    \emph{
    Escriba las ecuaciones backward de Kolmogorov para las probabilidades de transici\'on 
    $\imf{P_t}{x,y}$. Al argumentar por qu\'e $\imf{P_{t}}{x,x}=e^{-\lambda x}$, resuelva 
    las ecuaciones backward por medio de la t\'ecnica de factor integrante (comenzando con 
    $\imf{P_t}{x,x+1}$) y pruebe que
    \begin{esn}
        \imf{P_t}{x,y}=\binom{y-1}{y-x} e^{-\lambda x t}\paren{1-e^{-\lambda t}}^{y-x}.
    \end{esn}\pn
}
\afterstatement\pn

Esta vez, cada individuo tiene exáctamente un descendiente y no muere. Entonces, su matriz infinitesimal
es como en [\ref{problema5_6:inciso2}], sólo que $\mu_n = 0$ si $n \neq 1$ y $\mu_1 = 1$. Es decir
\begin{align}
    Q(i,j)  &=
            \begin{cases}
                \lambda i \mu_{1}       &   \text{si $j = i+1$ e $i \neq 0$}        \\
                -\lambda i              &   \text{si $i = j$}                       \\
                0                       &   \text{en cualquier otro caso}
            \end{cases}
\end{align}\pn

Entonces
\begin{align}
    \frac{d}{dt}    P_t(i, j)   &=  \sum_{k \in \N} Q(i,k)P_t(k,j)                  \\
                                &=  Q(i,i+1)P_t(i+1,j)  +   Q(i,i)P_t(i,j)          \\
                                &=  \lambda i P_t(i+1,j)  - \lambda i P_t(i,j)      \\
\end{align}\pn

Supongamos entonces $i = j$ y entonces 
\begin{align}
    \frac{d}{dt}    P_t(i, i)   &=  \lambda i P_t(i+1,i)  - \lambda i P_t(i,i)      \\
                                &=  -i \lambda P_(i, i).
\end{align}\pn

Donde, la solución a la ecuación diferencial

\begin{align}
    \frac{d}{dt}    P_t(i, i)   &=  - \lambda i P_(i, i).
\end{align}

es
\begin{align}
    P_t(i, i)   &=  e^{-t i \lambda}    
\end{align}

Supongamos ahora $j = i + 1$ y entonces
\begin{align}
     \frac{d}{dt}    P_t(i, i+1)    &=  \lambda i P_t(i+1,i+1)  - \lambda i P_t(i,i+1)          \\   
                                    &=  \lambda i e^{-t (i+1) \lambda}  - \lambda i P_t(i,i+1)  \\
                                    &\comment{la sustitución es por el caso recién demostrado}
\end{align}

De donde
\begin{align}
     \frac{d}{dt}P_t(i, i+1)  +  \lambda i P_t(i,i+1)                                       &&=&&&      \lambda i e^{-t (i+1) \lambda}                  \\
     (e^{\lambda i t}) \frac{d}{dt}P_t(i, i+1)  + (e^{\lambda i t}) \lambda i P_t(i,i+1)    &&=&&&      (e^{\lambda i t}) \lambda i e^{-t (i+1) \lambda}\\
     \comment{se multiplicó en ambos lados por lo mismo}                                    && &&&                                                      \\
     (e^{\lambda i t}) \frac{d}{dt}P_t(i, i+1)  + (e^{\lambda i t}) \lambda i P_t(i,i+1)    &&=&&&      \lambda i e^{-t \lambda}                        \\
      \frac{d}{dt} \bigg( (e^{\lambda i t}) P_t(i, i+1) \bigg)                              &&=&&&      \lambda i e^{-t \lambda}                        \\
     \comment{la suma se escribió como la derivada de un producto}                          && &&&                                                      \\
\end{align}

Este último término, lo podemos integrar suponiendo la condición inicial $P_0(i, i+1) = 0$
\begin{align}
       (e^{\lambda i t}) P_t(i, i+1)    &=  \int \frac{d}{dt} \bigg( (e^{\lambda i t}) P_t(i, i+1) \bigg) dt    \\      
                                        &=  \int_{0}^{t}    \lambda i e^{-t \lambda} dt                         \\    
                                        &=  i (1-e^{\lambda t})                         
\end{align}\pn

 Y por lo tanto
\begin{align}
       P_t(i, i+1)  &=  i (1-e^{\lambda t})(e^{-\lambda i t})                         
\end{align}\pn

Hagamos entonces la hipótesis de inducción (que mágicamente se ha cumplido hasta ahora, pero que para encontrar fue necesario hacer más cuentas)
\begin{align}
    P_t(i, i+k)     &=  \binom{i + k - 1}{k} e^{-\lambda i t} (1 - e^{- \lambda t})^k    
\end{align}\pn

Probemos entonces que para $j = i + k + 1$ también se cumple:
\begin{align}
    \frac{d}{dt} P_t(i, i + k + 1)  &=  \lambda i P_t(i + 1, i + 1 + k) - \lambda i P_t(i, i + 1 + k)                                    \\
                                    &=  \lambda i \binom{i + k }{k} e^{-\lambda (i + 1) t} (1 - e^{- \lambda t})^k - \lambda i P_t(i, i + 1 + k)   \\
\end{align}\pn

Y entonces
\begin{align}
    \frac{d}{dt} P_t(i, i + k + 1) - \lambda i P_t(i, i + 1 + k)                                    &=&&  \lambda i \binom{i + k}{k} e^{-\lambda (i + 1) t} (1 - e^{- \lambda t})^k                    \\
    (e^{\lambda i t}) \bigg( \frac{d}{dt} P_t(i, i + k + 1) - \lambda i P_t(i, i + 1 + k) \bigg)    &=&&  \lambda i \binom{i + k}{k} (e^{\lambda i t}) e^{-\lambda (i + 1) t} (1 - e^{- \lambda t})^k  \\
    \comment{se multiplicó en ambos lados por lo mismo, otra vez}                                   & &&                                                                                      \\
    \frac{d}{dt} \bigg((e^{\lambda i t}) P_t(i, i + k + 1)\bigg)                                    &=&&  \lambda i \binom{i + k}{k}  e^{-\lambda t} (1 - e^{- \lambda t})^k                           \\
     \comment{la suma se escribió como la derivada de un producto}                                  & &&
\end{align}\pn

Por lo tanto
\begin{align}
    (e^{\lambda i t}) P_t(i, i + k + 1)     &=  \int \frac{d}{dt} \bigg((e^{\lambda i t}) P_t(i, i + k + 1)\bigg)     dt            \\
                                            &=  \int_{0}^{t} \lambda i \binom{i + k}{k} e^{-\lambda t}(1 - e^{- \lambda t})^k    dt \\          
                                            &=  \binom{i + k}{k} i  \int_{0}^{t} \lambda  e^{-\lambda t} (1 - e^{- \lambda t})^k dt \\          
                                            &=  \binom{i + k}{k} \frac{i}{k+1}  (1-e^{- \lambda t})^{k+1}                           \\          
                                            &=  \binom{i + k + 1}{k + 1} (1-e^{- \lambda t})^{k+1}                                  \\          
\end{align}\pn

y entonces
\begin{align}
    P_t(i, i + k + 1)     &=  \binom{i + k + 1}{k + 1} (e^{-\lambda i t})(1-e^{- \lambda t})^{k+1}                                  \\
\end{align}\pn

como queríamos demostrar.

Entonces basta sustituir a $k = j - i$ para tener una fórmula en términos de $i$, $j$.\pn
\begin{align}
    P_t(i, j)     &=  \binom{j - 1}{j - i} e^{-\lambda i t} (1 - e^{- \lambda t})^k    
\end{align}\pn
    \newpage

    \subsection{Inciso (vi)} \label{problema5_6:inciso6}
    \emph{
    Al utilizar la f\'ormula para la esperanza de una variable binomial negativa, 
    pruebe que
    \begin{esn}
        \imf{\se_x}{Z_t}= xe^{\lambda t}.
    \end{esn}\pn
}

\afterstatement\pn

Del inciso anterior, tenemos que $P_t(i,j) = \P(W = i - j)$ donde $W$ es una variable aleatoria con
distribución binomial negativa de parámetros $r = x$ y $p = e^{-\lambda t}$ 
[véase \href{http://en.wikipedia.org/wiki/Negative_binomial_distribution}{distribución binomial negativa}].\pn

Entonces
\begin{align}
        \E_x(Z_t)   &=  \sum_{j} j \P_x(Z_t = j)   \\
                    &=  \sum_{j} j P_t(x, j)       \\
                    &=  \E(W)                      \\
                    &=  \frac{r}{p}                \\
                    &=  \frac{x}{e^{-\lambda t}}   \\
                    &=  x e^{\lambda t}.
\end{align}\pn

Como buscábamos demostrar.
    \newpage

    \subsection{Inciso (vii)} \label{problema5_6:inciso7}
    \emph{
    Pruebe que $e^{-\lambda t}Z_t$ es una martingala no-negativa y que por lo tanto 
    converge casi seguramente a una variable aleatoria $W$.\pn
}
\afterstatement\pn
	\newpage
	
    \subsection{Inciso (viii)} \label{problema5_6:inciso8}
    \emph{
    Al calcular la transformada de Laplace de $e^{-\lambda t}Z_t$, pruebe que $W$ tiene 
    distribuci\'on exponencial. Por lo tanto, argumente que casi seguramente $Z$ crece exponencialmente.
    %La distribuciÑn lÕmite està tomada de Beroin-Goldschmidt, ellos citan y corrigen un error de Athreya.
    \pn
}
\afterstatement\pn
\end{proof}
\newpage

\section{Problema 5.7}
\begin{problema}

(Tomado del examen general de conocimientos del \'area de Probabilidad del Posgrado en Ciencias Matem\'aticas, UNAM, \href{http://www.posgradomatematicas.unam.mx/contenidoEstatico/archivo/files/pdf/Examenes_Generales/Probabilidad/Probabilidad2011-2.pdf}{Agosto 2011})

Sea $N$ un proceso de Poisson homog\'eneo de par\'ametro $\lambda$. Sea $E=\paren{-1,1}$ y $X_0$ una variable aleatoria con valores en $E$ independiente de $N$. Se define el proceso\begin{esn}
X_t=X_0 \times \paren{-1}^{N_t}, \quad t\geq 0.
\end{esn}
\begin{enumerate}
\item Explique por qu\'e $X$ es una cadena de Markov a tiempo continuo con valores en $E$. 
\item Calcule sus probabilidades de transici\'on y su matriz infinitesimal. 
\item ?`Existe una distribuci\'on estacionaria para esta cadena? En caso afirmativo ?'Cu\'al es?
\end{enumerate}
\end{problema}

\newpage

\section{Problema 5.8}
\begin{problema}
Sea\begin{esn}
Q=\begin{pmatrix}
-2&2\\
3&-3
\end{pmatrix}.
\end{esn}\begin{enumerate}
\item Haga un programa en octave que permita simular las trayectorias de una cadena de Markov a tiempo continuo $X$ con matriz infinitesimal $Q$.
\item Utilice su programa para generar 10000 trayectorias en el intervalo de tiempo $[0,10]$ comenzando con probabilidad $1/2$ en cada estado y obtenga la distribuci\'on emp\'irica de $X_10$. 
\item Calcule $e^{10Q}$ (utilizando alg\'un comando adecuado) y contraste con la distribuci\'on emp\'irica del inciso anterior.
\item Codifique el siguiente esquema num\'erico, conocido como m\'etodo de Euler, para aproximar a $e^{10 Q}$: escoja $h>0$ peque\~no, defina a $P^h_0$ como la matriz identidad y recursivamente\begin{esn}
P^h_{i+1}=P^h_i+hQP^h_i. 
\end{esn}corra hasta que $i=\floor{10/h}$ y compare la matriz resultante con $e^{10Q}$. Si no se parecen escoja a $h$ m\'as peque\~no. ?`Con qu\'e $h$ puede aproximar a $e^{10Q}$ a 6 decimales?
\end{enumerate}
\end{problema}        
        \nqed
    
    \part{Tarea 6}
        \section{Problema 6.1}\label{problema6_1}
	\begin{problema}
	Un proceso estoc\'astico $B=\paren{B_t,t\geq 0}$ es un movimiento 
	browniano en ley si y s\'olo si es un proceso gaussiano centrado y $\esp{B_sB_t}=s\wedge t$. 
\end{problema}

\afterstatement\pn

Necesidad. Sea $B = (B_t, t \geq 0)$ un movimiento browniano en ley. Sabemos que los incrementos de un
movimiento browniano conforman un vector gaussiano y que son independientes. También sabemos que $\E(B_t) = 0$ (que el proceso es centrado), 
y que $\E(B_s B_t) = s \wedge t$.\pn

Sólo falta ver que se trata de un proceso gaussiano.\pn

Sean $0 \leq t_1 < t_2 < \dots < t_n$.\pn

Para cualquier combinación lineal 
\begin{align}
        \sum_{1 \leq i \leq n}  \alpha_i B_{t_i}
\end{align}\pn

Podemos encontrar $\beta_i$'s ($1 \leq i \leq n$) tales que:
\begin{align}
        \sum_{1 \leq i \leq n}  \alpha_i B_{t_i} = \sum_{1 \leq i \leq n}  \beta_i (B_{t_i} - B_{t_{i-1}}) 
\end{align}\pn

(Donde por comodidad de notación $B_{t_0} = 0$). Para el argumento de que podemos encontrar a las $\beta_i$'s, considérese el sistema
de ecuaciones lineales $\alpha_i = \beta_i - \beta_{i+1}$, donde $\beta_{n+1} = 0$ y entonces tenemos un sistema lineal de $n$ 
ecuaciones y $n$ variables.\pn

Entonces, cualquier combinación lineal del estilo $\sum_{1 \leq i \leq n}  \alpha_i B_{t_i}$ es una combinación linea de
variables aleatorias normales independientes y por lo tanto normal. Es decir que $(B_{t_1}, B_{t_2},\dots ,B_{t_n})$, es un vector gaussiano. Que es lo único
que hacía falta.\pn

Suficiencia. Sea $B =   \paren{B_t,t\geq 0}$ un proceso gaussiano centrado tal que $\E(B_s B_t) = s \wedge t$.\pn

Veamos que el proceso comienza en $0$. Que el proceso sea centrado nos dice que $\E(B_0) = 0$. Como la esperanza es $0$, entonces
la varianza es $Var(B_0) = \E(B_0^2) = \E(B_0 B_0) = 0 \wedge 0 = 0$. Entonces $B_0 = 0$ c.s.\pn


De manera similar, como $\E(B_t) = 0$, tenemos que $Var(B_t) = \E(B_t B_t) = t$. Como $B_t$ es una combinación linea de un proceso gaussiano,
entonces tiene distribución normal. Y como acabamos de ver, con media 0 y varianza $t$.

Igual que antes, para cualquier combinación lineal
\begin{align}
    \sum_{1 \leq i \leq n}  \beta_i (B_{t_i} - B_{t_{i-1}})    
\end{align}
(Donde por comodiad de notación $B_{t_0} = 0$). Podemos encontrar $\alpha_i$'s ($1 \leq i \leq n$) tales que

\begin{align}
    \sum_{1 \leq i \leq n}  \beta_i (B_{t_i} - B_{t_{i-1}}) = \sum_{1 \leq i \leq n}  \alpha_i B_{t_i}.
\end{align}\pn

(El argumento es el mismo que arriba). Y entonces tenemos que los incrementos siempre forman un vector gaussiano. 
Para ver la independencia de los incrementos sean $1 \leq i < j \leq n$ y entonces

\begin{align}
        &   \;\;\;\;\;\E((B_{t_i} - B_{t_{i-1}})(B_{t_j} - B_{t_{j-1}}))                                                \\
        &=  \E(B_{t_{i}}B_{t_{j}} - B_{t_{i}}B_{t_{j-1}} - B_{t_{i-1}}B_{t_{j}} + B_{t_{i-1}}B_{t_{j-i}})               \\
        &=  \E(B_{t_{i}}B_{t_{j}}) - \E(B_{t_{i}}B_{t_{j-1}}) - \E(B_{t_{i-1}}B_{t_{j}}) + \E(B_{t_{i-1}}B_{t_{j-i}})   \\
        &=  t_{i} \wedge t_{j} - t_{i} \wedge t_{j-1} - t_{i-1} \wedge t_{j} + t_{i-1} \wedge t_{j-i}                   \\
        &=  t_{i} - t_{i} - t_{i-1} + t_{i-1}                                                                           \\
        &=  0.
\end{align}\pn

Y por lo tanto, la correlación entre los incrementos, siempre es $0$. Por lo que son independientes.\pn

Ahora sean $s,t \geq 0$. Sabemos que $\E(B_{t+s} - B_t) =\E(B_{t+s}) - \E(B_t) = 0 - 0 = 0$. Así que $Var(B_{t+s} - B_t) = \E((B_{t+s} - B_t)^2)$. 
Desarrollando de manera similar a hace un momento
\begin{align}
    E((B_{t+s} - B_t)^2)    &=  \E(B_{t+s}B_{t+s}) - 2\E(B_{t+s}B_{t}) + \E(B_{t}B_{t})     \\
                            &=  t+s -2t + t                                                 \\
                            &=  s.
\end{align}

Además $B_{t+s} - B_{t}$ es una combinación lineal de variables de un proceso gaussiano, y por lo tanto tiene distribución gaussiana con media $0$ 
y varianza $s$ (por lo que acabamos de demostrar). Por lo tanto $(B_{t+s} - B_t) \sim B_s$. Entonces tenemos $B$ cumple con todas las hipótesis para ser
movimiento browniano en ley.
\newpage

\section{Problema 6.2}
	\begin{problema}
	El objetivo de este problema es construir, a partir de movimientos brownianos en $[0,1]$, al movimiento browniano en $[0,\infty)$.
	\begin{enumerate}
		\item 
			Pruebe que existe un espacio de probabilidad $\ofp$ en el que existe 
			una sucesi\'on $B^1,B^2,\ldots$ de movimientos brownianos en $[0,1]$ 
			independientes. (Sugerencia: utilice la construcci\'on del movimiento 
			browniano de L\'evy  para que la soluci\'on sea corta.)
			
		\item 
			Defina a $B_t=B^1_1+\cdots+B^{\floor{t}}_1+B^{\ceil{t}}_{t-\floor{t}}$ 
			para $t\geq 0$. Pruebe que $B$ es un movimiento browniano. 
		
		%\item Pruebe que $\paren{B_t}^2-t$ no tiene incrementos independientes. Sugerencia: 
		%En el ejercicio anterior identific\'o la distribuci\'on de $\paren{B_t}^2$; calcule 
		%la transformada de Laplace conjunta de dos incrementos.
	\end{enumerate}
\end{problema}

\begin{proof}
    \subsection{Inciso (i)} \label{problema6_2:inciso1}
    \emph{
    Pruebe que existe un espacio de probabilidad $\ofp$ en el que existe 
    una sucesi\'on $B^1,B^2,\ldots$ de movimientos brownianos en $[0,1]$ 
    independientes. (Sugerencia: utilice la construcci\'on del movimiento 
    browniano de L\'evy  para que la soluci\'on sea corta.)
}
\afterstatement\pn

La idea principal es utilizar que $\aleph_0 \times \aleph_0 = \aleph_0$ y aprovechar
la construcción que se dió en las notas [véase \ref{notas}].\pn

Se sabe que existe un espacio de probabilidad $(\Omega, \F, \P)$ donde están
definidas $\xi_{i,n}$ ($0 \leq i \leq 2^n$) con distribuciónes normales de media 0 y 
varianza 1 e independientes [ver \ref{notas}].\pn

Entonces, podemos para cada $m \in \N$ dar una sucesión de variables $\xi_{i,n}^m$
($0 \leq i \ 2^n$) con distribuciones normales de media 0 y varianza 1 tales que
son independientes entre sí y de las otras $\Xi_{i,n}^{m'}$.\pn

Entonces, para cada $m$ podemos construir un movimiento browniano $B^m$ en $[0, 1]$ utilizando
las $\xi_{i,n}^m$. Dichos movimientos serán independientes entre sí por la selección
de las $(\xi_{i,n}^m)_{m \in \N}$.
    \newpage

    \subsection{Inciso (ii)} \label{problema6_2:inciso2}
    \emph{
    Defina a $B_t=B^1_1+\cdots+B^{\floor{t}}_1+B^{\ceil{t}}_{t-\floor{t}}$ 
    para $t\geq 0$. Pruebe que $B$ es un movimiento browniano. 
}

\afterstatement\pn

$B_0 = B_0^1 = 0$. 

Veamos que $B_t$ es un proceso gaussiano. Sean $0 \leq t_1 < t_2 < \dots < t_n$ y sea 
\begin{align}
       \sum_{i \leq n}  \lambda_i B_{t_i}
\end{align}
una combinación lineal. Por ser los $B^m$'s movimientos brownianos independientes, entonces cada $B_{t_i}$
es una combinación lineal de normales independientes, por lo tanto cada $B_{t_i}$ es normal y por construcción, independientes.
Si sucediera que más de un $B_{t_i}$ pertenecieran al mismo $B^m$, basta argumentar que cada $B^m$ es por sí mismo un
proceso gaussiano. Entonces nuestra combinación lineal también es normal y con esto hemos dicho que $B_t$ es un proceso gaussiano.\pn

$B_{t}$ es suma de movimientos brownianos, los cuales son centrados. Por linealidad de la esperanza tenemos que
$\E(B_t) = 0$.\pn

Ahora, 
\begin{align}
        Var(B_t)    &=  Var\left( B^1_1+\cdots+B^{\floor{t}}_1+B^{\ceil{t}}_{t-\floor{t}} \right)                           \\
                    &=  \sum_{i \leq \lfloor t \rfloor} Var(B_1^i) + Var(B^{\ceil{t}}_{t-\floor{t}})                        \\
                    &\comment{por independencia de los movimientos}                                                         \\
                    &=  \lfloor t \rfloor + Var(B^{\ceil{t}}_{t-\floor{t}})                                                 \\
                    &\comment{porque las $B_1^i$ tienen distribución normal(0,1)}                                           \\
                    &=  \lfloor t \rfloor + t - \lfloor t \rfloor                                                           \\                    
                    &\comment{por que la distribución de $B^{\ceil{t}}_{t-\floor{t}}$ es normal(0, $t - \lfloor t \rfloor$)}\\
                    &= t.
\end{align}\pn

Entonces, $B_t$ tiene distribución normal(0,t).\pn

Veamos que $\E(B_t B_s) = t \wedge s$. Si $t = s$, aprovechamos que $B_t$ tiene distribución normal de media 0 y varianza 1 y entonces
$\E(B_t B_t) = Var(B_t) = t$. Si $t < s$, partiremos en dos casos, uno cuando $s \leq \lceil t \rceil$ y otro para $\leq \lceil t \rceil < s$.\pn

Para el primer caso
\begin{align}
        B_t =   B^1_1+\cdots+B^{\floor{t}}_1+B^{\ceil{t}}_{t-\floor{t}}   \\
        B_s =   B^1_1+\cdots+B^{\floor{t}}_1+B^{\ceil{t}}_{s-\floor{t}}
\end{align}\pn

Entonces
\begin{align}
        \E(B_t B_s) &=  \E\left(\left(B^1_1+\cdots+B^{\floor{t}}_1+B^{\ceil{t}}_{t-\floor{t}}\right) \cdot \left(B^1_1+\cdots+B^{\floor{t}}_1+B^{\ceil{t}}_{s-\floor{t}}\right)\right)  \\
                    &=  \sum_{i \leq \floor{t}} \E\left( (B_1^i)^2 \right) + \E\left( B^{\ceil{t}}_{t-\floor{t}} B^{\ceil{t}}_{s-\floor{t}}\right)                                      \\
                    &\comment{por independencia de los movimientos $B^m$}                                                                                                               \\
                    &=  \floor{t} + \E\left( B^{\ceil{t}}_{t-\floor{t}} B^{\ceil{t}}_{s-\floor{t}}\right)                                                                               \\
                    &=  \floor{t} + (t-\floor{t}) \wedge (s-\floor{t})                                                                                                                  \\
                    &=  \floor{t} + (t-\floor{t})                                                                                                                                       \\
                    &=  t.                                                                                                                                                          
\end{align}\pn

Para el segundo caso
\begin{align}
        B_t =   B^1_1+\cdots+B^{\floor{t}}_1+B^{\ceil{t}}_{t-\floor{t}}   \\
        B_s =   B^1_1+\cdots+B^{\floor{s}}_1+B^{\ceil{s}}_{s-\floor{s}}
\end{align}

Entonces
\begin{align}
    \E(B_t B_s) &=  \sum_{i \leq \floor{t}} \E\left( (B_1^i)^2 \right) + \E\left( B^{\ceil{t}}_{t-\floor{t}} B^{\ceil{t}}_1\right)                                      \\
                &\comment{por independencia de los movimientos $B^m$}                                                                                                               \\
                &=  \floor{t} + \E\left( B^{\ceil{t}}_{t-\floor{t}} B^{\ceil{t}}_1\right)                                                                             \\
                &=  \floor{t} + (t  - \floor{t})                                                                                                                                       \\
                &=  t.                                                                                                                                                          
\end{align}

Con todo esto dicho, basta utilizar [\ref{problema6_1}] para conculir que $B_t$ es un movimiento browniano en ley. Y la continuidad de las
trayectorias se dá por la construcción de $B_t$ puesto que los $B^m$ que lo componen son continuos. 
\end{proof}
\newpage

\section{Problema 6.3}
	\begin{problema}
	Pruebe que si $\tilde X$ es una modificaci\'on de $X$ entonces ambos procesos 
	tienen las mismas distribuciones finito-dimensionales. Concluya que si $B$ es 
	un movimiento browniano en ley y $\tilde B$ es una modificaci\'on de $B$ con 
	trayectorias continuas entonces $\tilde B$ es un movimiento browniano. 
\end{problema}
\newpage

\section{Problema 6.4}
	\begin{problema}
	Sea
	\begin{esn}
		M^\lambda_t=e^{\lambda B_t-\lambda^2t/2}.
	\end{esn}
		\begin{enumerate}
			\item Explique y pruebe formalmente por qu\'e, para toda $n\geq 1$, $\partial^n M^\lambda_t/\partial \lambda^n$ es una martingala. 
			\item Sea $\imf{H_n}{x}=\paren{-1}^ne^{x^2/2}\frac{d^n}{dx^n}e^{-x^2/2}$. A $H_n$ se le conoce como en\'esimo polinomio de Hermite. Calc\'ulelo para $n\leq 5$. Pruebe que $H_n$ es un polinomio para toda $n\in\na$ y que $\partial^n M^\lambda_t/\partial \lambda^n=t^{n/2}\imf{H_n}{B_t/\sqrt{t}}M^\lambda_t$. 
			\item Pruebe que $t^{n/2}\imf{H_n}{B_t/\sqrt{t}}$ es una martingala para toda $n$ y calc\'ulela para $n\leq 5$. 
			\item Aplique muestreo opcional a las martingalas anteriores al tiempo aleatorio $T_{a,b}=\min\set{t\geq 0:B_t\in\set{-a,b}}$ (para $a,b>0$) con $n=1,2$ para calcular $\proba{B_{T_{a,b}}=b}$ y $\esp{T_{a,b}}$, ËQu\'e concluye cuando $n=3,4$? ?` Cree que $T_{a,b}$ tenga momentos finitos de cualquier orden? Justifique su respuesta.
			\item Aplique el teorema de muestreo opcional a la martingala $M^\lambda $ al tiempo aleatorio $T_a=\inf\set{t\geq 0:B_t\geq a}$ si $\lambda>0$. Diga por qu\'e es necesaria la \'ultima hip\'otesis y calcule la transformada de Laplace de $T_a$. 
			\item Opcional (para subir calificaci\'on en esta u otra tarea): 
			\begin{enumerate}
				\item Modifique el ejercicio para que aplique al proceso Poisson.
				\item Resu\'elva el ejercicio modificado. 
			\end{enumerate}
		\end{enumerate}
\end{problema}
\newpage

\section{Problema 6.5}
	\begin{problema}
	\mbox{}
	\begin{enumerate}
	\item 
		Al aplicar la desigualdad maximal de Doob sobre los racionales de orden $n$ y pasar al l\'imite 
        conforme $n\to\infty$, pruebe que $\sup_{t\leq }\abs{B_t-B_1}$ es cuadrado integrable.\pn
	
	\item 
		Pruebe que la sucesi\'on de variables aleatorias
		\begin{esn}
			\paren{\sup_{t\in [0,1]}\abs{B_{n+t}-B_n},n\in\na}
		\end{esn}
		son independientes, id\'enticamente distribuidas y de media finita. (Utilice la propiedad de Markov.)\pn
	
	\item 
		Al utilizar Borel-Cantelli, pruebe que, para cualquier $C>0$ fija
		\begin{esn}
			\limsup_{n\to\infty}\sup_{t\in [0,1]}\abs{B_{n+t}-B_n}/n\leq C
		\end{esn} 
        casi seguramente.\pn
	
	\item 
		Pruebe que $\paren{B_n/n,n\geq 1}$ converge casi seguramente a $0$ y deduzca que
		\begin{esn}
			\lim_{t\to\infty }B_t/t=0.
		\end{esn}\pn
	\end{enumerate}
\end{problema}

\begin{proof}
    \subsection{Inciso (i)} \label{problema6_5:inciso1}
    \emph{
    Al aplicar la desigualdad maximal de Doob sobre los racionales de orden $n$ y pasar al l\'imite 
    conforme $n\to\infty$, pruebe que $\sup_{t\leq 1}\abs{B_t-B_1}$ es cuadrado integrable.\pn
}

\afterstatement\pn

La idea es partir el proceso en martingalas finitas que al irse refinando nos den un conjunto denso en $[0, 1]$ y
luego utilizar el argumento de continuidad para llenar los huecos.\pn

Llamemos 
\begin{align}
 D_n = \bigg\{ \frac{i}{2_n} : i = 0, 1, \dots, 2^n \bigg \}   
\end{align}

Con estos conjuntos, discretizamos a $B$ haciendo las martingalas finitas
$(B_j^n : j \in D_n)$ [donde $B_j^n = B_j$]. Por propiedades de la esperanza condicional y el hecho de que
$(B_j^n : j \in D_n)$ son martingalas. Tenemos que $(\abs{B_j^n} : j \in D_n)$ son sub-marginalas.\pn

La desigualdad maximal de Doob para el caso $L_2$ nos dice que
\begin{align}
        \| \max_{j \in D_n} \abs{B_j^n} \|_2 \leq 2 \| B_1^n\|_2
\end{align}

De donde, elevando al cuadrado y usando que $\E(B_1^n) = 0$.
\begin{align}
        \E\left( \left( \max_{j \in D_n} \abs{B_j^n} \right)^2 \right)  &=  4 \E( (B_1^n)^2 )   \\
                                                                        &=  4 Var(B_1^n)        \\
                                                                        &=  4 Var(B_1)          \\
                                                                        &=  4.
\end{align}\pn

Ahora tenemos que conforme $n \rightarrow \infty$, $\max_{j \in D_n} \abs{B_j^n}$ converge a $\sup \abs{B_t}$ c.s.
El teorema de convergencia monótona nos asegura que
\begin{align}
        E\left( \left( \sup \abs{B_t} \right)^2 \right) &= \lim_{n \rightarrow \infty}   \E\left( \left( max_{j \in D_n} \abs{B_j^n} \right)^2 \right)
\end{align}\pn

Como cada término del límite de la derecha está acotado por $4$ por la demostración de arriba.\pn

Entonces  $\sup_{t\leq 1}\abs{B_t-B_1}$ es cuadrado integrable.
    \newpage

    \subsection{Inciso (ii)} \label{problema6_5:inciso2}
    \emph{
    Pruebe que la sucesi\'on de variables aleatorias
    \begin{esn}
        \paren{\sup_{t\in [0,1]}\abs{B_{n+t}-B_n},n\in\na}
    \end{esn}
    son independientes, id\'enticamente distribuidas y de media finita. (Utilice la propiedad de Markov.)\pn
}

\afterstatement\pn

La propiedad de Markov nos asegura que para todo $n \in N$ el proceso dado obtenido ``desfasar'' el proceso 
en $n$ es independiente del pasado, es decir: $B_{n+t} - B_n$ es independiente de $\F_n = \sigma(B_t : t \leq n)$ y por lo tanto
también $\abs{B_{n+t} - B_n}$ es independiente de $\F_n$ para toda $n$ y por lo tanto son independientes.\pn

Además, sabemos que desfasar es únicamente volver a empezar el proceso, es decir, $B_{n+t} - B_n \sim B_t$ y entonces
$\abs{B_{n+t} - B_n} \sim \abs{B_t}$. Tomando supremos $\sup_{t \leq 1} \abs{B_{n+t} - B_n} \sim \sup_{t \leq 1} \abs{B_t}$.
Es decir que $\sup_{t \leq 1} \abs{B_{n+t} - B_n}$ tienen todas la misma distribución que $\sup_{t \leq 1} \abs{B_t}$.\pn

El en [\ref{problema6_5:inciso1}] se demostró que $\sup_{t\leq 1}\abs{B_t-B_1}$ es cuadrado integrable, y por lo tanto
tiene medida finita. Ese es el caso cuando $n = 1$, pero como para toda $n$ se tiene la misma distribución, todas
tienen media finita. 
    \newpage

    \subsection{Inciso (iii} \label{problema6_5:inciso3}
    \emph{
    Al utilizar Borel-Cantelli, pruebe que, para cualquier $C>0$ fija
    \begin{esn}
        \limsup_{n\to\infty}\sup_{t\in [0,1]}\abs{B_{n+t}-B_n}/n\leq C
    \end{esn} 
    casi seguramente.\pn
}
\afterstatement\pn 
    \newpage

    \subsection{Inciso (iv)} \label{problema6_5:inciso4}
    \emph{
    Pruebe que $\paren{B_n/n,n\geq 1}$ converge casi seguramente a $0$ y deduzca que
    \begin{esn}
        \lim_{t\to\infty }B_t/t=0.
    \end{esn}\pn
}

\afterstatement\pn

Escribimos
\begin{align}
        B_n = \sum_{i \leq n} (B_i - B_{i-1})
\end{align}

Recordemos que gracias a [\ref{problema6_5:inciso2}] tenemos que los sumandos de la derecha son variables aleatorias idependientes e idénticamente
distribuidas y que además, su distribución es normal de media $0$ y varianza $1$.\pn

De aquí, aplicando ley fuerte de los grandes números, tenemos que
\begin{align}
        \lim_{n \rightarrow \infty} \frac{B_n}{n}   &= \lim_{n \rightarrow \infty} \frac{\sum_{i \leq n} (B_i - B_{i-1})}{n}    \\
                                                    &= \E(B_1)                                                                  \\
                                                    &=  0.
\end{align}\pn

Sean ahora $t \geq 0$, $n = \lfloor t \rfloor$ y $s = t - n$ y entonces acotamos a $B_t/t$ por algo que ya conozcamos.
\begin{align}
    \frac{\abs{B_t}}{t} &=      \frac{\abs{B_{n+s}}}{n+s}                                                   \\
                        &\leq   \frac{\abs{B_{n+s}}}{n}                                                     \\
                        &=      \frac{\abs{B_{n+s} - B_n + B_n}}{n}                                         \\
                        &=      \frac{\abs{B_{n+s} - B_n}}{n} + \frac{\abs{B_n}}{n}                         \\
                        &=      \frac{\sup_{s \leq 1}\abs{B_{n+s} - B_n}}{n} + \frac{\abs{B_n}}{n}          \\
\end{align}\pn 

Ahora tomaremos límite superior, el primer sumando se hace $0$ por la conclusión de [\ref{problema6_5:inciso3}], y el sumando de la
derecha tiene límite y es 0, por lo tanto tiene límite superior y  también es 0. 
\end{proof}
\newpage



        
        \nqed

    \part{Parcial 2}
        Los problemas 1 y 2 se encuentran resueltos en [\ref{problema5_3:inciso2}] y [\ref{problema5_7}].

\section{Problema 3}
Sean $B_1, B_2, \dots$ variables aleatorias independientes con distribución Bernoulli de parámetro $p \in (0,1)$.
Sea $T_0 = 0$ y
\begin{align}
    T_{n+1} = \min\{k>T_n : B_k = 1\}
\end{align}
    \subsection{Inciso 1}
        \emph{
    Pruebe que $T_0, T_1,\dots$ son tiempos de renovación de un proceso de renovación aritmético.
}
\afterstatement\pn

    Pensemos en $B_1, B_2, \dots$ como volados con monedas cargadas donde $B_i = 1$ significa que ganamos el
    $i$-ésimo volado.\pn
    
    Con esta interpretación, $T_n$ no es otra cosa sino el tiempo en que ganamos por $n$-ésima vez. Por lo tanto\pn
    \begin{align}
        \P(T_{n} = k) = \binom{k-1}{n-1} p^n (1-p)^{k-n}
    \end{align}
    
    Es decir, $T_n$ tiene distribución binomial negativa de parámetros $n$ y $p$.\pn
    
    Definimos entonces a los tiempos de vida $S_0 = 0$ y $S_n = T_n - T_{n-1}$. Por tener $T_n$ y $T_{n-1}$ distribución
    binomial negativa de parámetros $n$, $p$ y $n-1$, $p$ respectivamente, tenemos que $S_n$ tiene distribución
    geométrica de parámetro $p$.\pn
    
    Otra manera de escribir a $S$ es
    \begin{align}
        S_n = \sum_{i = T_{n-1} + 1}^{T_n} i B_i
    \end{align}
    
    De donde tenemos que $S_n$ es suma de variables aleatorias independientes, todas distintas de las que suman $S_{n-1}$.
    Por lo tanto las $S_n$ son independientes y dado que $S_n$ tiene distribución geométrica de parámetro $p$ para toda
    $n$, tenemos que son idénticamente distribuidas.\pn
    
    Con esto, tenemos un proceso de renovación con tiempos de vida $S_n$ y tiempos de renovación $T_n$.

        \newpage
        
    \subsection{Inciso 2}
        \emph{
    Defina al proceso de contéo asociado $N$, expréselo en términos de las variables $B_1, B_2, \dots$ y pruebe
    diréctamente que $N_n/n$ converge casi seguramente conforme $n$ crece. Identifíque el límite.
}
\afterstatement\pn

Sea
\begin{align}
        N_n = \sum^n_{i = 1} B_i.
\end{align}

Y notemos que 
\begin{align}
        \min \{ m : T_{m+1} > n \} 
\end{align}
significa, el tiempo de volado más alto antes de pasarnos de $n$.\pn

Interpretado de esta forma tenemos que coinciden
\begin{align}
    \min \{ m : T_{m+1} > n \} = \sum^n_{i = 1} B_i = N_n.
\end{align}\pn

Por lo tanto tenemos que $N_n$ sí es un proceso de conteo para nuestro proceso de renovación.\pn

Por otra parte $N_n$ es suma de $n$ variables idependientes e idénticamente distribuidas
con esperanza $p$. $N_n / n$ es su promedio y la ley fuerte de los grandes números nos dice que
\begin{align}
    N_n/n \rightarrow p \;\; c.s.
\end{align}

        \newpage
        
    \subsection{Inciso 3}
        \emph{
    Calcule la distribución del tiempo residual al tiempo $n$
        \begin{align}
                R_n = \min\{k > n: B_k = 1\} - n
        \end{align}
    Asuma que dicho proceso es una cadena de Markov e identifique las probabilidades de transición.
    Pruebe que la distribución geométrica de parámetro $p$ es invariante para dicho proceso.
}

\afterstatement\pn

$R_n = k$ significa que si han pasado $n$ volados, en $k$ volados ganaremos uno, y antes de eso ninguno.
Dado que los $B_i$ son idénticamente distribuidos, tenemos que eso significa que
\begin{align}
        \P(R_n = k) = p (1-p)^{r-1}        
\end{align}\pn

Es decir, $R_n$ tiene distribución geométrica de parámetro $p$.\pn

Supongamos ahora que $R_n$ es una cadena de Markov. Y supongamos que al tiempo $n$, sabemos que
$R_n = k \neq 0$, es decir, nos faltan $k$ volados para volver a ganar. Entonces $R_{n+1} = k-1$ forzosamente.\pn
Entonces, en nuestra matriz de transición tendremos $P_{k, k-1} = 1$ cuando $k \neq 0$.\pn

Para el caso $R_n = 0$, la probabilidad de que pasemos de $0$ a $k>0$, es equivalente a perder los siguientes $k-1$ volados y luego
ganar el $k$-ésimo. Es decir, $P_{0, k} = p (1-p)^{k-1}$.\pn

Entonces la matriz de transición está determinada por
\begin{align}
        P_{i,j} =
                    \begin{cases}
                        1               &   \text{si $i > 0$ y $j = i - 1$}     \\
                        p (1-p)^{j-1}   &   \text{si $i = 0$ y $j>0$}           \\
                        0               &   \text{en cualquier otro caso}
                    \end{cases}
\end{align}\pn

Veamos que la distribución $\pi$, con $\pi_i = p (1-p)^{i-1}$ si $i>0$ y $\pi_0 = 0$ resulta invariante para este proceso.

\begin{align}
        (\pi \times P)_{j}      &=  \sum_{k \geq 0} \pi_{k} P_{k,j}     \\
                                &=  \pi_0 P_{0, j} + \pi_{j} P_{j+1, j} \\
                                &=  0 + \pi_{j} 1.
\end{align}\pn

Con lo que terminamos que $\pi$ es una distribución invariante para el proceso.
        \newpage
        
    \subsection{Inciso 4}
        \emph{
    Calcule la distribución del proceso de edad al tiempo $n$:
    \begin{align}
            A_n = n - \max{k \leq n: B_k = 1}
    \end{align}
    Calcula la distribución límite de $A_n$ conforme $n \rightarrow \infty$.
}

\afterstatement\pn

Supongamos $r < n$. Entonces $A_n = r$ significa que hace $r$ volados ganamos y que desde
entonces hemos perdido. Eso quiere decir que $\P(A_n = r) = p (1-p)^r$. Es decir que para el caso
$r < n$, $A_n$ se comporta como una variable con distribución geométrica de parámetro $p$.\pn

Para el caso $r = n$. $A_n = r$ significa que nunca hemos ganado un volado. Por lo tanto
$\P(A_n = r) = (1-p)^r$\pn

Recordemos para el caso $n < r$, tenemos que $\P(A_n = r) = 0$, pues estaríamos preguntando por la probabilidad
de haber perdido más veces que el número de volados jugados.\pn

Eso quiere decir que $A_n \sim Geo(p) \wedge n$.\pn

De esta última fórmula es fácil verificar que $A_n \rightarrow A_{\infty} \sim Geo(p)$.
        
\section{Problema 4}
Una matriz de transición $P$ se llama reversible si existe una distribución inicial $\pi$ tal que
\begin{align}
    \pi_i \P_{i,j} = \pi_j P_{j,i}.
\end{align}
A las identidades anteriores se les conoce como condición de balance detallado.
    \subsection{Inciso 1}
        \emph{
    Muestre que si $\pi$ y $P$ satisfacen la condición de balance detallado entonces $\pi$ es una
    distribución invariante para $P$.
}

\afterstatement\pn

Sencillamente verifiquemos la igualdad necesaria
\begin{align}
    (\pi \times P)_{j}  &= \sum{i \in E} \pi_{i}P_{i,j}     \\
                        &= \sum{i \in E} \pi_{j}P_{j,i}     \\
                        &= \pi_{j} \sum{i \in E} P_{j,i}    \\
                        &= \pi_{j} (1)                      \\
                        &= \pi_{j}.
\end{align}

Y con esto termina la demostración.
        \newpage
        
    \subsection{Inciso 2}
        \emph{
    Muestre que si $X$ es una cadena de Markov con distribución inicial $\pi$ y matriz de transición $P$ en balance detallado entonces
    \begin{align}
            (X_0, \dots, X_n) \sim (X_n, \dots, X_0).
    \end{align}
}

\afterstatement\pn

Por ser cadena de Markov con distribución inicial $\pi$, tenemos que
\begin{align}
        \P(X_0 = x_0, \dots, X_n = x_n) &=  \pi_{x_0} P_{x_0, x_1} P_{x_1, x_2} \dots P_{x_{n-1}, x_{n}}.
\end{align}\pn

Utilizando la hipótesis de que $\pi$ y $P$ están en balance detallado tenemos que
\begin{align}
    &   \pi_{x_0} P_{x_0, x_1} P_{x_1, x_2} \dots P_{x_{n-1}, x_{n}}    \\
    &=  P_{x_1, x_0} \pi_{x_1} P_{x_1, x_2} \dots P_{x_{n-1}, x_{n}}    \\
    &=  P_{x_1, x_0} P_{x_2, x_1} \pi_{x_2} \dots P_{x_{n-1}, x_{n}}    \\
    &\vdots                                                             \\
    &=  P_{x_1, x_0} P_{x_2, x_1} \dots P_{x_{n}, x_{n-1}} \pi_{x_n}    \\
    &= P(X_n = x_n, \dots, X_0 = x_0).
\end{align}\pn

Con lo que termina la demostración.
        \newpage
        
    \subsection{Inciso 3}
        \emph{
    Sea $G=(V,E)$ una gráfica y defina la matriz de transición $P$ sobre $V$ al estipular que
    $P_{x,y}$ es igual a 1 sobre el grado de $x$, denotado como $\delta_x$. Muestre que la distribución $\pi$
    Dada por $\pi_x = \frac{\delta_x}{\sum_{v \in V} \delta_v}$.
}

\afterstatement\pn

Si $x,y$ son tales que $\{x,y\} \not\in E$, tenemos que $P_{x,y} = 0 = P_{y,x}$. Por lo tanto
la condición de balance se da trivialmente en estos casos.\pn

Para, $x,y$ tales que $\{x, y\} \in E$, sencillamente verifiquemos la igualdad pertinente
\begin{align}
        \pi_x P_{x,y}   &=  \frac{\delta_x}{\sum_{v \in V} \delta_v} \frac{1}{\delta_x}          \\
                        &=  \frac{1}{\sum_{v \in V} \delta_v}                                    \\
                        &=  \frac{\delta_y}{\delta_y} \frac{1}{\sum_{v \in V} \delta_v}          \\
                        &\comment{es válido porque $\delta_y \geq 1$, pues $x$ es vecino de $y$}\\
                        &=  \frac{1}{\delta_y} \frac{\delta_y}{\sum_{v \in V} \delta_v}          \\
                        &=  P_{y,x} \pi_y.
\end{align}
Con lo que termina la demostración.
        \newpage
        
    \subsection{Inciso 4}
        \emph{
    Muestre que si $\pi$ y $P$ satisfacen las ecuaciones de balance detallado, entonces existe una medida de probabilidad
    $\P$ en $E^{\Z}$ tal que (con la noción adecuada deproceso canónico $X$),
    \begin{align}
            \P(X_m = x_m, \dots, X_n = x_n) = \pi_{x_m} P_{x_m, x_{m+1}} \cdots P_{x_{n-1}, x_{n}}
    \end{align}
    para cualesquier $m,n \in \Z$ con $m \leq n$ y para cualesquiera $x_m, \dots, x_n \in E$.
}

\afterstatement\pn


Notemos que si $\pi_{x_r} = 0$ con $m \leq r \leq n$, la probabilidad de arriba se hace $0$ gracias a la condición de
balance detallado (podemos ir ``voletando'', las entradas de $P$ hasta que eventualmente se atravezará $\pi_{x_r}$ en el producto).\pn

Entonces nos enfocaremos en $E' = \{x \in E : \pi_x > 0 \}$.\pn

Para cualesquier $m \leq n \in \Z$ y $x_m, x_n \in E'$, definamos $\P$ como
\begin{align}
    \P(X_m = x_m, X_n = x_n)    &= \sum_{x_{m+1}, x_{m+2},\dots x_{n-1} \in E'}   \P(X_m = x_m, \dots, X_n = x_n)                        \\
                                &= \sum_{x_{m+1}, x_{m+2},\dots x_{n-1} \in E'}   \pi_{x_m} P_{x_m, x_{m+1}} \cdots P_{x_{n-1}, x_{n}}.   \\
\end{align}\pn

Para $l \leq m \leq n \in \Z$ y $x_l, x_m, x_n \in E'$,
\begin{align}
    &\P(X_l = x_l, X_m = x_m, X_n = x_n) \\
    &=  \sum_{x_{l+1}, \dots, x_{m-1} x_{m-1},\dots x_{n-1} \in E'}   \P(X_l = x_l, \dots, X_m = x_m, \dots, X_n = x_n)
\end{align}\pn

 y así sucesivamente para ${n_1} \leq {n_2} \leq \dots \leq {n_z} \in \Z$ y $x_{n_1}, x_{n_2}, \dots, x_{n_z} \in E'$. Lo que estamos haciendo
es completar los huecos sumando sobre todos los valores posibles que dichos huecos puedan tener.\pn

Es fácil verificar que esta definición extiende a la mencionada en el enunciado. Para los casos donde tomamos infinitos $(n_i)'s \in \Z$ y su complemento
es finito, definimos a su probabilidad como $1$, menos la probabilidad de su complemento, que sí sabemos calcular.\pn

Definimos una $\sigma$-álgebra para $E^\Z$ donde los conjuntos de índices en $\Z$ tienen que ser finitos, o sus 
compelmentos tienen que ser finitos, de manera que las definiciones que hemos hecho hasta ahora, nos basten para 
calcular a los elementos de dicha $\sigma$-álgebra y que contengan a los casos mencionados en el enunciado.

Lo que hemos hecho hasta ahora no ha sido otra cosa más que forzar a $\P$ a satisfacer las hipótesis de ser medida de probabilidad.\pn
Con la primera parte garantizamos que la probabilidad del total sea $1$ y que $\P$ se porte bien sobre uniones disjuntas, con la segunda parte que 
los complementos se comporten bien bajo la medida de probabilidad.


        \newpage
        
    \subsection{Inciso 5}
        \emph{
    Muestre que si $\tilde X_n = X_{-n}$, entonces $\tilde X$ y $X$ tienen la misma distribución bajo $\P$.
}

\afterstatement\pn

Dadas las condiciones de balance detallado, y la ``condición de Markov'' dada para $\P$ en el inciso anterior, 
este ejercicio es exáctamente el mismo que el del segundo inciso de este problema y la demostración es 
completamente análoga (la nombrada ``condición de Markov'' fue la única parte de ser cadena de Markov que se utilizó en $2$).
        \newpage
        \pn
        \nqed

    \part{Ejercicios}
        \begin{ejercicio}[Ejercicio 1.1]
	Sea $X$ una variable aleatoria normal centrada de varianza 1. 
	Utilice el teorema de cambio de variable para calcular
	\begin{align}
		\E{X^{2n}}
	\end{align}
	para toda $n\in\na$.
\end{ejercicio}

Primera forma:
Utilizar este resultado: \href{http://en.wikipedia.org/wiki/Characteristic_function_(probability_theory)#Moments}{momentos y funci\'on caracter\'istica},

Se puede llegar f\'acilmente a que 
\begin{align}
	\E{X^{2n}} = \frac{(2n)!}{2^m m!}
\end{align}\pn

Segunda forma:
Integrar $x^2 \frac{1}{\sqrt{2 \pi} e^{-x^2 / 2}}$ y utilizar el cambio de variable $u = x^2/2$.


\pn---------------------------------------------------\pn
\begin{ejercicio}[Ejercicio 1.2]
Si $\paren{X,Y}$ son dos variables aleatorias con densidad
conjunta $f\paren{x,y}$, pruebe que:\begin{esn}
\espc{g\paren{Y}}{X}=\frac{\int f\paren{X,y}g\paren{y}\, dy}{\int {f\paren{X,y}\, dy}}.\end{esn}
\end{ejercicio}


\pn---------------------------------------------------\pn
\begin{ejercicio}[Ejercicio 1.3	]
Sean $X_1,X_2,\ldots$ vaiids. Sea $K$ una variable aleatoria independiente de $X_1,X_2,\ldots$ y con valores en $\na$. Cacule\begin{esn}
\espc{X_1+\cdots+X_K}{K}.
\end{esn}Sugerencia: ?`Qu\'e pasa cuando $K$ toma s\'olo un valor?
\end{ejercicio}
La respuesta es:
\begin{align}
	K\E(X_1)
\end{align}


\pn---------------------------------------------------\pn
Sea $U$ una variable aleatoria uniforme en $(0,1)$ y definamos a\begin{esn}
X_n=2^n\indi{U\leq 1/2^n}.
\end{esn}Entonces $X_0,X_1,\ldots$ es una martingala respecto de la filtraci\'on que genera. 
\begin{ejercicio}
Probar la afirmaci\'on anterior.
\end{ejercicio}


Consideremos el siguiente experimento aleatorio, se tiene una urna 
con $r$ bolas rojas y $v$ bolas verdes. Extraemos una bola, la 
reemplazamos junto con $c$ bolas del mismo color, revolvemos la 
urna y volvemos a realizar el experimento. Sea $X_0$ la fracci\'on 
inicial de bolas rojas en la urna  y $X_n$ la fracci\'on de bolas 
rojas  en la urna una vez realizado el experimento $n$ veces. 
Entonces $\paren{X_n}_{n\in\na}$ es una martingala con respecto a 
la filtraci\'on que genera esta sucesi\'on. Antes de proceder a verificar  
la afirmaci\'on anterior, debemos considerar el modelo matem\'atico preciso 
del experimento aleatorio en cuesti\'on, para poder calcular las esperanzas 
condicionales. 
Notemos que al momento de la $n$-\'esima extracci\'on hay
\begin{esn}
b_n=r+v+nc
\end{esn}
bolas en la urna. Sean $\paren{U_i}$ variables aleatorias independientes e id\'enticamente distribuidas, 
$r,v>0$ y definamos $X_0=r/(r+v)$ y para $n\geq 0$:
\begin{esn}
Y_{n+1}=\indi{U_{n+1}\leq X_n}\quad\text{y}\quad X_{n+1}=\frac{r+v+nc}{r+v+\paren{n+1}c}X_n+\frac{c}{r+v+\paren{n+1}c}Y_n.
\end{esn}
Esta es la descripci\'on matem\'atica que utilizaremos del experimento considerado anteriormente y en \'el, 
la variable $X_n$ es funci\'on  de $X_0,U_1,\ldots, U_n$ para $n\geq 1$ (de hecho es funci\'on de $X_{n-1}$ y $U_n$) y 
por lo tanto, $U_{n+1}$ es independiente de $\F_n$, la \sa\ generada por $X_0,\ldots, X_n$.
\begin{ejercicio}
Verificar que la sucesi\'on $X$ es una martingala respecto de $\paren{\F_n}$. 
\end{ejercicio}

\pn---------------------------------------------------\pn
\begin{ejercicio}[Descomposici\'on de Doob para submartingalas]
Sea \(X=\paren{X_n}_{n\in\na}\) una submartingala. Pruebe que \(X\) se puede descomponer de manera \'unica 
como \(X=M+A\) donde \(M\) es una martingala y \(A\) es un proceso previsible con \(A_0=0\). (Decimos que 
$A$ es previsible si $A_{n}$ es $\F_{n-1}$-medible para toda $n\geq 1$ y $A_0$ es $\F_0$-medible. Sugerencia: 
Asuma que ya tiene la descomposici\'on y calcule esperanza condicional de \(X_{n+1}\) dada \(X_n\).
\end{ejercicio}

\pn---------------------------------------------------\pn
\begin{ejercicio}
Sean \(X\) y \(Y\) dos martingalas (respecto de la misma filtraci\'on) y tales que 
\(\esp{X_i},\esp{Y_i}<\infty\) para toda \(i\). Pruebe la siguiente f\'ormula de integraci\'on 
por partes:
\begin{esn} 
\esp{X_nY_n}-\esp{X_0Y_0}=\sum_{i=1}^n \esp{\paren{X_i-X_{i-1}}\paren{Y_i-Y_{i-1}}} .
\end{esn}
\end{ejercicio}


\pn---------------------------------------------------\pn
\begin{ejercicio}[Ejercicio 1.8 p12]
Sea $X$ una supermartingala. Pruebe que si $T$ es un tiempo de paro acotado por $N$ entonces\begin{esn}
\esp{X_T}\geq \esp{X_N}.
\end{esn}
\end{ejercicio}


\pn---------------------------------------------------\pn
\begin{definicion}
Sea $C=\paren{C_n,n\geq 1}$ un proceso estoc\'astico. Decimos que $C$ es predecible respecto de $\paren{\F_n}$ si $C_n$ es $\F_{n-1}$-medible.
\end{definicion}
Si $C$ es un proceso predecible y acotado y $M$ es una martingala, formemos al nuevo proceso $C\cdot M$ como sigue:\begin{esn}
\paren{C\cdot M}_0=0\quad\text{y}\quad \paren{C\cdot M}_n=\sum_{i\leq n} C_{i}\paren{M_i-M_{i-1}}. 
\end{esn}
\begin{ejercicio}
Mostrar cuidadosamente que $C\cdot M$ es una martingala. Obtenga un enunciado an\'alogo si $M$ es una submartingala. 
\end{ejercicio}


\pn---------------------------------------------------\pn
\begin{ejercicio}[Extensiones del teorema de paro opcional]
Sea \(M=\paren{M_n,n\in\na}\) una (super)martingala respecto de una filtraci\'on \(\paren{\F_n,n\in\na}\) y sean \(S\) y \(T\) tiempos de paro.
\begin{enumerate}
		\item Pruebe que \(S\wedge T\), \(S+T\) y \(S\vee T\) son tiempos de paro.
		\item Sea \begin{esn}\F_T=\set{A\in\F:A\cap\set{T\leq n}\in\F_n\text{ para toda } n}\end{esn}es una \(\sigma\)-\'algebra, a la que nos referimos como la \(\sigma\)-\'algebra detenida en \(\tau\). Comente qu\'e puede fallar si \(T\) no es tiempo de paro. Pruebe que \(T\) es \(\F_T\)-medible. 
		\item Pruebe que si \(T\) es finito, entonces \(M_T\) es \(\F_T\)-medible.
		\item Pruebe que si \(S\leq T\leq n\) entonces \(\F_S\subset\F_T\). Si adem\'as \(T\) es acotado entonces \(X_S,X_T\in L_1\) y \begin{esn}\espc{M_T}{\F_S}\leq M_S.\end{esn}
		\item Si \(X=\paren{X_n,n\in\na}\) es un proceso estoc\'astico \(\paren{\F_n}\)-adaptado y tal que \(X_n\in L_1\) y tal que para cualesquiera tiempos de paro acotados \(S\) y \(T\) se tiene que \(\esp{X_S}=\esp{X_T}\) entonces \(X\) es una martingala. Sugerencia: considere tiempos de paro de la forma \(n\indi{A}+(n+1)\indi{A^c}\) con \(A\in\F_n\).
\end{enumerate}
\end{ejercicio}

\pn---------------------------------------------------\pn
\begin{ejercicio}
Suponga que $p>1-p$. 
\begin{enumerate}
\item Sea $\imf{\phi}{x}=\paren{p/q}^x$ y pruebe que $\paren{\imf{\phi}{S_n}}_{n\in\na}$ es martingala respecto a la filtraci\'on que genera.
\item Note que al aplicar el teorema de muestreo opcional de Doob al tiempo de paro acotado $T_{-a}\wedge T_b\wedge n$ se obtiene\begin{esn}
1=\esp{\imf{\phi}{S_{T_{-a}\wedge T_b\wedge n}}}.
\end{esn}Utilice alguna propiedad de la esperanza para pasar al l\'imite conforme $n\to\infty$ y concluir que\begin{esn}
1=\esp{\imf{\phi}{S_{T_{-a}\wedge T_b}}}=\imf{\phi}{-a}\proba{T_{-a}<T_b}+\imf{\phi}{b}\proba{T_b<T_{-a}}. 
\end{esn}Concluya con el c\'alculo expl\'icito de $\proba{T_b<T_{-a}}$.
\item Pruebe que $\paren{S_n-n\paren{2p-1}}_{n\in\na}$ es una martingala.
\item Note que al aplicar muestreo opcional al tiempo de paro $T_{-a}\wedge T_b\wedge n$ se obtiene\begin{esn}
\esp{S_{T_{-a}\wedge T_b\wedge n}}=\paren{2p-1}\esp{T_{-a}\wedge T_b\wedge n}.
\end{esn}Aplique propiedades de la esperenza al lado derecho y de la probabilidad al lado derecho que permitan pasar al l\'imite conforme $n\to\infty$ en la expresi\'on anterior y obtener:
\begin{align*}
\esp{T_{-a}\wedge T_b}&=\frac{1}{2p-1}\esp{S_{T_{-a}\wedge T_b}}
\\&=\frac{1}{2p-1}\paren{-a\proba{T_{-a}<T_b}+b\proba{T_b<T_{-a}}}
\end{align*}y calcule expl\'icitamente $\esp{T_{-a}\wedge T_b}$.
\end{enumerate}
\end{ejercicio}

\pn---------------------------------------------------\pn
\begin{teorema}
Sea $M$ una (sub)martingala y $C$ un proceso predecible y acotado entonces $C\cdot M$ es una (sub)martingala.
\end{teorema}
\begin{ejercicio}
Pruebe el teorema anterior. 
\end{ejercicio}

\pn---------------------------------------------------\pn
\begin{ejercicio}
Sea $U_n$ la cantidad de cruces hacia arriba que hace el proceso $M$ en el intervalo $[a,b]$ antes de $n$. Argumente que\begin{esn}
Y_n\geq \paren{b-a}U_n+\paren{M_n-a}^-.
\end{esn}Al tomar esperanzas verifique que se satisface la desigualdad de cruces de Doob\begin{esn}
\esp{U_n}\leq \frac{1}{b-a}\esp{\paren{a-M_n}^+}.
\end{esn}
\end{ejercicio}

\pn---------------------------------------------------\pn
\begin{proposicion}
Las variables aleatorias $\paren{Y_n}$ son intercambiables. De hecho, si $i_1,\ldots,i_n\in\set{0,1}$ y $s_n=i_1+\cdots+i_n$ entonces
\begin{align*}
\proba{Y_1=i_1,\ldots, Y_n=i_n}
&=\frac{\fa{\paren{r/c}}{s_n}\fa{\paren{v/c}}{n-s_n}}{\fa{\paren{\paren{r+v}/c}}{n}}.
%\\&=\frac{\frac{\imf{\Gamma}{r/c+s_n}\imf{\Gamma}{v/c+n-s_n}}{\imf{\Gamma}{r/c}\imf{\Gamma}{v/c}}}{\frac{\imf{\Gamma}{\paren{r+v}/c+n}}{\imf{\Gamma}{\paren{r+v}/c}}}
%\\&=\frac{\imf{\Beta}{r/c+n,v/c}}{\imf{\Beta}{r/c,v.c}}.
\end{align*}
\end{proposicion}
\begin{ejercicio}
Pruebe la proposici\'on anterior. Sugerencia, utilice el principio de inducci\'on. 
\end{ejercicio}

\pn---------------------------------------------------\pn
Una cadena de Markov es un proceso estoc\'astico de un tipo especial; se encuentra caracterizado por satisfacer la propiedad de Markov. Una manera coloquial de expresar la propiedad de Markov es que el futuro del proceso es independiente del pasado cuando se conoce el presente. Esto se puede traducir matem\'aticamente mediante el concepto de independencia condicional.
\begin{definicion}
%Sea $\ofp$ un espacio de probabilidad y $\mc{H}_1,\mc{H}_2,\G$ sub\sa s de \F. Decimos que $\mc{H}_1$ y $\mc{H}_2$ son condicionalmente independientes dada $\G$, denotado por $\condind{\mc{H}_1}{\mc{H}_2}{\G}$, si para cualesquiera variables aleatorias acotadas $H_1$ y $H_2$ tal que $H_i$ es $\mc{H}_i$-medible se tiene que
\begin{esn}
\espc{H_1H_2}{\G}=\espc{H_1}{\G}\espc{H_2}{\G}.
\end{esn}
\end{definicion}
Si $X=\paren{X_n}_{n\in\na}$ es un proceso estoc\'astico, $\paren{\F_n}_{n\in\na}$ es su filtraci\'on can\'onica asociada y\begin{esn}
\F^n=\sag{X_{n},X_{n+1},\ldots},
\end{esn}diremos que $X$ satisface la propiedad de Markov (inhomog\'enea) si $\F_n$ y $\F^n$ son condicionalmente independientes dada $X_n$. 
A trav\'es de una equivalencia del concepto de independencia condicional, podremos obtener la expresi\'on usual de la propiedad de Markov.
\begin{proposicion}
Las sub\sa s $\mc{H}_1$ y $\mc{H}_2$ son condicionalmente independientes dada $\G$ si y s\'olo si para cualquier variable aleatoria $H_1$ que sea $\mc{H}_1$-medible y acotada:
\begin{esn}
\espc{H_1}{\G,\mc{H}_1}=\espc{H_1}{\G}.
\end{esn}
\end{proposicion}
\begin{ejercicio}
Probar la proposici\'on anterior.
\end{ejercicio}

\pn---------------------------------------------------\pn
Tiempos de arribo para la caminata aleatoria simple unidimensional
Sea $\paren{\p_k,k\in\z}$ la familia Markoviana asociada a la caminata aleatoria simple con matriz de transici\'on $P_{i,i+1}=1-P_{i,i-1}=p\in (0,1)$. Sea $T_0$ el primer arribo a cero dado por\begin{esn}
T_0=\min\set{n:X_n=0}.
\end{esn}Nuestro objetivo ser\'a determinar a\begin{esn}
\imf{\phi}{s}=\imf{\se_1}{s^{T_0}}.
\end{esn}

Comenzando en $2$, la caminata aleatoria simple debe pasar por uno para llegar a cero, y la trayectoria de $2$ a $1$ tiene la misma distribuci\'on que la de $1$ a cero. Por lo tanto:\begin{esn}
\imf{\se_2}{s^{T_0}}=\imf{\phi}{s}^2.
\end{esn}Por otra parte, la propiedad de Markov al instante $1$ nos dice que\begin{esn}
\imf{\phi}{s}=\imf{\se_1}{s^{T_0}}=(1-p)s+ps\imf{\se_2}{s^{T_0}}=\paren{1-p}s+ps\imf{\phi}{s}^2.
\end{esn}As\'i, puesto que $\imf{\phi}{s}\in (0,1)$ para $s\in (0,1)$, vemos que\begin{esn}
\imf{\phi}{s}=\frac{1-\sqrt{1-4p\paren{1-p}s^2}}{2ps}.
\end{esn}Esto tiene varias implicaciones. La primera es que\begin{esn}
\imf{\p_1}{T_0<\infty}=\lim_{s\to 1}\imf{\se_1}{s^{T_0}}=\frac{1-\abs{1-2p}}{2p}
=\begin{cases}
1&p<1/2\\
\frac{q}{p}& p\geq 1/2
\end{cases}.
\end{esn}La segunda es el c\'alculo, para $p\leq 1/2$ de la esperanza de $T_0$: al derivar la ecuaci\'on que satisface $\phi$ vemos que\begin{esn}
0=1-\paren{2p-1}\imf{\phi'}{1-}
\end{esn}

La segunda consecuencia es la determinaci\'on expl\'icita de la distribuci\'on de $T_0$ bajo $\p_1$.
\begin{ejercicio}
Haga un desarrollo en serie de $\phi$ para obtener el valor exacto de $\imf{\p_1}{T_0=2n+1}$. \emph{Sugerencia: } La serie bin\'omica ser\'a de utilidad. Pruebe que\begin{esn}
\proba{T=2n+1}=\frac{1}{2n+1}\imf{\p_1}{S_{2n+1}=-1}.
\end{esn}
\end{ejercicio}
\pn---------------------------------------------------\pn
\begin{ejercicio}
Pruebe que $T_{N_t+1}-t,S_{N_t+2},S_{N_t+3},\ldots$ son variables aleatorias independientes y que $S_{N_t+2},S_{N_t+3},\ldots$ tienen la misma distribuci\'on que $S_1$. 

Pruebe que si $S_1$ tiene distribuci\'on exponencial entonces $T_{N_t+1}-t$ tambi\'en tiene distribuci\'on exponencial y por lo tanto los procesos $N$ y $N^t$ dado por $N^t_s=N_{t+s}-N_t$ tienen la misma distribuci\'on. 
\end{ejercicio}
\pn---------------------------------------------------\pn
Recordemos que el proceso de Poisson de intensidad $\lambda$ es estacionario en el siguiente sentido: el proceso $N^t$ dado por  $N^t_s=N_{t+s}-N_t,s\geq 0$ es tambi\'en un proceso de Poisson de intensidad $\lambda$. A\'un m\'as es cierto: de hecho $N^t$ es independiente de $\F^N_t=\sag{N_s:s\leq t}$. Antes de ver este resultado pasaremos por uno m\'as sencillo.
\begin{definicion}
Un proceso estoc\'astico $X=\paren{X_t,t\geq 0}$ tiene \defin{incrementos independientes} si cuando $0=t_0<t_0<\cdots<t_n$ se tiene que $X_{t_1}-X_{t_0},\ldots, X_{t_{n}}-X_{t_{n-1}}$ son variables aleatorias independientes.

Un proceso estoc\'astico $X=\paren{X_t,t\geq 0}$ tiene \defin{incrementos estacionarios} si $X_{t+s}-X_t$ tiene la misma distribuci\'on que $X_s$. 

Un \defin{proceso de L\'evy} es un proceso estoc\'astico $X$ con trayectorias \cadlag\ que comienza en cero y tiene incrementos independientes y estacionarios.
\end{definicion}
En otras palabras, un proceso de L\'evy es la versi\'on a tiempo continuo de una caminata aleatoria. 
\begin{ejercicio}
\label{IncrementosIndependientesImplicanIncrementosIndepDelPasadoEjercicio}
Pruebe que si $X$ tiene incrementos independientes entonces el proceso $X^t$ dado por $X^t_s=X_{t+s}-X_t$ es independiente de $\F^X_t=\sag{X_s:s\geq 0}$.
\end{ejercicio}
\pn---------------------------------------------------\pn
\begin{ejercicio}
Calcular la esperanza y varianza del proceso de Poisson y de Poisson compuesto (en t\'erminos de la intensidad y la distribuci\'on de salto). Probar que si $X$ es\begin{esn}
\esp{e^{iu Z_t}}=e^{-\lambda t\paren{1-\imf{\psi}{u}}}\quad\text{donde}\quad \imf{\psi}{u}=\esp{e^{iu \xi_1}}. 
\end{esn}
\end{ejercicio}
\pn---------------------------------------------------\pn
\begin{ejercicio}
Sea $N$ un proceso de L\'evy tal que $N_t$ tiene distribuci\'on de par\'ametro $\lambda t$. 
\begin{enumerate}
\item Pruebe que casi seguramente las trayectorias de $N$ son no-decrecientes.
\item Sea $\Xi$ la \'unica medida en $\mc{B}_{\re_+}$ tal que $\imf{\Xi}{[0,t]}=N_t$. Pruebe que $\Xi$ es una medida de Poisson aleatoria de intensidad $\lambda \times\leb$.
\item Concluya que $N$ es un proceso de Poisson de intensidad $\lambda$. 
\end{enumerate}
\end{ejercicio}
\pn---------------------------------------------------\pn
\pn---------------------------------------------------\pn
\pn---------------------------------------------------\pn
\pn---------------------------------------------------\pn
\pn---------------------------------------------------\pn
\pn---------------------------------------------------\pn
\pn---------------------------------------------------\pn
\pn---------------------------------------------------\pn
\pn---------------------------------------------------\pn
\pn---------------------------------------------------\pn
1471
\pn        
        \nqed
    
    \part*{Referencias}
        \bibliography{GenBib}
        \bibliographystyle{amsalpha}
        \newpage

        
        
    \part*{Notas}\label{notas}
        \embedfile{notas.pdf}
        Se ha incluido una versión de las notas del curso como archivo adjunto en este \texttt{pdf}. No todos los lectores de \texttt{pdf} soportan 
        archivos adjuntos o no se comportan como es esperado. Adobe Reader y Evince (versiones para Windows) no se comportan como es esperado, 
        pero sí son capaces de accesar a los archivos adjuntos de alguna manera (la manera comprobada [en Windows] es extraer el archivo diciendo 
        ``guardar como'' y luego, como se hace con cualquier otro \texttt{pdf}, abrir el archivo recién creado).\pn
        
        El motivo de adjuntar el archivo, es porque se han hecho algunas citas a dichas notas. Si las notas cambian (lo cual es probable), 
        es posible que las referencias dejen de ser correctas.\pn
        
        Si no es posible recompilar este documento gracias a la librería para manejar archivos adjuntos, comente las lineas $50$ y $100$ del archivo 
        principal (\texttt{tareas.tex}) y vuelva a intentar.\pn
        
        Para el creador de las notas: Sería muy útil antes de cada sección mencionar en qué otra fuente se puede consultar el material de
        la sección. A veces se omiten detalles que para los novatos en probabilidad (como yo) no son obvios. Personalmente, añadiría recomendaciones 
        a los libros de $J.\,R.\,Norris$ y $Krishna\,B.\,Athreya$.
\end{document}