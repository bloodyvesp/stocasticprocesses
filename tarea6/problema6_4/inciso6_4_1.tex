\emph{
    Explique y pruebe formalmente por qu\'e, para toda 
    $n\geq 1$, $\partial^n M^\lambda_t/\partial \lambda^n$ es una martingala.\pn
}

\afterstatement\pn

Analicemos las primeras derivadas.

\begin{align}
    \frac{d}{d\lambda}  M_t^\lambda &=  \frac{d}{d\lambda} \left( e^{\lambda B_t - \lambda^2 t/2} \right)                       \\
                                    &=  \frac{d}{d\lambda} \left( e^{\lambda B_t} e^{- \lambda^2 t/2} \right)                   \\
                                    &=  B_t e^{\lambda B_t} e^{- \lambda^2/2}  - \lambda t e^{\lambda B_t} e^{- \lambda^2 t/2}  \\
                                    &=  B_t M_t^\lambda - \lambda t M_t^\lambda                                                 \\
                                    &=  M_t^\lambda (B_t - \lambda t).                                                          \\
\end{align}

La de segundo orden
\begin{align}
    \frac{d^2}{d\lambda^2}  M_t^\lambda     &=  \frac{d}{d\lambda} \left( M_t^\lambda (B_t - \lambda t) \right)                     \\
                                            &=  M_t^\lambda (B_t - \lambda t) (B_t - \lambda t) -  t M_t^\lambda                    \\ 
                                            &=  \left( \frac{d}{d\lambda} M_t^\lambda \right) (B_t - \lambda t) -  t M_t^\lambda         
\end{align}\pn

Como hipótesis de inducción digamos que
\begin{align}
    \frac{d^n}{d\lambda^n}  M_t^\lambda     &=  \left( \frac{d^{n-1}}{d\lambda^{n-1}} M_t^\lambda \right) (B_t - \lambda t) - 
                                            (n-1)t \left( \frac{d^{n-2}}{d\lambda^{n-2}} M_t^\lambda \right)
\end{align}
(donde $n \geq 1$, derivada de orden $0$ significa no derivar, y derivada de orden $-1$ significa $0$).\pn

Verifiquemos el paso inductivo.

\begin{align}
    &   \;\;\;\;\;\frac{d^{n+1}}{d\lambda^{n+1}}  M_t^\lambda                                                           \\
    &=  \frac{d}{d \lambda}\left(\left( \frac{d^{n-1}}{d\lambda^{n-1}} M_t^\lambda \right) (B_t - \lambda t)\right) - 
        \frac{d}{d \lambda}\left((n-1)t \left( \frac{d^{n-2}}{d\lambda^{n-2}} M_t^\lambda \right)\right)                \\
    &=  \left( \frac{d^{n}}{d\lambda^{n}} M_t^\lambda \right) (B_t - \lambda t) + 
        \left( \frac{d^{n-1}}{d\lambda^{n-1}} M_t^\lambda \right) \left(\frac{d}{d \lambda}(B_t - \lambda t)\right)     \\
    &   \;\;\;\;\;-(n-1)t \left( \frac{d^{n-1}}{d\lambda^{n-1}} M_t^\lambda \right)                                     \\
    &=  \left( \frac{d^{n}}{d\lambda^{n}} M_t^\lambda \right) (B_t - \lambda t) + 
        \left( \frac{d^{n-1}}{d\lambda^{n-1}} M_t^\lambda \right) (-t)                                                  \\
    &   \;\;\;\;\;-(n-1)t \left( \frac{d^{n-1}}{d\lambda^{n-1}} M_t^\lambda \right)                                     \\
    &=  \left( \frac{d^{n}}{d\lambda^{n}} M_t^\lambda \right) (B_t - \lambda t) - (n)t \left( \frac{d^{n-1}}{d\lambda^{n-1}} M_t^\lambda \right)                                     \\
\end{align}\pn

Una variable aleatoria con distribución normal es acotada casi seguramente y 
por lo tanto tiene momentos de todos los órdenes. Por el mismo argumento, el valor 
absoluto de una variable aleatoria con distribución normal tiene momentos de todos los
órdenes.

