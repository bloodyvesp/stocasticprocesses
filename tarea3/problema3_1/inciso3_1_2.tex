\emph{
    $T$ es integrable y la sucesi\'on $\paren{M_n-M_{n-1}}$ es acotada.
}

Que la sucesión $\paren{M_n-M_{n-1}}$ sea acotada, significa que existe $r \in R$ tal que
$\abs{M_n - M_{n-1}} < r$ para todo $n \geq 1$.\par\null

Que $T$ sea integrable significa que $\E(T) < \infty$. Recordemos que el que $T$ sea finito implica que 
$T \wedge n \underset{c.s.}\longrightarrow T$ (la justificación está en \ref{problema3_1:inciso1}). \par\null

Otra vez, $T \wedge n$ es tiepo de paro acotado, y por Teorema del Muestreo Opcional de Doob tenemos que 
$\E(M_{T \wedge n}) = \E(M_0)$. En este caso nos interesa escribirlo de la siguiente manera, 
$\E(M_{T \wedge n} - M_0) = 0$.\par\null

Ahora, 
\begin{align}
    \abs{M_{T \wedge n} - M_0}  &=      \abs{\sum_{1 \leq i \leq (T \wedge n)} (M_i - M_{i-1})}     \\
                                &\leq   \sum_{1 \leq i \leq (T \wedge n)} \abs{(M_i - M_{i-1})}     \\
                                &\leq   (T \wedge n) r                                              \\      
                                &\leq   T r                                                    
\end{align}

De donde 

\begin{align}
        \E(\abs{M_{T \wedge n} - M_0})  &\leq   \E(T r) \\
                                        &= r \E(T)      \\
                                        &< \infty.
\end{align} \par\null

Con esto tenemos que la suceción $(M_{T \wedge n} - M_0)_{n \in \N}$ es dominada por $Tr$ y que cada 
elemento de la suceción es integrable. Entonces tenemos todas las hipótesis del Teorema de convergencia dominada.
Y por lo tanto

\begin{align}
        \E(M_T - M_0)   &=  \E(\lim_{n \rightarrow \infty} M_{T \wedge n} - M_0)        \\
                        &=  \lim_{n \rightarrow \infty} \E(M_{T \wedge n} - M_0)        \\
                        &=  \lim_{n \rightarrow \infty} 0                               \\
                        &=  0.                        
\end{align}\par\null

Lo cual implica $\E(M_T) = \E(M_0)$, como queríamos demostrar.