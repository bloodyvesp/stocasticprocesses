\begin{problema}
	\mbox{}
	\begin{enumerate}
	\item 
		Al aplicar la desigualdad maximal de Doob sobre los racionales de orden $n$ y pasar al l\'imite 
        conforme $n\to\infty$, pruebe que $\sup_{t\leq }\abs{B_t-B_1}$ es cuadrado integrable.\pn
	
	\item 
		Pruebe que la sucesi\'on de variables aleatorias
		\begin{esn}
			\paren{\sup_{t\in [0,1]}\abs{B_{n+t}-B_n},n\in\na}
		\end{esn}
		son independientes, id\'enticamente distribuidas y de media finita. (Utilice la propiedad de Markov.)\pn
	
	\item 
		Al utilizar Borel-Cantelli, pruebe que, para cualquier $C>0$ fija
		\begin{esn}
			\limsup_{n\to\infty}\sup_{t\in [0,1]}\abs{B_{n+t}-B_n}/n\leq C
		\end{esn} 
        casi seguramente.\pn
	
	\item 
		Pruebe que $\paren{B_n/n,n\geq 1}$ converge casi seguramente a $0$ y deduzca que
		\begin{esn}
			\lim_{t\to\infty }B_t/t=0.
		\end{esn}\pn
	\end{enumerate}
\end{problema}

\begin{proof}
    \subsection{Inciso (i)} \label{problema6_5:inciso1}
    \emph{
    Al aplicar la desigualdad maximal de Doob sobre los racionales de orden $n$ y pasar al l\'imite 
    conforme $n\to\infty$, pruebe que $\sup_{t\leq 1}\abs{B_t-B_1}$ es cuadrado integrable.\pn
}

\afterstatement\pn

La idea es partir el proceso en martingalas finitas que al irse refinando nos den un conjunto denso en $[0, 1]$ y
luego utilizar el argumento de continuidad para llenar los huecos.\pn

Llamemos 
\begin{align}
 D_n = \bigg\{ \frac{i}{2_n} : i = 0, 1, \dots, 2^n \bigg \}   
\end{align}

Con estos conjuntos, discretizamos a $B$ haciendo las martingalas finitas
$(B_j^n : j \in D_n)$ [donde $B_j^n = B_j$]. Por propiedades de la esperanza condicional y el hecho de que
$(B_j^n : j \in D_n)$ son martingalas. Tenemos que $(\abs{B_j^n} : j \in D_n)$ son sub-marginalas.\pn

La desigualdad maximal de Doob para el caso $L_2$ nos dice que
\begin{align}
        \| \max_{j \in D_n} \abs{B_j^n} \|_2 \leq 2 \| B_1^n\|_2
\end{align}

De donde, elevando al cuadrado y usando que $\E(B_1^n) = 0$.
\begin{align}
        \E\left( \left( \max_{j \in D_n} \abs{B_j^n} \right)^2 \right)  &=  4 \E( (B_1^n)^2 )   \\
                                                                        &=  4 Var(B_1^n)        \\
                                                                        &=  4 Var(B_1)          \\
                                                                        &=  4.
\end{align}\pn

Ahora tenemos que conforme $n \rightarrow \infty$, $\max_{j \in D_n} \abs{B_j^n}$ converge a $\sup \abs{B_t}$ c.s.
El teorema de convergencia monótona nos asegura que
\begin{align}
        E\left( \left( \sup \abs{B_t} \right)^2 \right) &= \lim_{n \rightarrow \infty}   \E\left( \left( max_{j \in D_n} \abs{B_j^n} \right)^2 \right)
\end{align}\pn

Como cada término del límite de la derecha está acotado por $4$ por la demostración de arriba.\pn

Entonces  $\sup_{t\leq 1}\abs{B_t-B_1}$ es cuadrado integrable.
    \newpage

    \subsection{Inciso (ii)} \label{problema6_5:inciso2}
    \emph{
    Pruebe que la sucesi\'on de variables aleatorias
    \begin{esn}
        \paren{\sup_{t\in [0,1]}\abs{B_{n+t}-B_n},n\in\na}
    \end{esn}
    son independientes, id\'enticamente distribuidas y de media finita. (Utilice la propiedad de Markov.)\pn
}

\afterstatement\pn

La propiedad de Markov nos asegura que para todo $n \in N$ el proceso dado obtenido ``desfasar'' el proceso 
en $n$  es decir: $B_{n+t} - B_n$ es independiente de $\F_n = \sigma(B_t : t \leq n)$ y por lo tanto
también $\abs{B_{n+t} - B_n}$ es independiente de $\F_n$ para toda $n$ y por lo tanto son independientes.\pn

Además, sabemos que desfasar es únicamente volver a empezar el proceso, es decir, $B_{n+t} - B_n \sim \B_t$ y entonces
$\abs{B_{n+t} - B_n} \sim \abs{B_t}$. Tomando supremos $\sup_{t \leq 1} \abs{B_{n+t} - B_n} \sim \sup_{t \leq 1} \abs{B_t}$.
Es decir que $\sup_{t \leq 1} \abs{B_{n+t} - B_n}$ tienen todas la misma distribución que $\sup_{t \leq 1} \abs{B_t}$.\pn

El en [\ref{problema6_5:inciso1}] se demostró que $\sup_{t\leq 1}\abs{B_t-B_1}$ es cuadrado integrable, y por lo tanto
tiene medida finita. Ese es el caso cuando $n = 1$, pero como para toda $n$ se tiene la misma distribución, todas
tienen medida finita. 
    \newpage

    \subsection{Inciso (iii} \label{problema6_5:inciso3}
    \emph{
    Al utilizar Borel-Cantelli, pruebe que, para cualquier $C>0$ fija
    \begin{esn}
        \limsup_{n\to\infty}\sup_{t\in [0,1]}\abs{B_{n+t}-B_n}/n\leq C
    \end{esn} 
    casi seguramente.\pn
}
\afterstatement\pn 
    \newpage

    \subsection{Inciso (iv)} \label{problema6_5:inciso4}
    \emph{
    Pruebe que $\paren{B_n/n,n\geq 1}$ converge casi seguramente a $0$ y deduzca que
    \begin{esn}
        \lim_{t\to\infty }B_t/t=0.
    \end{esn}\pn
}

\afterstatement\pn

Escribimos
\begin{align}
        B_n = \sum_{i \leq n} (B_i - B_{i-1})
\end{align}

Recordemos que gracias a [\ref{problema6_5:inciso2}] tenemos que los sumandos de la derecha son variables aleatorias idependientes e idénticamente
distribuidas y que además, su distribución es normal de media $0$ y varianza $1$.\pn

De aquí, aplicando ley fuerte de los grandes números, tenemos que
\begin{align}
        \lim_{n \rightarrow \infty} \frac{B_n}{n}   &= \lim_{n \rightarrow \infty} \frac{\sum_{i \leq n} (B_i - B_{i-1})}{n}    \\
                                                    &= \E(B_1)                                                                  \\
                                                    &=  0.
\end{align}\pn

Sean ahora $t \geq 0$, $n = \lfloor t \rfloor$ y $s = t - n$ y entonces acotamos a $B_t/t$ por algo que ya conozcamos.
\begin{align}
    \frac{\abs{B_t}}{t} &=      \frac{\abs{B_{n+s}}}{n+s}                                                   \\
                        &\leq   \frac{\abs{B_{n+s}}}{n}                                                     \\
                        &=      \frac{\abs{B_{n+s} - B_n + B_n}}{n}                                         \\
                        &=      \frac{\abs{B_{n+s} - B_n}}{n} + \frac{\abs{B_n}}{n}                         \\
                        &=      \frac{\sup_{s \leq 1}\abs{B_{n+s} - B_n}}{n} + \frac{\abs{B_n}}{n}          \\
\end{align}\pn 

Ahora tomaremos límite superior, el primer sumando se hace $0$ por la conclusión de [\ref{problema6_5:inciso3}], y el sumando de la
derecha tiene límite y es 0, por lo tanto tiene límite superior y  también es 0. 
\end{proof}