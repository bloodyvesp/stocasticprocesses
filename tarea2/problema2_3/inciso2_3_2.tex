\emph{
    Descomposici\'on de Doob para submartingalas: Sea $X=\paren{X_n}_{n\in\na}$ una submartingala. 
    Pruebe que $X$ se puede descomponer de manera \'unica como $X=M+A$ donde $M$ es una martingala y $A$ 
    es un proceso previsible con $A_0=0$. Sugerencia: Asuma que ya tiene la descomposici\'on y calcule 
    esperanza condicional de $X_{n+1}$ dada $X_n$. 
}

\afterstatement\par\null

$\paren{X_n}_{n\in\na}$ es $(\F_n)_{n \in \N}$-adaptada por definición de submartingala. Utilizando el inciso anterior tenemos
que $M_n = \sum_{k=1}^n X_k-\espc{X_k}{\F_{k-1}}, n \in \N$ es martingala.\par\null

%Ahora sea $M_n = N_n + X_0$. Veamos que $(M_n)_{n \in \N}$ sigue siendo martingala.
%
%\begin{itemize}
	%\item 
        %Veamos que $(M_n)_{n \in \N}$ es $(\F_n)_{n \in \N}$-adaptada.\par\null
        %
        %Por ser $(N_n)_{n \in \N}$ una martingala, $N_n$ es $\F_n$-medible y $X_0$ es 
        %$\F_0 \subset \F_n$-medible. Entonces $M_n$ es suma finita de variables $\F_n$-medibles y por lo tanto $\F_n$-medible.\par\null
    %
    %\item 
        %Veamos que $M_n \in L_1$ para toda $n \in \N$.\par\null
        %
        %Por ser $(N_n)_{n \in \N}$ una martingala, $N_n \in L_1$ para todo $n \in \N$. 
        %Por ser $(X_n)_{n \in \N}$ una subpartingala, $X_0 \in  L_1$.\par\null
        %
        %Por lo tanto $M_n$ es suma finita de variables en $L_1$ y por lo tanto pertenece a $L_1$.\par\null
    %
    %\item
        %Veamos que $\E(M_n | F_{n-1}) = M_{n-1}$.\par\null
        %
        %\begin{align}
            %\E(M_n | F_{n-1})   &=  \E(N_n + X_0 | F_{n-1})                     \\
                                %&=  \E(N_n | F_{n-1}) + \E( X_0 | F_{n-1})      \\
                                %&=  N_{n-1} + X_0                               \\
                                %&=  M_{n-1}.
        %\end{align}
        %
        %Y con esto terminamos de demostrar que $(M_n)_{n \in \N}$ es martingala.
%\end{itemize}

Si definimos a $A_n = X_n - M_n$ tendremos que $M_n + A_n = M_n + X_n - M_n = X_n$. Probemos ahora que
$(A_n)_{n \in \N}$ es previsible.\par\null

\begin{align}
    A_n     &=      X_n - M_n                                                                   \\
            &=      X_n - \sum_{k=1}^n X_k-\espc{X_k}{\F_{k-1}}                                 \\
            &=      X_n - X_n + \E(X_n | \F_{n-1}) + \sum_{k=1}^{n-1} X_k-\E(X_k|\F_{k-1})      \\     
            &=      \E(X_n | \F_{n-1}) + \sum_{k=1}^{n-1} X_k-\E(X_k|\F_{k-1}).     
\end{align}\par\null

Donde por ser $(X_n){n \in \N}$ submartingala tenemos que $\E(X_k|\F_{k-1})$ es $F_{n-1}$ siempre que $0 \leq k \leq n-1$.
Y por lo tanto $A_n$ es suma finita de variables $F_{n-1}$-medibles y por lo tanto $F_{n-1}$-medible.\par\null

Hemos demostrado hasta ahora que una submartingala se puede descomponer como la suma de una martingala y un proceso previsible.
Demostremos ahora la unicidad.\par\null

Sean entonces $M'_n$ y $A'_n$ una martingala y un proceso previsible respectivamente tal que $X_n = M'_n + A'_n$.\par\null

Entonces para toda $n \in \N$ tenemos que $M'_n + A'_n = M_n + A_n$. Por lo tanto 

\begin{align}
    (M'_n + A'_n) - (M'_{n+1} + A'_{n+1}) = (M_n + A_n) - (M_{n+1} + A_{n+1})
\end{align}\par\null

Tomemos esperanzas condicionales sobre el lado derecho.

\begin{align}
                                                    & \E((M_n + A_n) -( M_{n+1} + A_{n+1}) | \F_n)                                              \\ 
                                                    &=  \E((M_n + A_n) -( M_{n+1} + A_{n+1}) | \F_n)                                            \\
                                                    &=  \E(M_n - M_{n+1}| \F_n) + \E(A_n - A_{n+1} | \F_n)                                      \\
                                                    &=  M_n - M_{n} + \E(A_n - A_{n+1} | \F_n) \label{problema2_3:hipotesis_de_martingala}      \\
                                                    &=  \E(A_n - A_{n+1} | \F_n)                                                                \\
                                                    &=  \E(A_n| \F_n) - \E(A_{n+1} | \F_n)                                                      \\
                                                    &=  A_n - A_{n+1} \label{problema2_3:hipotesis_de_proceso_previsible}                                                                               
\end{align}\par\null

Donde \eqref{problema2_3:hipotesis_de_martingala} es gracias a que $(M_n)_{n\in\N}$ es martingala y \eqref{problema2_3:hipotesis_de_proceso_previsible}
es gracias a que $(A_n)_{n\in\N}$ es un proceso previsible.\par\null

De manera análoga obtenemos que $\E((M'_n + A'_n) - (M'_{n+1} + A'_{n+1}) | \F_n) = A'_n - A'_{n+1}$ y por lo tanto $A_n - A_{n+1} = A'_n - A'_{n+1}$.\par\null

Lo que esto último dice, es que $(A_n)_{n\in\N}$ y $(A'_n)_{n\in\N}$ crecen igual. Recordemos que por hipótesis $A'_0, A_0 = 0$.\par\null

Entonces
\begin{align}
    A_0 - A_1 &= A'_0 - A'_1 &\Rightarrow A_1 &= A'_1
\end{align}\par\null

Inductivamente, si $A_n = A'_n$, entonces
           
\begin{align}
    A_n - A_{n+1} &= A'_n - A'_{n+1} &\Rightarrow A_{n+1} &= A'_{n+1}
\end{align}                                                                                                         

Y con esto queda demostrada la unicidad de $(A_n)_{n\in\N}$. Para la demostración de la unicidad de $M_n$, sólo recordemos que 
$M_n = X_n - A_n$ y que $M'_n = X_n - A'_n$, por ser $A_n$ y $A'_n$ iguales, entonces $M_n$ y $M'_n$ también lo son.