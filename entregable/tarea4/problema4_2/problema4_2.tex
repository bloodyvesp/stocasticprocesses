\begin{problema}
	Sea $\mu$ una distribuci\'on de progenie y defina $\tilde \mu_j=\mu_{j+1}$. 
	Sea $S=\paren{S_n}$ una caminata aleatoria con distribuci\'on de salto $\tilde\mu$. 
	Sea $k$ un entero no-negativo y defina recursivamente

	\begin{esn}
		Z_0=k=C_0,\quad Z_{n+1}=k+S_{C_n} \quad \text{y} \quad C_{n+1}=C_n+Z_{n+1}.
	\end{esn}

	\begin{enumerate}
		\item[(i)]		[\ref{problema4_2:inciso1}]
			Pruebe que $Z_n\geq 0$ para toda $n$ y que si $Z_n=0$ entonces $Z_{n+1}=0$.\pn
		
		\item[(ii)]		[\ref{problema4_2:inciso2}]
			Pruebe que $C_n$ es un tiempo de paro para la filtraci\'on can\'onica asociada a $S$.\pn
		
		\item[(iii)] 	[\ref{problema4_2:inciso3}]
			Pruebe que $Z$ es un proceso de Galton-Watson con ley de progenie $\mu$.\pn
		
		\item[(iv)]		[\ref{problema4_2:inciso4}] 
			Pruebe que si $S$ alcanza $-1$ entonces existe $n$ tal que $Z_n=0$. Deduzca que si la media de $\mu$ es $1$ entonces $Z$ se extingue. (Sugerencia: utilice un ejercicio anterior sobre martingalas con saltos acotados hacia abajo.) 
	\end{enumerate}

	\defin{Categor\'ias: } Caminatas aleatorias, Procesos de Galton-Watson%, Propiedad de Markov fuerte.
\end{problema}

\begin{proof}
    \subsection{Inciso (i)} \label{problema4_2:inciso1}
    \emph{
    Pruebe que $Z_n\geq 0$ para toda $n$ y que si $Z_n=0$ entonces $Z_{n+1}=0$.
}

\afterstatement\pn

Sean $(\xi_i)_{i \in \N}$ v.a.i.i.d con distribución $\tilde\mu$ 
(es decir, $\mw(\xi_i = s) = \tilde\mu_s$ donde $-1 \leq s$).\pn

Es decir que podemos denotar a $S_n = \sum_{i \leq n} \xi_i$.\pn

Primero demostraremos que $Z_n \geq 0$. La demostración se hará por inducción sobre $n$.

\begin{itemize}
	\item 
        \textbf{Base De Inducción.}
        
        Para $n = 0$ tenemos $Z_0 = k \geq 0$.\pn
        
    \item
        \textbf{Hipótesis De Inducción.}
        
        Sea $n$ tal que $Z_i \geq 0$ para toda $i \leq n$.\pn
        
    \item
        \textbf{Paso Inductivo.}
        
        \begin{align} \label{problema4_2:descomposicion_de_Z_n+1}
            Z_{n+1}     &=  k + S_{C_n}                                                                             \\
                        &=  k + \sum_{i \leq C_n} \xi_i                                                             \\
                        &=  k + \sum_{i \leq C_{n-1} + Z_{n}} \xi_i                                                 \\
                        &=  k + \sum_{i \leq C_{n-1}} \xi_i + \sum_{C_{n-1} < i \leq  C_{n-1} + Z_{n}} \xi_i        \\
                        &=  k + S_{C_n-1} + \sum_{C_{n-1} < i \leq  C_{n-1} +Z_{n}} \xi_i                           \\
                        &=  Z_n + \sum_{C_{n-1} < i \leq  C_{n-1} + Z_{n}} \xi_i                                    \\                        
        \end{align}
        
        En la suma del lado derecho tenemos exáctamente $Z_n$ sumandos, y cada sumando es mayor o igual a $-1$. Entonces
        en el peor de los casos la suma de la derecha es igual a $-Z_n$. Es decir, 
        
        \begin{align}
            \sum_{C_{n-1} < i \leq  C_{n-1} + Z_{n}} \xi_i \geq -Z_n.
        \end{align}
        
        Y combinando estos resultados tenemos
        
        \begin{align}
            Z_{n+1} &=      Z_n + \sum_{C_{n-1} < i \leq  C_{n-1} + Z_{n}} \xi_i    \\
                    &\geq   Z_n - Z_n                                               \\
                    &=      0.
        \end{align}
        
        Y con esto termina la demostración.
\end{itemize}

Ahora demostraremos que si $Z_n = 0$, entonces $Z_{n+1} = 0$.\pn

Notemos los siguientes hechos. 
\begin{itemize}
	\item 
        $Z_n = k + S_{C_{n-1}} = 0$ implica que $S_{C_{n-1}} = -k$.
    
    \item
        $C_n = C_{n-1} + Z_n = C_{n-1}$.
\end{itemize}

Con esto presente, basta recordar la definición de $Z_{n+1}$:

\begin{align}
    Z_{n+1} &=  k   +   S_{C_n}         \\
            &=  k   +   S_{C_{n-1}}     \\
            &=  k   -   k               \\
            &=  0.
\end{align}

Que era lo que buscábamos demostrar.
    \newpage

    \subsection{Inciso (ii)} \label{problema4_2:inciso2}
    \emph{
	Pruebe que $C_n$ es un tiempo de paro para la filtraci\'on can\'onica asociada a $S$.
}

\afterstatement\pn

Sea $(\F_n)_{n \in \N}$ la filtración canónica asociada a $(S_n)_{n \in \N}$.
Tenemos que demostrar que para $m \in \Z$, $\{ C_n = m \} \in \F_m$.\pn

Usemos la definición de $C_n$ para descomponerlo, 

\begin{align}\label{problema4_2:descomposicion_de_C_n}
    C_n &=  C_{n-1} + Z_n                   \\
        &=  C_{n-2} + Z_{n-1} + Z_n         \\
        &\vdots                             \\
        &= k + \sum_{1 \leq i \leq n} Z_i   
\end{align}

Ahora, recordando que $0 \leq Z_i$, tenemos que $k \leq C_n$, es decir que
$\{ C_n < k\} = \emptyset$, en particular, si $m < k$, $\{C_n = m\} = \emptyset \in \F_m$.\pn

Supongamos entonces que $k \leq m$. De nuevo gracias a que $0 \leq Z_i$, tenemos que la sucesión $(C_i)_{i \in \N}$
es creciente. Entonces, dentro del conjunto $\{ C_n = m\}$ tenemos que $C_i \leq m$ para $i \leq n$. Y por lo tanto,
dentro del conjunto $\{ C_n = m\}$, $S_{C_i}$ es $\F_m$-medible (donde dentro del conjunto se debería interpretar como
al intesectar con dicho conjunto, las preimagenes de $S_{C_i}$ pertenecen a $\F_m$).\pn

Entonces $Z_i = S_{C_{i-1}}$ también resulta ser $\F_m$-medible (dentro de $\{ C_n = m \}$ y por lo tanto. Entonces nos
basta con descomponer $\{ C_n = m \}$ en conjuntos que sean $\F_m$-medibles usando lo que hemos demostrado ahora.
Utilizando \eqref{problema4_2:descomposicion_de_C_n} tenemos que

\begin{align}
    \{ C_n = m \}   &=  \bigg\{ k + \sum_{1 \leq i \leq n} Z_i  = m \bigg\}.                         
\end{align}

Y por todo lo anteriormente dicho, el lado derecho de la igualdad resulta ser $\F_m$-medible.

    \newpage

    \subsection{Inciso (iii)} \label{problema4_2:inciso3}
    \emph{
	Pruebe que $Z$ es un proceso de Galton-Watson con ley de progenie $\mu$.
}

\afterstatement\par\null

Tenemos que encontrar variables $\zeta_{i,n}$ con distribución $\mu$ tales que

\begin{align}
    Z_{n+1} =   \sum_{1 \leq i \leq Z_n} \zeta_{i,n}. \label{problema4_2:expresion_Galton-Watson}
\end{align}

Sean $\xi_{i}$ con $-1 \leq i$ definidas como en el primer inciso de este ejercicio [\ref{problema4_2:inciso1}],
estas estaban definidas bajo una distribución $\tilde\mu$ que era exactamente igual a $\mu$ pero desfasada por $-1$.
Es decir $\tilde\mu_{-1} = \mu_{0}, \tilde\mu_{0} = \mu_{1}, \dots, \tilde\mu_{j} = \mu_{j+1}$. Entonces, 
$\xi_{i} + 1$ tendrá distribución $\mu$.\par\null

Ya teníamos una descomposición útil de $Z_{n+1}$ en \eqref{problema4_2:descomposicion_de_Z_n+1}. La reescribiremos cambiando $\xi_{i}$
por $\xi_{i} + 1 - 1$.

\begin{align}
    Z_{n+1} &=  Z_n + \paren{\sum_{C_{n-1} < i \leq  C_{n-1} + Z_{n}} \xi_i}                \\
            &=  Z_n + \paren{\sum_{C_{n-1} < i \leq  C_{n-1} + Z_{n}} (\xi_i + 1 - 1)}      \\
            &=  Z_n + \paren{\sum_{C_{n-1} < i \leq  C_{n-1} + Z_{n}} (\xi_i + 1)} - Z_n    \\
            &=  \paren{\sum_{C_{n-1} < i \leq  C_{n-1} + Z_{n}} (\xi_i + 1)}                
\end{align}

Ahora que tenemos a $Z_{n}$ expresado como suma de variables con distribución $\mu$, bastará reescribirlo de manera que
tenga la forma \eqref{problema4_2:expresion_Galton-Watson}.\par\null

Definamos entonces $\zeta_{i, n} = \xi_{C_{n-1} + 1$, y entonces

\begin{align}
    Z_{n+1} &=  \paren{\sum_{C_{n-1} < i \leq  C_{n-1} + Z_{n}} (\xi_i + 1)}    \\
            &=  \paren{\sum_{1 \leq i \leq Z_{n}} (\zeta_{i, n})}.
\end{align}

Y con esto hemos demostrado que $Z$ es un proceso Galton-Watson con distribución de progenie $\mu$ y, como $Z_0 = k$, se trata
de un proceso Galton-Watson con población inicial $k$.
	\newpage
	
    \subsection{Inciso (iv)} \label{problema4_2:inciso4}
    \emph{
    Pruebe que si $S$ alcanza $-1$ entonces existe $n$ tal que $Z_n=0$. 
    Deduzca que si la media de $\mu$ es $1$ entonces $Z$ se extingue. 
    (Sugerencia: utilice un ejercicio anterior sobre martingalas con saltos acotados hacia abajo.) 
}

\afterstatement\par\null

Suponiendo que la media de $\mu$ es $1$, significa que $\E(\xi_i + 1) = 1$ y por lo tanto $\E(\xi_i) = 0$.\par\null

Esto convierte a $(S_n)_{n \in \N}$ en una caminata aleatoria no trivial y centrada (la media de sus saltos es 0).
Por lo dicho en [\ref{problema3_2}] tenemos que es una caminata que oscila y por lo tanto $\liminf S_n = -\infty$.\par\null

Esto quiere decir que existe un $n$ tal que $S_{C_{n}} = -k$ y entonces 
$Z_{n + 1} = k + S_{C_{n}} = k - k = 0$. Lo cual se traduce en que la población se extinge casi seguramente.\par\null
\end{proof}