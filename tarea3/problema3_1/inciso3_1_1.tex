\emph{
    $M$ es acotada.
}

\afterstatement\par\null

Que $M$ sea acotada significa que existen $a \in  \R$ tal que $\abs{M_n} < r$ para todo $n \in \N$. En particular 
tenemos que $\abs{M_{T \wedge n}} < r$.\par\null

Por otro lado, que $T$ sea finito implica que $T \wedge n  \underset{c.s.}\longrightarrow T$ conforme $n \rightarrow \infty$.
(Para todo $\omega \in \Omega$, $T(\omega) = n_0 < \infty$ y por lo tanto $T \wedge n (\omega) = T(\omega) = n_0$ para todo
$n \geq n_0$). De donde afirmamos que $M_{T \wedge n} \underset{c.s.}\longrightarrow M_T$. \par\null

Ahora, $T \wedge n$ es tiepo de paro acotado, y por Teorema del Muestreo Opcional de Doob tenemos que 
$\E(M_{T \wedge n}) = \E(M_0)$.\par\null

Tenemos todas las hipótesis para aplicar el Teorema de Convergencia Acotada y por lo tanto:

\begin{align}
    \E(M_T)     &=  \E(\lim_{n \rightarrow \infty} M_{T \wedge n})     \\
                &=  \lim_{n \rightarrow \infty} \E(M_{T \wedge n})     \\
                &=  \E(M_0).
\end{align}\par\null

Con lo que termina la demostración.
 