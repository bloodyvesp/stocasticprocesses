\begin{problema}[Tomado del examen general de probabilidad del Posgrado en Ciencias Matem\'aticas, UNAM, \href{http://www.posgradomatematicas.unam.mx/contenidoEstatico/archivo/files/pdf/Examenes_Generales/Probabilidad/Probabilidad2011-1.pdf}{Febrero 2011}]
Una planta de producci\'on toma su energ\'ia de dos generadores. La cantidad de generadores al tiempo 
$t$ est\'a representado por una cadena de Markov a tiempo continuo $\set{X_t,t\geq 0}$ con espacio de 
estados $E=\set{0,1,2}$ y matriz infinit\'esimal $Q$ dada por
\begin{esn}
    Q=
        \begin{pmatrix}
            -6  &6  &   0  \\
            1   &-7 &   6  \\
            0   &2  &   -2
        \end{pmatrix}.
\end{esn}
    \begin{enumerate}
        \item[(i)]      [\ref{problema5_5:inciso1}]
            Encuentre la matriz de transici\'on de la cadena de Markov de los estados distintos que toma $X$, 
            clasifique los estados, diga si existe una \'unica distribuci\'on invariante y en caso afirmativo, 
            encu\'entrela. Calcule expl\'icitamente las potencias de la matriz de transici\'on. 
            (Recuerde que de ser posible diagonalizar, esta es una buena estrategia.)\pn
            
        \item[(ii)]     [\ref{problema5_5:inciso2}] 
            ?`Cu\'al es la probabilidad de que ambos generadores est\'en trabajando al tiempo $t$ 
            si s\'olo uno trabaja al tiempo cero?\pn
            
        \item[(iii)]    [\ref{problema5_5:inciso3}] 
            Si $\rho_2$ denota la primera vez que ambos generadores est\'an trabajando al mismo tiempo, 
            encuentre la distribuci\'on de $\rho_2$ cuando s\'olo un generador est\'a trabajando al tiempo cero.\pn
            
        \item[(iv)]     [\ref{problema5_5:inciso4}]
            Encuentre la proporci\'on de tiempo asint\'otica en que los dos generadores est\'an trabajando. 
            Si cada generador produce 2.5 MW de energ\'ia por unidad de tiempo, ?`Cu\'al es la cantidad promedio 
            de energ\'ia producida a largo plazo por unidad de tiempo?\pn
    \end{enumerate}
\end{problema}

\begin{proof}
    \subsection{Inciso (i)} \label{problema5_5:inciso1}
    \emph{
    Encuentre la matriz de transici\'on de la cadena de Markov de los estados distintos que toma $X$, 
    clasifique los estados, diga si existe una \'unica distribuci\'on invariante y en caso afirmativo, 
    encu\'entrela. Calcule expl\'icitamente las potencias de la matriz de transici\'on. 
    (Recuerde que de ser posible diagonalizar, esta es una buena estrategia.)\pn
}

\afterstatement\pn

Como no hay estados absorbentes, podemos utilizar la fórmula
\begin{align}
        P(i,j)  &=  \frac{Q(i, j)}{c(i)} (i-\delta_{i,j})
\end{align}

donde $c(i)$ es la suma de las entradas positivas de la $i$-ésima fila de $Q$ y $\delta_{i,j} = 1$
si $i = j$ y $\delta_{i,j} = 0$ en cualquier otro caso.\pn

Después de algunas cuentas se obtiene:
\begin{align}
     P=
        \begin{pmatrix}
            0           &   1   &   0               \\
            \frac{1}{7} &   0   &   \frac{6}{7}     \\
            0           &   1   &   0
        \end{pmatrix}.   
\end{align}\pn

De $P$ se puede todos los estados son alcanzables desde cualquier otro (de los estados $0$ y $2$ 
se puede llegar al $1$, y del $1$ a cualquiera de los otros). Es decir, es irreducible.\pn

Existe un teorema que nos asegura que por ser irreducible y de estados finitos, los estados son positivos
recurrentes y existe una distribución invariante. También existe otro teorema que nos asegura que por tener
estados positivos recurrentes, la distribución invariante es única.\pn

Entonces, buscar una distribución invariante se reduce a resolver el sistema de ecuaciones

\begin{align}
        \frac{1}{7}x_2              &=      x_1 \\
        x_1 + x_3                   &=      x_2 \\
        \frac{6}{7}x_2              &=      x_3 \\
\end{align}

Donde $x_1 + x_2 + x_3 = 1$. Después de algunas cuentas, se llega a que la solución del sistema es
$x_1 = \frac{1}{14}$, $x_2 = \frac{7}{14}$, $x_3 = \frac{6}{14}$.\pn

Para calcular explícitamente las potencias de $P$, primero encontramos el polinomio característico de $P$.
\begin{align}
        p(x)    =   -\lambda(\lambda + 1)(\lambda - 1).
\end{align}

De donde, los valores propios son $0, -1, 1$. Después de resolver los sistemas de ecuaciones correspondientes 
(lo cual son muchas cuentas) se obtiene que los vectores propios correspondientes son $(-6, 0 ,1), (1, 1, 1), (1, -1, 1)$.

Por lo tanto, $P$ se puede diagonalizar como
\begin{align}
    A^{-1} P A  = D
\end{align}

Donde 
\begin{align}
     A  =
        \begin{pmatrix}
            -6  &   1   &   1      \\
            0   &   1   &   -1     \\
            1   &   1   &   1
        \end{pmatrix}.   
\end{align}\pn

y 


\begin{align}
     D  =
        \begin{pmatrix}
            0   &   0   &   0      \\
            0   &   1   &   0      \\
            0   &   0   &   -1
        \end{pmatrix}.   
\end{align}\pn

y Entonces

\begin{align}
        P^n = A D^n A^{-1}
\end{align}

Ahora notemos que 

\begin{align}
     D^n  =
            \begin{pmatrix}
                0   &   0   &   0      \\
                0   &   1   &   0      \\
                0   &   0   &   -1^n
            \end{pmatrix}.   
\end{align}\pn

Por lo tanto
\begin{align}
     P^{2n + 1} = A D^{2n+1} A^{-1} = A D A^{-1} = P
\end{align}

Y 
\begin{align}
     P^{2n} = A D^{2} A^{-1} = 
                            \begin{pmatrix}
                                \frac{1}{7}     &   0   &   \frac{6}{7}     \\
                                0               &   1   &   0               \\
                                \frac{1}{7}     &   0   &   \frac{6}{7}
                            \end{pmatrix}.
\end{align}


    \newpage

    \subsection{Inciso (ii)} \label{problema5_5:inciso2}
    \emph{
    ?`Cu\'al es la probabilidad de que ambos generadores est\'en trabajando al tiempo $t$ 
    si s\'olo uno trabaja al tiempo cero?\pn
}

\afterstatement\pn

Ahora la estrategia consiste en diagonalizar $Q$. 
Su polinomio característico es
\begin{align}
        q(x) =  \lambda (\lambda + 5) (\lambda + 10).
\end{align}

Por lo tanto sus valores propios son $0, -5, -10$. Después de resolver los
sistemas de ecuaciones correspondientes se obtiene que los vectores propios correspondientes son
$(1,1,1), (18, 3, -2), (-6, 4, -1)$.

Con lo que definimos a la matriz
\begin{align}
     A      =
                \begin{pmatrix}
                    1   &   18  &   -6      \\
                    1   &   3   &   4       \\
                    1   &   -2  &   -1
                \end{pmatrix}   
\end{align}
la cual tiene inversa
\begin{align}
     A^{-1}     =   \frac{1}{25}
                    \begin{pmatrix}
                        1       &   6   &   18      \\
                        1       &   1   &   -2      \\
                        -1      &   4   &   -3
                    \end{pmatrix}.   
\end{align}

Ahora definimos
\begin{align}
    D = A^{-1} Q A  =
                        \begin{pmatrix}
                            0   &   0   &   0      \\
                            0   &   -5  &   0      \\
                            0   &   0   &   -10
                        \end{pmatrix}.
\end{align}

Para calcular la probablidad de que $2$ generadores estén funcionando al tiempo $t$ dado que se empezó con $1$, necesitamos 
encontrar la entrada $(2,3)$ de $P_t$ (recordemos que el estado representado en el renglón $1$ significa ningún generador 
está funcionando, el representado por el renglón $2$ significa que un generador está funcionando y el representado por el 
$3$ significa que ambos generadores están funcionando).\pn
  
Para ello utilizamos que $P_t = e^{tQ} = e^{t(ADA^{-1})} = Ae^{tD}A^{-1}$ y que
\begin{align}
    e^{tD}  =   
                \begin{pmatrix}
                    e^{0 t} &   0           &   0      \\
                    0       &   e^{-5 t}    &   0      \\
                    0       &   0           &   e^{-10 t}
                \end{pmatrix}
            =   
                \begin{pmatrix}
                    1       &   0           &   0      \\
                    0       &   e^{-5 t}    &   0      \\
                    0       &   0           &   e^{-10 t}
                \end{pmatrix}.
\end{align}

Entonces, 
\begin{align}
        P_t(2,3) &= \frac{1}{25} (1,3,4) \cdot (18, -2e^{-5t}, -3e^{-10t})                                                                                 \\
                 &= \frac{1}{25} (18 - 6e^{-5t} - 12e^{-10t}).                                                                                                \\
\end{align}
    \newpage

    \subsection{Inciso (iii)} \label{problema5_5:inciso3}
    \emph{
    Si $\rho_2$ denota la primera vez que ambos generadores est\'an trabajando al mismo tiempo, 
    encuentre la distribuci\'on de $\rho_2$ cuando s\'olo un generador est\'a trabajando al tiempo cero.\pn
}

\afterstatement\pn

Este problema es similar al anterior. En el inciso anterior $P_t(2,3)$ representa la probabilidad de pasar del estado $2$ al $3$
en $t$, pero en la cadena cualquier estado es alcanzable desde cualquier otro, es decir que $P_t(2,3)$ concidera la posibilidad
de que se haya pasado varias veces por $3$ antes de que terminara $t$. Lo único que hay que suponer entonces, es que el estado 
``dos generadores están funcionando'' es absorbente y entonces lo que buscamos es la $\P(\rho_2 \leq t)$.\pn

Entonces, sustituimos el tercer renglón de $Q$ por puros ceros para convertir al estado ``dos generadores están funcionando''
en un estado absorvente y dejar a todas las distribuciones que no implican salir del estado ``dos generadores están 
funcionando'' sin cambios.\pn

Definimos entonces $Q'$ como
\begin{align}
    Q'  =
        \begin{pmatrix}
            -6  &6  &   0  \\
            1   &-7 &   6  \\
            0   &0  &   0
        \end{pmatrix}.
\end{align}

De ahora en adelante, la el problema es idéntico al del inciso anterior. Buscamos el polinomio característico de $Q'$.
\begin{align}
    q'(\lambda)   &=  \lambda(4+\lambda)(9+\lambda)
\end{align}

Vemos que sus valores propios son $0, -4 y -9$. Después de cuentas, encontramos los vectores propios asociados
$(1, 1, 1),(3, 1, 0),(-2, 1, 0)$ y entonces escribimos
\begin{align}
    Q'  = \frac{1}{5}
        \begin{pmatrix}
            1   &3  &   -2 \\
            1   &1  &   1  \\
            1   &0  &   0
        \end{pmatrix}
        \begin{pmatrix}
            0   &0  &   0  \\
            0   &-4 &   0  \\
            0   &0  &   -9
        \end{pmatrix}        
        \begin{pmatrix}
            0   &0  &   5 \\
            1   &2  &   -3  \\
            -1  &3  &   -2
        \end{pmatrix}.
\end{align}

De donde
\begin{align}
    P_t  = \frac{1}{5}
            \begin{pmatrix}
                1   & 3  &   -2 \\
                1   & 1  &   1  \\
                1   & 0  &   0
            \end{pmatrix}
            \begin{pmatrix}
                1   & 0          &   0           \\
                0   & e^{-4t}    &   0           \\
                0   & 0          &   e^{-9t} 
            \end{pmatrix}        
            \begin{pmatrix}
                0   & 0  &   5   \\
                1   & 2  &   -3  \\
                -1  & 3  &   -2
            \end{pmatrix}.
\end{align}

Lo que buscamos es $\P(\rho_2 \leq t) = P_t(2, 3)$, es decir
\begin{align}
    P_t(2, 3)   &=  \frac{1}{5} (1,1,1)\cdot(5, -3e^{-4t}, -2e^{-9t})           \\
                &=  1 - \frac{3}{5}e^{-4t} - \frac{2}{5} e^{-9t}.
\end{align}

	\newpage
	
    \subsection{Inciso (iv)} \label{problema5_5:inciso4}
    \emph{
    Encuentre la proporci\'on de tiempo asint\'otica en que los dos generadores est\'an trabajando. 
    Si cada generador produce 2.5 MW de energ\'ia por unidad de tiempo, ?`Cu\'al es la cantidad promedio 
    de energ\'ia producida a largo plazo por unidad de tiempo?\pn
}

\afterstatement\pn
\end{proof}