\begin{problema}
		Sea $S_n=X_1+\cdots+X_n$ una caminata aleatoria con saltos $X_i\in \{-1,0,1,\ldots\}$. 
		Sea \(C_p\) una variable aleatoria geom\'etrica de par\'ametro $p$ independiente de \(S\) y definimos 
		
		\begin{align}
			M_p=-\min_{n\leq C_p} S_n. 
		\end{align}
		
		El objetivo del ejercicio es determinar la distribuci\'on de $M_p$.

		(A las caminatas aleatorias como $S$ se les ha denominado Skip-free random walks Para aplicaciones de este tipo 
		de procesos, ver \cite{MR1978607}. Tambi\'en aparecen en el estudio de Procesos Galton-Watson. 
		Este ejercicio es el resultado b\'asico del estudio de sus extremos, denominado teor\'ia de fluctuaciones.)

	\begin{enumerate}
		
		\item[(i)] 	
					Sea$$g(\lambda)=E(e^{- \lambda X_1}).$$Pruebe que \(g(\lambda)\in (0,\infty)\) y que$$M_n=e^{-\lambda S_n}g(\lambda)^{-n},n\geq 0$$es una martingala.
		
		\item[(ii)] 
					Pruebe que \(g\) es log-convexa al aplicar la desigualdad de H\"older. Pruebe que si \(P(X_1=-1)>0\) (hip\'otesis que se utilizar\'a desde ahora) 
					entonces \(g(\lambda)\to\infty\) conforme \(\lambda\to\infty\). Utilice esta informaci\'on para esbozar la gr\'afica de \(g\). 
					Defina \( f(s)=\inf \{ \lambda>0:g(\lambda)^{-1} < s\} \). Note que \(1/g\circ f=Id\) en \((0,1)\). Pruebe que si \(g(\lambda)>1\), 
					la martingala \(M\) es acotada hasta el tiempo de arribo de \(S\) a \(-k\) dado por $$ T_k =\min \{n\in\na:S_n=-k\} $$
					(donde se utiliza la convenci\'on \(\inf\emptyset=\infty\) ). Aplique el teorema de muestreo opcional de Doob para mostrar que 
					\begin{align}
						E(s^{T_k})=e^{-k f(s)}.
					\end{align}
					
					Justifique MUY bien por qu\'e la f\'ormula es válida aún cuando \(T_k\) puede tomar el valor \(\infty\) y deduzca que de hecho \(\p (T_k=\infty)=0\).
		
		\item[(iii)] 
					Argumente que$$ P(M_p\geq n)=P(T_n\leq C_p)=E((1-p)^{T_n})$$ para demostrar que \(M_p\) tiene distribuci\'on geom\'etrica de par\'ametro \(1-e^{-f(1-p)}\)
		
		\item[(iv)] 
					Tome el límite conforme \(p\to 0\) para mostrar que la variable aleatoria 
					\begin{align}
						M=-\min_{n\geq 0}S_n
					\end{align}
					tiene una distribuci\'on geom\'etrica de par\'ametro $1-e^{-f(1)}$. Interprete esto cuando $f(1)=0$.
	\end{enumerate}

		\defin{Categor\'ias:} Caminatas aleatorias, muestreo opcional, fluctuaciones.
\end{problema}
