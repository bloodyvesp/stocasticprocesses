\begin{problema}
Sea $P_t$ la probabilidad de transici\'on en $t$ unidades de tiempo para el proceso de Poisson de par\'ametro $\lambda$. 

Al utilizar el teorema del biniomio, pruebe directamente que las probabilidades de transici\'on del proceso de Poisson satisfacen las ecuaciones de Chapman-Kolmogorov $P_{t+s}=P_tP_s$. D\'e adem\'as un argumento probabil\'istico, basado en condicionar con lo que sucede al tiempo $s$, para probar dicha ecuaci\'on. 

Sea\begin{esn}
\imf{Q}{i,j}=\begin{cases}
-\lambda&j=i\\
\lambda&j=i+1\\
0&j\neq i,i+1
\end{cases}.
\end{esn}Pruebe directamente que se satisfacen las ecuaciones de Kolmogorov\begin{equation*}
%\label{CKEquationsForPoisson}
\frac{d}{dt}\imf{P_t}{i,j}=\imf{QP_t}{i,j}=\imf{P_tQ}{i,j},
\end{equation*}donde $QP_t$ es el producto de las matrices $Q$ y $P_t$.
\end{problema}
