\begin{problema}
Sea $M$ una $\paren{\F_n}$-martingala con saltos acotados. Sean\begin{esn}
C=\set{\limsup M_n=\liminf M_n\in\re}\quad\text{y}\quad D=\set{\limsup M_n=-\infty\text{ y }\limsup M_n=\infty}.
\end{esn}Pruebe que $\proba{C\cup D}=1$. Deduzca que las caminatas aleatorias centradas con saltos acotados oscilan. Sugerencia: Para cada $K>0$ defina\begin{esn}
T=\min\set{n\geq 0: \abs{M_n}\geq K}
\end{esn}y aplique el teorema de convergencia de martingalas a $M^T$. 

Sea $M$ una caminata aleatoria no trivial con saltos integrables en $-1,0,1,\ldots$ y  media cero. Pruebe que $\proba{M\text{ converge en }\na}=0$ y  concluya que $\liminf M_n=-\infty$ casi seguramente. (Este resultado permitir\'a dar una prueba adicional de que un Galton-Watson cr\'itico se extingue).  Sugerencia: proceda como en el p\'arrafo anterior y pruebe la integrabilidad uniforme de $M_{T\wedge n},n\in\na$.

\defin{Categor\'ias: } Teoremas de convergencia de martingalas
\end{problema}