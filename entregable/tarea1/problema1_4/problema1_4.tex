\begin{problema}[Extensiones del teorema de paro opcional]
	Sea $M=\paren{M_n,n\in\na}$ una (super)martingala respecto de una filtraci\'on $\paren{\F_n,n\in\na}$ y sean $S$ y $T$ tiempos de paro.
	
	\begin{enumerate}
					\item[(i)]	[\ref{problema1_4:inciso1}] 
						Pruebe que $S\wedge T$, $S+T$ y $S\vee T$ son tiempos de paro.
					
					\item[(ii)]	[\ref{problema1_4:inciso2}]
						\begin{align}
							\F_T=\set{A\in\F:A\cap\set{T\leq n}\in\F_n\text{ para toda } n}
						\end{align}
						es una $\sigma$-\'algebra, a la que nos referimos como la $\sigma$-\'algebra 
						detenida en $\tau$. Comente qu\'e puede fallar si $T$ no es tiempo de paro. 
						Pruebe que $T$ es $F_T$-medible. 
					
					\item[(iii)][\ref{problema1_4:inciso3}] 
						Pruebe que si $T$ es finito, entonces $M_T$ es $\F_T$-medible.
					
					\item[(iv)]	[\ref{problema1_4:inciso4}] 
						Pruebe que si $S\leq T\leq n$ entonces $\F_S\subset\F_T$. Si adem\'as $T$ es acotado entonces $M_S, M_T\in L_1$ y 
						\begin{align}
							\espc{M_T}{\F_S} = M_S &\mbox{\;\;\;(En el caso de que M sea martingala)} \\
							\espc{M_T}{\F_S}\leq M_S &\mbox{\;\;\;(En el caso de que M sea supermartingala)}                 	
						\end{align}

					\item[(v)]	[\ref{problema1_4:inciso5}]
						Si $X=\paren{X_n,n\in\na}$ es un proceso estoc\'astico $\paren{\F_n}$-adaptado y tal que $X_n\in L_1$ y tal que 
						para cualesquiera tiempos de paro acotados $S$ y $T$ se tiene que $\esp{X_S}=\esp{X_T}$ entonces $X$ es una 
						martingala. Sugerencia: considere tiempos de paro de la forma $n\indi{A}+(n+1)\indi{A^c}$ con $A\in\F_n$.
						
					\item[(vi)]	[\ref{problema1_4:inciso6}]
						Pruebe que el proceso $M^T$ obtenido al detener a $M$ al instante $T$ y dado por $M^T_n=M_{T\wedge n}$ es una 
						martingala respecto de $\paren{\F_{T\wedge n},n\geq 0}$ pero tambi\'en respecto de $\paren{\F_{n},n\geq 0}$. 
						Sugerencia: basta probar el resultado respecto de $\paren{\F_n}$ y para esto es \'util el inciso anterior.
	\end{enumerate}

\defin{Categor\'ias: }Tiempos de paro, Muestreo opcional
\end{problema}

\begin{proof}
	\subsection{Inciso (i)} \label{problema1_4:inciso1}
	\emph{
	Pruebe que $ S \wedge T $, $ S + T $ y $ S \vee T$ son tiempos de paro.\\
}
\afterstatement
	\begin{itemize}
		\item 
			Comprobemos que:
			\begin{align}
				\{ S \wedge T \leq n \} = \{ T \leq n \} \cup \{ S \leq n\}.
			\end{align}
			
			Sea $\omega \in \{ S \wedge T \leq n \}$. Entonces $T(\omega) \leq n$ ó $S(\omega) \leq n$.
			Por lo tanto:
			\begin{align}
					\{ S \wedge T \leq n \} \subset \{ T \leq n \} \cup \{ S \leq n\}.							
			\end{align}
			
			Por otro lado, si $\omega \in \{ T \leq n \} \cup \{ S \leq n\}$ entonces $T(\omega) \leq n$ ó $S(\omega) \leq n$.
			En particular, el mínimo tendrá que ser menor que $n$ y por lo tanto:
			\begin{align}
					\{ T \leq n \} \cup \{ S \leq n\} \subset \{ S \wedge T \leq n \}.						
			\end{align}
			
			Por último, $\{ T \leq n \} \in \F_n$ y $\{ S \leq n \} \in \F_n$. Por lo tanto	\\	
			$\{ T \leq n \} \cup \{ S \leq n\} = \{ S \wedge T \leq n \} \in \F_n$ y con esto demostramos que 
			$ S \wedge T$ es tiempo de paro.\\
			
		\item
			Comprobemos que:
			\begin{align}
				\{ S \vee T \leq n \} = \{ T \leq n \} \cap \{ S \leq n\}.
			\end{align}
			
			Sea $\omega \in \{ S \vee T \leq n \}$. Entonces $T(\omega) \leq n$ y $S(\omega) \leq n$.
			Por lo tanto:
			\begin{align}
					\{ S \vee T \leq n \} \subset \{ T \leq n \} \cap \{ S \leq n\}.							
			\end{align}
			
			Por otro lado, si $\omega \in \{ T \leq n \} \cap \{ S \leq n\}$ entonces $T(\omega) \leq n$ y $S(\omega) \leq n$.
			En particular, el máximo tendrá que ser menor que $n$ y por lo tanto:
			\begin{align}
					\{ T \leq n \} \cap \{ S \leq n\} \subset \{ S \vee T \leq n \}.						
			\end{align}
			
			Por último, $\{ T \leq n \} \in \F_n$ y $\{ S \leq n \} \in \F_n$. Por lo tanto	\\	
			$\{ T \leq n \} \cap \{ S \leq n\} = \{ S \vee T \leq n \} \in \F_n$ y con esto demostramos que 
			$ S \vee T$ es tiempo de paro.\\
			
		\item
			Comprobemos que:
			\begin{align}
				\{ S + T = n \} = \bigcup_{i = 1}^{n-1} \bigg( \{S = n - i\} \cap \{T = i\} \bigg).
			\end{align}
			
			Si $\omega \in \{ S + T = n \}$, entonces $S(\omega) + T(\omega) = n$, como $S$ y $T$ son positivas,
			entonces $1 \leq S(\omega) \leq n-1$. Entonces basta elegir $i = S(\omega)$ para afirmar que 
			$\omega \in \{S = n - i\} \cap \{T = i\}$ y por lo tanto
			\begin{align}
			\{ S + T = n \} \subset \bigcup_{i = 1}^{n-1} \bigg( \{S = n - i\} \cap \{T = i\} \bigg).
			\end{align}
			
			Por otro lado, si $\omega \in \bigcup_{i = 1}^{n-1} \bigg( \{S = n - i\} \cap \{T = i\} \bigg)$
			significa que existe un $i$ tal que $1\leq i \leq n-1$ y $\omega \in \{S = n - i\} \cap \{T = i\}$.
			Y por lo tanto: $S(\omega) = n-i$ y $T(\omega) = i$. De donde $(T + S)(\omega) = n$.
			
			De aquí que
			\begin{align}
				\bigcup_{i = 1}^{n-1} \bigg( \{S = n - i\} \cap \{T = i\} \bigg) \subset \{ S + T = n \}. 
			\end{align}
			
			Aohra, para cada $i$ tal que $1 \leq i \leq n-1$, $\{S = n - i\} \cap \{T = i\} \in \F_n$.
			Por lo tanto $\{ S + T = n\} \in \F_n$ y con esto queda demostrado que $ S + T $ es tiempo de paro.
	\end{itemize}
	
	\newpage
	
	\subsection{Inciso (ii)}\label{problema1_4:inciso2}
	\emph{
	\begin{align}
		\F_T=\set{A\in\F:A\cap\set{T\leq n}\in\F_n\text{ para toda } n}
	\end{align}
	es una $\sigma$-\'algebra, 
	a la que nos referimos como la $\sigma$-\'algebra detenida en $\tau$. Comente qu\'e puede fallar si $T$ no es tiempo de paro. 
	Pruebe que $T$ es $\F_T$-medible.\\		
}			
\afterstatement		
	Primero hay que demostrar que $\F_T$ es $\sigma$-algebra.\pn
	
	\begin{itemize}
		\item $\Omega \in \F_T$.\pn
		
			Notemos que 
            
			\begin{align}
				\Omega \cap \{T \leq n\} = \{T \leq n\} \in \F_n.
			\end{align}\pn
            
			Donde la pertenencia a $\F_n$ es gracias a que $T$ es tiempo de paro. (Esta parte podria fallar si 
			$T$ no fuera tiempo de paro).\pn
		
		\item $\F_T$ es cerrado bajo complementación.\pn
		
			Sea $A \in \F_T$. Eso significa que para todo $n \in  \N$ ocurre que $B = A \cap \{ T \leq n \} \in \F_n$. 
			Por ser $\F_n$ una $\sigma$-algebra tenemos que el complemento de $B$ también debe estar en $\F_n$. 
			Escrito en símbolos:
			
			\begin{align}
				B^c 	&= (A   \cap \{ T \leq n \})^c      \\
						&=  A^c \cup \{ T > n \} \in \F_n
			\end{align}
			
			Dado que $B^c$ y $\{ T \leq n \}$ se encuentran en $\F_n$, también \\
			$B^c \cap \{ T \leq n \} \cap \{ T \leq n \}$.\pn
			
			Escribiendo esto último de otra manera:
            
			\begin{align}
				B^c \cap \{ T \leq n \} \cap \{ T \leq n \} 	&=		A^c \cup \{ T > n \} \cap \{ T \leq n \} \cap \{ T \leq n \} \\
																&= 		A^c \cap \{ T \leq n \} \cup \{ T > n \} \cap \{ T \leq n \} \\
																&= 		\bigg(A^c \cap \{ T \leq n \}\bigg) 
																			\bigcup 
																		\bigg(\{ T > n \} \cap \{ T \leq n \}\bigg) \\
																&=		\bigg(A^c \cap \{ T \leq n \}\bigg)	\bigcup \emptyset \\
																&=		A^c \cap \{ T \leq n \} \in \F_n
			\end{align}
			
			Y por lo tanto $A^c \in \F_T$.\pn
			
		\item $\F_T$ es cerrado bajo uniones numerables.
		
			Sean $A_m \in \F_T$ para $m \in \N$.
			\begin{align}
				\bigg( \bigcup_{k=1}^\infty A_k \bigg) \cap \{ T \leq n \} 	&=	\bigcup_{k=1}^\infty \bigg( A_k \cap \{ T \leq n \} \bigg)
			\end{align}								
			
			La última parte, por definición de $\F_T$ es una unión numerable de elementos de $\F_n$ y por lo tanto, dicha unión
			tambien pertenece a $\F_n$. Y por último $\cup A_k \in \F_T$.
	\end{itemize}\pn
	
	Ahora veamos que $T$ es $\F_T$-medible. Dado un $k \in \N$, ocurre que:
	
	\begin{align}
		\{T \leq k\} \cap \{T \leq n\} = \{T \leq min(k,n)\} \in \F_{min(k, n)} \subset \F_n \forall n \in \N.
	\end{align}

	Y por lo tanto $\{T \leq k\} \in \F_T$.
	\newpage
		
	\subsection{Inciso (iii)}\label{problema1_4:inciso3}
	\emph{
	Pruebe que si $T$ es finito, entonces $M_T$ es $\F_T$-medible.
}

\afterstatement\pn

	Sea $A \in \B(\R)$. Por ser $T$ finito, tenemos que $\{ T \in \N \} = \Omega$. Entonces
	podemos descomponer a $\{ M_T \in A \}$ como:
	
	\begin{align}
		\{ M_T \in A \}		&=		\bigcup_{i \in \N} \bigg( \{ M_T \in A \} \cap \{ T = i \} \bigg)\\
							&=		\bigcup_{i \in \N} \bigg( \{ M_i \in A \} \cap \{ T = i \} \bigg).
	\end{align}

	Ahora sea $n \in \N$. 
	
	\begin{align}
		\{ M_T \in A \} \cap \{ T \leq n \}		&=		\bigcup_{i \in \N} \bigg( \{ M_i \in A \} \cap \{ T = i \} \bigg) \cap \{ T \leq n \} \\
												&=		\bigcup_{i \leq n} \bigg( \{ M_i \in A \} \cap \{ T = i \} \bigg).
	\end{align}
	
	Por ser $M_i$ martingala con respecto a la filtración $(\F_n)_{n \in \N}$ y $T$ tiempo de paro. Tenemos que $\{ T = i\} \in \F_i$ y
	$\{ M_i \in A \} \in \F_i$. Por lo tanto $\{ T = i\} \cap \{ M_i \in A \} \in \F_i \subset \F_n$ para todo $i \leq n$. Por lo que
	$\{ M_T \in A \} \cap \{ T \leq n \} \in \F_n$. Como esto es cierto para toda $n \in \N$, tenemos que $\{ M_T \in A \} \in \F_T$. Con
	lo que terminamos de demostrar que $M_T$ es $\F_T$-medible.\pn
	
	\newpage
	
	\subsection{Inciso (iv)}\label{problema1_4:inciso4}
	\emph{
	Pruebe que si $S\leq T\leq n$ entonces $\F_S\subset\F_T$. Si adem\'as $T$ es acotado entonces $M_S,M_T \in L_1$ y 
	\begin{align}
		\espc{M_T}{\F_S} = M_S &\mbox{\;\;\;(En el caso de que M sea martingala)} \label{problema1_4:caso_martingala}\\
		\espc{M_T}{\F_S}\leq M_S &\mbox{\;\;\;(En el caso de que M sea supermartingala)} \label{problema1_4:caso_supermartingala}
	\end{align}	
}

	Primero probaremos que $\F_S \subset \F_T$. Sea $n \in \N$. Notemos que $\{ T \leq n \} \subset \{ S \leq n \}$ pues
	si $\omega \in \{ T \leq n \}$, entonces $S(\omega) \leq T(\omega) \leq n $ y por lo tanto $\omega \in \{ S \leq n \}$.\\

	Ahora sea $A \in \F_S$. Entonces

	\begin{align}
		A \cap \{ T \leq n \} 	&=		A  \cap \{ T \leq n \} \cap \{ S \leq n \} \\
								&=		(A  \cap \{ S \leq n \}) \cap \{ T \leq n \} \in \F_n
	\end{align}

	Como escogimos $n$ arbitrariamente, esto es cierto para toda $n \in \N$. Y por lo tanto $A \in \F_T$ y $\F_S \subset \F_T$.\\

	Ahora supongamos que $T$ es acotado y sea $N$ una cota para $T$. Demostraremos que $M_S,M_T \in L_1$.
	
	\begin{align}
		\E(|M_T|) 	&=		\E\bigg(	\bigg| \sum_{i \in \N} M_i \cdot \indic_{T = i} \bigg|	\bigg) \\ 
					&\leq	\E\bigg(	\sum_{i \in \N} |M_i \cdot \indic_{T = i} |\bigg) \\
					&\leq	\E\bigg(	\sum_{i \in \N} |M_i|\bigg) \\
					&=		\E\bigg(	\sum_{i \leq N} |M_i|\bigg) \\
					&< 		\infty
	\end{align}
	
	Y con esto queda demostrado que $M_T \in L_1$. La demostración de que $M_S \in L_1$ es completamente análoga.\\
	
	Nos queda por demostrar (\ref{problema1_4:caso_martingala}) y (\ref{problema1_4:caso_supermartingala}).
	\newpage
	
	\subsection{Inciso (v)} \label{problema1_4:inciso5}
	\emph{
	Si $X=\paren{X_n,n\in\N}$ es un proceso estoc\'astico $\paren{\F_n}$-adaptado y tal que $X_n\in L_1$ y tal que 
	para cualesquiera tiempos de paro acotados $S$ y $T$ se tiene que $\esp{X_S}=\esp{X_T}$ entonces $X$ es una 
	martingala. Sugerencia: considere tiempos de paro de la forma $n\indi{A}+(n+1)\indi{A^c}$ con $A\in\F_n$.\\
}
\afterstatement
	Ya tenemos dos de las hipótesis necesarias para demostrar que algo es martingala.
\begin{enumerate}
	\item 
		$(X_n)_{n\in\N}$ es $(\F_n)_{n \in \N}$-adaptado.

	\item 
		$X_n \in L_1$ para toda $n \in \N$. 
\end{enumerate}
\null

	Sólo falta demostrar que $\espc{X_{n+1}}{\F_n} = X_n$ para toda $n \in \N$.\\

	Es decir, dada una $n \in \N$, queremos demostar que para todo $A \in \F_n$,
ocurre que $\E(X_{n+1} \cdot \indic_{A}) = \E(X_{n} \cdot \indic_{A})$.\\

	Sean entonces $n \in \N$ y $A \in \F_n$. Definimos $T_{A,n} = n\cdot\indic_{A} + (n+1) \cdot \indic_{A^c}$. 
Veamos que $T_{A, n}$ es tiempo de paro.\\

	Sea $k \in \N$.

\begin{itemize}
	\item 
		Si $k = n$, entonces 
		\begin{align}
			\{T_{A, n} = k\} 	&= \{T_{A, n} = n\} \\
								&= A \in F_n = F_k.
		\end{align}
		
	\item
		Si $k = n + 1$, entonces 
		\begin{align}
			\{T_{A, n} = k\} 	&= \{T_{A, n} = n + 1\} \\
								&= A^c \in F_n \subset \F_{n+1} = F_k.
		\end{align}

	\item 
		Si $k \not= n$ y $k \not= n+1$, entonces
		\begin{align}
			\{T_{A, n} = k\} 	&= \emptyset \in \F_k.
		\end{align}
\end{itemize}

	Entonces, $\{ T_{A,n} = k\} \in \F_k$ para toda $k \in \N$ y por lo tanto $T_{A,n}$ es tiempo de paro.\\

    Ahora notemos lo siguiente
    \begin{align}
    	X_{T_{A,n}} 	&=		\sum_{i \in \N} X_i \cdot  \indic_{\{T=i\}}						\\
    					&=		X_n \cdot  \indic_{\{T=n\}} + X_{n+1} \cdot  \indic_{\{T=n+1\}}	\\
    					&=		X_n \cdot  \indic_{A} + X_{n+1} \cdot  \indic_{A^c}.
    \end{align}
    
    Eso quiere decir que
    \begin{align}
    	\E(X_{T_{A,n}}) 	&=	\E(X_n \cdot  \indic_{A} + X_{n+1} \cdot  \indic_{A^c})		\\
							&=	\E(X_n \cdot  \indic_{A}) + \E(X_{n+1} \cdot  \indic_{A^c}).
    \end{align}
    
    Ahora tomemos la función constante $S_n = n+1$. Por ser constante \\
    $\{S_n = k\} \in \{ \Omega, \emptyset\} \subset \F_k$ para toda $k \in \N$
    y por lo tanto $S_n$ es tiempo de paro.\\
    
    Notemos entonces que
    \begin{align}
    	\E(X_{S_n}) 	&= 		\E(X_{n+1}) 													\\
    					&=		\E(X_{n+1} \cdot \indic_A + X_{n+1} \cdot \indic_{A^c}) 		\\
    					&=		\E(X_{n+1} \cdot \indic_A) +\E( X_{n+1} \cdot \indic_{A^c}). 
    \end{align}
    
    Dado que tanto $T_{a,n}$ como $S_n$ son variables que toman una cantidad finita de valores y por lo tanto son acotadas, 
    nuestra hipótesis dice que \\
    $\E(X_{S_n}) = \E(X_{T_{A,n}})$. \\
    
    Entonces
    \begin{align}
		\E(X_{n+1} \cdot \indic_A) +\E( X_{n+1} \cdot \indic_{A^c}) &=  \E(X_n \cdot  \indic_{A}) + \E(X_{n+1} \cdot  \indic_{A^c}).
    \end{align}
    
    De donde
    \begin{align}
			\E(X_{n+1} \cdot \indic_A) &=	\E(X_n \cdot  \indic_{A}).
    \end{align}
    
    Como esto lo hicimos para $n \in \N$ y $A \in \F_n$ arbitrarios, con esto terminamos la demostración de que $(X_n)_{n \in \N}$ es
    martingala.
	\newpage
	
	\subsection{Inciso (vi)}\label{problema1_4:inciso6}
	\emph{
	Pruebe que el proceso $M^T$ obtenido al detener a $M$ al instante $T$ y dado por $M^T_n=M_{T\wedge n}$ es una 
	martingala respecto de $\paren{\F_{T\wedge n},n\geq 0}$ pero tambi\'en respecto de $\paren{\F_{n},n\geq 0}$. 
	Sugerencia: basta probar el resultado respecto de $\paren{\F_n}$ y para esto es \'util el inciso anterior.
}
\afterstatement
	Es claro que $M^T$ es $(\F_n)_{n \in \N}$-adaptado.\\

	Pues
	\begin{align}
		M_n^T 		&= 		M_{n \wedge T} 									\\
					&= 		\sum_{i \in \N} M_i \cdot \indic_{n \wedge T = i}		\\
					&=      \sum_{i \leq n} M_i \cdot \indic_{T = i}.
	\end{align}		

	Que resulta ser suma finita de variables que son $F_n$-medibles y por lo tanto,
	es $\F_n$-medible.	\\
	
	De aquí mismo podemos notar que $M_n^T$ es suma finita de variables en $L_1$ y por
	lo tanto, pertenece a $L_1$.\\
	
	Ahora, sean $S$ y $U$ dos tiempos de paro acotados.\\
	
	Por [\ref{problema1_4:inciso1}] sabemos que $S \wedge T$ y $S \wedge U$ también son tiempos de paro.\\
	
	Además por ser $S$ y $U$ acotados, $S \wedge T$ y $S \wedge U$ son acotados también.\\

	Sabiendo que $M$ es martingala, con el teorema del muestreo opcional de Doob, podemos notar que
	
	\begin{align}
		\E(M_S^T) 	&=	\E(M_{S \wedge T}) \\
					&= \E(M_0)
	\end{align}
	
	y
	
	\begin{align}
		\E(M_U^T) 	&=	\E(M_{U \wedge T}) \\
					&= \E(M_0).
	\end{align}
	
	Por lo tanto, para cualesquiera dos tiempo de paro acotados $S$ y $U$ tenemos que $\E(M_S^T) = \E(M_U^T)$.
	Entonces, por lo visto en [\ref{problema1_4:inciso5}], $M^T$ es una martingala con respecto de la filtración
	$(F_n)_{n \in \N}$.\\
	
	Para probar que $M_T$ es una martingala con respecto a $(\F_{T \wedge n})_{n \in \N}$, primero notemos que
	$(\F_{T \wedge n})_{n \in \N}$ es efectivamente una filtración.\\
	
	Dado $n \in \N$, gracias a [\ref{problema1_4:inciso2}], tenemos que $\F_{T \wedge n}$ es una $\sigma$-álgebra.\\

	Además tenemos que $T \wedge n \leq T \wedge (n+1)$. El resultado de [\ref{problema1_4:inciso4}] implica que 
	$\F_{T \wedge n} \subset \F_{T \wedge (n+1)}$ y por lo tanto $(\F_{T \wedge n})_{n \in \N}$ sí es una filtración.\\
	
	Ahora veamos que $M^T$ es adaptado a la filtración $(\F_{T \wedge n})_{n \in \N}$.\\

	Dado $n \in \N$ como $T \wedge n$ es acotado, por [\ref{problema1_4:inciso3}] $M_n^T = M_{T \wedge n}$ es $\F_{T \wedge n}$-medible.
	Y con esto tenemos que $M^T$ sí es adaptado a la filtración $(\F_{T \wedge n})_{n \in \N}$.\\
	
	Ya sabíamos que $M_n^T$ pertenecía a $L_1$.\\
	
	Gracias [\ref{problema1_4:inciso5}] y a la primera parte que se demostró en este ejercicio, se sigue que $M_T$ también es
	martingala con respecto a la filtración $(\F_{T \wedge n})_{n \in \N}$.\\
\end{proof}