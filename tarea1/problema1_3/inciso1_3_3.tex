\emph{
	Sea $M_n$ la martingala obtenida al detener a $-S$ en $T_1$. Utilice la solución al
	Problema de la Ruina para probar que $\mw(max_n M_n \geq M) = 1/(M+1)$ para todo $M \geq 1$. Concluya que
	$\E(max_m M_n) = \infty$ y que por lo tanto $\E(max_{m\leq n} M_m) \rightarrow \infty$ conforme 
	$n \rightarrow \infty$. Finalmente, deduzca que no puede haber una desigualdad de tipo Doob cuando $p=1$.\\
}	

	Definimos $M_n = -S_{T_1 \wedge n}$. Notemos que $M_n$ únicamente toma valores en $[-1, \infty]$.
	Para calcular $\mw(max_n M_n \geq M)$ notemos primero que:
	\begin{align}
		\mw(max_n M_n \geq M) = 1 - \mw(max_n M_n < M).
	\end{align}\\
	
	$max_n M_n < M$ significa que $M_n$ nunca alcanza el valor $M$.\\
	 
	Intentando hacer analogía con el problema de la ruina, pensemos en dos concursantes,
	uno con $1$ peso y otro con $M$ pesos. Nunca alcanzar $M$ significa que nunca gana el que tiene $1$ peso.\\
	
	Esta probabilidad ya la conocemos y es 
	
	\begin{align*}
		\mw(max_n M_n < M) = \frac{M}{M + 1}
	\end{align*}
		
	Por lo tanto
	
	\begin{align}
		\mw(max_n M_n \geq M) 	&= 1 - \mw(max_n M_n < M) \\
								&= 1 - \frac{M}{M + 1}\\
								&= \frac{M+1}{M+1} - \frac{M}{M + 1}\\
								&= \frac{1}{M+1}
	\end{align}
	
	Utilizando este resultado:
	\begin{align} \label{problema1_3:esperanza_del_maximo_de_M_n}
		\E(max_n M_n) 	&= - \mw(max_n M_n = -1) + \sum_{M=1}^{\infty} \mw(max_n M_n \geq M) \\
						&= - \mw(max_n M_n = -1) + \sum_{M=1}^{\infty} \frac{1}{M+1} \\ 
						&= - \mw(max_n M_n = -1) + \infty \\
						&= \infty
	\end{align}						
	
	Ahora, tenemos que:
	\begin{align}
		\|\overline{M_{n}^{+}}\|_1  &=    \E{\overline{M_{n}^{+}}} \\
									&=    \E{\max_{m \leq n}M_m^+} \\
									&\geq \E{\max_{m \leq n}M_m}										
	\end{align}
		
	Donde, el último término, tiende a infinito en base al resultado 
	(\ref{problema1_3:esperanza_del_maximo_de_M_n}).

	Por otro lado:
	\begin{align}
		\|M_n^+\|_1=\|-S_{T_{1\wedge n}}^{+}\|_1  \longrightarrow  \|-S_{T_1}^+\|_1 = 0 < \infty
	\end{align}
	
	Por lo tanto, no existe número $K$, tal que
	\begin{align}
		 \|\overline{M_n^+}\|_1 \leq  K \|M_n^+\|_1
	\end{align}
	
	En otras palabras, no tenemos una desigualdad de tipo Doob para $p=1$.\\
	