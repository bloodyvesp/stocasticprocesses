\begin{problema}
	Sea
	\begin{esn}
		M^\lambda_t=e^{\lambda B_t-\lambda^2t/2}.
	\end{esn}
		\begin{enumerate}
			\item[(i)]      [\ref{problema6_4:inciso1}]
                Explique y pruebe formalmente por qu\'e, para toda 
                $n\geq 1$, $\partial^n M^\lambda_t/\partial \lambda^n$ es una martingala.\pn
                
			\item[(ii)]     [\ref{problema6_4:inciso2}]
                Sea $\imf{H_n}{x}=\paren{-1}^ne^{x^2/2}\frac{d^n}{dx^n}e^{-x^2/2}$.
                A $H_n$ se le conoce como en\'esimo polinomio de Hermite. Calc\'ulelo para 
                $n\leq 5$. Pruebe que $H_n$ es un polinomio para toda $n\in\na$ y que 
                $\partial^n M^\lambda_t/\partial \lambda^n=t^{n/2}\imf{H_n}{B_t/\sqrt{t}}M^\lambda_t$.\pn
                
			\item[(iii)]    [\ref{problema6_4:inciso3}]
                Pruebe que $t^{n/2}\imf{H_n}{B_t/\sqrt{t}}$ es una martingala para 
                toda $n$ y calc\'ulela para $n\leq 5$.\pn
                
			\item[(iv)]     [\ref{problema6_4:inciso4}]
                Aplique muestreo opcional a las martingalas anteriores al tiempo 
                aleatorio $T_{a,b}=\min\set{t\geq 0:B_t\in\set{-a,b}}$ (para $a,b>0$) 
                con $n=1,2$ para calcular $\proba{B_{T_{a,b}}=b}$ y $\esp{T_{a,b}}$,
                ËQu\'e concluye cuando $n=3,4$? ?` Cree que $T_{a,b}$ tenga momentos 
                finitos de cualquier orden? Justifique su respuesta.\pn
			
            \item[(v)]      [\ref{problema6_4:inciso5}]
                Aplique el teorema de muestreo opcional a la martingala 
                $M^\lambda $ al tiempo aleatorio $T_a=\inf\set{t\geq 0:B_t\geq a}$ 
                si $\lambda>0$. Diga por qu\'e es necesaria la \'ultima 
                hip\'otesis y calcule la transformada de Laplace de $T_a$. 
			%\item 
                %Opcional (para subir calificaci\'on en esta u otra tarea): 
			%\begin{enumerate}
				%\item Modifique el ejercicio para que aplique al proceso Poisson.
				%\item Resu\'elva el ejercicio modificado. 
			%\end{enumerate}
		\end{enumerate}
\end{problema}

\begin{proof}
    \subsection{Inciso (i)}     \label{problema6_4:inciso1}
    \emph{
    Explique y pruebe formalmente por qu\'e, para toda 
    $n\geq 1$, $\partial^n M^\lambda_t/\partial \lambda^n$ es una martingala.\pn
}

\afterstatement\pn

Analicemos las primeras derivadas.

\begin{align}
    \frac{d}{d\lambda}  M_t^\lambda &=  \frac{d}{d\lambda} \left( e^{\lambda B_t - \lambda^2 t/2} \right)                       \\
                                    &=  \frac{d}{d\lambda} \left( e^{\lambda B_t} e^{- \lambda^2 t/2} \right)                   \\
                                    &=  B_t e^{\lambda B_t} e^{- \lambda^2/2}  - \lambda t e^{\lambda B_t} e^{- \lambda^2 t/2}  \\
                                    &=  B_t M_t^\lambda - \lambda t M_t^\lambda                                                 \\
                                    &=  M_t^\lambda (B_t - \lambda t).                                                          \\
\end{align}

La de segundo orden
\begin{align}
    \frac{d^2}{d\lambda^2}  M_t^\lambda     &=  \frac{d}{d\lambda} \left( M_t^\lambda (B_t - \lambda t) \right)                     \\
                                            &=  M_t^\lambda (B_t - \lambda t) (B_t - \lambda t) -  t M_t^\lambda                    \\ 
                                            &=  M_t^\lambda \left[ (B_t - \lambda t)^2 -  t \right]                                 \\    
                                            &=  t M_t^\lambda \left[ (B_t - \lambda t)^2 (t^{-1/2})^2 -  1 \right]                       
\end{align}\pn

La de tercer orden
\begin{align}
    &   \;\;\;\;\frac{d^3}{d\lambda^3}  M_t^\lambda                                                                                                                                                         \\
    &=  \frac{d}{d\lambda} \left(t M_t^\lambda  \left[ (B_t - \lambda t)^2 (t^{-1/2})^2 -  1 \right] \right)                                                                                                \\ 
    &=  t \frac{d}{d\lambda} \left( M_t^\lambda  \left[ (B_t - \lambda t)^2 (t^{-1/2})^2 -  1 \right] \right)                                                                                               \\
    &=  t   M_t^\lambda \left[ (B_t - \lambda t)^3 (t^{-1/2})^2 -  (B_t - \lambda t) \right]    +   M_t^\lambda \frac{d}{d\lambda} \left[ (B_t - \lambda t)^2 (t^{-1/2})^2 -  1 \right]                  \\
    &=  t   M_t^\lambda \left(\left[ (B_t - \lambda t)^3 (t^{-1/2})^2 -  (B_t - \lambda t) \right]  +  \frac{d}{d\lambda} \left[ (B_t - \lambda t)^2 (t^{-1/2})^2 -  1 \right]\right)                  \\
    &=  t   M_t^\lambda \left(\left[ (B_t - \lambda t)^3 (t^{-1/2})^2 -  (B_t - \lambda t) \right]  +  \left[ -2t(B_t - \lambda t) (t^{-1/2})^2 \right]\right)                  \\
    &=  t   M_t^\lambda \left(\left[ (B_t - \lambda t)^3 (t^{-1/2})^2 -  (B_t - \lambda t) \right]  +  \left[ -2(B_t - \lambda t) \right]\right)                  \\
    &=  t   M_t^\lambda \left[ (B_t - \lambda t)^3 (t^{-1/2})^2 -  3(B_t - \lambda t) \right]                    \\
    &=  t \, t^{1/2} t^{-1/2}  M_t^\lambda \left[ (B_t - \lambda t)^3 (t^{-1/2})^2 -  3(B_t - \lambda t) \right]                   \\
    &=  t^{3/2}   M_t^\lambda \left[ (B_t - \lambda t)^3 (t^{-1/2})^3 -  3 (t^{-1/2})(B_t - \lambda t) \right]                    \\
\end{align}\pn

(Estamos haciendo un poquito de trampa para facilitar el inciso siguiente). Piénsese en el factor 
que se encuentra entre corchetes como un polinomio en $(B_t - \lambda t) (t^{-1/2})$ (en la parte 
anterior, se trata de un polinomio de grado $3$, con sólo los términos de grado $3$ y $1$, donde 
los coeficientes son $1$ y $3$ respectivamente).\pn

Para facilitar la notación, definimos $B_t^\lambda = (B_t - \lambda t) (t^{-1/2})$ y sólo hay 
que tener presente que $\frac{d}{d\lambda} B_t^\lambda = -t^{1/2}$ y en general, que 
$\frac{d}{d\lambda} (B_t^\lambda)^n = -t^{1/2} n (B_t^\lambda)^{n-1}$.\pn

Con esta nueva notación, tenemos
\begin{align}
    \frac{d}{d\lambda}  M_t^\lambda     &= t^{1/2} M_t^\lambda \left[ B_t^\lambda \right]                       \\
    \frac{d^2}{d\lambda^2}  M_t^\lambda &= t M_t^\lambda \left[ (B_t^\lambda)^2 - 1  \right]                    \\
    \frac{d^3}{d\lambda^3}  M_t^\lambda &= t^{3/2} M_t^\lambda \left[ (B_t^\lambda)^3 - 3 B_t^\lambda  \right]  \\
\end{align}\pn

Hagamos la siguiente hipótesis de inducción. Para cierta $n \geq 1$, se tiene que 
\begin{align}
    \frac{d^n}{d\lambda^n} M_t^\lambda = t^{n/2} M_t^\lambda \left[ r_n (B_t^\lambda)^n + r_{n-1} (B_t^\lambda)^{n-1} + \dots + r_1 B_t^\lambda + r_0 \right]
\end{align}
Donde $r_0, r_1, \dots, r_n \in \R$ y $r_n \neq 0$.\pn

Derivemos una vez más para ver qué ocurre con la derivada de orden $n+1$
\tiny
\begin{align}
    & \;\;\;\;\;\frac{d^{n+1}}{d\lambda^{n+1}} M_t^\lambda                                                                                                                                                                                      \\
    &= t^{n/2} \frac{d}{d \lambda} \left( M_t^\lambda \left[ r_n (B_t^\lambda)^n + \dots + r_1 B_t^\lambda + r_0 \right] \right)                                                                                                                \\
    &= t^{n/2}  M_t^\lambda \left( t^{1/2} \left[ r_n (B_t^\lambda)^{n+1}  + \dots + r_1 (B_t^\lambda)^2 + r_0 B_t^\lambda \right] -t^{1/2} \left[ r_n n (B_t^\lambda)^{n-1} + \dots + r_1 \right]\right)                                       \\
    &= t^{(n+1)/2}  M_t^\lambda \left( \left[ r_n (B_t^\lambda)^{n+1}  + \dots + r_1 (B_t^\lambda)^2 + r_0 B_t^\lambda \right] - \left[ r_n n (B_t^\lambda)^{n-1} + \dots + r_1 \right]\right)                                                  \\
    &= t^{(n+1)/2}  M_t^\lambda \left( \left[ r_n (B_t^\lambda)^{n+1}  + \dots + (r_{m-2} - r_{m})(B_t^\lambda)^{m-1} + \dots + r_1 (B_t^\lambda)^2 + r_0 B_t^\lambda \right]\right) 
\end{align}\pn
\normalsize

En resumen, tenemos la siguiente relación recursiva:
\tiny
\begin{align}\label{problema6_4:equivalencia_recursiva}
    &\;\;\;\;\;\frac{d^{n+1}}{d\lambda^{n+1}} M_t^\lambda     \\
    &= t^{(n+1)/2}  M_t^\lambda \left( \left[ r_n (B_t^\lambda)^{n+1}  + \dots + (r_{m-2} - r_{m})(B_t^\lambda)^{m-1} + \dots + r_1 (B_t^\lambda)^2 + r_0 B_t^\lambda \right]\right) 
\end{align}\pn
\normalsize

De esta fórmula recursiva, vemos que basta que $M_t^\lambda (B_t)^m$ sea integrable (para toda $m$, es decir, cada sumando de la parte de arriba) para que 
$\frac{d^{n}}{d\lambda^{n}} M_t^\lambda$ lo sea. Veamos que para que $M_t^\lambda (B_t)^m = e^{\lambda B_t}e^{-\lambda^2t/2} (B_t)^m$ sea integrable,
basta que $e^{\lambda B_t} (B_t)^m$ lo sea.\pn

Recordemos entonces que $B_t$ es una variable aleatoria con distribución normal de media 0 y varianza 1, entonces, tiene momentos
de todos los ordenes y además su función generadora de momentos $\E(e^{\alpha B_t}) = e^{\frac{1}{2} \alpha^2}$ 
(véase \href{https://en.wikipedia.org/wiki/Normal_distribution#Moment_and_cumulant_generating_functions}{Normal distribution : Moment and culminant generating functions})
es finita para toda $\alpha$. Haciendo la elección $\alpha = \lambda + 1$ tenemos\pn

\begin{align}
        \E(e^{(\lambda + 1) B_t})   &=  \E(e^{\lambda B_t} e^{B_t}) \\
                                    &=  \E\left(e^{\lambda B_t} \sum \frac{(B_t)^m}{m!}\right) \\
                                    &=  \sum \E\left(e^{\lambda B_t} \frac{(B_t)^m}{m!}\right) \\
                                    &<  \infty.
\end{align}\pn

De donde se tiene que cada sumando tiene que ser finito y por lo tanto las derivadas de todos los órdenes son integrables.\pn

El mismo argumento aplica para la integrabilidad de $e^{\abs{\lambda B_t}} \abs{B_t}^m$.
Entonces, los valores absolutos de las derivadas de cada orden están acotadas por una función integrable 
(es decir la función $e^{\abs{\lambda B_t}} \abs{B_t}^m$).\pn

Existe un resultado que nos permite intercambiar los operadores derivada y esperanza condicional si el valor absoluto de la derivada de la
función está dominado por una función integrable. Entonces ahora podemos hacer:
\begin{align}
        \E\left(\frac{d^n}{d \lambda^n} M_t^\lambda \bigg| \F_{s}\right)   &=   \frac{d}{d \lambda} \E\left(\frac{d^{n-1}}{d \lambda^{n-1}} M_t^\lambda \bigg| \F_{s}\right)
\end{align}\pn
Donde $s<t$ y $(\F_s)_{s \geq 0}$ es la filtración generada por $\sigma(B_t : t\leq s)$.\pn

Suponiendo la hipótesis de inducción
\begin{align}
    \E\left(\frac{d^{n-1}}{d \lambda^{n-1}} M_t^\lambda \bigg| \F_{s}\right) &= \frac{d^{n-1}}{d \lambda^{n-1}} M_s^\lambda
\end{align}

tendremos que
\begin{align}
    \E\left(\frac{d^n}{d \lambda^n} M_t^\lambda \bigg| \F_{s}\right)   &=    \frac{d^{n}}{d \lambda^{n}} M_s^\lambda
\end{align}\pn

Dado que las derivdas de todos los ordenes son integrables y que las derivadas de $M_t^\lambda$ son límite de funciones 
$\F_t$-medibles (y por lo tanto $\F_t$-medibles), basta demostrar la base de inducción para concluir que se trata de una martingala.\pn

La base de inducción la haremos sobre $n = 0$, entonces sean $s < t$
\begin{align}
        \E\left(\frac{d^0}{d \lambda^0} M_t^\lambda \bigg| \F_s \right)  &=  \E\left( M_t^\lambda \bigg| \F_s \right)                                               \\
                                                                    &=  \E\left( e^{-\lambda^2t/2} e^{\lambda B_t} \bigg| \F_s \right)                              \\
                                                                    &=  e^{-\lambda^2 t/2} \E\left( e^{\lambda B_t} \bigg| \F_s \right)                             \\
                                                                    &=  e^{-\lambda^2 t/2} \E\left( e^{\lambda (B_t - B_s + B_s)} \bigg| \F_s \right)               \\
                                                                    &=  e^{-\lambda^2 t/2} \E\left( e^{\lambda (B_t - B_s)} e^{\lambda B_s} \bigg| \F_s \right)     \\
                                                                    &=  e^{-\lambda^2 t/2} e^{\lambda B_s} \E\left( e^{\lambda (B_t - B_s)}  \bigg| \F_s \right)    \\
                                                                    &=  e^{-\lambda^2 t/2} e^{\lambda B_s} \E\left( e^{\lambda B_{t-s}}  \bigg| \F_s \right)        \\
                                                                    &\comment{los incrementos son estacionarios}                                                    \\
                                                                    &=  e^{-\lambda^2 t/2} e^{\lambda B_s)} \E\left( e^{\lambda B_{t-s}} \right)                    \\
                                                                    &\comment{los incrementos son independientes}                                                   \\
                                                                    &=  e^{-\lambda^2 t/2} e^{\lambda B_s} e^{\lambda^2 (t-s)/2}                                    \\
                                                                    &\comment{de la igualdad de la función generadora de momentos}                                  \\
                                                                    &=  e^{-\lambda^2 s/2} e^{\lambda B_s}                                                          \\
                                                                    &=  M_s^\lambda.
\end{align}\pn

Y con esto se termina la demostración.
 


    \newpage

    \subsection{Inciso (ii)}    \label{problema6_4:inciso2}
    \emph{
    Sea $\imf{H_n}{x}=\paren{-1}^ne^{x^2/2}\frac{d^n}{dx^n}e^{-x^2/2}$.
    A $H_n$ se le conoce como en\'esimo polinomio de Hermite. Calc\'ulelo para 
    $n\leq 5$. Pruebe que $H_n$ es un polinomio para toda $n\in\na$ y que 
    $\partial^n M^\lambda_t/\partial \lambda^n=t^{n/2}\imf{H_n}{B_t/\sqrt{t}}M^\lambda_t$.\pn
}

\afterstatement\pn
    \newpage
    
    \subsection{Inciso (iii)}   \label{problema6_4:inciso3}
    \emph{
    Pruebe que $t^{n/2}\imf{H_n}{B_t/\sqrt{t}}$ es una martingala para 
    toda $n$ y calc\'ulela para $n\leq 5$.\pn
}

\afterstatement\pn
    \newpage

    \subsection{Inciso (iv)}    \label{problema6_4:inciso4}
    \emph{
    Aplique muestreo opcional a las martingalas anteriores al tiempo 
    aleatorio $T_{a,b}=\min\set{t\geq 0:B_t\in\set{-a,b}}$ (para $a,b>0$) 
    con $n=1,2$ para calcular $\proba{B_{T_{a,b}}=b}$ y $\esp{T_{a,b}}$,
    ËQu\'e concluye cuando $n=3,4$? ?` Cree que $T_{a,b}$ tenga momentos 
    finitos de cualquier orden? Justifique su respuesta.\pn
}

\afterstatement\pn
    \newpage
    
    \subsection{Inciso (v)}     \label{problema6_4:inciso5}
    \emph{
    Aplique el teorema de muestreo opcional a la martingala 
    $M^\lambda $ al tiempo aleatorio $T_a=\inf\set{t\geq 0:B_t\geq a}$ 
    si $\lambda>0$. Diga por qu\'e es necesaria la \'ultima 
    hip\'otesis y calcule la transformada de Laplace de $T_a$
}

\afterstatement\pn

Sabemos que $T_a$ es finito casi seguramente dado que la martingala $B$ oscila.\pn

Aplicando el teorema de muestreo opcional de Doob con el tiempo de paro acotado
$T_a \wedge s$ tenemos que
\begin{align}
    \E(M_{T_a \wedge s}^\lambda) = \E(M_0^\lambda) = 1
\end{align}\pn

Por otro lado, $M_{T_a}^\lambda = e^{\lambda B_{T_a} - \lambda^2 B_{T_a}/2}$. Argumentando que $B$ empieza en 0, 
para el caso $s \leq T_a$, tenemos que $B_s \leq B_{T_a} = a$ tenemos 
\begin{align}
    M_{T_a \wedge s}^\lambda    &=  M_{s}^\lambda                       \\
                                &=  e^{\lambda B_{s} - \lambda^2 s/2}   \\
                                &=  e^{\lambda a - \lambda^2 s/2}       \\
                                &\leq e^{\lambda a}.   
\end{align}
Aquí fue necesaria la suposición $\lambda > 0$.\pn

Para el caso $T_a < s$, tenemos
\begin{align}
    M_{T_a \wedge s}^\lambda    &=      M_{T_a}^\lambda                 \\
                                &=      e^{\lambda a - \lambda^2 T_a/2} \\
                                &\leq   e^{\lambda a}.
\end{align}\pn

De este análisis de casos, tenemos que $M_{T_a \wedge s}^\lambda$ es una martingala acotada por $e^{\lambda a}$.
Dado que $M_{T_a \wedge s}^lambda \rightarrow M_{T_a}^\lambda$, el teorema de convergencia acotada nos dice que
\begin{align}
    \E(M_{T_a}^\lambda) &=  \lim_{s \rightarrow \infty} \E(M_{T_a \wedge s}^lambda)   \\
                        &=  1.
\end{align}

Es decir
\begin{align}
    \E(M_{T_a}^\lambda) &=  \E(e^{\lambda a - \lambda^2 T_a/2})         \\
                        &=  \E(e^{\lambda a} e^{-\lambda^2 T_a/2})      \\
                        &=  e^{\lambda a} \E(e^{-\lambda^2 T_a/2})      \\
                        &=  1
\end{align}

De donde $\E(e^{-\lambda^2 T_a/2}) = e^{-\lambda a}$. Para concular la transformada de Laplace de $T_a$, 
\begin{align}
    L_{T_a}(\alpha) &=  \E(e^{-\alpha T_a})
\end{align}
(véase \href{http://en.wikipedia.org/wiki/Laplace_transform#Formal_definition}{Laplace transform: Formal definition})
Haciendo $\alpha = \lambda^2/2$ y por lo tanto, $\lambda = (2\alpha)^{1/2}$ tenemos
\begin{align}
    L_{T_a}(\alpha) &=  \E(e^{-\alpha T_a})         \\
                    &=  \E(e^{-\lambda^2 T_a/2})    \\
                    &=  e^{-\lambda a}              \\
                    &=  e^{-(2\alpha)^{1/2} a}.
\end{align}

\end{proof}