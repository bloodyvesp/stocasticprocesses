\begin{problema}
	Sea
	\begin{esn}
		M^\lambda_t=e^{\lambda B_t-\lambda^2t/2}.
	\end{esn}
		\begin{enumerate}
			\item[(i)]      [\ref{problema6_4:inciso1}]
                Explique y pruebe formalmente por qu\'e, para toda 
                $n\geq 1$, $\partial^n M^\lambda_t/\partial \lambda^n$ es una martingala.\pn
                
			\item[(ii)]     [\ref{problema6_4:inciso2}]
                Sea $\imf{H_n}{x}=\paren{-1}^ne^{x^2/2}\frac{d^n}{dx^n}e^{-x^2/2}$.
                A $H_n$ se le conoce como en\'esimo polinomio de Hermite. Calc\'ulelo para 
                $n\leq 5$. Pruebe que $H_n$ es un polinomio para toda $n\in\na$ y que 
                $\partial^n M^\lambda_t/\partial \lambda^n=t^{n/2}\imf{H_n}{B_t/\sqrt{t}}M^\lambda_t$.\pn
                
			\item[(iii)]    [\ref{problema6_4:inciso3}]
                Pruebe que $t^{n/2}\imf{H_n}{B_t/\sqrt{t}}$ es una martingala para 
                toda $n$ y calc\'ulela para $n\leq 5$.\pn
                
			\item[(iv)]     [\ref{problema6_4:inciso4}]
                Aplique muestreo opcional a las martingalas anteriores al tiempo 
                aleatorio $T_{a,b}=\min\set{t\geq 0:B_t\in\set{-a,b}}$ (para $a,b>0$) 
                con $n=1,2$ para calcular $\proba{B_{T_{a,b}}=b}$ y $\esp{T_{a,b}}$,
                ËQu\'e concluye cuando $n=3,4$? ?` Cree que $T_{a,b}$ tenga momentos 
                finitos de cualquier orden? Justifique su respuesta.\pn
			
            \item[(v)]      [\ref{problema6_4:inciso5}]
                Aplique el teorema de muestreo opcional a la martingala 
                $M^\lambda $ al tiempo aleatorio $T_a=\inf\set{t\geq 0:B_t\geq a}$ 
                si $\lambda>0$. Diga por qu\'e es necesaria la \'ultima 
                hip\'otesis y calcule la transformada de Laplace de $T_a$. 
			%\item 
                %Opcional (para subir calificaci\'on en esta u otra tarea): 
			%\begin{enumerate}
				%\item Modifique el ejercicio para que aplique al proceso Poisson.
				%\item Resu\'elva el ejercicio modificado. 
			%\end{enumerate}
		\end{enumerate}
\end{problema}

\begin{proof}
    \subsection{Inciso (i)}     \label{problema6_4:inciso1}
    \emph{
    Explique y pruebe formalmente por qu\'e, para toda 
    $n\geq 1$, $\partial^n M^\lambda_t/\partial \lambda^n$ es una martingala.\pn
}

\afterstatement\pn

Analicemos las primeras derivadas.

\begin{align}
    \frac{d}{d\lambda}  M_t^\lambda &=  \frac{d}{d\lambda} \left( e^{\lambda B_t - \lambda^2 t/2} \right)                       \\
                                    &=  \frac{d}{d\lambda} \left( e^{\lambda B_t} e^{- \lambda^2 t/2} \right)                   \\
                                    &=  B_t e^{\lambda B_t} e^{- \lambda^2/2}  - \lambda t e^{\lambda B_t} e^{- \lambda^2 t/2}  \\
                                    &=  B_t M_t^\lambda - \lambda t M_t^\lambda                                                 \\
                                    &=  M_t^\lambda (B_t - \lambda t).                                                          \\
\end{align}

La de segundo orden
\begin{align}
    \frac{d^2}{d\lambda^2}  M_t^\lambda     &=  \frac{d}{d\lambda} \left( M_t^\lambda (B_t - \lambda t) \right)                     \\
                                            &=  M_t^\lambda (B_t - \lambda t) (B_t - \lambda t) -  t M_t^\lambda                    \\ 
                                            &=  M_t^\lambda \left[ (B_t - \lambda t)^2 -  t \right]                                 \\    
                                            &=  t M_t^\lambda \left[ (B_t - \lambda t)^2 (t^{-1/2})^2 -  1 \right]                       
\end{align}\pn

La de tercer orden
\begin{align}
    &   \;\;\;\;\frac{d^3}{d\lambda^3}  M_t^\lambda                                                                                                                                                         \\
    &=  \frac{d}{d\lambda} \left(t M_t^\lambda  \left[ (B_t - \lambda t)^2 (t^{-1/2})^2 -  1 \right] \right)                                                                                                \\ 
    &=  t \frac{d}{d\lambda} \left( M_t^\lambda  \left[ (B_t - \lambda t)^2 (t^{-1/2})^2 -  1 \right] \right)                                                                                               \\
    &=  t   M_t^\lambda \left[ (B_t - \lambda t)^3 (t^{-1/2})^2 -  (B_t - \lambda t) \right]    +   M_t^\lambda \frac{d}{d\lambda} \left[ (B_t - \lambda t)^2 (t^{-1/2})^2 -  1 \right]                  \\
    &=  t   M_t^\lambda \left(\left[ (B_t - \lambda t)^3 (t^{-1/2})^2 -  (B_t - \lambda t) \right]  +  \frac{d}{d\lambda} \left[ (B_t - \lambda t)^2 (t^{-1/2})^2 -  1 \right]\right)                  \\
    &=  t   M_t^\lambda \left(\left[ (B_t - \lambda t)^3 (t^{-1/2})^2 -  (B_t - \lambda t) \right]  +  \left[ -2t(B_t - \lambda t) (t^{-1/2})^2 \right]\right)                  \\
    &=  t   M_t^\lambda \left(\left[ (B_t - \lambda t)^3 (t^{-1/2})^2 -  (B_t - \lambda t) \right]  +  \left[ -2(B_t - \lambda t) \right]\right)                  \\
    &=  t   M_t^\lambda \left[ (B_t - \lambda t)^3 (t^{-1/2})^2 -  3(B_t - \lambda t) \right]                    \\
    &=  t \, t^{1/2} t^{-1/2}  M_t^\lambda \left[ (B_t - \lambda t)^3 (t^{-1/2})^2 -  3(B_t - \lambda t) \right]                   \\
    &=  t^{3/2}   M_t^\lambda \left[ (B_t - \lambda t)^3 (t^{-1/2})^3 -  3 (t^{-1/2})(B_t - \lambda t) \right]                    \\
\end{align}\pn

(Estamos haciendo un poquito de trampa para facilitar el inciso siguiente). Piénsese en el factor 
que se encuentra entre corchetes como un polinomio en $(B_t - \lambda t) (t^{-1/2})$ (en la parte 
anterior, se trata de un polinomio de grado $3$, con sólo los términos de grado $3$ y $1$, donde 
los coeficientes son $1$ y $3$ respectivamente).\pn

Para facilitar la notación, definimos $B_t^\lambda = (B_t - \lambda t) (t^{-1/2})$ y sólo hay 
que tener presente que $\frac{d}{d\lambda} B_t^\lambda = -t^{1/2}$ y en general, que 
$\frac{d}{d\lambda} (B_t^\lambda)^n = -t^{1/2} n (B_t^\lambda)^{n-1}$.\pn

Con esta nueva notación, tenemos
\begin{align}
    \frac{d}{d\lambda}  M_t^\lambda     &= t^{1/2} M_t^\lambda \left[ B_t^\lambda \right]                       \\
    \frac{d^2}{d\lambda^2}  M_t^\lambda &= t M_t^\lambda \left[ (B_t^\lambda)^2 - 1  \right]                    \\
    \frac{d^3}{d\lambda^3}  M_t^\lambda &= t^{3/2} M_t^\lambda \left[ (B_t^\lambda)^3 - 3 B_t^\lambda  \right]  \\
\end{align}\pn

Hagamos la siguiente hipótesis de inducción. Para cierta $n \geq 1$, se tiene que 
\begin{align}
    \frac{d^n}{d\lambda^n} M_t^\lambda = t^{n/2} M_t^\lambda \left[ r_n (B_t^\lambda)^n + r_{n-1} (B_t^\lambda)^{n-1} + \dots + r_1 B_t^\lambda + r_0 \right]
\end{align}
Donde $r_0, r_1, \dots, r_n \in \R$ y $r_n \neq 0$.\pn

Derivemos una vez más para ver qué ocurre con la derivada de orden $n+1$
\tiny
\begin{align}
    & \;\;\;\;\;\frac{d^{n+1}}{d\lambda^{n+1}} M_t^\lambda                                                                                                                                                                                      \\
    &= t^{n/2} \frac{d}{d \lambda} \left( M_t^\lambda \left[ r_n (B_t^\lambda)^n + \dots + r_1 B_t^\lambda + r_0 \right] \right)                                                                                                                \\
    &= t^{n/2}  M_t^\lambda \left( t^{1/2} \left[ r_n (B_t^\lambda)^{n+1}  + \dots + r_1 (B_t^\lambda)^2 + r_0 B_t^\lambda \right] -t^{1/2} \left[ r_n n (B_t^\lambda)^{n-1} + \dots + r_1 \right]\right)                                       \\
    &= t^{(n+1)/2}  M_t^\lambda \left( \left[ r_n (B_t^\lambda)^{n+1}  + \dots + r_1 (B_t^\lambda)^2 + r_0 B_t^\lambda \right] - \left[ r_n n (B_t^\lambda)^{n-1} + \dots + r_1 \right]\right)                                                  \\
    &= t^{(n+1)/2}  M_t^\lambda \left( \left[ r_n (B_t^\lambda)^{n+1}  + \dots + (r_{m-2} - r_{m})(B_t^\lambda)^{m-1} + \dots + r_1 (B_t^\lambda)^2 + r_0 B_t^\lambda \right]\right) 
\end{align}\pn
\normalsize

En resumen, tenemos la siguiente relación recursiva:
\tiny
\begin{align}\label{problema6_4:equivalencia_recursiva}
    &\;\;\;\;\;\frac{d^{n+1}}{d\lambda^{n+1}} M_t^\lambda     \\
    &= t^{(n+1)/2}  M_t^\lambda \left( \left[ r_n (B_t^\lambda)^{n+1}  + \dots + (r_{m-2} - r_{m})(B_t^\lambda)^{m-1} + \dots + r_1 (B_t^\lambda)^2 + r_0 B_t^\lambda \right]\right) 
\end{align}\pn
\normalsize

Una variable aleatoria con distribución normal es acotada casi seguramente y 
por lo tanto tiene momentos de todos los órdenes. Por el mismo argumento, el valor 
absoluto de una variable aleatoria con distribución normal tiene momentos de todos los
órdenes.


    \newpage

    \subsection{Inciso (ii)}    \label{problema6_4:inciso2}
    \emph{
    Sea $\imf{H_n}{x}=\paren{-1}^ne^{x^2/2}\frac{d^n}{dx^n}e^{-x^2/2}$.
    A $H_n$ se le conoce como en\'esimo polinomio de Hermite. Calc\'ulelo para 
    $n\leq 5$. Pruebe que $H_n$ es un polinomio para toda $n\in\na$ y que 
    $\partial^n M^\lambda_t/\partial \lambda^n=t^{n/2}\imf{H_n}{B_t/\sqrt{t}}M^\lambda_t$.\pn
}

\afterstatement\pn

\begin{align}
        H_1(x)  &=  x                       \\
        H_2(x)  &=  x^2 - 1                 \\
        H_3(x)  &=  x^3 - 3x                \\
        H_4(x)  &=  x^4 - 6x^2 + 3          \\
        H_5(x)  &=  x^5 - 10x^3 + 15x       \\
\end{align}

Supongamos que para cierto $n \in \N$ tenemos que $H_n$ es un polinomio
de orden $n$. Es decir, exísten $r_0, r_1, r_2, \dots, r_n \in \R$ tales que
\begin{align}
        H_n &=  r_n x^n + \dots + r_1 x + r_0. 
\end{align}

Entonces
\scriptsize
\begin{align}
    H_{n+1} &=  (-1)^{n+1}  e^{x^2/2}   \frac{d^n}{dx^{n+1}}e^{-x^2/2}                                                                              \\
            &=  (-1)^{n+1}  e^{x^2/2}   \frac{d}{dx}\left(\frac{d^n}{dx^{n}}e^{-x^2/2}\right)                                                       \\
            &=  (-1)^{n+1}  e^{x^2/2}   \frac{d}{dx}\left((-1)^n e^{-x^2/2}(r_n x^n + \dots + r_1 x + r_0)\right)                                   \\
            &\comment{hemos despejado la derivada de orden $n$ en terminos del supuesto polinomio $H_n$}                                            \\
            &=  (-1)^{n+1}  e^{x^2/2} (-1)^n \frac{d}{dx}\left(e^{-x^2/2}(r_n x^n + \dots + r_1 x + r_0)\right)                                     \\
            &=  (-1)^{n+1}  e^{x^2/2} (-1)^n \left(-x e^{-x^2/2}(r_n x^n + \dots + r_1 x + r_0) + e^{-x^2/2}(r_n n x^{n-1} + \dots + r_1)\right)    \\
            &=  (-1)^{n+1}  e^{x^2/2} e^{-x^2/2} (-1)^n \left(-x (r_n x^n + \dots + r_1 x + r_0) + (r_n n x^{n-1} + \dots + r_1)\right)             \\
            &=  (-1)^{n+1} (-1)^n \left(-x (r_n x^n + \dots + r_1 x + r_0) + (r_n n x^{n-1} + \dots + r_1)\right)                                   \\
            &=  (-1) \left(-x (r_n x^n + \dots + r_1 x + r_0) + (r_n n x^{n-1} + \dots + r_1)\right)                                                \\
            &=  (-1) \left(-(r_n x^{n+1} + \dots + r_1 x^2 + r_0 x) + (r_n n x^{n-1} + \dots + r_1)\right)                                          \\
            &=  (r_n x^{n+1} + \dots + r_1 x^2 + r_0 x) - (r_n n x^{n-1} + \dots + r_1)                                                             \\
            &=  r_n x^{n+1} + \dots + (r_{m-2} - m r_m) x^{m-1} + \dots - r_1                                                                       \\
\end{align}
\normalsize

Hemos demostrado que si $H_n$ es un polinomio de grado $n$, entonces $H_{n+1}$ es 
uno de grado $n+1$. Lo que demostramos al inicio de este inciso no es otra 
cosa sino la base de inducción. Por lo tanto tenemos que para todo $n \in \N$, 
$H_n$ es un polinomio de grado $n$.\pn

Para verificar la equivalencia $d^n M^\lambda_t/d \lambda^n=t^{n/2}\imf{H_n}{(B_t - \lambda t)/\sqrt{t}}M^\lambda_t$, basta sustituir en la ecuación anterior
$B_t^\lambda$ (definido en el inciso anterior) en vez de $x$, y comparar con $\eqref{problema6_4:equivalencia_recursiva}$. (Nótese que está mal escrito
el enunciado del problema, en vez de escribir $B_t - \lambda t$ se escribe únicamente $B_t$).
    \newpage
    
    \subsection{Inciso (iii)}   \label{problema6_4:inciso3}
    \emph{
    Pruebe que $t^{n/2}\imf{H_n}{B_t/\sqrt{t}}$ es una martingala para 
    toda $n$ y calc\'ulela para $n\leq 5$.\pn
}

\afterstatement\pn
    \newpage

    \subsection{Inciso (iv)}    \label{problema6_4:inciso4}
    \emph{
    Aplique muestreo opcional a las martingalas anteriores al tiempo 
    aleatorio $T_{a,b}=\min\set{t\geq 0:B_t\in\set{-a,b}}$ (para $a,b>0$) 
    con $n=1,2$ para calcular $\proba{B_{T_{a,b}}=b}$ y $\esp{T_{a,b}}$,
    ?`Qu\'e concluye cuando $n=3,4$? ?` Cree que $T_{a,b}$ tenga momentos 
    finitos de cualquier orden? Justifique su respuesta.\pn
}

\afterstatement\pn
    \newpage
    
    \subsection{Inciso (v)}     \label{problema6_4:inciso5}
    \emph{
    Aplique el teorema de muestreo opcional a la martingala 
    $M^\lambda $ al tiempo aleatorio $T_a=\inf\set{t\geq 0:B_t\geq a}$ 
    si $\lambda>0$. Diga por qu\'e es necesaria la \'ultima 
    hip\'otesis y calcule la transformada de Laplace de $T_a$
}

\afterstatement\pn
\end{proof}