\emph{
    Muestre que si $X$ es una cadena de Markov con distribución inicial $\pi$ y matriz de transición $P$ en balance detallado entonces
    \begin{align}
            (X_0, \dots, X_n) \sim (X_n, \dots, X_0).
    \end{align}
}

\afterstatement\pn

Por ser cadena de Markov con distribución inicial $\pi$, tenemos que
\begin{align}
        \P(X_0 = x_0, \dots, X_n = x_n) &=  \pi_{x_0} P_{x_0, x_1} P_{x_1, x_2} \dots P_{x_{n-1}, x_{n}}.
\end{align}\pn

Utilizando la hipótesis de que $\pi$ y $P$ están en balance detallado tenemos que
\begin{align}
    &   \pi_{x_0} P_{x_0, x_1} P_{x_1, x_2} \dots P_{x_{n-1}, x_{n}}    \\
    &=  P_{x_1, x_0} \pi_{x_1} P_{x_1, x_2} \dots P_{x_{n-1}, x_{n}}    \\
    &=  P_{x_1, x_0} P_{x_2, x_1} \pi_{x_2} \dots P_{x_{n-1}, x_{n}}    \\
    &\vdots                                                             \\
    &=  P_{x_1, x_0} P_{x_2, x_1} \dots P_{x_{n}, x_{n-1}} \pi_{x_n}    \\
    &= P(X_n = x_n, \dots, X_0 = x_0).
\end{align}\pn

Con lo que termina la demostración.