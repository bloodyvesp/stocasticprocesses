\emph{
    Encuentre la matriz de transici\'on de la cadena de Markov de los estados distintos que toma $X$, 
    clasifique los estados, diga si existe una \'unica distribuci\'on invariante y en caso afirmativo, 
    encu\'entrela. Calcule expl\'icitamente las potencias de la matriz de transici\'on. 
    (Recuerde que de ser posible diagonalizar, esta es una buena estrategia.)\pn
}

\afterstatement\pn

Como no hay estados absorbentes, podemos utilizar la fórmula
\begin{align}
        P(i,j)  &=  \frac{Q(i, j)}{c(i)} (i-\delta_{i,j})
\end{align}

donde $c(i)$ es la suma de las entradas positivas de la $i$-ésima fila de $Q$ y $\delta_{i,j} = 1$
si $i = j$ y $\delta_{i,j} = 0$ en cualquier otro caso.\pn

Después de algunas cuentas se obtiene:
\begin{align}
     P=
        \begin{pmatrix}
            0           &   1   &   0               \\
            \frac{1}{7} &   0   &   \frac{6}{7}     \\
            0           &   1   &   0
        \end{pmatrix}.   
\end{align}\pn

De $P$ se puede todos los estados son alcanzables desde cualquier otro (de los estados $0$ y $2$ 
se puede llegar al $1$, y del $1$ a cualquiera de los otros). Es decir, es irreducible.\pn

Existe un teorema que nos asegura que por ser irreducible y de estados finitos, los estados son positivos
recurrentes y existe una distribución invariante. También existe otro teorema que nos asegura que por tener
estados positivos recurrentes, la distribución invariante es única.\pn

Entonces, buscar una distribución invariante se reduce a resolver el sistema de ecuaciones

\begin{align}
        \frac{1}{7}x_2              &=      x_1 \\
        x_1 + x_3                   &=      x_2 \\
        \frac{6}{7}x_2              &=      x_3 \\
\end{align}

Donde $x_1 + x_2 + x_3 = 1$. Después de algunas cuentas, se llega a que la solución del sistema es
$x_1 = \frac{1}{14}$, $x_2 = \frac{7}{14}$, $x_3 = \frac{6}{14}$.\pn

Para calcular explícitamente las potencias de $P$, primero encontramos el polinomio característico de $P$.
\begin{align}
        p(x)    =   -\lambda(\lambda + 1)(\lambda - 1).
\end{align}

De donde, los valores propios son $0, -1, 1$. Después de resolver los sistemas de ecuaciones correspondientes 
(lo cual son muchas cuentas) se obtiene que los vectores propios correspondientes son $(-6, 0 ,1), (1, 1, 1), (1, -1, 1)$.

Por lo tanto, $P$ se puede diagonalizar como
\begin{align}
    A^{-1} P A  = D
\end{align}

Donde 
\begin{align}
     A  =
        \begin{pmatrix}
            -6  &   1   &   1      \\
            0   &   1   &   -1     \\
            1   &   1   &   1
        \end{pmatrix}.   
\end{align}\pn

y 


\begin{align}
     D  =
        \begin{pmatrix}
            0   &   0   &   0      \\
            0   &   1   &   0      \\
            0   &   0   &   -1
        \end{pmatrix}.   
\end{align}\pn

y Entonces

\begin{align}
        P^n = A D^n A^{-1}
\end{align}

Ahora notemos que 

\begin{align}
     D^n  =
            \begin{pmatrix}
                0   &   0   &   0      \\
                0   &   1   &   0      \\
                0   &   0   &   -1^n
            \end{pmatrix}.   
\end{align}\pn

Por lo tanto
\begin{align}
     P^{2n + 1} = A D^{2n+1} A^{-1} = A D A^{-1} = P
\end{align}

Y 
\begin{align}
     P^{2n} = A D^{2} A^{-1} = 
                            \begin{pmatrix}
                                \frac{1}{7}     &   0   &   \frac{6}{7}     \\
                                0               &   1   &   0               \\
                                \frac{1}{7}     &   0   &   \frac{6}{7}
                            \end{pmatrix}.
\end{align}

