\emph{
    $\paren{M_{n\wedge T}}$ es uniformemente integrable. 
}

Sabemos que $M_{T \wedge n} \underset{c.s.}\longrightarrow M_T$ gracias a que $T$ es finito 
(la justificación está en \ref{problema3_1:inciso1}).\par\null

Existe un teorema que garantiza que si $M_{T \wedge n} \underset{c.s.}\longrightarrow M_T$ y 
$\paren{M_{n\wedge T}}$ es uniformemente integrable, entonces 
$M_T \in L_1$ y $M_{T \wedge n} \underset{L_1}\longrightarrow M_T$.\par\null

Es decir que $\E(M_{T \wedge n}) \longrightarrow \E(M_T)$, y de nueva cuenta, por Teorema del Muestreo Opcional de Doob, 
en vista que $T \wedge n$ es tiempo de paro acotado, tenemos que $\E(M_{T \wedge n}) = \E(M_0)$ y por lo tanto

\begin{align}
    \E(M_T)     &=  \lim_{n \rightarrow \infty} \E(M_{T \wedge n})      \\
                &=  \lim_{n \rightarrow \infty} \E(M_0)                 \\
                &=  \E(N_0).
\end{align}\par\null

Que es lo que buscábamos demostrar.
