\emph{
    Defina al proceso de contéo asociado $N$, expréselo en términos de las variables $B_1, B_2, \dots$ y pruebe
    diréctamente que $N_n/n$ converge casi seguramente conforme $n$ crece. Identifíque el límite.
}
\afterstatement\pn

Sea
\begin{align}
        N_n = \sum^n_{i = 1} B_i.
\end{align}

Y notemos que 
\begin{align}
        \min \{ m : T_{m+1} > n \} 
\end{align}
significa, el tiempo de volado más alto antes de pasarnos de $n$.\pn

Interpretado de esta forma tenemos que coinciden
\begin{align}
    \min \{ m : T_{m+1} > n \} = \sum^n_{i = 1} B_i = N_n.
\end{align}\pn

Por lo tanto tenemos que $N_n$ sí es un proceso de conteo para nuestro proceso de renovación.\pn

Por otra parte $N_n$ es suma de $n$ variables idependientes e idénticamente distribuidas
con esperanza $p$. $N_n / n$ es su promedio y la ley fuerte de los grandes números nos dice que
\begin{align}
    N_n/n \rightarrow p \;\; c.s.
\end{align}
