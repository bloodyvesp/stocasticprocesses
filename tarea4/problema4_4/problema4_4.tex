\begin{problema}
	Sean $\p$ y $\q$ dos medidas de probabilidad en el espacio can\'onico $E^\na$ 
	para sucesiones con valores en un conjunto a lo m\'as numerable $E$. 
	Decimos que $\q$ es \defin{localmente absolutamente continua} respecto de $\p$ 
	si para cada $n\in\na$, $\q|_{\F_n}\ll\p|_{\F_n}$. Sea
	
	\begin{esn}
		D_n=\frac{d \q|_{\F_n}}{d \p|_{\F_n}}.
	\end{esn}
	
	\begin{enumerate}
		\item[(i)]		[\ref{problema4_4:inciso1}]
			Pruebe que $D$ es una martingala bajo $\p$. Pruebe que si 
			$D$ es uniformemente integrable entonces $\q\ll\p$.\pn
			
		\item[(ii)]		[\ref{problema4_4:inciso2}]
			Pruebe que si $T$ es un tiempo de paro finito 
			entonces $\q|_{\F_T}\ll\p|_{\F_T}$.\pn
			
		\item[(iii)]	[\ref{problema4_4:inciso3}]	
			Sea $\p^p$ la distribuci\'on de una caminata aleatoria simple 
			que comienza en $0$ y va de $k$ a $k+1$ con probabilidad $p$, 
			donde $p\in (0,1)$. Pruebe que $\p^p$ es localmente absolutamente 
			continua respecto de $\p^{1/2}$ y encuentre la martingala $D_n$ asociada.\pn
		
		\item[(iv)]		[\ref{problema4_4:inciso4}]
			Para $a,b>0$, sea $T=\min\set{n\in\na: X_n\in \set{-a,b}}$. 
			Pruebe que $T$ y $X_T$ son independientes bajo $\p^{1/2}$. Al utilizar 
			la continuidad absoluta local, pruebe que $T$ y $X_T$ tambi\'en son 
			independientes bajo $\p^p$. Utilice alguna martingala de ejercicios 
			anteriores para calcular $\esp{T^2}$.			
	\end{enumerate}
\defin{Categor\'ias: }Cambio de medida, Caminata aleatoria simple.
\end{problema}

\begin{proof}
    \subsection{Inciso (i)} \label{problema4_4:inciso1}
    \emph{
	Pruebe que $D$ es una martingala bajo $\p$. Pruebe que si 
	$D$ es uniformemente integrable entonces $\q\ll\p$.
}

\afterstatement\pn

Primero veamos que se trata de una martingala bajo $\P$.

\begin{itemize}
	\item   (\textbf{Adaptada a la filtración})
        Por definición de derivada de Radon-Nykodim, $D_n$ es $\F_n$-medible.\pn
    
    \item   (\textbf{Cada término es integrable})
        Por ser $\Q$ una medida de probabilidad, es no negativa. Por lo tanto cualquier versión que se
        escoja para la derivada de Radon-Nykodim, es no negativa casi seguramente y por lo tanto
        $\E(\abs{D_n}) = \E(D_n)$. Ahora
        \begin{align}
            \E(D_n)     &=  \int D_n d\P                                    \\
                        &=  \int D_n d\P|_{\F_n}                            \\
                        &\comment{gracias a que $D_n$ es $\F_n$-medible}    \\
                        &=  \Q|_{\F_n}(\Omega)                              \\
                        &=  \Q(\Omega)                                      \\
                        &=  1                                               \\
                        &<  \infty.
        \end{align}\pn
        
    \item   (\textbf{Condición de martingala})
        Sea $A \in \F_n$. Entonces
        \begin{align}
            \E(D_{n+1} \indic_A)    &=  \int D_{n+1} \indic_A d\P                                       \\
                                    &=  \int D_{n+1} \indic_A d\P|_{\F_{n+1}}                           \\
                                    &\comment{porque $D_{n+1}, \indic_A$ son $\F_{n+1}$-medibles}       \\
                                    &=  \Q|_{\F_{n+1}}(A)                                               \\
                                    &=  \Q|_{\F_n}(A)                                                   \\
                                    &\comment{porque $D_n, \indic_A$ son $\F_n$-medibles}               \\
                                    &=  \int D_n \indic_A d\P|_{\F_n}                                   \\
                                    &=  \int D_n \indic_A d\P                                           
                                    &=  \E(D_n \indic_A).   
        \end{align}
        
        Lo que demuestra, que bajo $\F_n$, tanto $D_{n+1}$ como $D_n$ integran lo mismo. Es decir
        $\E(D_{n+1} | \F_{n}) = \F_n$, como queríamos demostrar.\pn
\end{itemize}

Ahora supongamos que $D$ es uniformemente integrable. Tenemos un teorema que dice que si un martingala es 
uniformemente integrable entonces existe una variable $D_\infty$ tal que $\E(D_\infty | \F_n)$.
(Este teorema es el 1.9 de la versión de las notas de la clase que se incluye junto con esta tarea, 
ver [\ref{notas}]).\pn

Definimos $\F_\infty = \sigma\paren(\bigcup \F_i)$. El mismo teorema asegura que $D_n$ converge casi seguramente y en
$L_1$ a $\E(D_\infty | \F_\infty)$. 

Ahora definimos
\begin{align}
    P   =   \bigcup_{i \in N} \F_i.
\end{align}\pn

Es fácil ver que $P$ es un álgebra. Claramente $P$ es no vacío, todos los $\F_i$ son subconjuntos de él. 
Si $A_1, A_2 \in P$ entonces existen $i, j \in \N$ tales que $A_1 \in \F_i$ y $A_2 \in \F_j$ y por lo 
tanto $A_1 \cap A_2, A_1 \cup A_2 \in \F_{\max(i,j)}$. Que sea cerrado bajo complementos se hereda directamente de
que cada $\F_i$ es una álgebra.\pn

Ahora sea 
\begin{align}
    M   =   \bigg\{ A \in \F_infty   :   \Q(A) = \int_A D_\infty d\P \bigg\}
\end{align}

Veamos que $M$ es una clase monótona. Es decir, cerrada bajo sucesiones crecientes y  decrecientes.\pn

Sean $A_1, A_2, \dots \in M$ tales que $A_i \subset A_{i+1}$.

\begin{align}
    \Q\paren{\bigcup A_i}   &=  \lim_{i \rightarrow \infty} \Q(A_i)                                         \\
                            &=  \lim_{i \rightarrow \infty} \int_{A_i}  D_\infty d\P                        \\
                            &=  \lim_{i \rightarrow \infty} \int  D_\infty \indic_{A_i} d\P                 \\
                            &=  \int  D_\infty \lim_{i \rightarrow \infty} \indic_{A_i} d\P                 \\
                            &\lcomment{por que $A_n$ es creciente}                                          \\
                            &\rcomment{ y por teorema de convergencia mónotona}                             \\
                            &=  \int  D_\infty \indic_{\cup A_i} d\P                                        \\
                            &=  \int_{\cup A_i}  D_\infty d\P                                               \\
\end{align}

Tomando complementos, y haciendo una demostración análoga, se tiene que $M$ también es cerrada bajo 
sucesiones decrecientes.\pn

Ahora veamos que $P \subset M$. Sea $A \in P$. Entonces existe $n \in \N$ tal que $A \in \F_n$. Entonces

\begin{align}
    \Q(A)       &=  \Q|_{F_n}(A)                                    \\
                &=  \int D_n \indic_A d\P|_{\F_n}                   \\
                &=  \int \E(D_\infty | \F_n) \indic_A d\P|_{\F_n}   \\
                &=  \int D_\infty \indic_A d\P|_{\F_n}              \\
                &\comment{la integración ya se hace sobre $\F_n$}   \\
                &=  \int D_\infty \indic_A d\P.              
\end{align} 

Y por lo tanto $A \in M$. El teorema de clases monótonas garantiza que $\sigma(P) \subset M$. Pero $\sigma(P) = \F_\infty$
y $M \subset \F_\infty$, por lo tanto $\sigma(P) = \F_\infty = M$. Entonces, tenemos que para todo $A \in \F_\infty$

\begin{align}
    \Q(A)   =   \int_{A}    D_\infty    d\P.
\end{align}

Por lo tanto $\Q \ll \P$. Como buscábamos demostrar.
    \newpage

    \subsection{Inciso (ii)} \label{problema4_4:inciso2}
    \emph{
	Pruebe que si $T$ es un tiempo de paro finito 
	entonces $\q|_{\F_T}\ll\p|_{\F_T}$.
}
    \newpage

    \subsection{Inciso (iii)} \label{problema4_4:inciso3}
    \emph{
	Sea $\p^p$ la distribuci\'on de una caminata aleatoria simple 
	que comienza en $0$ y va de $k$ a $k+1$ con probabilidad $p$, 
	donde $p\in (0,1)$. Pruebe que $\p^p$ es localmente absolutamente 
	continua respecto de $\p^{1/2}$ y encuentre la martingala $D_n$ asociada.
}
	\newpage
	
    \subsection{Inciso (iv)} \label{problema4_4:inciso4}
    \emph{
	Para $a,b>0$, sea $T=\min\set{n\in\na: X_n\in \set{-a,b}}$. 
	Pruebe que $T$ y $X_T$ son independientes bajo $\p^{1/2}$. Al utilizar 
	la continuidad absoluta local, pruebe que $T$ y $X_T$ tambi\'en son 
	independientes bajo $\p^p$. Utilice alguna martingala de ejercicios 
	anteriores para calcular $\esp{T^2}$.
}
\end{proof}