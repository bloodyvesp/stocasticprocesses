\begin{problema}
	Sea
	\begin{esn}
		M^\lambda_t=e^{\lambda B_t-\lambda^2t/2}.
	\end{esn}
		\begin{enumerate}
			\item Explique y pruebe formalmente por qu\'e, para toda $n\geq 1$, $\partial^n M^\lambda_t/\partial \lambda^n$ es una martingala. 
			\item Sea $\imf{H_n}{x}=\paren{-1}^ne^{x^2/2}\frac{d^n}{dx^n}e^{-x^2/2}$. A $H_n$ se le conoce como en\'esimo polinomio de Hermite. Calc\'ulelo para $n\leq 5$. Pruebe que $H_n$ es un polinomio para toda $n\in\na$ y que $\partial^n M^\lambda_t/\partial \lambda^n=t^{n/2}\imf{H_n}{B_t/\sqrt{t}}M^\lambda_t$. 
			\item Pruebe que $t^{n/2}\imf{H_n}{B_t/\sqrt{t}}$ es una martingala para toda $n$ y calc\'ulela para $n\leq 5$. 
			\item Aplique muestreo opcional a las martingalas anteriores al tiempo aleatorio $T_{a,b}=\min\set{t\geq 0:B_t\in\set{-a,b}}$ (para $a,b>0$) con $n=1,2$ para calcular $\proba{B_{T_{a,b}}=b}$ y $\esp{T_{a,b}}$, ËQu\'e concluye cuando $n=3,4$? ?` Cree que $T_{a,b}$ tenga momentos finitos de cualquier orden? Justifique su respuesta.
			\item Aplique el teorema de muestreo opcional a la martingala $M^\lambda $ al tiempo aleatorio $T_a=\inf\set{t\geq 0:B_t\geq a}$ si $\lambda>0$. Diga por qu\'e es necesaria la \'ultima hip\'otesis y calcule la transformada de Laplace de $T_a$. 
			\item Opcional (para subir calificaci\'on en esta u otra tarea): 
			\begin{enumerate}
				\item Modifique el ejercicio para que aplique al proceso Poisson.
				\item Resu\'elva el ejercicio modificado. 
			\end{enumerate}
		\end{enumerate}
\end{problema}