\emph{
    Sea
        \begin{align}
                g(\lambda)=E(e^{- \lambda X_1}).
        \end{align} 
    Pruebe que 
        \begin{align}
                g(\lambda)    &\in (0,\infty)\label{problema2_1:parte1}
        \end{align}        
    y que
        \begin{align}
                M_n=e^{-\lambda S_n}g(\lambda)^{-n},n\geq 0
        \end{align}
    es una martingala.\\
}
    
    Para probar \eqref{problema2_1:parte1} notemos que
    \begin{align}
        e^x > 0 \;\; \forall x \in \R.
    \end{align}
    por lo tanto
    \begin{align}
        e^{-\lambda X_1} > 0
    \end{align}
    y entonces
    \begin{align}
        \E(e^{-\lambda X_1}) >\E(0) = 0
    \end{align}

    Falta ver que $\E(e^{-\lambda X_1})$ está acotada. Recordemos que $e^x$ es creciente y por lo tanto
    \begin{align}
            X_1                &\geq     -1                       & \Rightarrow\\
            1                  &\geq     - X_1                    & \Rightarrow\\
            \lambda            &\geq     - \lambda X_1            & \Rightarrow\\
            e^{\lambda}        &\geq     e^{- \lambda X_1}        & \Rightarrow\\
            \E(e^{\lambda})    &\geq     \E(e^{- \lambda X_1})
    \end{align}
    
    Pero $\E(e^{\lambda}) = e^{\lambda}$. Y por lo tanto
    \begin{align}
        e^{\lambda} \geq  \E(e^{- \lambda X_1})
    \end{align}    
    Entonces $0 < \E(e^{-\lambda X_1}) < \infty$, como queríamos demostrar.