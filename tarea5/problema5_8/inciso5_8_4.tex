\emph{
	Codifique el siguiente esquema num\'erico, conocido como m\'etodo de Euler, 
	para aproximar a $e^{10 Q}$: escoja $h>0$ peque\~no, defina a $P^h_0$ como 
	la matriz identidad y recursivamente
	\begin{esn}
		P^h_{i+1}=P^h_i+hQP^h_i. 
	\end{esn}
	corra hasta que $i=\floor{10/h}$ y compare la matriz resultante con $e^{10Q}$. 
	Si no se parecen escoja a $h$ m\'as peque\~no. 
	?`Con qu\'e $h$ puede aproximar a $e^{10Q}$ a 6 decimales?
}

\afterstatement\pn

El siguiente código es la implementación.

\small
\texttt{
	\lstinputlisting[inputencoding=utf8]{tarea5/problema5_8/AproxE10Q.m}
}\pn
\normalsize

En realidad, la fórmula recursiva no es necesaria. Podríamos escribir:
\begin{align}
 P^h_{i}    = (I + hQ)^i   
\end{align}

Y de aquí, que si el máximo de los valores absolutos de las entradas de
$I + hQ$ es menor o igual que $1$, la convergencia de $P^h_{i}$ está garantizada.\pn

Haciendo algunas cuentas sencillas, se encuentra que si $h = 1/3$ ocurre lo comentado en
el párrafo anterior. \verb|(I + hQ)^33| resulta estar a menos de $6$ decimales de diferencia
de $e^{10Q}$. Sin embargo, el tope que se le pone al algoritmo es \texttt{floor(10/(1/3)) = 30}.\pn

Con el siguiente algoritmo, se buscó una $h$ chiquita que hiciera que el resultado del
algoritmo se acercara a $e^{10Q}$ lo suficiente.

\small
\texttt{
	\lstinputlisting[inputencoding=utf8]{tarea5/problema5_8/IteraAproxE10Q.m}
}\pn
\normalsize

El resultado fue \texttt{h = 0.328304729443698}.
