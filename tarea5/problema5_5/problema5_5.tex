\begin{problema}[Tomado del examen general de probabilidad del Posgrado en Ciencias Matem\'aticas, UNAM, \href{http://www.posgradomatematicas.unam.mx/contenidoEstatico/archivo/files/pdf/Examenes_Generales/Probabilidad/Probabilidad2011-1.pdf}{Febrero 2011}]
Una planta de producci\'on toma su energ\'ia de dos generadores. La cantidad de generadores al tiempo $t$ est\'a representado por una cadena de Markov a tiempo continuo $\set{X_t,t\geq 0}$ con espacio de estados $E=\set{0,1,2}$ y matriz infinit\'esimal $Q$ dada por\begin{esn}
Q=\begin{pmatrix}
-6&6&0\\
1&-7&6\\
0&2&-2
\end{pmatrix}.
\end{esn}
\begin{enumerate}
\item Encuentre la matriz de transici\'on de la cadena de Markov de los estados distintos que toma $X$, clasifique los estados, diga si existe una \'unica distribuci\'on invariante y en caso afirmativo, encu\'entrela. Calcule expl\'icitamente las potencias de la matriz de transici\'on. (Recuerde que de ser posible diagonalizar, esta es una buena estrategia.)
\item ?`Cu\'al es la probabilidad de que ambos generadores est\'en trabajando al tiempo $t$ si s\'olo uno trabaja al tiempo cero? 
\item Si $\rho_2$ denota la primera vez que ambos generadores est\'an trabajando al mismo tiempo, encuentre la distribuci\'on de $\rho_2$ cuando s\'olo un generador est\'a trabajando al tiempo cero. 
\item Encuentre la proporci\'on de tiempo asint\'otica en que los dos generadores est\'an trabajando. Si cada generador produce 2.5 MW de energ\'ia por unidad de tiempo, ?`Cu\'al es la cantidad promedio de energ\'ia producida a largo plazo por unidad de tiempo?
\end{enumerate}
\end{problema}
