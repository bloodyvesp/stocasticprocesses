\begin{problema}
Sea $M$ una $\paren{\F_n}$-martingala con saltos acotados. Sean

\begin{align}
C   &=  \set{\limsup M_n=\liminf M_n\in\re}                 \\
    &y                                                      \\
D   &=  \set{\limsup M_n=-\infty\text{ y }\limsup M_n=\infty}.
\end{align}\par\null

Pruebe que $\proba{C\cup D}=1$. Deduzca que las caminatas aleatorias centradas 
con saltos acotados oscilan. Sugerencia: Para cada $K>0$ defina

\begin{align}
T=\min\set{n\geq 0: \abs{M_n}\geq K}
\end{align}

y aplique el teorema de convergencia de martingalas a $M^T$.\par\null

Sea $M$ una caminata aleatoria no trivial con saltos integrables en 
$-1,0,1,\ldots$ y  media cero.\par\null 

Pruebe que $\proba{M\text{ converge en }\na}=0$ y  concluya que $\liminf M_n=-\infty$ 
casi seguramente. (Este resultado permitir\'a dar una prueba adicional de que un Galton-Watson cr\'itico se extingue).  
Sugerencia: proceda como en el párrafo anterior y pruebe la integrabilidad uniforme de $(M_{T\wedge n})_{n \in \N}$.

\defin{Categor\'ias: } Teoremas de convergencia de martingalas
\end{problema}