\emph{
	Sea $\paren{S_n}_{n\in\na}$ una cadena de Markov con espacio de estados $\z$ y matriz de transici\'on
	\begin{esn}
	P_{i,i+1}=p\quad P_{i,i-1}=1-p
	\end{esn}	
	donde $p\in [0,1]$. D\'e una condici\'on necesaria y suficiente para que $\paren{\abs{S_n},n\in\na}$ 
	sea una cadena de Markov.
}

Tenemos que $\abs{S_{n - 1} - S_n} = 1$ con probabilidad $1$. Esto significa, los saltos del proceso siempre son
de tamaño $1$. Esto significa que, la paridad de un estado al siguiente siempre cambia con probabilidad $1$. En
otras palabras, la paridad se preserva después de una cantidad par de pasos.\pn

Es decir que si $r_1, r_2 \in \Z$ entonces:
\begin{align}
        \mw(S_{2n + n_0} = r_2 | S_{n_0} = r_1 ) =
                                                    \begin{cases}
                                                        1,  \text{si $2$ divide a $r_1 - r_2$} \\
                                                        0,  \text{si $2$ no divide a $r_1 - r_2$} 
                                                    \end{cases}
\end{align}

La paridad se preserva bajo valor absoluto, así que lo mismo ocurre en el caso de $\abs{S_n}$