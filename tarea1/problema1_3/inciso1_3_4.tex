\emph
{
	Sea $T=\min\set{n\geq 2:S_n=S_{n-2} + 2}$ y $U=T-2$. ?`Son $T$ y $U$ 
	tiempos de paro? Justifique su respuesta.\\
}

	Intuitivamente, $T$ significa, el primer tiempo tal que ganamos en dos volados consecutivos.
	También intuitivamente, esto debería ser un tiempo de paro.
	
	Veamos que efectivamente así ocurre. Utilizando la siguiente prueba por inducción:\\
	
	\textbf{Base de inducción:}		
		\begin{align}
			\{T = 0\} 		&= \emptyset  				& 	\in \F_0 \\
			\{T = 1\} 		&= \emptyset  				& 	\in \F_1 \\
			\{T = 2\} 		&= \{ X_1 = 1, X_2 = 1\} 	&	\in \F_2
		\end{align}	\\					
	
	\textbf{Hipótesis de inducción:}\\
	
		Supongamos que $\{T = n\} \in \F_n$ para cierto $n \geq 2$.\\
		
	\textbf{Paso inductivo:}
		
		\begin{align}
			\{T = n + 1 \} = \{ X_n = 1, X_{n+1} = 1\} \setminus \bigcup_{i=0}^{n} \{T = i\}.
		\end{align}				
	
		Es claro que $\{ X_n = 1, X_{n+1} = 1\} \in \F_{n + 1}$ y que por hipótesis de inducción
		$\bigcup_{i=0}^{n} {T = i} \in \F_n \subset \F_{n + 1}$. Por lo tanto
		$\{T = n + 1 \} \in \F_{n+1}$ para toda $n \geq 2$ y con esto termina la demostración.\\
		
	Ahora, intuitivamente $U$ significa el momento justo antes de ganar dos volados consecutivos.
	Esto, quedría decir que tenemos información sobre eventos que aún no ocurren. Así que intuitivamente
	esto no debería ser un tiempo de paro.\\
	
	Efectivamente, si tomamos como ejemplo el conjunto: 
		\begin{align}
			\{ U = 1 \} = \{ T - 2 = 1\} = \{ T = 3\} = \{X_1 = -1, X_2 = 1, X_3 = 1\}
		\end{align}		\\
			
	Es fácil notar que es un conjunto que pertenece a $\F_3$, pero no a $\F_1$. Pues $\F_1$
	no contiene información alguna sobre $X_2$ y $X_3$. Así que el conjunto más pequeño de $\F_1$ 
	que contiene a $\{ U = 1 \}$ es $\{ X_1 = -1 \}$.\\
