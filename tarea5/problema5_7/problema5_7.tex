\begin{problema}
    (Tomado del examen general de conocimientos del \'area de Probabilidad del Posgrado 
    en Ciencias Matem\'aticas, UNAM, 
    \href{http://www.posgradomatematicas.unam.mx/contenidoEstatico/archivo/files/pdf/Examenes_Generales/Probabilidad/Probabilidad2011-2.pdf}{Agosto 2011})

    Sea $N$ un proceso de Poisson homog\'eneo de par\'ametro $\lambda$. Sea $E=\paren{-1,1}$ y $X_0$ una 
    variable aleatoria con valores en $E$ independiente de $N$. Se define el proceso
    \begin{esn}
        X_t=X_0 \times \paren{-1}^{N_t}, \quad t\geq 0.
    \end{esn}

    \begin{enumerate}
        \item[(i)]      [\ref{problema5_7:inciso1}]
            Explique por qu\'e $X$ es una cadena de Markov a tiempo continuo con valores en $E$.\pn 
        \item[(ii)]     [\ref{problema5_7:inciso1}]
            Calcule sus probabilidades de transici\'on y su matriz infinitesimal.\pn
        \item[(iii)]    [\ref{problema5_7:inciso1}]
            ?`Existe una distribuci\'on estacionaria para esta cadena? En caso afirmativo ?`Cu\'al es?\pn
    \end{enumerate}
\end{problema}

\begin{proof}
    \subsection{Inciso (i)} \label{problema5_7:inciso1}
    \emph{
    Explique por qu\'e $X$ es una cadena de Markov a tiempo continuo con valores en $E$.\pn 
}

\afterstatement\pn

Sea $T_n$ el tiempo en el que ocurre el $n$-ésimo salto del proceso de Poisson
($T_0 = 0$). Definimos $W_{T_n} = X_{T_n}$. Veamos que $W_n$ es una cadena de Markov.

\begin{align}
    & \;\;\;\;\; \P\left(W_n = i_n | W_{n-1} = i_{n-1}, \dots, W_0 = i_0 \right)                \\
    & =\P\left(X_0 \cdot (-1)^n  = i_n | X_0 \cdot (-1)^{n-1}  = i_{n-1}, X_0 = i_{0} \right)   \\
    & =\P\left( -(i_{n-1})  = i_n | X_0 \cdot (-1)^{n-1}  = i_{n-1}, X_0 = i_{0} \right)        \\
    & =\indic_{-(i_{n-1})  = i_n}                                                               \\
    & =\P\left(X_0 \cdot (-1)^n  = i_n | X_0 \cdot (-1)^{n-1}  = i_{n-1}\right)                 \\
    & =\P\left(W_n  = i_n | W_{n-1}  = i_{n-1}\right)                                                 
\end{align}\pn

Con esto hemos demostrado que $W_n$ es una cadena de Markov a tiempo discreto. Como además,
el tiempo entre saltos tiene distribución exponencial (porque está definido por un proceso de Poisson),
concluimos que $X$, es una cadena de Markov.
    \newpage

    \subsection{Inciso (ii)} \label{problema5_7:inciso2}
    \emph{
    Calcule sus probabilidades de transici\'on y su matriz infinitesimal.\pn
}
\afterstatement\pn
    \newpage

    \subsection{Inciso (iii)} \label{problema5_7:inciso3}
    \emph{
    ?`Existe una distribuci\'on estacionaria para esta cadena? En caso afirmativo ?`Cu\'al es?\pn
}

\afterstatement\pn

Sí existe. Dado que se trata de un proceso de estados finitos y recurrentes, sabemos que es positivo
recurrente y por lo tanto existe una distribución invariante que además es única.\pn

Existe un teorema que nos asegura que una distribución $\nu$ es invariante para una cadena continua si y solo
si $cv = (c_x \nu_x, x \in E)$ es invariante para la cadena discreta asociada.\pn

Entonces estamos buscando un vector $(\pi_1, \pi_2)$ tal que $(\pi_1, \pi_2) P = (\pi_1, \pi_2)$.\pn

Pero
\begin{align}
        (\pi_1, \pi_2) P &= 
                        (\pi_1, \pi_2) 
                        \begin{pmatrix}
                            0   &   1 \\
                            1   &   0 \\
                        \end{pmatrix} \\
                        &=              
                        (\pi_2, \pi_1)
\end{align}

Así que $\pi_1 = \pi_2$. Además $\pi_1 + \pi_2 = 1$ por tratarse de una distribución. Así que $\pi_1 = \pi_2 = \frac{1}{2}$.
\end{proof}