
\begin{problema}[Extensiones del teorema de paro opcional]
Sea \(M=\paren{M_n,n\in\na}\) una (super)martingala respecto de una filtraci\'on \(\paren{\F_n,n\in\na}\) y sean \(S\) y \(T\) tiempos de paro.
\begin{enumerate}
                \item Pruebe que \(S\wedge T\), \(S+T\) y \(S\vee T\) son tiempos de paro.
                \item Sea \begin{esn}\F_T=\set{A\in\F:A\cap\set{T\leq n}\in\F_n\text{ para toda } n}\end{esn}es una \(\sigma\)-\'algebra, a la que nos referimos como la \(\sigma\)-\'algebra detenida en \(\tau\). Comente qu\'e puede fallar si \(T\) no es tiempo de paro. Pruebe que \(T\) es \(F_T\)-medible. 
                \item Pruebe que si \(T\) es finito, entonces \(M_T\) es \(\F_T\)-medible.
                \item Pruebe que si \(S\leq T\leq n\) entonces \(\F_S\subset\F_T\). Si adem\'as \(T\) es acotado entonces \(X_S,X_T\in L_1\) y \begin{esn}\espc{M_T}{\F_S}\leq M_S.\end{esn}
                \item Si \(X=\paren{X_n,n\in\na}\) es un proceso estoc\'astico \(\paren{\F_n}\)-adaptado y tal que \(X_n\in L_1\) y tal que para cualesquiera tiempos de paro acotados \(S\) y \(T\) se tiene que \(\esp{X_S}=\esp{X_T}\) entonces \(X\) es una martingala. Sugerencia: considere tiempos de paro de la forma \(n\indi{A}+(n+1)\indi{A^c}\) con \(A\in\F_n\).
\end{enumerate}

\defin{Categor\'ias: }Tiempos de paro, Muestreo opcional
\end{problema}