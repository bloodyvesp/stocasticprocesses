\emph
{
	Para la variable $T$ que hemos definido, calcule $\esp{T}$.\\
}

	Primero, platicaré de manera intuitiva cómo vamos a proceder para solucionar este problema.\\

	Imaginemos un juego de casino a base de un juego de volados con las siguientes reglas:\\
	\begin{itemize}
			\item En cada turno, cada jugador tiene que apostar todo el dinero que tiene.
			\item Si un jugador se queda sin dinero, tiene que abandonar el juego.
			\item Si cae ``sol" cada jugador recibe el doble de lo que había apostado en ese turno.  
	\end{itemize}
   
	\;El casino tiene dinero infinito y cada habitante cuenta con exactamente $1$ peso antes de
	iniciar el juego.\\
   
	Además, en cada nuevo turno entra exactamente un nuevo jugador al juego.\\
   
	Para nuestro problema, supongamos que $X_n = 1$ significa que en el $n$-ésimo turno, salió sol.
	Entonces, $T$ nos indica cuando es la primera vez que caen dos soles consecutivos.\\
   
	Sea $D_n$ la variable que indica cuánto dinero ha ganado el casino para el tiempo $n$.\\
   
	Observemos que en el momento que cae ``águila", todo jugador pierde todo su dinero y abandona el juego.
	Y que por cada jugador que pierde, el casino gana exáctamente $1$ peso (pues cada jugador en cada
	turno apuesta todo el dinero que posee, es decir el peso con el que empezó y todo lo que le habia
	ganado al casino).\\
   
	Entonces, al tiempo $T-2$, todo mundo había perdido. Es decir que al tiempo $T-2$ el casino ha ganado
	$T-2$ pesos.\\
   
	Luego, al tiempo $T-1$, ha caido un sol y hay exactamente un jugador al que el casino tuvo 
	que pagar $1$ peso.\\
   
	Al tiempo $T$, al jugador del turno pasado el casino tuvo que darle $2$ pesos y al jugador del nuevo
	turno tuvo que darle $1$ peso.\\
   
	Entonces, ya podemos decir cuanto dinero ha ganado el casino al tiempo $T$.
	\begin{align}\label{problema1_3:Dinero_al_tiempo_T}
		D_T = T-2 - 1 - 3 = T - 6. 
	\end{align}					   
   
	Notemos que además el juego es justo, en cada turno cada jugador tiene $1/2$ de probabilidad de
	ganar $2^t$ y $1/2$ de probabilidad de perder $2^t$. Es decir, la esperanza es $0$.\\
	
	$D_n$ es suma de este tipo de variables y por lo tanto su esperanza también será $0$.\\
   
	Esto, nos da la intuición de que $D_n$ es martingala, pero esa es la parte que demostraremos más adelante.\\
   
	Si logramos demostrar que $\E(D_T) = 0$. De \eqref{problema1_3:Dinero_al_tiempo_T} concluimos
	\begin{align}
		0 = \E(T - 6) = \E(T) - 6
	\end{align}
	
	De donde $\E(T) = 6$.\\
	
	Ahora, para terminar con las formalidades, definamos bien a $D$ y comprobemos que es martingala y 
	que podemos utilizar el Teoremoa del muestreo opcional de Doob como lo hemos hecho.\\
	
	Sean entonces $(Y_n)_{n \in \N}$ variables aleatorias Bernulli de parámetro $1/2$ independientes.
	Y sea $Z_n^m$ la cantidad de dinero que el jugador $m$ ha dado al casino definidas como:
	
	\begin{itemize}
		\item 
			Si $n < m$ entonces $Z_n^m = 0$ . (El jugador $m$ no participa en el juego sino hasta el turno $m$).
		\item
			$Z_{n+1}^{m} = (Z_n^{m} - 1) \cdot 2(Y_{n+1}) + 1$. (Si $Y_{n + 1} = 1$ [El jugador gana el volado], entonces el casino
			pierde la cantidad apostada, que para el turno $n+1$ es $Z_n^n + 1$). Nótese que en cuanto un $Y_{n_0}$ se hace cero,
			$Z_{n_0}^{n}$ y todos los que le sigan son todos iguales a $1$ (Como el jugador deja el juego después de haber perdido
			un volado, deja su peso en el casino y entonces de ahí en adelante la cantidad que ha dado al casino es exactamente 1).
	\end{itemize}
	
	Veamos que para cada $m \in \N$, $(Z_n^m)_{n \in \N}$ forma una martingala con respecto a la filtración $(\G_n)_{n \in \N}$ 
	definida por los $Y_n$.\\
	
	Cada $Z_n^m$ es $\G_m$-medible por definición.\\
	
	Cada $Z_n^m$ es suma finita de variables acotadas. Por lo tanto cada una pertenece a $L_1$.\\
	
	Sólo nos falta verificar la propiedad de martingala.
	
	\begin{align}\label{problema1_3:Propiedad_de_martingala_para_el_dinero_perdido_por_un_apostador}
		\E(Z_{n+1}^{m} | \G_n) &= \E((Z_n^{m} - 1) \cdot 2(Y_{n+1}) + 1 | \G_n)\\
							   &= \E((Z_n^{m} -1) \cdot 2(Y_{n+1}) | \G_n) + \E( 1 | \G_n)\\
							   &\;\;\;\;\mbox{[Por ser $(Z_n^{m} - 1)$ una variable $\G_n$-medible.]}\\
							   &= (Z_n^{m} -1) \cdot \E(  2(Y_{n+1}) | \G_n) + \E( 1 | \G_n)\\
							   &= (Z_n^{m} -1) \cdot 2\E( Y_{n+1} | \G_n) + \E( 1 | \G_n)\\
							   &\;\;\;\;\mbox{[Por ser $Y_{n + 1}$ independiente $\G_n$.]}\\
							   &= (Z_n^{m} -1) \cdot 2\E( Y_{n+1}) + \E( 1 | \G_n)\\
							   &= (Z_n^{m} -1) \cdot 2 \frac{1}{2} + \E( 1 | \G_n)\\
							   &= (Z_n^{m} -1) \cdot + 1 \\
							   &= Z_n^m.
	\end{align}
	
	Ahora definamos a $D_n = \sum_{i=1}^n (Z_n^i)$. Que significa, La cantidad de dinero que el casino ha ganado al tiempo $n$. 
	Justo como lo habíamos dicho en la ``demostración intuitiva".\\
	
	Ver que $D$ es martingala es fácil. Cada $D_n$ es suma finita de variables finitas y por lo tanto pertenece a $L_1$.
	$D_n$ es suma de variables $\G_n$-medibles y por lo tanto también lo es. Y la propiedad de martingala se sigue directamente de
	\eqref{problema1_3:Propiedad_de_martingala_para_el_dinero_perdido_por_un_apostador}.\\
	
	Es cierto que podemos aplicar Doob sobre el tiempo $T \wedge n$ por ser acotado y de aquí que:
	$\E(D_{T \wedge n}) = \E(D_1) = 0$.\\
	
	Notemos que $D_{T \wedge n} \longrightarrow D_T \; c.s.$\\
	
	Nos gustaría poder decir lo mismo de sus esperanzas y para eso utilizaremos teroema de convergencia dominada.\\
	
	Es claro que al tiempo $T \wedge n$ el casino a lo más pudo haber ganado $T \wedge n$ pesos.
	De aquí que $D_{T \wedge n} \geq T \wedge n \geq T$.\\
	
	Además, por definición de $T$, para el tiempo $T$ el casino a lo más ha perdido $4$ pesos y es la primera vez que 
	pierde tanto. Así que $-4 \leq D_{T \wedge n}$.\\ 
	
	Entonces, nuestra martingala $D$ esta dominada por $\max(T, 4) < T + 4$. Bastaría demostrar que $\E(T+4) < \infty$ para poder
	utilizar el teorema de convergencia dominada.\\
	
	Notemos que 
	\begin{align}\label{problema1_3:Acotando_T}
		\mw(T > n) &= \frac{1}{2} \mw(T > n-1) + \frac{1}{4} \mw(T > n-2). 
	\end{align}
	
	Pues queremos garantizar que en los primeros $n$ turnos, no pierde dos veces consecutivas el casino.\\
	
	Si en el turno 1, gana el casino (de aquí el $\frac{1}{2}$), en el resto de los $n-1$ turnos tenemos que garantizar que 
	el casino no pierde dos veces consecutivas, como las variables están idénticamente distribuidas, esto es equivalente a que
	$T>n-1$. (De aqui el $\mw(T > n-1)$.\\
	
	Si el casino pierde en el turno 1, necesariamente tiene que ganar en el turno 2 (de aquí el $\frac{1}{4})$. Y la probabilidad
	de que no pierda dos veces consecutivas en los siguientes $n-2$ turnos es $\mw(T > n-2)$.\\					    
	
	De \eqref{problema1_3:Acotando_T} podemos notar fácilmente que $\mw(T > 2) \leq \frac{3}{4}$\\
	
	También notemos que $\mw(T > n ) \geq \mw(T > n + 1)$ (El primer evento contiene al segundo).
	
	Ahora
	\begin{align}
		\mw(T > 2 (2)) &=     \frac{1}{2} \mw(T > 2+1) + \frac{1}{4} \mw(T > 2) \\
					   &\leq  \frac{1}{2} \mw(T > 2) + \frac{1}{4} \mw(T > 2) \\
					   &\leq  \frac{1}{2} \cdot \frac{3}{4} + \frac{1}{4} \cdot \frac{3}{4} = \bigg(\frac{3}{4}\bigg)^2.
	\end{align}
	
	De manera recursiva tenemos que
	\begin{align}
		\mw(T > 2(n+1)) &=     \frac{1}{2} \mw(T > 2(n+1) - 1) + \frac{1}{4} \mw(T > 2(n+1) - 2) \\
						&=     \frac{1}{2} \mw(T > 2(n) + 1) + \frac{1}{4} \mw(T > 2(n)) \\
					   &\leq  \frac{1}{2} \mw(T > 2n) + \frac{1}{4} \mw(T > 2n) \\
					   &\leq  \frac{1}{2} \cdot \bigg(\frac{3}{4}\bigg)^{n} + \frac{1}{4} \cdot \bigg(\frac{3}{4}\bigg)^{n} = 		
					   \bigg(\frac{3}{4}\bigg)^{n+1}.
	\end{align}
	
	Entonces,
	\begin{align}
		\E(T) 	&= 		\sum_{n=0}^\infty \mw(T > n) \\
				&\leq 	\sum_{n=0}^\infty 2\mw(T > 2n) \\
				&= 		2 \sum_{n=0}^\infty \mw(T > 2n) \\
				&\leq 	2 \sum_{n=0}^\infty \bigg(\frac{3}{4}\bigg)^{n} < \infty.
	\end{align}
	
	Por lo tanto $\E(T + 4) \leq \infty$. Y entonces nuestra martingala $D$ está dominada por una variable integrable y finalmente
	podemos aplicar teorema de convergencia dominada para concluir que:
	\begin{align}
		0 = \lim_{n \longrightarrow \infty} \E(D_{T \wedge n}) = \E(D_T).
	\end{align}
	
	Y con esto terminamos de demostrar todas las formalidades que nos hacían falta.