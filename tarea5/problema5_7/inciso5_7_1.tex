\emph{
    Explique por qu\'e $X$ es una cadena de Markov a tiempo continuo con valores en $E$.\pn 
}

\afterstatement\pn

Sea $T_n$ el tiempo en el que ocurre el $n$-ésimo salto del proceso de Poisson
($T_0 = 0$). Definimos $W_{n} = X_{T_n}$. Veamos que $W_n$ es una cadena de Markov.

\begin{align}
    & \;\;\;\;\; \P\left(W_n = i_n | W_{n-1} = i_{n-1}, \dots, W_0 = i_0 \right)                \\
    & =\P\left(X_0 \cdot (-1)^n  = i_n | X_0 \cdot (-1)^{n-1}  = i_{n-1}, X_0 = i_{0} \right)   \\
    & =\P\left( -(i_{n-1})  = i_n | X_0 \cdot (-1)^{n-1}  = i_{n-1}, X_0 = i_{0} \right)        \\
    & =\indic_{-(i_{n-1})  = i_n}                                                               \\
    & =\P\left(X_0 \cdot (-1)^n  = i_n | X_0 \cdot (-1)^{n-1}  = i_{n-1}\right)                 \\
    & =\P\left(W_n  = i_n | W_{n-1}  = i_{n-1}\right)                                                 
\end{align}\pn

Con esto hemos demostrado que $W_n$ es una cadena de Markov a tiempo discreto. Como además,
el tiempo entre saltos tiene distribución exponencial (porque está definido por un proceso de Poisson),
concluimos que $X$, es una cadena de Markov.