\begin{problema}
    Sea $S_n=\sum_{i=1}^n X_i$ donde $X_1,X_2,\ldots$ son iid. Sea
    
    \begin{align}
        \imf{\phi}{\lambda}=\esp{e^{\lambda S_n}}\in (0,\infty].
    \end{align}

    \begin{enumerate}
        \item[(i)]		[\ref{problema2_4:inciso1}]
            Pruebe que si existen $\lambda_1<0<\lambda_2$ tales que $\imf{\phi}{\lambda_i}<\infty$ entonces $\imf{\phi}{\lambda}<\infty$ 
			para toda $\lambda\in [\lambda_1,\lambda_2]$. Sugerencia: escriba $\lambda=a\lambda_1+(1-a)\lambda_2$ para alg\'un $a\in [0,1]$ 
			y aplique la desigualdad de H\"older. A partir de ahora se asume la premisa de este inciso.\par\null
			
        \item[(ii)] 	[\ref{problema2_4:inciso2}]
            Pruebe que $\esp{\abs{S_n}^k}<\infty$ para toda $k\geq 0$.\par\null
			
        \item[(iii)] 	[\ref{problema2_4:inciso3}]
            Sea $M^\lambda_t=e^{\lambda S_t}/\imf{\phi}{\lambda}$. Argumente que si $M^n$ es el proceso dado por
            
            \begin{align}
                M^n_t=\left.\frac{\partial^n}{\partial \lambda^n}\right|_{\lambda=0}M^\lambda_t,
            \end{align}
            
            entonces $M^n$ es una martingala para toda $n$. \par\null
        \item[(iv)] 	[\ref{problema2_4:inciso4}]
            Calcule las primeras $4$ martingalas resultantes si $\proba{X_i=\pm 1}=1/2$. Util\'icelas para calcular el valor de $\esp{T^2}$ donde

            \begin{align}
                T=\min\set{n\geq 0: S_n\in\set{-a,b}}
            \end{align}y $a,b>0$.\par\null 
    \end{enumerate}

    \defin{Categor\'ias:} Caminatas aleatorias, muestreo opcional, ejemplos de martingalas. 
\end{problema}

\begin{proof}
    \subsection{Inciso (i)}     \label{problema2_4:inciso1}
    \emph{
	Pruebe que si existen $\lambda_1<0<\lambda_2$ tales que $\imf{\phi}{\lambda_i}<\infty$ entonces $\imf{\phi}{\lambda}<\infty$ 
	para toda $\lambda\in [\lambda_1,\lambda_2]$. Sugerencia: escriba $\lambda=a\lambda_1+(1-a)\lambda_2$ para alg\'un $a\in [0,1]$ 
	y aplique la desigualdad de H\"older. A partir de ahora se asume la premisa de este inciso.
}

\afterstatement\pn

Sea $\lambda = \lambda_1 (1-t) + \lambda_2 t$ con $t \in [0, 1]$. Por lo tanto 

\begin{align}
    \phi(\lambda)   &= \E(e^{\lambda S_n})                                              \\
                    &= \E\paren{e^{(\lambda_1 (1-t) + \lambda_2 t) S_n}}                \\
                    &= \E\paren{e^{(\lambda_1 (1-t) S_n + \lambda_2 tS_n) }}            \\  
                    &= \E\paren{e^{\lambda_1 (1-t) S_n} e^{\lambda_2 tS_n) }}           \\  
                    &= \E\paren{{e^{(\lambda_1 S_n)}}^{1-t} {e^{(\lambda_2 S_n)}}^t}      
\end{align}

Si definimos $p = 1/(1-t)$ y $q = 1/t$, tenemos que
${e^{(\lambda_1 S_n)}}^{1-t} \in L_p$ pues $\paren{{e^{(\lambda_1 S_n)}}^{1-t}}^p = e^{(\lambda_1 S_n)}$ quien 
pertenece a $L_1$ por hipótesis. Análgomanente tenemos que  ${e^{(\lambda_2 S_n)}}^t \in L_q$.\pn

Ahora, recordando que $|e^x| = e^x$ y utilizando la desigualdad de Hölder \eqref{Desigualdad_de_Holder}

\begin{align}
    \phi(\lambda)   &=      \E\paren{{e^{(\lambda_1 S_n)}}^{1-t} {e^{(\lambda_2 S_n)}}^t}           \\      
                    &\leq   \E\paren{e^{(\lambda_1 S_n)}}^{1-t} \E\paren{e^{(\lambda_2 S_n)}}^t     
\end{align}\pn

Donde cada uno de los factores es finito por hipótesis y por lo tanto $\phi(\lambda) < \infty$. Como queríamos demostrar.
    \newpage
    
    \subsection{Inciso (ii)}    \label{problema2_4:inciso2}
    \emph{
	Pruebe que $\esp{\abs{S_n}^k}<\infty$ para toda $k\geq 0$.
}

\afterstatement\par\null

Ahora que sabemos que si $\lambda \in [\lambda_1, \lambda_2]$ entonces $\phi(\lambda) < \infty$. Elijamos 
$\lambda = \min(-\lambda_1, \lambda_2)$ y de esta manera, tenemos que $\phi(-\lambda), \phi(\lambda) < \infty$.\par\null

Entonces

\begin{align}
    \E(e^{|\lambda S_n|})   &=      \E(e^{\lambda S_n} \indic_{\lambda S_n \geq 0} + e^{-\lambda S_n} \indic_{\lambda S_n < 0} )        \\ 
                            &=      \E(e^{\lambda S_n} \indic_{\lambda S_n \geq 0}) + \E(e^{-\lambda S_n} \indic_{\lambda S_n < 0})     \\
                            &\leq   \E(e^{\lambda S_n} ) + \E(e^{-\lambda S_n})                                                         \\
                            &=      \phi(\lambda) + \phi(-\lambda)                                                                      \\
                            &<      \infty.                                                  
\end{align}\par\null

Este hecho lo necesitaremos más adelante.\par\null
    \newpage
        
    \subsection{Inciso (iii)}    \label{problema2_4:inciso3}
    \emph{
	Sea $M^\lambda_t=e^{\lambda S_t}/\imf{\phi}{\lambda}$. Argumente que si $M^n$ es el proceso dado por
	\null
	\begin{align}
		M^n_t=\left.\frac{\partial^n}{\partial \lambda^n}\right|_{\lambda=0}M^\lambda_t,
	\end{align}
	\null
	entonces $M^n$ es una martingala para toda $n$.
}

\afterstatement\pn
    \newpage
    
    \subsection{Inciso (iv)}    \label{problema2_4:inciso4}
    \emph{
	Si $\proba{X_i=\pm 1}=1/2$, calcule las primeras $4$ martingalas resultantes.
    Util\'icelas para calcular el valor de $\esp{T^2}$ donde
    \null
	\begin{align}
		T=\min\set{n\geq 0: S_n\in\set{-a,b}}
	\end{align}
    \null
    y $a,b>0$.
}

\afterstatement\par\null
\end{proof}