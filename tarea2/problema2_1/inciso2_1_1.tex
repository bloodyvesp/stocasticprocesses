\emph{
    Sea
        \begin{align}
                g(\lambda)=E(e^{- \lambda X_1}).
        \end{align} 
    Pruebe que 
    \null
        \begin{align}
                g(\lambda)    &\in (0,\infty)\label{problema2_1:parte1}
        \end{align}
    \null
    y que
    \null
        \begin{align}
                M_n=e^{-\lambda S_n}g(\lambda)^{-n},n\geq 0
        \end{align}
    \null
    es una martingala.
}

\afterstatement

    Para probar \eqref{problema2_1:parte1} notemos que

    \begin{align}
        e^x > 0 \;\; \forall x \in \R.
    \end{align}\par\null

    por lo tanto

    \begin{align}
        e^{-\lambda X_1} > 0
    \end{align}\par\null

    y entonces

    \begin{align}
        \E(e^{-\lambda X_1}) >\E(0) = 0
    \end{align}\par\null

    Falta ver que $\E(e^{-\lambda X_1})$ está acotada. Recordemos que $e^x$ es creciente y por lo tanto

    \begin{align}
        X_1                &\geq     -1                       & \Rightarrow\\
        1                  &\geq     - X_1                    & \Rightarrow\\
        \lambda            &\geq     - \lambda X_1            & \Rightarrow\\
        e^{\lambda}        &\geq     e^{- \lambda X_1}        & \Rightarrow\\
        \E(e^{\lambda})    &\geq     \E(e^{- \lambda X_1})
    \end{align}\par\null
    
    Pero $\E(e^{\lambda}) = e^{\lambda}$. De donde
    
    \begin{align}
        e^{\lambda} \geq  \E(e^{- \lambda X_1})
    \end{align}\par\null    
    
    Entonces $0 < \E(e^{-\lambda X_1}) < \infty$, y con esto queda demostrado \eqref{problema2_1:parte1}.\par\null
    
    Probemos ahora que $(M_n)_{n \in \N}$ es martingala.
    
    \begin{itemize}
        \item 
            Veamos que $M_n$ es $\F_n$-medible.\par\null
            
            Para esto, recordemos que si $\G$ es una $\sigma$-algebra, $f$ es $\G$-medible y $g$ es continua
            con la topología usual de $\R$ entonces $g \circ f$ es $\G$-medible.\par\null
            
            $S_n$ es $\F_n$-medible. Multiplicar por constantes es función continua. Dividir entre constantes 
            es función continua. La función exponencial es continua. Y por lo tanto $M_n=e^{-\lambda S_n}g(\lambda)^{-n}$ 
            es composición de funciones continuas con $S_n$. De donde $M_n$ es $F_n$-medible.\par\null
         
        \item 
            Ahora veamos que $M_n \in L_1$.\par\null
            
            De la misma demostración que dimos para probar la primera parte, se sigue 
            que $0 < e^{-\lambda X_n} < \infty$ y que $e^{-\lambda X_n} \in \N$ para toda $n \in \N$. Por lo tanto
            
            \begin{align}
                M_n         &=      e^{-\lambda S_n}g(\lambda)^{-n} \\
                            &=      \prod_{1 \leq i \leq n} e^{-\lambda X_i} g(\lambda)^{-n}. 
            \end{align}\par\null
            
            Entonces, $M_n$ es producto finito de funciones en $L_1$, de donde tenemos que $M_n \in L_1$.\par\null
            
         \item
            Únicamente falta demostrar que $\E(M_{n+1} | \F_n) = M_n$.\par\null
            
            \begin{align}
                \E(M_{n+1} | \F_n)      &=      \E(e^{-\lambda S_{n+1}}g(\lambda)^{-(n+1)}|\F_n)                                        \\
                                        &=      g(\lambda)^{-(n+1)}\E(e^{-\lambda S_{n+1}}|\F_n)                                        \\
                                        &\comment{Esto último es gracias a que $g(\lambda)^{-(n+1)}$ es constante}               		\\
                                        &=      g(\lambda)^{-(n+1)}\E(e^{-\lambda S_{n+1}}|\F_n)                                        \\
                                        &=      g(\lambda)^{-(n+1)}\E(e^{-\lambda S_{n} \cdot X_{n+1}}|\F_n)                            \\
                                        &=      g(\lambda)^{-(n+1)}\E(e^{-\lambda S_{n}} \cdot e^{-\lambda X_{n+1}}|\F_n)               \\
                                        &=      g(\lambda)^{-(n+1)} e^{-\lambda S_{n}} \E(e^{-\lambda X_{n+1}}|\F_n)                    \\                                       
                                        &\comment{Esto último es gracias a que $S_n$ es $F_n$-medible}                           		\\
                                        &=      g(\lambda)^{-(n+1)} e^{-\lambda S_{n}} \E(e^{-\lambda X_{n+1}})                         \\
                                        &\comment{Esto último es gracias a que $X_{n+1}$ es independiente de $F_n$}              		\\
                                        &=      g(\lambda)^{-(n+1)} e^{-\lambda S_{n}} \E(e^{-\lambda X_{1}})                           \\
                                        &\rcomment{Esto último es gracias a que $X_{n+1}$ y $X_1$ tienen la}                       	    \\ 
                                        &\lcomment{misma distribución}   															    \\
                                        &=      g(\lambda)^{-(n+1)} e^{-\lambda S_{n}} g(\lambda)                                       \\
                                        &=      g(\lambda)^{-(n)} e^{-\lambda S_{n}}                                                    \\
                                        &=      M_n.
            \end{align}\par\null
    \end{itemize}
    
    Y con esto terminamos de demostrar que $(M_n)_{n \in \N}$ es martingala.