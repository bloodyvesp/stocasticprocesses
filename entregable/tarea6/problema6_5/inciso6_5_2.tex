\emph{
    Pruebe que la sucesi\'on de variables aleatorias
    \begin{esn}
        \paren{\sup_{t\in [0,1]}\abs{B_{n+t}-B_n},n\in\na}
    \end{esn}
    son independientes, id\'enticamente distribuidas y de media finita. (Utilice la propiedad de Markov.)\pn
}

\afterstatement\pn

La propiedad de Markov nos asegura que para todo $n \in N$ el proceso dado obtenido ``desfasar'' el proceso 
en $n$  es decir: $B_{n+t} - B_n$ es independiente de $\F_n = \sigma(B_t : t \leq n)$ y por lo tanto
también $\abs{B_{n+t} - B_n}$ es independiente de $\F_n$ para toda $n$ y por lo tanto son independientes.\pn

Además, sabemos que desfasar es únicamente volver a empezar el proceso, es decir, $B_{n+t} - B_n \sim \B_t$ y entonces
$\abs{B_{n+t} - B_n} \sim \abs{B_t}$. Tomando supremos $\sup_{t \leq 1} \abs{B_{n+t} - B_n} \sim \sup_{t \leq 1} \abs{B_t}$.
Es decir que $\sup_{t \leq 1} \abs{B_{n+t} - B_n}$ tienen todas la misma distribución que $\sup_{t \leq 1} \abs{B_t}$.\pn

El en [\ref{problema6_5:inciso1}] se demostró que $\sup_{t\leq 1}\abs{B_t-B_1}$ es cuadrado integrable, y por lo tanto
tiene medida finita. Ese es el caso cuando $n = 1$, pero como para toda $n$ se tiene la misma distribución, todas
tienen medida finita.