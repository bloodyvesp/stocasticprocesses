\emph{
    Aplique muestreo opcional a las martingalas anteriores al tiempo 
    aleatorio $T_{a,b}=\min\set{t\geq 0:B_t\in\set{-a,b}}$ (para $a,b>0$) 
    con $n=1,2$ para calcular $\proba{B_{T_{a,b}}=b}$ y $\esp{T_{a,b}}$,
    ?`Qu\'e concluye cuando $n=3,4$? ?` Cree que $T_{a,b}$ tenga momentos 
    finitos de cualquier orden? Justifique su respuesta.\pn
}

\afterstatement\pn

Primero, trabajemos con $n=1$. Como de costumbre, trabajaremos con los tiempos acotados $T_{a,b} \wedge s$, donde podemos aplicar
el teorema del muestreo opcional de Doob y obtener que $\E(B_{T_{a,b} \wedge s}) = \E(B_0) = 0$.\pn

Ahora, $B_{T_{a,b} \wedge s}$ está acotado por $\max\{ a, b\}$ y además $B_{T_{a,b} \wedge s} \longrightarrow B_{T_{a,b}}$
conforme $s$ crece. Así que podemos utilizar el teorema de convergencia acotada para obtener que 
$\E(B_{T_{a,b}})  = \lim_{s \rightarrow \infty} \E(B_{T_{a,b} \wedge s}) = 0$.\pn

Al igual que en el problema de la ruina, como $B_{T_{a,b}} \in \{-a, b\}$ casi seguramente (porque la martingala $B$ oscila), 
tenemos que 
\begin{align}
    0   &=  \E(B_{T_{a,b}})                                     \\
        &= -a \P(B_{T_{a,b}} = -a) + b \P(B_{T_{a,b}} = b)      \\
        &= -a \P(B_{T_{a,b}} = -a) + b (1-\P(B_{T_{a,b}} = -a)) \\   
        &= -a \P(B_{T_{a,b}} = -a) + b - b\P(B_{T_{a,b}} = -a)) \\   
        &= b - (a+b)\P(B_{T_{a,b}} = -a)) \\   
\end{align}

De donde $\P(B_{T_{a,b}} = -a) = \frac{b}{a+b}$ y análogamente $\P(B_{T_{a,b}} = b) = \frac{a}{a+b}$.\pn

De manera complétamente análoga, tomando $n=2$, la martingala $(B_t)^2 -t$, tenemos que
\begin{align}
    0   &=  \E((B_{T_{a,b}})^2 -{T_{a,b}}).
\end{align}

Descomponiendo la esperanza tenemos
\begin{align}
    0   &=  \E((B_{T_{a,b}})^2 -{T_{a,b}})                                      \\
        &=  a^2 \P(B_{T_{a,b}} = -a) + b^2 \P(B_{T_{a,b}} = b) - \E(T_{a,b})    \\
        &=  a^2 \frac{b}{a+b} + b^2 \frac{a}{a+b} - \E(T_{a,b})
\end{align}

De donde concluimos que 
\begin{align}
    \E(T_{a,b}) &=  \frac{a^2b +b^2a}{a+b}  \\
                &=  \frac{ab (a+b)}{a+b}    \\
                &=  ab.
\end{align}
es finito.

Para $n=3$, con los mismos argumentos tenemos que
\begin{align}
    0   &=  \E((B_{T_{a,b}})^3 - 3 B_{T_{a,b}}T_{a,b})              \\
        &=  \frac{b^3 a - a^3 b}{a + b} - 3\E(B_{T_{a,b}}T_{a,b})   \\
\end{align}\pn

De donde $\E(B_{T_{a,b}}T_{a,b})$ es finito.

Para $n=4$, con los mismos argumentos de nuevo, tenemos que
\begin{align}
    0   &=  \E((B_{T_{a,b}})^4 - 6(B_{T_{a,b}})^2T_{a,b} + 3 (T_{a,b})^2)                   \\
        &=  \E((B_{T_{a,b}})^4) - 6\E((B_{T_{a,b}})^2T_{a,b}) + 3 \E((T_{a,b})^2))          \\
        &=  \frac{a^4 b + b^4 a}{a+b} - 6\E((B_{T_{a,b}})^2T_{a,b}) + 3 \E((T_{a,b})^2))
\end{align}\pn

Luego, argumentando que de nuevo que $B_{T_{a,b}} \in \{-a, b\}$ casi seguramente, podemos partir la esperanza $\E((B_{T_{a,b}})^2T_{a,b})$
en los casos $B_{T_{a,b}} = -a$ y $B_{T_{a,b}} = b$ y luego aplicar el resultado del caso $n=3$ para concluir que $\E((T_{a,b})^2))$ es finito.\pn

De \eqref{problema6_4:equivalencia_recursiva} podemos ver que para cada $n$ par, en $H_n(x)$ queda un término constante que en $t^{n/2}H_n(B_t t^{-1/2})$
se convierte en un término de $t^{n/2}$ por una constante. Para el resto de los términos, recursivamente se conoce el momento que resulta ser siempre finito 
(para los términos del tipo $(B_{T_{a,b}})^j (T_{a,b})^k$ basta con descomponer en los casos $B_{T_{a,b}} = -a$ y $B_{T_{a,b}} = b$ e iterar el procedimiento
sobre lo que queda). Despejando, obtendremos que $\E((T_{a,b})^{n/2})$ resulta ser finito y por lo tanto, tenemos momentos de todos los ordenes para $T_{a,b}$. 