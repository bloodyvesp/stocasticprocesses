\emph{
	Si $X=\paren{X_n,n\in\N}$ es un proceso estoc\'astico $\paren{\F_n}$-adaptado y tal que $X_n\in L_1$ y tal que 
	para cualesquiera tiempos de paro acotados $S$ y $T$ se tiene que $\esp{X_S}=\esp{X_T}$ entonces $X$ es una 
	martingala. Sugerencia: considere tiempos de paro de la forma $n\indi{A}+(n+1)\indi{A^c}$ con $A\in\F_n$.\\
}

	Ya tenemos dos de las hipótesis necesarias para demostrar que algo es martingala.
\begin{enumerate}
	\item 
		$(X_n)_{n\in\N}$ es $(\F_n)_{n \in \N}$-adaptado.

	\item 
		$X_n \in L_1$ para toda $n \in \N$. 
\end{enumerate}
\null

	Sólo falta demostrar que $\espc{X_{n+1}}{\F_n} = X_n$ para toda $n \in \N$.\\

	Es decir, dada una $n \in \N$, queremos demostar que para todo $A \in \F_n$,
ocurre que $\E(X_{n+1} \cdot \indic_{A}) = \E(X_{n} \cdot \indic_{A})$.\\

	Sean entonces $n \in \N$ y $A \in \F_n$. Definimos $T_{A,n} = n\cdot\indic_{A} + (n+1) \cdot \indic_{A^c}$. 
Veamos que $T_{A, n}$ es tiempo de paro.\\

	Sea $k \in \N$.

\begin{itemize}
	\item 
		Si $k = n$, entonces 
		\begin{align}
			\{T_{A, n} = k\} 	&= \{T_{A, n} = n\} \\
								&= A \in F_n = F_k.
		\end{align}
		
	\item
		Si $k = n + 1$, entonces 
		\begin{align}
			\{T_{A, n} = k\} 	&= \{T_{A, n} = n + 1\} \\
								&= A^c \in F_n \subset \F_{n+1} = F_k.
		\end{align}

	\item 
		Si $k \not= n$ y $k \not= n+1$, entonces
		\begin{align}
			\{T_{A, n} = k\} 	&= \emptyset \in \F_k.
		\end{align}
\end{itemize}

	Entonces, $\{ T_{A,n} = k\} \in \F_k$ para toda $k \in \N$ y por lo tanto $T_{A,n}$ es tiempo de paro.\\

    Ahora notemos lo siguiente
    \begin{align}
    	X_{T_{A,n}} 	&=		\sum_{i \in \N} X_i \cdot  \indic_{\{T=i\}}						\\
    					&=		X_n \cdot  \indic_{\{T=n\}} + X_{n+1} \cdot  \indic_{\{T=n+1\}}	\\
    					&=		X_n \cdot  \indic_{A} + X_{n+1} \cdot  \indic_{A^c}.
    \end{align}
    
    Eso quiere decir que
    \begin{align}
    	\E(X_{T_{A,n}}) 	&=	\E(X_n \cdot  \indic_{A} + X_{n+1} \cdot  \indic_{A^c})		\\
							&=	\E(X_n \cdot  \indic_{A}) + \E(X_{n+1} \cdot  \indic_{A^c}).
    \end{align}
    
    Ahora tomemos la función constante $S_n = n+1$. Por ser constante \\
    $\{S_n = k\} \in \{ \Omega, \emptyset\} \subset \F_k$ para toda $k \in \N$
    y por lo tanto $S_n$ es tiempo de paro.\\
    
    Notemos entonces que
    \begin{align}
    	\E(X_{S_n}) 	&= 		\E(X_{n+1}) 													\\
    					&=		\E(X_{n+1} \cdot \indic_A + X_{n+1} \cdot \indic_{A^c}) 		\\
    					&=		\E(X_{n+1} \cdot \indic_A) +\E( X_{n+1} \cdot \indic_{A^c}). 
    \end{align}
    
    Dado que tanto $T_{a,n}$ como $S_n$ son variables que toman una cantidad finita de valores y por lo tanto son acotadas, 
    nuestra hipótesis dice que \\
    $\E(X_{S_n}) = \E(X_{T_{A,n}})$. \\
    
    Entonces
    \begin{align}
		\E(X_{n+1} \cdot \indic_A) +\E( X_{n+1} \cdot \indic_{A^c}) &=  \E(X_n \cdot  \indic_{A}) + \E(X_{n+1} \cdot  \indic_{A^c}).
    \end{align}
    
    De donde
    \begin{align}
			\E(X_{n+1} \cdot \indic_A) &=	\E(X_n \cdot  \indic_{A}).
    \end{align}
    
    Como esto lo hicimos para $n \in \N$ y $A \in \F_n$ arbitrarios, con esto terminamos la demostración de que $(X_n)_{n \in \N}$ es
    martingala.