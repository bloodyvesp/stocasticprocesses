Los problemas 1 y 2 se encuentran resueltos en [\ref{problema5_3:inciso2}] y [\ref{problema5_7}].

\section{Problema 3}
Sean $B_1, B_2, \dots$ variables aleatorias independientes con distribución Bernoulli de parámetro $p \in (0,1)$.
Sea $T_0 = 0$ y
\begin{align}
    T_{n+1} = \min\{k>T_n : B_k = 1\}
\end{align}
    \subsection{Inciso 1}
        \emph{
    Pruebe que $T_0, T_1,\dots$ son tiempos de renovación de un proceso de renovación aritmético.
}
\afterstatement\pn

    Pensemos en $B_1, B_2, \dots$ como volados con monedas cargadas donde $B_i = 1$ significa que ganamos el
    $i$-ésimo volado.\pn
    
    Con esta interpretación, $T_n$ no es otra cosa sino el tiempo en que ganamos por $n$-ésima vez. Por lo tanto\pn
    \begin{align}
        \P(T_{n} = k) = \binom{k-1}{n-1} p^n (1-p)^{k-n}
    \end{align}
    
    Es decir, $T_n$ tiene distribución binomial negativa de parámetros $n$ y $p$.\pn
    
    Definimos entonces a los tiempos de vida $S_0 = 0$ y $S_n = T_n - T_{n-1}$. Por tener $T_n$ y $T_{n-1}$ distribución
    binomial negativa de parámetros $n$, $p$ y $n-1$, $p$ respectivamente, tenemos que $S_n$ tiene distribución
    geométrica de parámetro $p$.\pn
    
    Otra manera de escribir a $S$ es
    \begin{align}
        S_n = \sum_{i = T_{n-1} + 1}^{T_n} i B_i
    \end{align}
    
    De donde tenemos que $S_n$ es suma de variables aleatorias independientes, todas distintas de las que suman $S_{n-1}$.
    Por lo tanto las $S_n$ son independientes y dado que $S_n$ tiene distribución geométrica de parámetro $p$ para toda
    $n$, tenemos que son idénticamente distribuidas.\pn
    
    Con esto, tenemos un proceso de renovación con tiempos de vida $S_n$ y tiempos de renovación $T_n$.

        \newpage
        
    \subsection{Inciso 2}
        \emph{
    Defina al proceso de contéo asociado $N$, expréselo en términos de las variables $B_1, B_2, \dots$ y pruebe
    diréctamente que $N_n/n$ converge casi seguramente conforme $n$ crece. Identifíque el límite.
}
\afterstatement\pn

Sea
\begin{align}
        N_n = \sum^n_{i = 1} B_i.
\end{align}

Y notemos que 
\begin{align}
        \min \{ m : T_{m+1} > n \} 
\end{align}
significa, el tiempo de volado más alto antes de pasarnos de $n$.\pn

Interpretado de esta forma tenemos que coinciden
\begin{align}
    \min \{ m : T_{m+1} > n \} = \sum^n_{i = 1} B_i = N_n.
\end{align}\pn

Por lo tanto tenemos que $N_n$ sí es un proceso de conteo para nuestro proceso de renovación.\pn

Por otra parte $N_n$ es suma de $n$ variables idependientes e idénticamente distribuidas
con esperanza $p$. $N_n / n$ es su promedio y la ley fuerte de los grandes números nos dice que
\begin{align}
    N_n/n \rightarrow p \;\; c.s.
\end{align}

        \newpage
        
    \subsection{Inciso 3}
        \emph{
    Calcule la distribución del tiempo residual al tiempo $n$
        \begin{align}
                R_n = \min\{k > n: B_k = 1\} - n
        \end{align}
    Asuma que dicho proceso es una cadena de Markov e identifique las probabilidades de transición.
    Pruebe que la distribución geométrica de parámetro $p$ es invariante para dicho proceso.
}

\afterstatement\pn

$R_n = k$ significa que si han pasado $n$ volados, en $k$ volados ganaremos uno, y antes de eso ninguno.
Dado que los $B_i$ son idénticamente distribuidos, tenemos que eso significa que
\begin{align}
        \P(R_n = k) = p (1-p)^{r-1}        
\end{align}\pn

Es decir, $R_n$ tiene distribución geométrica de parámetro $p$.\pn

Supongamos ahora que $R_n$ es una cadena de Markov. Y supongamos que al tiempo $n$, sabemos que
$R_n = k \neq 0$, es decir, nos faltan $k$ volados para volver a ganar. Entonces $R_{n+1} = k-1$ forzosamente.\pn
Entonces, en nuestra matriz de transición tendremos $P_{k, k-1} = 1$ cuando $k \neq 0$.\pn

Para el caso $R_n = 0$, la probabilidad de que pasemos de $0$ a $k>0$, es equivalente a perder los siguientes $k-1$ volados y luego
ganar el $k$-ésimo. Es decir, $P_{0, k} = p (1-p)^{k-1}$.\pn

Entonces la matriz de transición está determinada por
\begin{align}
        P_{i,j} =
                    \begin{cases}
                        1               &   \text{si $i > 0$ y $j = i - 1$}     \\
                        p (1-p)^{j-1}   &   \text{si $i = 0$ y $j>0$}           \\
                        0               &   \text{en cualquier otro caso}
                    \end{cases}
\end{align}\pn

Veamos que la distribución $\pi$, con $\pi_i = p (1-p)^{i-1}$ si $i>0$ y $\pi_0 = 0$ resulta invariante para este proceso.

\begin{align}
        (\pi \times P)_{j}      &=  \sum_{k \geq 0} \pi_{k} P_{k,j}     \\
                                &=  \pi_0 P_{0, j} + \pi_{j} P_{j+1, j} \\
                                &=  0 + \pi_{j} 1.
\end{align}\pn

Con lo que terminamos que $\pi$ es una distribución invariante para el proceso.
        \newpage
        
    \subsection{Inciso 4}
        \emph{
    Calcule la distribución del proceso de edad al tiempo $n$:
    \begin{align}
            A_n = n - \max{k \leq n: B_k = 1}
    \end{align}
    Calcula la distribución límite de $A_n$ conforme $n \rightarrow \infty$.
}

\afterstatement\pn

Supongamos $r < n$. Entonces $A_n = r$ significa que hace $r$ volados ganamos y que desde
entonces hemos perdido. Eso quiere decir que $\P(A_n = r) = p (1-p)^r$. Es decir que para el caso
$r < n$, $A_n$ se comporta como una variable con distribución geométrica de parámetro $p$.\pn

Para el caso $r = n$. $A_n = r$ significa que nunca hemos ganado un volado. Por lo tanto
$\P(A_n = r) = (1-p)^r$\pn

Recordemos para el caso $n < r$, tenemos que $\P(A_n = r) = 0$, pues estaríamos preguntando por la probabilidad
de haber perdido más veces que el número de volados jugados.\pn

Eso quiere decir que $A_n \sim Geo(p) \wedge n$.\pn

De esta última fórmula es fácil verificar que $A_n \rightarrow A_{\infty} \sim Geo(p)$.
        
\section{Problema 4}
Una matriz de transición $P$ se llama reversible si existe una distribución inicial $\pi$ tal que
\begin{align}
    \pi_i \P_{i,j} = \pi_j P_{j,i}.
\end{align}
A las identidades anteriores se les conoce como condición de balance detallado.
    \subsection{Inciso 1}
        \emph{
    Muestre que si $\pi$ y $P$ satisfacen la condición de balance detallado entonces $\pi$ es una
    distribución invariante para $P$.
}

\afterstatement\pn

Sencillamente verifiquemos la igualdad necesaria
\begin{align}
    (\pi \times P)_{j}  &= \sum{i \in E} \pi_{i}P_{i,j}     \\
                        &= \sum{i \in E} \pi_{j}P_{j,i}     \\
                        &= \pi_{j} \sum{i \in E} P_{j,i}    \\
                        &= \pi_{j} (1)                      \\
                        &= \pi_{j}.
\end{align}

Y con esto termina la demostración.
        \newpage
        
    \subsection{Inciso 2}
        \emph{
    Muestre que si $X$ es una cadena de Markov con distribución inicial $\pi$ y matriz de transición $P$ en balance detallado entonces
    \begin{align}
            (X_0, \dots, X_n) \sim (X_n, \dots, X_0).
    \end{align}
}

\afterstatement\pn

Por ser cadena de Markov con distribución inicial $\pi$, tenemos que
\begin{align}
        \P(X_0 = x_0, \dots, X_n = x_n) &=  \pi_{x_0} P_{x_0, x_1} P_{x_1, x_2} \dots P_{x_{n-1}, x_{n}}.
\end{align}\pn

Utilizando la hipótesis de que $\pi$ y $P$ están en balance detallado tenemos que
\begin{align}
    &   \pi_{x_0} P_{x_0, x_1} P_{x_1, x_2} \dots P_{x_{n-1}, x_{n}}    \\
    &=  P_{x_1, x_0} \pi_{x_1} P_{x_1, x_2} \dots P_{x_{n-1}, x_{n}}    \\
    &=  P_{x_1, x_0} P_{x_2, x_1} \pi_{x_2} \dots P_{x_{n-1}, x_{n}}    \\
    &\vdots                                                             \\
    &=  P_{x_1, x_0} P_{x_2, x_1} \dots P_{x_{n}, x_{n-1}} \pi_{x_n}    \\
    &= P(X_n = x_n, \dots, X_0 = x_0).
\end{align}\pn

Con lo que termina la demostración.
        \newpage
        
    \subsection{Inciso 3}
        \emph{
    Sea $G=(V,E)$ una gráfica y defina la matriz de transición $P$ sobre $V$ al estipular que
    $P_{x,y}$ es igual a 1 sobre el grado de $x$, denotado como $\delta_x$. Muestre que la distribución $\pi$
    Dada por $\pi_x = \frac{\delta_x}{\sum_{v \in V} \delta_v}$.
}

\afterstatement\pn

Si $x,y$ son tales que $\{x,y\} \not\in E$, tenemos que $P_{x,y} = 0 = P_{y,x}$. Por lo tanto
la condición de balance se da trivialmente en estos casos.\pn

Para, $x,y$ tales que $\{x, y\} \in E$, sencillamente verifiquemos la igualdad pertinente
\begin{align}
        \pi_x P_{x,y}   &=  \frac{\delta_x}{\sum_{v \in V} \delta_v} \frac{1}{\delta_x}          \\
                        &=  \frac{1}{\sum_{v \in V} \delta_v}                                    \\
                        &=  \frac{\delta_y}{\delta_y} \frac{1}{\sum_{v \in V} \delta_v}          \\
                        &\comment{es válido porque $\delta_y \geq 1$, pues $x$ es vecino de $y$}\\
                        &=  \frac{1}{\delta_y} \frac{\delta_y}{\sum_{v \in V} \delta_v}          \\
                        &=  P_{y,x} \pi_y.
\end{align}
Con lo que termina la demostración.
        \newpage
        
    \subsection{Inciso 4}
        \emph{
    Muestre que si $\pi$ y $P$ satisfacen las ecuaciones de balance detallado, entonces existe una medida de probabilidad
    $\P$ en $E^{\Z}$ tal que (con la noción adecuada deproceso canónico $X$),
    \begin{align}
            \P(X_m = x_m, \dots, X_n = x_n) = \pi_{x_m} P_{x_m, x_{m+1}} \cdots P_{x_{n-1}, x_{n}}
    \end{align}
    para cualesquier $m,n \in \Z$ con $m \leq n$ y para cualesquiera $x_m, \dots, x_n \in E$.
}

\afterstatement\pn


Notemos que si $\pi_{x_r} = 0$ con $m \leq r \leq n$, la probabilidad de arriba se hace $0$ gracias a la condición de
balance detallado (podemos ir ``voletando'', las entradas de $P$ hasta que eventualmente se atravezará $\pi_{x_r}$ en el producto).\pn

Entonces nos enfocaremos en $E' = \{x \in E : \pi_x > 0 \}$.\pn

Para cualesquier $m \leq n \in \Z$ y $x_m, x_n \in E'$, definamos $\P$ como
\begin{align}
    \P(X_m = x_m, X_n = x_n)    &= \sum_{x_{m+1}, x_{m+2},\dots x_{n-1} \in E'}   \P(X_m = x_m, \dots, X_n = x_n)                        \\
                                &= \sum_{x_{m+1}, x_{m+2},\dots x_{n-1} \in E'}   \pi_{x_m} P_{x_m, x_{m+1}} \cdots P_{x_{n-1}, x_{n}}.   \\
\end{align}\pn

Para $l \leq m \leq n \in \Z$ y $x_l, x_m, x_n \in E'$,
\begin{align}
    &\P(X_l = x_l, X_m = x_m, X_n = x_n) \\
    &=  \sum_{x_{l+1}, \dots, x_{m-1} x_{m-1},\dots x_{n-1} \in E'}   \P(X_l = x_l, \dots, X_m = x_m, \dots, X_n = x_n)
\end{align}\pn

 y así sucesivamente para ${n_1} \leq {n_2} \leq \dots \leq {n_z} \in \Z$ y $x_{n_1}, x_{n_2}, \dots, x_{n_z} \in E'$. Lo que estamos haciendo
es completar los huecos sumando sobre todos los valores posibles que dichos huecos puedan tener.\pn

Es fácil verificar que esta definición extiende a la mencionada en el enunciado. Para los casos donde tomamos infinitos $(n_i)'s \in \Z$ y su complemento
es finito, definimos a su probabilidad como $1$, menos la probabilidad de su complemento, que sí sabemos calcular.\pn

Definimos una $\sigma$-álgebra para $E^\Z$ donde los conjuntos de índices en $\Z$ tienen que ser finitos, o sus 
compelmentos tienen que ser finitos, de manera que las definiciones que hemos hecho hasta ahora, nos basten para 
calcular a los elementos de dicha $\sigma$-álgebra y que contengan a los casos mencionados en el enunciado.

Lo que hemos hecho hasta ahora no ha sido otra cosa más que forzar a $\P$ a satisfacer las hipótesis de ser medida de probabilidad.\pn
Con la primera parte garantizamos que la probabilidad del total sea $1$ y que $\P$ se porte bien sobre uniones disjuntas, con la segunda parte que 
los complementos se comporten bien bajo la medida de probabilidad.


        \newpage
        
    \subsection{Inciso 5}
        \emph{
    Muestre que si $\tilde X_n = X_{-n}$, entonces $\tilde X$ y $X$ tienen la misma distribución bajo $\P$.
}

\afterstatement\pn

Dadas las condiciones de balance detallado, y la ``condición de Markov'' dada para $\P$ en el inciso anterior, 
este ejercicio es exáctamente el mismo que el del segundo inciso de este problema y la demostración es 
completamente análoga (la nombrada ``condición de Markov'' fue la única parte de ser cadena de Markov que se utilizó en $2$).
        \newpage