\emph{
	\begin{align}
		\F_T=\set{A\in\F:A\cap\set{T\leq n}\in\F_n\text{ para toda } n}
	\end{align}
	es una ${\sigma}$-\'algebra, 
	a la que nos referimos como la ${\sigma}$-\'algebra detenida en ${\tau}$. Comente qu\'e puede fallar si ${T}$ no es tiempo de paro. 
	Pruebe que ${T}$ es ${\F_T}$-medible.\\		
}			
		
	Primero hay que demostrar que ${\F_T}$ es ${\sigma}$-algebra.\\
	
	\begin{itemize}
		\item ${\Omega \in \F_T}$. \\
		
			Notemos que 
			\begin{align}
				\Omega \cap \{T \leq n\} = \{T \leq n\} \in \F_n.
			\end{align}
			Donde la pertenencia a ${\F_n}$ es gracias a que ${T}$ es tiempo de paro. (Esta parte podria fallar si 
			${T}$ no fuera tiempo de paro).\\
		
		\item ${\F_T}$ es cerrado bajo complementación.\\
		
			Sea ${A \in \F_T}$. Eso significa que para todo ${n \in  \N}$ ocurre que ${B = A \cap \{ T \leq n \} \in \F_n}$. 
			Por ser ${\F_n}$ una ${\sigma}$-algebra tenemos que el complemento de ${B}$ también debe estar en ${\F_n}$. 
			Escrito en símbolos:
			
			\begin{align}
				B^c 	&= (A   \cap \{ T \leq n \})^c \\
						&=  A^c \cup \{ T > n \} \in \F_n
			\end{align}
			
			Dado que ${B^c}$ y ${\{ T \leq n \}}$ se encuentran en ${\F_n}$, también ${B^c \cap \{ T \leq n \} \cap \{ T \leq n \}}$.
			
			Escribiendo esto último de otra manera:
			\begin{align}
				B^c \cap \{ T \leq n \} \cap \{ T \leq n \} 	&=		A^c \cup \{ T > n \} \cap \{ T \leq n \} \cap \{ T \leq n \} \\
																&= 		A^c \cap \{ T \leq n \} \cup \{ T > n \} \cap \{ T \leq n \} \\
																&= 		\bigg(A^c \cap \{ T \leq n \}\bigg) 
																			\bigcup 
																		\bigg(\{ T > n \} \cap \{ T \leq n \}\bigg) \\
																&=		\bigg(A^c \cap \{ T \leq n \}\bigg)	\bigcup \emptyset \\
																&=		A^c \cap \{ T \leq n \} \in \F_n
			\end{align}
			
			Y por lo tanto ${A^c \in \F_T}$.\\
			
		\item ${\F_T}$ es cerrado bajo uniones numerables.
		
			Sean ${A_m \in \F_T}$ para ${m \in \N}$.
			\begin{align}
				\bigg( \bigcup_{k=1}^\infty A_k \bigg) \cap \{ T \leq n \} 	&=	\bigcup_{k=1}^\infty \bigg( A_k \cap \{ T \leq n \} \bigg)
			\end{align}								
			
			La última parte, por definición de ${\F_T}$ es una unión numerable de elementos de ${\F_n}$ y por lo tanto, dicha unión
			tambien pertenece a ${\F_n}$. Y por último ${\cup A_k \in \F_T}$.
	\end{itemize}
	
	Ahora veamos que $T$ es $\F_T$-medible. Dado un $k \in \N$, ocurre que:
	
	\begin{align}
		\{T \leq k\} \cap \{T \leq n\} = \{T \leq min(k,n)\} \in \F_{min(k, n)} \subset \F_n \forall n \in \N.
	\end{align}

	Y por lo tanto $\{T \leq k\} \in \F_T$.
	