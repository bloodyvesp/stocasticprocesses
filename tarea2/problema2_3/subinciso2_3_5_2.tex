\emph{
    Pruebe que si $M$ es una martingala nula en cero tal que para algunas constantes $\paren{c_n,n\in\na}$ se tiene que
    \begin{align}
        \abs{M_n-M_{n-1}}\leq c_n\quad\forall n                                             \label{problema2_3:cota_para_diferencias}
    \end{align}
    entonces, para $x>0$
    \begin{align}
        \proba{\max_{k\leq n} M_k\geq x}\leq \imf{\exp}{\frac{x^2}{2\sum_{k=1}^n c_k^2}}.   \label{problema2_3:desiguladad_por_demostrar}
    \end{align}
}

\afterstatement\par\null

Lo primero que sugiere \eqref{problema2_3:desiguladad_por_demostrar} es utilizar la desigualdad maximal de Doob que dice que si $N=(N_n)_{n\in\N}$
es una $(\G_n)_{n\in\N}$-submartingala, entonces para toda $\lambda > 0$ se tiene

\begin{align}
        \lambda \mw\left(\overline{N}_n^+ > \lambda \right) \leq \E(N_n^+) \label{problema2_3:desigualdad_maximal_de_Doob}
\end{align}

Donde $\overline{N}_n^+ = \max\limits_{1 \leq i \leq n} N_n^+$.\par\null

Ahora construiremos una submartingala conveniente. En clase se vio que por ser $M$ es una martingala, al aplicarle la función convexa $e^{\theta y}$, el
proceso resultante $(e^{\theta M_n})_{n \in \N}$ es una submartingala positiva con respecto de la filtración $(\F_n)_{n\in\N}$.\par\null

Aplicando \eqref{problema2_3:desigualdad_maximal_de_Doob} con $\lambda = e^{\theta x}$ obtenemos

\begin{align}
    e^{\theta x} \mw\left( \max_{1 \leq k \leq n} e^{\theta M_k} > x \right) \leq \E(e^{\theta M_n})
\end{align}\par\null
 
Y por lo tanto

\begin{align}
    \mw\left( \max_{1 \leq k \leq n} e^{\theta M_k} > e^{\theta x} \right) \leq e^{- \theta x} \E(e^{\theta M_n})
\end{align}\par\null

Ahora notemos que si $\theta > 0$, entonces $e^{\theta y}$ es inyectiva y por lo tanto los eventos
$\{ \max_{1 \leq i \leq n} e^{\theta M_k} > e^{\theta x} \}$ y $\{ \max_{1 \leq i \leq n} M_k > x \}$ son iguales.
Supongamos entonces que $\theta > 0$ y por lo tanto

\begin{align}
    \mw\left( \max_{1 \leq k \leq n} M_k > x \right) \leq e^{- \theta x} \E(e^{\theta M_n})
\end{align}\par\null

Ahora, gracias a \eqref{problema2_3:cota_para_diferencias} tenemos que

\begin{align}
        |M_n|   &=      \bigg| \sum_{1 \leq k \leq n} M_k - M_{k-1} \bigg|       \\
                &\leq   \sum_{1 \leq k \leq n} \bigg| M_k - M_{k-1} \bigg|       \\
                &\leq   \sum_{1 \leq k \leq n} c_k.                               
\end{align}

Entonces, gracias a lo demostrado en [\ref{problema2_3:subinciso5_1}] y usando que \par$\sum_{1 \leq k \leq n} c_k^2 \leq (\sum_{1 \leq k \leq n} c_k)^2$ tenemos

\begin{align}
    e^{\theta M_n}  &\leq   e^{\left( \frac{1}{2} \theta^2 \left( \sum_{1 \leq k \leq n} c_k \right)^2\right)}  \\
                    &\leq   e^{\left( \frac{1}{2} \theta^2 \left( \sum_{1 \leq k \leq n} c_k^2 \right)\right)}.
\end{align}\par\null

Definamos entonces

\begin{align}
    \theta &= \frac{x}{\sum_{1 \leq k \leq n} c_k^2}.
\end{align}\par\null

Definido así, $\theta > 0$ y por lo tanto, todo lo mencionado anteriormente es válido. Conectemos todo lo dicho hasta ahora.\par\null

\begin{align}
    \mw\left( \max_{1 \leq k \leq n} M_k > x \right)    &\leq   e^{- \theta x} \E(e^{\theta M_n})                                                                                                                                                           \\
                                                        &\leq   e^{- \theta x} e^{\left( \frac{1}{2} \theta^2 \left( \sum_{1 \leq k \leq n} c_k^2 \right)\right)}                                                                                           \\
                                                        &=      e^{- \left(\frac{x}{\sum_{1 \leq k \leq n} c_k^2}\right) x} e^{\left( \frac{1}{2} \left(\frac{x}{\sum_{1 \leq k \leq n} c_k^2}\right)^2 \left( \sum_{1 \leq k \leq n} c_k^2 \right)\right)} \\
                                                        &=      e^{- \left(\frac{x^2}{2 \sum_{1 \leq k \leq n} c_k^2}\right)}.
\end{align}\par\null

Que es precisamente lo que búscabamos demostrar.

