\emph{
	Pruebe que si $S\leq T\leq n$ entonces $\F_S\subset\F_T$. Si además $T$ es acotado entonces $M_S,M_T \in L_1$ y 
	\begin{align}
		&\espc{M_T}{\F_S} = M_S \label{problema1_4:caso_martingala} 			\\
		&\mbox{(En el caso de que M sea martingala)} 
		\\
		\\
		&\espc{M_T}{\F_S}\leq M_S \label{problema1_4:caso_supermartingala}	\\
		&\mbox{(En el caso de que M sea supermartingala)} 
	\end{align}	
}

\afterstatement\par\null

	Primero probaremos que $\F_S \subset \F_T$. Sea $n \in \N$. Notemos que $\{ T \leq n \} \subset \{ S \leq n \}$ pues
	si $\omega \in \{ T \leq n \}$, entonces $S(\omega) \leq T(\omega) \leq n $ y por lo tanto $\omega \in \{ S \leq n \}$.\par\null

	Ahora sea $A \in \F_S$. Entonces

	\begin{align}
		A \cap \{ T \leq n \} 	&=		A  \cap \{ T \leq n \} \cap \{ S \leq n \} \\
								&=		(A  \cap \{ S \leq n \}) \cap \{ T \leq n \} \in \F_n
	\end{align}\par\null

	Como escogimos $n$ arbitrariamente, esto es cierto para toda $n \in \N$. Y por lo tanto $A \in \F_T$ y $\F_S \subset \F_T$.\par\null

	Ahora supongamos que $T$ es acotado y sea $N$ una cota para $T$. Demostraremos que $M_S,M_T \in L_1$.
	
	\begin{align}
		\E(|M_T|) 	&=		\E\bigg(	\bigg| \sum_{i \in \N} M_i \cdot \indic_{T = i} \bigg|	\bigg) \\ 
					&\leq	\E\bigg(	\sum_{i \in \N} |M_i \cdot \indic_{T = i} |				\bigg) \\
					&\leq	\E\bigg(	\sum_{i \in \N} |M_i|									\bigg) \\
					&=		\E\bigg(	\sum_{i \leq N} |M_i|									\bigg) \\
					&< 		\infty
	\end{align}\par\null
	
	Y con esto queda demostrado que $M_T \in L_1$. La demostración de que $M_S \in L_1$ es completamente análoga.\par\null
	
	Nos queda por demostrar \eqref{problema1_4:caso_martingala} y \eqref{problema1_4:caso_supermartingala}.\par\null
	
	Demostraremos primero \eqref{problema1_4:caso_martingala}.\par\null
	
	 Dado que $S$ es finito (en particular, es acotado), por lo visto en [\ref{problema1_4:inciso3}], 
	 tenemos que $S$ es $\F_S$-medible. Así que únicamente falta demostrar que para todo $A \in \F_S$ se tiene que 
	 $\E(M_T \cdot \indic_A) = \E(M_S \cdot \indic_A)$ lo cual es equivalente a que $\E((M_T - M_S) \cdot \indic_A) = 0$.
	 Intentaremos demostrar esto último.\par\null
	 
	 Sea $A \in \F_S$.\par\null
	 
	 Notemos que	 
    
	 \begin{align}
	 	M_T - M_S &= \sum_{ i \in \N} ( M_{i+1} - M_i) \cdot \indic_{S \leq i < T }. 
	 \end{align}\par\null

	 También notemos que 
    
	 \begin{align}
	 	\{S \leq i < T \} \cap A 	&=		\{S \leq i < T \} \cap A 						\\
	 								&=		\{S \leq i \} \cap \{ i < T \} \cap A 			\\
	 								&=		(\{S \leq i \} \cap A) \cap \{ i < T \}.  		\\
	 								&=		(\{S \leq i \} \cap A) \cap \{ T \leq i \}^c.
	 \end{align}\par\null
    
     De esta última parte hay que puntualizar que $(\{S \leq i \} \cap A) \in \F_i$ por que $A \in \F_S$ y también
	 hay que puntualizar que $\{ T \leq i + 1 \}^c \in F_{i}$.\par\null
	 
	 Con estos detalles listos procedemos a la parte que nos interesa:
	 
	 \begin{align}
	 	\E((M_T - M_S) \cdot \indic_A)	
	 									&=		
	 	\E\bigg( \sum_{ i \in \N} ( M_{i+1} - M_i) \cdot \indic_{S \leq i < T } \cdot \indic_A \bigg)									\\
	 									&=		
	 	\E	\bigg( 
	 			\sum_{ i \in \N} 
	 				( M_{i+1} - M_i) \cdot \indic_{(\{S \leq i \} \cap A)} \cdot \indic_{\{ T \leq i \}^c} 
	 		\bigg)																														\\
	 									&=		
	 	\sum_{ i \in \N} 
	 		\E	\bigg(
	 				( M_{i+1} - M_i) \cdot \indic_{(\{S \leq i \} \cap A)} \cdot \indic_{\{ T \leq i \}^c} 
	 			\bigg)																													
	\end{align}
	\begin{align}
	 									&=		
	 	\sum_{ i \in \N} 
	 		\E	\bigg(
		 			\E	\bigg(
		 					( M_{i+1} - M_i) \cdot \indic_{(\{S \leq i \} \cap A)} \cdot \indic_{\{ T \leq i \}^c}  \bigg| F_i	
						\bigg)
	 			\bigg)																													\\
										&=
	 	\sum_{ i \in \N} 
	 		\E	\bigg(
					\indic_{(\{S \leq i \} \cap A)} \cdot \indic_{\{ T \leq i \}^c} \cdot
		 			\E	\bigg(
		 					 ( M_{i+1} - M_i)  \bigg| F_i	 												
						\bigg)
	 			\bigg)																													\\
										&=
	 	\sum_{ i \in \N} 
	 		\E	\bigg(
					\indic_{(\{S \leq i \} \cap A)} \cdot \indic_{\{ T \leq i \}^c} \cdot
		 			\E	(
		 					 M_{i+1} | F_i	 												
						)
					-
					\E	(
		 					 M_i  | F_i	 												
						)
	 			\bigg)																													\\
										&=
	 	\sum_{ i \in \N} 
	 		\E	\bigg(
					\indic_{(\{S \leq i \} \cap A)} \cdot \indic_{\{ T \leq i \}^c} \cdot
		 					 M_i
					-
					\E	(
		 					 M_i  | F_i	 												
						)
	 			\bigg)\label{problema1_4:aplicacion_de_que_M_es_martingala}																\\
										&\;\;\;\;\;\mbox{(Esto último por ser M una martingala.)}									    \\
										&=
	 	\sum_{ i \in \N} 
	 		\E	\bigg(
					\indic_{(\{S \leq i \} \cap A)} \cdot \indic_{\{ T \leq i \}^c} \cdot
		 			 M_i
						-
					 M_i
	 			\bigg)																													\\
										&\;\;\;\;\;\mbox{(Esto último por que $M_i$ es $F_i$-medible.)}								\\
										&=
		\sum_{ i \in \N} 
	 		\E	\bigg(
					\indic_{(\{S \leq i \} \cap A)} \cdot \indic_{\{ T \leq i \}^c} \cdot
		 			 0
	 			\bigg)																													\\
										&=
		0
	 \end{align}\par\null
	
	Que es justo lo que queríamos demostrar.\par\null
	
	Para demostrar \eqref{problema1_4:caso_supermartingala}. Hay que notar que $ \espc{M_T}{\F_S}\leq M_S $ es cierto sí y sólo sí para todo $A \in \F_S$
	se tiene que $ \E((M_T - M_S) \indic_{A}) \leq 0 $.\par\null
	
	Demostrar esto último lo podemos hacer de manera análoga a como demostramos el caso en el que $M$ es martingala. En las cuentas previamente hechas,
	utilizando el hecho de que $M$ es supermartingala en vez de martingala, la iguladad de \eqref{problema1_4:aplicacion_de_que_M_es_martingala} 
	se transforma en una desigualdad de tipo ``$\leq$". El resto se queda exactamente igual y entonces tenemos la demostración que necesitabamos.