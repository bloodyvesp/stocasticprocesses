\emph{
    Sea $\imf{H_n}{x}=\paren{-1}^ne^{x^2/2}\frac{d^n}{dx^n}e^{-x^2/2}$.
    A $H_n$ se le conoce como en\'esimo polinomio de Hermite. Calc\'ulelo para 
    $n\leq 5$. Pruebe que $H_n$ es un polinomio para toda $n\in\na$ y que 
    $\partial^n M^\lambda_t/\partial \lambda^n=t^{n/2}\imf{H_n}{B_t/\sqrt{t}}M^\lambda_t$.\pn
}

\afterstatement\pn

\begin{align}
        H_1(x)  &=  x                       \\
        H_2(x)  &=  x^2 - 1                 \\
        H_3(x)  &=  x^3 - 3x                \\
        H_4(x)  &=  x^4 - 6x^2 + 3          \\
        H_5(x)  &=  x^5 - 10x^3 + 15x       \\
\end{align}

Supongamos que para cierto $n \in \N$ tenemos que $H_n$ es un polinomio
de orden $n$. Es decir, exísten $r_0, r_1, r_2, \dots, r_n \in \R$ tales que
\begin{align}
        H_n &=  r_n x^n + \dots + r_1 x + r_0. 
\end{align}

Entonces
\scriptsize
\begin{align}
    H_{n+1} &=  (-1)^{n+1}  e^{x^2/2}   \frac{d^n}{dx^{n+1}}e^{-x^2/2}                                                                              \\
            &=  (-1)^{n+1}  e^{x^2/2}   \frac{d}{dx}\left(\frac{d^n}{dx^{n}}e^{-x^2/2}\right)                                                       \\
            &=  (-1)^{n+1}  e^{x^2/2}   \frac{d}{dx}\left((-1)^n e^{-x^2/2}(r_n x^n + \dots + r_1 x + r_0)\right)                                   \\
            &\comment{hemos despejado la derivada de orden $n$ en terminos del supuesto polinomio $H_n$}                                            \\
            &=  (-1)^{n+1}  e^{x^2/2} (-1)^n \frac{d}{dx}\left(e^{-x^2/2}(r_n x^n + \dots + r_1 x + r_0)\right)                                     \\
            &=  (-1)^{n+1}  e^{x^2/2} (-1)^n \left(-x e^{-x^2/2}(r_n x^n + \dots + r_1 x + r_0) + e^{-x^2/2}(r_n n x^{n-1} + \dots + r_1)\right)    \\
            &=  (-1)^{n+1}  e^{x^2/2} e^{-x^2/2} (-1)^n \left(-x (r_n x^n + \dots + r_1 x + r_0) + (r_n n x^{n-1} + \dots + r_1)\right)             \\
            &=  (-1)^{n+1} (-1)^n \left(-x (r_n x^n + \dots + r_1 x + r_0) + (r_n n x^{n-1} + \dots + r_1)\right)                                   \\
            &=  (-1) \left(-x (r_n x^n + \dots + r_1 x + r_0) + (r_n n x^{n-1} + \dots + r_1)\right)                                                \\
            &=  (-1) \left(-(r_n x^{n+1} + \dots + r_1 x^2 + r_0 x) + (r_n n x^{n-1} + \dots + r_1)\right)                                          \\
            &=  (r_n x^{n+1} + \dots + r_1 x^2 + r_0 x) - (r_n n x^{n-1} + \dots + r_1)                                                             \\
            &=  r_n x^{n+1} + \dots + (r_{m-2} - m r_m) x^{m-1} + \dots - r_1                                                                       \\
\end{align}
\normalsize

Hemos demostrado que si $H_n$ es un polinomio de grado $n$, entonces $H_{n+1}$ es 
uno de grado $n+1$. Lo que demostramos al inicio de este inciso no es otra 
cosa sino la base de inducción. Por lo tanto tenemos que para todo $n \in \N$, 
$H_n$ es un polinomio de grado $n$.\pn

Para verificar la equivalencia $d^n M^\lambda_t/d \lambda^n=t^{n/2}\imf{H_n}{(B_t - \lambda t)/\sqrt{t}}M^\lambda_t$, basta sustituir en la ecuación anterior
$B_t^\lambda$ (definido en el inciso anterior) en vez de $x$, y comparar con $\eqref{problema6_4:equivalencia_recursiva}$.\pn 

Nótese que está mal escrito el enunciado del problema, en vez de escribir $B_t - \lambda t$ se escribe únicamente $B_t$. Es necesario hacer la aclaración
$d^n M^\lambda_t/d \lambda^n \bigg|_{\lambda = 0}=t^{n/2}\imf{H_n}{(B_t)/\sqrt{t}}$ (la cual es fácil de verificar al evaluar), 
