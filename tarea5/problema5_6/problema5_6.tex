\begin{problema}[Procesos de ramificaci\'on a tiempo continuo]
Sea $\mu$ una distribuci\'on en $\na$. A $\mu_k$ lo interpretamos como la probabilidad de que un individuo tenga $k$ hijos. Nos imaginamos la din\'amica de la poblaci\'on como sigue: a tasa $\lambda$, los individuos de una poblaci\'on se reproducen. Entonces tienen $k$ hijos con probabilidad $\mu_k$. Se pueden introducir dos modelos: uno en que el individuo que se reproduce es retirado de la poblaci\'on (nos imaginamos que muere) y otro en que no es retirado de la poblaci\'on (por ejemplo cuando se interpreta a la poblaci\'on como especies y a sus descendientes como mutaciones). En el caso particular del segundo modelo en que $\mu_1=1$, se conoce como proceso de Yule. 
\begin{enumerate}
\item Especifique un modelo de cadenas de Markov a tiempo continuo para cada uno de los modelos anteriores. A estos procesos se les conoce como procesos de ramificaci\'on a tiempo continuo.
\end{enumerate}
Nuestro primer objetivo ser\'a encontrar una relaci\'on entre procesos de ramificaci\'on a tiempo continuo y procesos de Poisson compuestos. Sea $N$ un proceso de Poisson  y $S$ una caminata aleatoria independiente de $N$ tal que $\proba{S_1=j}=\mu_{j-1}$ \'o $\mu_{j}$ dependiendo de si estamos en el primer caso o en el segundo. Sea $k\geq 0$ y definamos a $X_t=k+S_{N_t}$.
\begin{enumerate}[resume]
\item Diga brevemente por qu\'e $X$ es una cadena de Markov a tiempo continuo e identifique su matriz infinitesimal para ambos modelos.
\end{enumerate}
Sea ahora $\tau=\min\set{t\geq 0: X_t=0}$ y $Y_t=X_{t\wedge \tau}$. 
\begin{enumerate}[resume]
\item Argumente por qu\'e $Y$ es una cadena de Markov a tiempo continuo e identifique su matriz infinitesimal.
\item Argumente por qu\'e existe un \'unico proceso $Z$ que satisface\begin{esn}
Z_t=Y_{\int_0^t Z_s\, ds}
\end{esn}y que dicho proceso es un proceso de ramificaci\'on a tiempo continuo. Sugerencia: Recuerde que las trayectorias de $Y$ son constantes por pedazos.
\end{enumerate}
Ahora nos enfocaremos en el proceso de Yule. 
\begin{enumerate}[resume]
\item Escriba las ecuaciones backward de Kolmogorov para las probabilidades de transici\'on $\imf{P_t}{x,y}$. Al argumentar por qu\'e $\imf{P_{t}}{x,x}=e^{-\lambda x}$, resuelva las ecuaciones backward por medio de la t\'ecnica de factor integrante (comenzando con $\imf{P_t}{x,x+1}$) y pruebe que\begin{esn}
\imf{P_t}{x,y}=\binom{y-1}{y-x} e^{-\lambda x t}\paren{1-e^{-\lambda t}}^{y-x}.
\end{esn}
\item Al utilizar la f\'ormula para la esperanza de una variable binomial negativa, pruebe que\begin{esn}
\imf{\se_x}{Z_t}= xe^{\lambda t}.
\end{esn}
\item Pruebe que $e^{-\lambda t}Z_t$ es una martingala no-negativa y que por lo tanto converge casi seguramente a una variable aleatoria $W$.
\item Al calcular la transformada de Laplace de $e^{-\lambda t}Z_t$, pruebe que $W$ tiene distribuci\'on exponencial. Por lo tanto, argumente que casi seguramente $Z$ crece exponencialmente.
%La distribuci�n l�mite est� tomada de Beroin-Goldschmidt, ellos citan y corrigen un error de Athreya.
\end{enumerate}
\end{problema}
