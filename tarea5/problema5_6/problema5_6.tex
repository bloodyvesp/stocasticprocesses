\begin{problema}[Procesos de ramificaci\'on a tiempo continuo]
    Sea $\mu$ una distribuci\'on en $\na$. A $\mu_k$ lo interpretamos como la probabilidad de 
    que un individuo tenga $k$ hijos. Nos imaginamos la din\'amica de la poblaci\'on como sigue: 
    a tasa $\lambda$, los individuos de una poblaci\'on se reproducen. Entonces tienen $k$ hijos 
    con probabilidad $\mu_k$. Se pueden introducir dos modelos: uno en que el individuo que se 
    reproduce es retirado de la poblaci\'on (nos imaginamos que muere) y otro en que no es retirado 
    de la poblaci\'on (por ejemplo cuando se interpreta a la poblaci\'on como especies y a sus 
    descendientes como mutaciones). En el caso particular del segundo modelo en que $\mu_1=1$, 
    se conoce como proceso de Yule. 

    \begin{enumerate}
        \item[(i)]      [\ref{problema5_6:inciso1}]
            Especifique un modelo de cadenas de Markov a tiempo continuo para cada uno de los
            modelos anteriores. A estos procesos se les conoce como procesos de ramificaci\'on
            a tiempo continuo.\pn
    \end{enumerate}

        Nuestro primer objetivo ser\'a encontrar una relaci\'on entre procesos de ramificaci\'on a
        tiempo continuo y procesos de Poisson compuestos. Sea $N$ un proceso de Poisson  y $S$ una
        caminata aleatoria independiente de $N$ tal que $\proba{S_1=j}=\mu_{j-1}$ \'o $\mu_{j}$
        dependiendo de si estamos en el primer caso o en el segundo. Sea $k\geq 0$ y definamos a
        $X_t=k+S_{N_t}$.
        
    \begin{enumerate}[resume]
        \item[(ii)]     [\ref{problema5_6:inciso2}] 
            Diga brevemente por qu\'e $X$ es una cadena de Markov a tiempo continuo e identifique 
            su matriz infinitesimal para ambos modelos.\pn
    \end{enumerate}

        Sea ahora $\tau=\min\set{t\geq 0: X_t=0}$ y $Y_t=X_{t\wedge \tau}$. 
        
    \begin{enumerate}[resume]
        \item[(iii)]    [\ref{problema5_6:inciso3}] 
            Argumente por qu\'e $Y$ es una cadena de Markov a tiempo continuo e identifique su 
            matriz infinitesimal.\pn
            
        \item[(iv)]     [\ref{problema5_6:inciso4}] 
            Argumente por qu\'e existe un \'unico proceso $Z$ que satisface
            \begin{esn}
                Z_t=Y_{\int_0^t Z_s\, ds}
            \end{esn}
            y que dicho proceso es un proceso de ramificaci\'on a tiempo continuo. Sugerencia: Recuerde que las 
            trayectorias de $Y$ son constantes por pedazos.\pn
    \end{enumerate}

    Ahora nos enfocaremos en el proceso de Yule. 

    \begin{enumerate}[resume]
        \item[(v)]      [\ref{problema5_6:inciso5}]
            Escriba las ecuaciones backward de Kolmogorov para las probabilidades de transici\'on 
            $\imf{P_t}{x,y}$. Al argumentar por qu\'e $\imf{P_{t}}{x,x}=e^{-\lambda x}$, resuelva 
            las ecuaciones backward por medio de la t\'ecnica de factor integrante (comenzando con 
            $\imf{P_t}{x,x+1}$) y pruebe que
            \begin{esn}
                \imf{P_t}{x,y}=\binom{y-1}{y-x} e^{-\lambda x t}\paren{1-e^{-\lambda t}}^{y-x}.
            \end{esn}\pn
        
        \item[(vi)]     [\ref{problema5_6:inciso6}]
            Al utilizar la f\'ormula para la esperanza de una variable binomial negativa, 
            pruebe que
            \begin{esn}
                \imf{\se_x}{Z_t}= xe^{\lambda t}.
            \end{esn}\pn
        
        \item[(vii)]    [\ref{problema5_6:inciso7}]
            Pruebe que $e^{-\lambda t}Z_t$ es una martingala no-negativa y que por lo tanto 
            converge casi seguramente a una variable aleatoria $W$.\pn
        
        \item[(viii)]   [\ref{problema5_6:inciso8}]
            Al calcular la transformada de Laplace de $e^{-\lambda t}Z_t$, pruebe que $W$ tiene 
            distribuci\'on exponencial. Por lo tanto, argumente que casi seguramente $Z$ crece exponencialmente.
            %La distribuci�n l�mite est� tomada de Beroin-Goldschmidt, ellos citan y corrigen un error de Athreya.
            \pn
    \end{enumerate}
\end{problema}

\begin{proof}
    \subsection{Inciso (i)} \label{problema5_6:inciso1}
    \emph{
    Especifique un modelo de cadenas de Markov a tiempo continuo para cada uno de los
    modelos anteriores. A estos procesos se les conoce como procesos de ramificaci\'on
    a tiempo continuo.\pn
}
\afterstatement\pn
    \newpage

    \subsection{Inciso (ii)} \label{problema5_6:inciso2}
    \emph{
    Diga brevemente por qu\'e $X$ es una cadena de Markov a tiempo continuo e identifique 
    su matriz infinitesimal para ambos modelos.\pn
}
\afterstatement\pn
    \newpage

    \subsection{Inciso (iii)} \label{problema5_6:inciso3}
    \emph{
    Argumente por qu\'e $Y$ es una cadena de Markov a tiempo continuo e identifique su 
    matriz infinitesimal.\pn
}
\afterstatement\pn
	\newpage
	
    \subsection{Inciso (iv)} \label{problema5_6:inciso4}
    \emph{
    Argumente por qu\'e existe un \'unico proceso $Z$ que satisface
    \begin{esn}
        Z_t=Y_{\int_0^t Z_s\, ds}
    \end{esn}
    y que dicho proceso es un proceso de ramificaci\'on a tiempo continuo. Sugerencia: Recuerde que las 
    trayectorias de $Y$ son constantes por pedazos.\pn
}

\afterstatement\pn

Definamos $T_n$ como en [\ref{problema5_6:inciso1}]. Nos interesa ver cómo se comporta $Y_t$ en los intervalos
$[T_{n-1}, T_{n})$, pues en estos intervalos $Y_t$ es constante.\pn

Analizando el primer segumento, sea $t \in [T_{0}, T_{1})$, donde la población no cambia, tenemos que 
$int_{0}^{t} Y_t = tk$
    \newpage
    
    \subsection{Inciso (v)} \label{problema5_6:inciso5}
    \emph{
    Escriba las ecuaciones backward de Kolmogorov para las probabilidades de transici\'on 
    $\imf{P_t}{x,y}$. Al argumentar por qu\'e $\imf{P_{t}}{x,x}=e^{-\lambda x}$, resuelva 
    las ecuaciones backward por medio de la t\'ecnica de factor integrante (comenzando con 
    $\imf{P_t}{x,x+1}$) y pruebe que
    \begin{esn}
        \imf{P_t}{x,y}=\binom{y-1}{y-x} e^{-\lambda x t}\paren{1-e^{-\lambda t}}^{y-x}.
    \end{esn}\pn
}
\afterstatement\pn

Esta vez, cada individuo tiene exáctamente un descendiente y no muere. Entonces, su matriz infinitesimal
es como en [\ref{problema5_6:inciso2}], sólo que $\mu_n = 0$ si $n \neq 1$ y $\mu_1 = 1$. Es decir
\begin{align}
    Q(i,j)  &=
            \begin{cases}
                \lambda i \mu_{1}       &   \text{si $j = i+1$ e $i \neq 0$}        \\
                -\lambda i              &   \text{si $i = j$}                       \\
                0                       &   \text{en cualquier otro caso}
            \end{cases}
\end{align}\pn

Entonces
\begin{align}
    \frac{d}{dt}    P_t(i, j)   &=  \sum_{k \in \N} Q(i,k)P_t(k,j)                  \\
                                &=  Q(i,i+1)P_t(i+1,j)  +   Q(i,i)P_t(i,j)          \\
                                &=  \lambda i P_t(i+1,j)  - \lambda i P_t(i,j)      \\
\end{align}\pn

Supongamos entonces $i = j$ y entonces 
\begin{align}
    \frac{d}{dt}    P_t(i, i)   &=  \lambda i P_t(i+1,i)  - \lambda i P_t(i,i)      \\
                                &=  -i \lambda P_(i, i).
\end{align}\pn

Donde, la solución a la ecuación diferencial

\begin{align}
    \frac{d}{dt}    P_t(i, i)   &=  - \lambda i P_(i, i).
\end{align}

es
\begin{align}
    P_t(i, i)   &=  e^{-t i \lambda}    
\end{align}

Supongamos ahora $j = i + 1$ y entonces
\begin{align}
     \frac{d}{dt}    P_t(i, i+1)    &=  \lambda i P_t(i+1,i+1)  - \lambda i P_t(i,i+1)          \\   
                                    &=  \lambda i e^{-t (i+1) \lambda}  - \lambda i P_t(i,i+1)  \\
                                    &\comment{la sustitución es por el caso recién demostrado}
\end{align}

De donde
\begin{align}
     \frac{d}{dt}P_t(i, i+1)  +  \lambda i P_t(i,i+1)                                       &&=&&&      \lambda i e^{-t (i+1) \lambda}                  \\
     (e^{\lambda i t}) \frac{d}{dt}P_t(i, i+1)  + (e^{\lambda i t}) \lambda i P_t(i,i+1)    &&=&&&      (e^{\lambda i t}) \lambda i e^{-t (i+1) \lambda}\\
     \comment{se multiplicó en ambos lados por lo mismo}                                    && &&&                                                      \\
     (e^{\lambda i t}) \frac{d}{dt}P_t(i, i+1)  + (e^{\lambda i t}) \lambda i P_t(i,i+1)    &&=&&&      \lambda i e^{-t \lambda}                        \\
      \frac{d}{dt} \bigg( (e^{\lambda i t}) P_t(i, i+1) \bigg)                              &&=&&&      \lambda i e^{-t \lambda}                        \\
     \comment{la suma se escribió como la derivada de un producto}                          && &&&                                                      \\
\end{align}

Este último término, lo podemos integrar suponiendo la condición inicial $P_0(i, i+1) = 0$
\begin{align}
       (e^{\lambda i t}) P_t(i, i+1)    &=  \int \frac{d}{dt} \bigg( (e^{\lambda i t}) P_t(i, i+1) \bigg) dt    \\      
                                        &=  \int_{0}^{t}    \lambda i e^{-t \lambda} dt                         \\    
                                        &=  i (1-e^{\lambda t})                         
\end{align}\pn

 Y por lo tanto
\begin{align}
       P_t(i, i+1)  &=  i (1-e^{\lambda t})(e^{-\lambda i t})                         
\end{align}\pn

Hagamos entonces la hipótesis de inducción (que mágicamente se ha cumplido hasta ahora, pero que para encontrar fue necesario hacer más cuentas)
\begin{align}
    P_t(i, i+k)     &=  \binom{i + k - 1}{k} e^{-\lambda i t} (1 - e^{- \lambda t})^k    
\end{align}\pn

Probemos entonces que para $j = i + k + 1$ también se cumple:
\begin{align}
    \frac{d}{dt} P_t(i, i + k + 1)  &=  \lambda i P_t(i + 1, i + 1 + k) - \lambda i P_t(i, i + 1 + k)                                    \\
                                    &=  \lambda i \binom{i + k }{k} e^{-\lambda (i + 1) t} (1 - e^{- \lambda t})^k - \lambda i P_t(i, i + 1 + k)   \\
\end{align}\pn

Y entonces
\begin{align}
    \frac{d}{dt} P_t(i, i + k + 1) - \lambda i P_t(i, i + 1 + k)                                    &=&&  \lambda i \binom{i + k}{k} e^{-\lambda (i + 1) t} (1 - e^{- \lambda t})^k                    \\
    (e^{\lambda i t}) \bigg( \frac{d}{dt} P_t(i, i + k + 1) - \lambda i P_t(i, i + 1 + k) \bigg)    &=&&  \lambda i \binom{i + k}{k} (e^{\lambda i t}) e^{-\lambda (i + 1) t} (1 - e^{- \lambda t})^k  \\
    \comment{se multiplicó en ambos lados por lo mismo, otra vez}                                   & &&                                                                                      \\
    \frac{d}{dt} \bigg((e^{\lambda i t}) P_t(i, i + k + 1)\bigg)                                    &=&&  \lambda i \binom{i + k}{k}  e^{-\lambda t} (1 - e^{- \lambda t})^k                           \\
     \comment{la suma se escribió como la derivada de un producto}                                  & &&
\end{align}\pn

Por lo tanto
\begin{align}
    (e^{\lambda i t}) P_t(i, i + k + 1)     &=  \int \frac{d}{dt} \bigg((e^{\lambda i t}) P_t(i, i + k + 1)\bigg)     dt            \\
                                            &=  \int_{0}^{t} \lambda i \binom{i + k}{k} e^{-\lambda t}(1 - e^{- \lambda t})^k    dt \\          
                                            &=  \binom{i + k}{k} i  \int_{0}^{t} \lambda  e^{-\lambda t} (1 - e^{- \lambda t})^k dt \\          
                                            &=  \binom{i + k}{k} \frac{i}{k+1}  (1-e^{- \lambda t})^{k+1}                           \\          
                                            &=  \binom{i + k + 1}{k + 1} (1-e^{- \lambda t})^{k+1}                                  \\          
\end{align}\pn

y entonces
\begin{align}
    P_t(i, i + k + 1)     &=  \binom{i + k + 1}{k + 1} (e^{-\lambda i t})(1-e^{- \lambda t})^{k+1}                                  \\
\end{align}\pn

como queríamos demostrar.

Entonces basta sustituir a $k = j - i$ para tener una fórmula en términos de $i$, $j$.\pn
\begin{align}
    P_t(i, j)     &=  \binom{j - 1}{j - i} e^{-\lambda i t} (1 - e^{- \lambda t})^k    
\end{align}\pn
    \newpage

    \subsection{Inciso (vi)} \label{problema5_6:inciso6}
    \emph{
    Al utilizar la f\'ormula para la esperanza de una variable binomial negativa, 
    pruebe que
    \begin{esn}
        \imf{\se_x}{Z_t}= xe^{\lambda t}.
    \end{esn}\pn
}

\afterstatement\pn

Del inciso anterior, tenemos que $P_t(i,j) = \P(W = i - j)$ donde $W$ es una variable aleatoria con
distribución binomial negativa de parámetros $r = x$ y $p = e^{-\lambda t}$ 
[véase \href{http://en.wikipedia.org/wiki/Negative_binomial_distribution}{distribución binomial negativa}].\pn

Entonces
\begin{align}
        \E_x(Z_t)   &=  \sum_{j} j \P_x(Z_t = j)   \\
                    &=  \sum_{j} j P_t(x, j)       \\
                    &=  \E(W)                      \\
                    &=  \frac{r}{p}                \\
                    &=  \frac{x}{e^{-\lambda t}}   \\
                    &=  x e^{\lambda t}.
\end{align}\pn

Como buscábamos demostrar.
    \newpage

    \subsection{Inciso (vii)} \label{problema5_6:inciso7}
    \emph{
    Pruebe que $e^{-\lambda t}Z_t$ es una martingala no-negativa y que por lo tanto 
    converge casi seguramente a una variable aleatoria $W$.\pn
}
\afterstatement\pn
	\newpage
	
    \subsection{Inciso (viii)} \label{problema5_6:inciso8}
    \emph{
    Al calcular la transformada de Laplace de $e^{-\lambda t}Z_t$, pruebe que $W$ tiene 
    distribuci\'on exponencial. Por lo tanto, argumente que casi seguramente $Z$ crece exponencialmente.
    %La distribuciÑn lÕmite està tomada de Beroin-Goldschmidt, ellos citan y corrigen un error de Athreya.
    \pn
}
\afterstatement\pn
\end{proof}