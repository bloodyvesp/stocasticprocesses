\emph{
	Considere el hipercubo $n$-dimensional $E=\set{0,1}^n$. A $E$ lo pensaremos como la composici\'on 
	de la primera de dos urnas que tienen en total $n$ bolas etiquetadas del $1$ al $n$. 
	Si $x=\paren{x_1,\ldots, x_n}\in E$, interpretaremos $x_i=1$ como que la bola $i$ est\'a en la urna $1$. 
	Considere el siguiente experimento aleatorio: inicialmente la composici\'on de las urnas est\'a dada por 
	$x$ y a cada instante de tiempo escogemos una bola al azar y la cambiamos de urna. 
	Modele esta situaci\'on por medio de una cadena de Markov $X$ en $E$. Sea $\fun{f}{E}{\set{0,\ldots, n}}$ 
	dada por $\imf{f}{x}=\sum_i x_i$. Pruebe que $\imf{f}{X}=\paren{\imf{f}{X_n},n\in\na}$ es una cadena de 
	Markov cuya matriz de transici\'on determinar\'a.
}
\afterstatement\pn

Sean $i,j \in E$. Definamos las siguientes entradas de una matriz de transición:

\begin{align}
        p_{i,j} &=   
                    \begin{cases}
                        \frac{1}{n},    &       \|i-j\| = 1 \comment{donde $\|\cdot\|$ es la norma euclidiana}  \\
                        0,              &       \text{en cualquier otro caso}
                    \end{cases}
\end{align}

Dado $i \in E$, sólo existen $n$ vectores que distan exáctamente $1$ de $i$. Entonces 
$\sum_{j \in E} \frac{1}{n} = n \frac{1}{n} = 1$ y nuestra matriz sí es de transición.\pn

Ya que nuestro proceso está restringido a comenzar en $x$, nuestra distribución inicial
será $\lambda = (\lambda_i : i \in E)$ tal que $\lambda_x = 1$ y $\lambda_i = 0$ si $i \not= x$.\pn

Es fácil notar que la evolución del proceso depende únicamente del presente. Para llegar a una configuración
de bolas en particular, únicamente es importante en qué configuración se encuentra actualmente.\pn

Dado que tenemos una matriz de transición y una distribución inicial bien definidas, existe una cadena de
Markov $X$ con dichas matriz de transcición y distribución inicial que describen la evolución del proceso.\pn

El proceso $(f(X_n))_{n \in N}$ es el proceso que cuenta la cantidad de bolas en la urna $1$ que equivale al proceso
de Ehrenfest, el cual es una cadena de Markov y las entradas de su matriz de transición están dadas por

\begin{align}
        p_{i,j}     &=  
                        \begin{cases}
                            \frac{n-i}{n},  &   j=i+1   \\
                            \frac{i}{n},    &   j=i-1   \\
                            0,              &   \text{en cualquier otro caso.}
                        \end{cases}
\end{align}