\emph{
    Sea $S_n=\xi_1+\cdots+\xi_n$ donde las variables $\xi$ son independientes y $\xi_i$ tiene 
    media cero y varianza finita $\sigma_i^2$. Pruebe que si $\sum_i \sigma_i^2<\infty$ entonces 
    $S_n$ converge casi seguramente y en $L_2$ conforme $n\to\infty$. Construya un ejemplo de 
    variables aleatorias $\xi_i$ tales que la serie $\sum_i \xi_i$ sea casi seguramente absolutamente 
    divergente y casi seguramente condicionalmente convergente (considere ejemplos simples!). 
    Explique heur\'isticamente por qu\'e cree que suceda esto.
    %Ser\'a que \sum_i\abs{x_i}=\infty casi seguramente si \sum_i\abs\esp{\xi_i}=\infty? 
}

\afterstatement\par\null

Sea $\F_n = \sigma( \xi_0, \xi_1, \dots, \xi_n)$.\par\null


Probemos ahora que $(S_n)_{n \in \N}$ es martingala con respecto de la \newline 
filtración $(\F_n)_{n \in \N}$.\par\null

Por como se definió $\F_n$, $S_n$ es $\F_n$-medible. Por otra parte, $S_n$ es suma finita de variables en $L_1$ 
(pues las $\xi_i$'s tienen esperanza finita).\par\null

Ahora
\begin{align}
    \E(S_{n+1} | \F_n)  &=  \E(\xi_{n+1} | \F_n) + \E(S_n | \F_n)                       \\
                        &=  \E(\xi_{n+1} | \F_n) + S_n                                  \\
                        &\;\;\;\;\mbox{Porque $S_n$ es $F_n$-medible}                   \\
                        &=  \E(\xi_{n+1}) + S_n                                         \\
                        &\comment{Porque $\xi_{n+1}$ es independiente de$F_n$}        	\\
                        &=  0 + S_n.                                                    \\
\end{align}\par\null

Y con esto hemos demostrado que $S_n$ es martingala.\par\null

Ahora, como $\E(\xi_n) = 0$ para todo $n \in \N$, $\E(S_n) = 0$ para todo $n \in \N$.

\begin{align}
        \var{S_n}       &=  \E(S_n^2) - \E(S_n)^2                                               \\
                        &=  \E(S_n^2) - 0                                                       \\
                        &=  \sum_{i \leq n} \sigma_i^2                                          \\
                        &\comment{Esto último gracias a que los $\xi_i$'s son independientes}.
\end{align}

Ahora $\| S_n \|_2 = \sqrt{\var{S_n}} = \sqrt{\sum_{i \leq n} \sigma_i^2}$. Recordemos que por ser  $\sqrt{\cdot}$
función continua, manda sucesiones convergentes en suceciones convergentes. Como 
$\lim\limits_{n \rightarrow \infty} \sum\limits_{i < n} \sigma_i^2 < \infty$, 
entonces $\lim\limits_{n \rightarrow \infty}\| S_n \|_2 < \infty$.\par\null

En particular, $\sup\limits_n \| S_n \|_2 < \infty$.\par\null

El teorema de convergencia en $L_p$ de Doob nos garantiza que $S_n$ converge casi seguramente y en $L_2$. Como queríamos
hacer ver.\par\null

Definamos ahora $\xi_i$ con $\mw(\xi_i = \pm \frac{1}{i+1}) = \frac{1}{2}$.