\begin{problema}
    \begin{enumerate}
        \item[(i)] 
            Instale \href{www.octave.org}{Octave} en su computadora
        \item[(ii)] 
            \'Echele un ojo a la documentaci\'on
        \item[(iii)] 
   
            Ejecute el siguiente c\'odigo linea por linea: 
            \textsl{
                    \lstinputlisting[caption=]{tarea2/problema2_2/polya1.R}
                    }
        \item[(iv)] 
            Lea las secciones sobre 
            \href{http://www.gnu.org/software/octave/doc/interpreter/Simple-Examples.html#Simple-Examples}{simple examples}, 
            \href{http://www.gnu.org/software/octave/doc/interpreter/Ranges.html#Ranges}{ranges}, 
            \href{http://www.gnu.org/software/octave/doc/interpreter/Random-Number-Generation.html#Random-Number-Generation}{random number generation} 
            y 
            \href{http://www.gnu.org/software/octave/doc/interpreter/Comparison-Ops.html#Comparison-Ops}{comparison operators} 
            y escriba su interpretaci\'on de lo que hace el c\'odigo anterior. Nota: est\'a relacionado con uno de los ejemplos del curso.
        \item[(v)] 
            Vuelva a correr el c\'odigo varias veces y escriba sus impresiones sobre lo que est\'a sucediendo.
    \end{enumerate}
\end{problema}

\afterstatement\par\null

En el código de arriba, se implementó una ejecución del proceso de las urnas de Poyla.\par\null

El vector \textsl{x} representa la proporción de las bolas ``verdes'' en el contenedor conforme el tiempo avanza. Que
al tiempo inicial se tenga \textsl{x=[1/2]} quiere decir que al inicio había tantas bolas verdes como rojas.\par\null

Para octave el booleano \textsl{u(i)$<$x(i)} puede operarse numéricamente, \textsl{true} es equivalente a \textsl{1} y
\textsl{false} es equivalente a \textsl{0}. Entonces, la parte que dice \\
\textsl{+(u(i)$<$x(i))} significa que se está sumando
uno o cero. Que quiere decir que la constante de bolas que se agregan a la urna es $1$.\par\null

El vector \textsl{u} representa el resultado de sacar una bola. Si \textsl{u(i)$<$x(i)} significa que en el turno \textsl{i}, 
se obtuvo una bola verde.\par\null

El \textsl{2} misterioso que se encuentra en varias partes significa que en el turno inicial existían \textsl{2} bolas en total.
Como al inicio la proporción era \textsl{1/2}, esto significa que al inicio existían una bola verde y una roja exactamente.\par\null

A continuación una gráfica obtenida de ejcutar el código.

\begin{center}
    \includegraphics[width=10cm]{tarea2/problema2_2/poyla.PNG}
\end{center}
\begin{center}
    Gráfica de una ejecución de urnas de Poyla \par\null
    1 bola verde inicial, 1 bola roja inicial y constante 1.
\end{center}\par\null