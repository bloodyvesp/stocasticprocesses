\begin{problema}
    Sea

    \begin{esn}
        Q=
            \begin{pmatrix}
                -2  &   2   \\
                3   &   -3
            \end{pmatrix}.
    \end{esn}

    \begin{enumerate}
        \item 
            Haga un programa en octave que permita simular las trayectorias de una 
            cadena de Markov a tiempo continuo $X$ con matriz infinitesimal $Q$.\pn
            
        \item 
            Utilice su programa para generar 10000 trayectorias en el intervalo de tiempo $[0,10]$ 
            comenzando con probabilidad $1/2$ en cada estado y obtenga la distribuci\'on emp\'irica 
            de $X_{10}$.\pn
            
        \item 
            Calcule $e^{10Q}$ (utilizando alg\'un comando adecuado) y contraste con la 
            distribuci\'on emp\'irica del inciso anterior.\pn
        
        \item 
            Codifique el siguiente esquema num\'erico, conocido como m\'etodo de Euler, 
            para aproximar a $e^{10 Q}$: escoja $h>0$ peque\~no, defina a $P^h_0$ como 
            la matriz identidad y recursivamente
            \begin{esn}
                P^h_{i+1}=P^h_i+hQP^h_i. 
            \end{esn}
            corra hasta que $i=\floor{10/h}$ y compare la matriz resultante con $e^{10Q}$. 
            Si no se parecen escoja a $h$ m\'as peque\~no. 
            ?`Con qu\'e $h$ puede aproximar a $e^{10Q}$ a 6 decimales?
    \end{enumerate}
\end{problema}