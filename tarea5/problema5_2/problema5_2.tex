\begin{problema}
%Simulaci\'on de un proceso Poisson puntual... subordinador...
    Sea $\Xi$ una medida de Poisson aleatoria en $(0,\infty)\times (0,\infty)$ cuya 
    medida de intensidad $\nu$ est\'a dada por $\imf{\nu}{ds,dx}=\indi{x>0}C/x^{1+\alpha}\, ds\,dx$. 
    
    \begin{enumerate}
        \item Determine los valores de $\alpha$ para los cuales $\int 1\wedge x\,\imf{\nu}{dx}<\infty$. 
    \end{enumerate}
    
    Nos restringimos ahora a valores de $\alpha$ para los cuales la integral anterior sea finita. Sean $\imf{f_t}{s,x}=\indi{s\leq t}x$ y $X_t=\Xi f_t$. 
    
    \begin{enumerate}[resume]
        \item Determine los valores  de $\alpha$ para los cuales $X_t<\infty$ para toda $t\geq 0$ casi seguramente.
    \end{enumerate}

    Nos restringiremos a dichos valores de $\alpha$. 
    
    \begin{enumerate}[resume]
        \item Calcule $\esp{e^{-\lambda X_t}}$ y pruebe que $X_{t}$ tiene la misma distribuci\'on que $t^{1/\alpha}X_1$. 
        \item Diga por qu\'e el siguiente c\'odigo en Octave simula la trayectoria aproximada del proceso $X$ en el intervalo $[0,1]$.
        \lstinputlisting{tarea5/problema5_2/SuborEst.m}
    \end{enumerate}
\end{problema}