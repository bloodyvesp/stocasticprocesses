\begin{problema}[Extensiones del teorema de paro opcional]
	Sea ${M=\paren{M_n,n\in\na}}$ una (super)martingala respecto de una filtraci\'on ${\paren{\F_n,n\in\na}}$ y sean ${S}$ y ${T}$ tiempos de paro.
	
\begin{enumerate}
                \item[(i)] 
                	Pruebe que ${S\wedge T}$, ${S+T}$ y ${S\vee T}$ son tiempos de paro.
                
                \item[(ii)] 
                	\begin{esn}
                		\F_T=\set{A\in\F:A\cap\set{T\leq n}\in\F_n\text{ para toda } n}
                	\end{esn}
                	es una ${\sigma}$-\'algebra, a la que nos referimos como la ${\sigma}$-\'algebra 
                	detenida en ${\tau}$. Comente qu\'e puede fallar si ${T}$ no es tiempo de paro. 
                	Pruebe que ${T}$ es ${F_T}$-medible. 
                
                \item[(iii)] 
                	Pruebe que si ${T}$ es finito, entonces ${M_T}$ es ${\F_T}$-medible.
                
                \item[(iv)] 
                	Pruebe que si ${S\leq T\leq n}$ entonces ${\F_S\subset\F_T}$. Si adem\'as ${T}$ es acotado entonces ${X_S,X_T\in L_1}$ y 
                	\begin{esn}
	                	\espc{M_T}{\F_S}\leq M_S.
                	\end{esn}

                \item[(v)] 
                	Si ${X=\paren{X_n,n\in\na}}$ es un proceso estoc\'astico ${\paren{\F_n}}$-adaptado y tal que ${X_n\in L_1}$ y tal que 
                	para cualesquiera tiempos de paro acotados ${S}$ y ${T}$ se tiene que ${\esp{X_S}=\esp{X_T}}$ entonces ${X}$ es una 
                	martingala. Sugerencia: considere tiempos de paro de la forma ${n\indi{A}+(n+1)\indi{A^c}}$ con ${A\in\F_n}$.
\end{enumerate}

\defin{Categor\'ias: }Tiempos de paro, Muestreo opcional
\end{problema}

\begin{proof}
	\subsubsection{Inciso (i)} 
	\emph{
		Pruebe que ${ S \wedge T }$, ${ S + T }$ y ${ S \vee T}$ son tiempos de paro.\\
	}
	
		\begin{itemize}
			\item 
				Comprobemos que:
				\begin{align}
					\{ S \wedge T \leq n \} = \{ T \leq n \} \cup \{ S \leq n\}.
				\end{align}
				
				Sea ${\omega \in \{ S \wedge T \leq n \}}$. Entonces ${T(\omega) \leq n}$ ó ${S(\omega) \leq n}$.
				Por lo tanto:
				\begin{align}
						\{ S \wedge T \leq n \} \subset \{ T \leq n \} \cup \{ S \leq n\}.							
				\end{align}
				
				Por otro lado, si ${\omega \in \{ T \leq n \} \cup \{ S \leq n\}}$ entonces ${T(\omega) \leq n}$ ó ${S(\omega) \leq n}$.
				En particular, el mínimo tendrá que ser menor que ${n}$ y por lo tanto:
				\begin{align}
						\{ T \leq n \} \cup \{ S \leq n\} \subset \{ S \wedge T \leq n \}.						
				\end{align}
				
				Por último, ${\{ T \leq n \} \in \F_n}$ y ${\{ S \leq n \} \in \F_n}$. Por lo tanto	\\	
				${\{ T \leq n \} \cup \{ S \leq n\} = \{ S \wedge T \leq n \} \in \F_n}$ y con esto demostramos que 
				${ S \wedge T}$ es tiempo de paro.\\
				
			\item
				Comprobemos que:
				\begin{align}
					\{ S \vee T \leq n \} = \{ T \leq n \} \cap \{ S \leq n\}.
				\end{align}
				
				Sea ${\omega \in \{ S \vee T \leq n \}}$. Entonces ${T(\omega) \leq n}$ y ${S(\omega) \leq n}$.
				Por lo tanto:
				\begin{align}
						\{ S \vee T \leq n \} \subset \{ T \leq n \} \cap \{ S \leq n\}.							
				\end{align}
				
				Por otro lado, si ${\omega \in \{ T \leq n \} \cap \{ S \leq n\}}$ entonces ${T(\omega) \leq n}$ y ${S(\omega) \leq n}$.
				En particular, el máximo tendrá que ser menor que ${n}$ y por lo tanto:
				\begin{align}
						\{ T \leq n \} \cap \{ S \leq n\} \subset \{ S \vee T \leq n \}.						
				\end{align}
				
				Por último, ${\{ T \leq n \} \in \F_n}$ y ${\{ S \leq n \} \in \F_n}$. Por lo tanto	\\	
				${\{ T \leq n \} \cap \{ S \leq n\} = \{ S \vee T \leq n \} \in \F_n}$ y con esto demostramos que 
				${ S \vee T}$ es tiempo de paro.\\
				
			\item
				Comprobemos que:
				\begin{align}
					\{ S + T = n \} = \bigcup_{i = 1}^{n-1} \bigg( \{S = n - i\} \cap \{T = i\} \bigg).
				\end{align}
				
				Si ${\omega \in \{ S + T = n \}}$, entonces ${S(\omega) + T(\omega) = n}$, como ${S}$ y ${T}$ son positivas,
				entonces ${1 \leq S(\omega) \leq n-1}$. Entonces basta elegir ${i = S(\omega)}$ para afirmar que 
				${\omega \in \{S = n - i\} \cap \{T = i\}}$ y por lo tanto
				\begin{align}
				\{ S + T = n \} \subset \bigcup_{i = 1}^{n-1} \bigg( \{S = n - i\} \cap \{T = i\} \bigg).
				\end{align}
				
				Por otro lado, si ${\omega \in \bigcup_{i = 1}^{n-1} \bigg( \{S = n - i\} \cap \{T = i\} \bigg)}$
				significa que existe un ${i}$ tal que ${1\leq i \leq n-1}$ y ${\omega \in \{S = n - i\} \cap \{T = i\}}$.
				Y por lo tanto: ${S(\omega) = n-i}$ y ${T(\omega) = i}$. De donde ${(T + S)(\omega) = n}$.
				
				De aquí que
				\begin{align}
					\bigcup_{i = 1}^{n-1} \bigg( \{S = n - i\} \cap \{T = i\} \bigg) \subset \{ S + T = n \}. 
				\end{align}
				
				Ahra, para cada ${i}$ tal que ${1 \leq i \leq n-1}$, ${\{S = n - i\} \cap \{T = i\} \in \F_n}$.
				Por lo tanto ${\{ S + T = n\} \in \F_n}$ y con esto queda demostrado que ${ S + T }$ es tiempo de paro.
		\end{itemize}
		
	\subsubsection{Inciso (ii)}
	\emph{
		\begin{align}
			\F_T=\set{A\in\F:A\cap\set{T\leq n}\in\F_n\text{ para toda } n}
		\end{align}
		es una ${\sigma}$-\'algebra, 
		a la que nos referimos como la ${\sigma}$-\'algebra detenida en ${\tau}$. Comente qu\'e puede fallar si ${T}$ no es tiempo de paro. 
		Pruebe que ${T}$ es ${\F_T}$-medible.\\		
	}			
			
		Primero hay que demostrar que ${\F_T}$ es ${\sigma}$-algebra.\\
		
		\begin{itemize}
			\item ${\Omega \in \F_T}$. \\
			
				Notemos que 
				\begin{align}
					\Omega \cap \{T \leq n\} = \{T \leq n\} \in \F_n.
				\end{align}
				Donde la pertenencia a ${\F_n}$ es gracias a que ${T}$ es tiempo de paro. (Esta parte podria fallar si ${T}$ no fuera tiempo de paro).\\
			
			\item ${\F_T}$ es cerrado bajo complementación.\\
			
				Sea ${A \in \F_T}$. Eso significa que para todo ${n \in  \N}$ ocurre que ${B = A \cap \{ T \leq n \} \in \F_n}$. Por ser ${\F_n}$
								una ${\sigma}$-algebra tenemos que el complemento de ${B}$ también debe estar en ${\F_n}$. Escrito en símbolos:
				
				\begin{align}
					B^c 	&= (A   \cap \{ T \leq n \})^c \\
							&=  A^c \cup \{ T > n \} \in \F_n
				\end{align}
				
				Dado que ${B^c}$ y ${\{ T \leq n \}}$ se encuentran en ${\F_n}$, también ${B^c \cap \{ T \leq n \} \cap \{ T \leq n \}}$.
				
				Escribiendo esto último de otra manera:
				\begin{align}
					B^c \cap \{ T \leq n \} \cap \{ T \leq n \} 	&=		A^c \cup \{ T > n \} \cap \{ T \leq n \} \cap \{ T \leq n \} \\
																	&= 		A^c \cap \{ T \leq n \} \cup \{ T > n \} \cap \{ T \leq n \} \\
																	&= 		\bigg(A^c \cap \{ T \leq n \}\bigg) 
																				\bigcup 
																			\bigg(\{ T > n \} \cap \{ T \leq n \}\bigg) \\
																	&=		\bigg(A^c \cap \{ T \leq n \}\bigg)	\bigcup \emptyset \\
																	&=		A^c \cap \{ T \leq n \} \in \F_n
				\end{align}
				
				Y por lo tanto ${A^c \in \F_T}$.\\
				
			\item ${\F_T}$ es cerrado bajo uniones numerables.
			
				Sean ${A_m \in \F_T}$ para ${m \in \N}$.
				\begin{align}
					\bigg( \bigcup_{k=1}^\infty A_k \bigg) \cap \{ T \leq n \} 	&=	\bigcup_{k=1}^\infty \bigg( A_k \cap \{ T \leq n \} \bigg)
				\end{align}								
				
				La última parte, por definición de ${\F_T}$ es una unión numerable de elementos de ${\F_n}$ y por lo tanto, dicha unión
				tambien pertenece a ${\F_n}$. Y por último ${\cup A_k \in \F_T}$.
		\end{itemize}
		
		Ahora veamos que $T$ es $\F_T$-medible. Dado un $k \in \N$, ocurre que:
		
		\begin{align}
			\{T \leq k\} \cap \{T \leq n\} = \{T \leq min(k,n)\} \in \F_{min(k, n)} \subset \F_n \forall n \in \N.
		\end{align}

		Y por lo tanto $\{T \leq k\} \in \F_T$.
		
	\subsubsection{Inciso (iii)}
	\emph{
		Pruebe que si ${T}$ es finito, entonces ${M_T}$ es ${\F_T}$-medible.\\
	}
	
		Sea $A \in \B(\R)$. Por ser $T$ finito, tenemos que $\{ T \in \N \} = \Omega$. Entonces
		podemos descomponer a $\{ M_T \in A \}$ como:
		
		\begin{align}
			\{ M_T \in A \}		&=		\bigcup_{i \in \N} \bigg( \{ M_T \in A \} \cap \{ T = i \} \bigg)\\
								&=		\bigcup_{i \in \N} \bigg( \{ M_i \in A \} \cap \{ T = i \} \bigg).
		\end{align}
	
		Ahora sea $n \in \N$. 
		
		\begin{align}
			\{ M_T \in A \} \cap \{ T \leq n \}		&=		\bigcup_{i \in \N} \bigg( \{ M_i \in A \} \cap \{ T = i \} \bigg) \cap \{ T \leq n \} \\
													&=		\bigcup_{i \leq n} \bigg( \{ M_i \in A \} \cap \{ T = i \} \bigg).
		\end{align}
		
		Por ser ${M_i}$ martingala con respecto a la filtración ${(\F_n)_{n \in \N}}$ y ${T}$ tiempo de paro. Tenemos que ${\{ T = i\} \in \F_i}$ y
		${\{ M_i \in A \} \in \F_i}$. Por lo tanto ${\{ T = i\} \cap \{ M_i \in A \} \in \F_i \subset \F_n}$ para todo ${i \leq n}$. Por lo que
		${\{ M_T \in A \} \cap \{ T \leq n \} \in \F_n}$. Como esto es cierto para toda $n \in \N$, tenemos que ${\{ M_T \in A \} \in \F_T}$. Con
		lo que terminamos de demostrar que ${M_T}$ es ${\F_T}$-medible.\\
		
	\subsubsection{Inciso (iv)} 
	\emph{
		Pruebe que si ${S\leq T\leq n}$ entonces ${\F_S\subset\F_T}$. Si adem\'as ${T}$ es acotado entonces ${X_S,X_T\in L_1}$ y 
		\begin{esn}
			\espc{M_T}{\F_S}\leq M_S.
		\end{esn}	
	}

	\subsubsection{Inciso (v)}
	\emph{
		Si ${X=\paren{X_n,n\in\na}}$ es un proceso estoc\'astico ${\paren{\F_n}}$-adaptado y tal que ${X_n\in L_1}$ y tal que 
		para cualesquiera tiempos de paro acotados ${S}$ y ${T}$ se tiene que ${\esp{X_S}=\esp{X_T}}$ entonces ${X}$ es una 
		martingala. Sugerencia: considere tiempos de paro de la forma ${n\indi{A}+(n+1)\indi{A^c}}$ con ${A\in\F_n}$.
	}
\end{proof}