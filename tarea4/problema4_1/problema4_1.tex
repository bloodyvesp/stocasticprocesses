\begin{problema}
	Sean $\F_1,\F_2,\ldots $ y $\G$ sub\sa s de $\F$. 
	Decimos que $\F_1,\F_2,\ldots$ son condicionalmente independientes dada $\G$ si para 
	cualquier $H_i$ que sea $\F_i$ medible y acotada se tiene que
	
	\begin{esn}
		\espc{H_1\cdots H_n}{\G}=\espc{H_1}{\G}\cdots \espc{H_n}{\G}.
	\end{esn}
	
	\begin{enumerate}
		\item[(i)]	[\ref{problema4_1:inciso1}]
			?`Qu\'e quiere decir la independencia condicional cuando $\G=\set{\oo,\emptyset}$?
			\pn

		\item[(ii)] 	[\ref{problema4_1:inciso2}]
			Pruebe que $\F_1$ y $\F_2$ son condicionalmente independientes dada $\G$ 
			(denotado $\condind{\F_1}{\F_2}{\G}$) si y s\'olo si para 
			cualquier $H$ que sea $\F_1$-medible y acotada se tiene que
			\begin{esn}
				\espc{H}{\F_2,\G}=\espc{H}{\G}.
			\end{esn}

		\item[(iii)]	[\ref{problema4_1:inciso3}]
			Pruebe que $\F_1,\F_2,\ldots, $ son condicionalmente independientes dada 
			$\G$ si y s\'olo si para cada $n\geq 1$, $\F_{n+1}$ es condicionalmente 
			independiente de $\F_1,\ldots, \F_n$ dada $\G$. 
	\end{enumerate}

	\defin{Categor\'ias: } Esperanza condicional, Independencia condicional.
\end{problema}

\begin{proof}
    \subsection{Inciso (i)} \label{problema4_1:inciso1}
    \emph{
	?`Qu\'e quiere decir la independencia condicional cuando $\G=\set{\oo,\emptyset}$?
}
    \newpage

    \subsection{Inciso (ii)} \label{problema4_1:inciso2}
    \emph{
		Pruebe que $F_1$ y $\F_2$ son condicionalmente independientes dada $\G$ 
		(denotado $\condind{\F_1}{\F_2}{\G}$) si y s\'olo si para 
		cualquier $H$ que sea $\F_1$-medible y acotada se tiene que
		\begin{esn}
			\espc{H}{\F_2,\G}=\espc{H}{\G}.
		\end{esn}
}
    \newpage

    \subsection{Inciso (iii)} \label{problema4_1:inciso3}
    \emph{
	Pruebe que $\F_1,\F_2,\ldots, $ son condicionalmente independientes dada 
	$\G$ si y s\'olo si para cada $n\geq 1$, $\F_{n+1}$ es condicionalmente 
	independiente de $\F_1,\ldots, \F_n$ dada $\G$. 
}

\afterstatement\par\null

Primero demostraremos la suficiencia. Para ello emplearemos el ejercicio anterior.
Sean $H_1, \dots, H_{n+1}$ variables aleatorias $\F_1, \dots, \F_{n+1}$-medibles (respectivamente) 
y acotadas.\par\null

Sea $P = \{ A : A = F_1 \cap \dots \cap \F_n \cap \G \text{ donde } F_i \in \F_i \text{ y } G \in \G \}$,
Es claro que $P \subset \sigma(\F_1, \dots, \F_n, \G)$. También es fácil ver que no es vacía ($\Omega$ es uno de sus elementos), 
y que es cerrada bajo intersecciones finitas. En otras palabras $C$ es un $\pi$-sistema. Además, justo
igual que antes, tenemos que $\sigma(P) = \sigma(\F_1, \dots, \F_n, \G)$.\par\null

Ahora sea $D = \{ A \in \sigma(\F_1, \dots, \F_n, \G) : \E(H_{n+1} \indic_A) = \E(\E(H_{n+1} | \G) \indic_A) \}$




\end{proof}