\emph
{	
	Mediante el inciso anterior, construya una martingala que converge 
	casi seguramente pero no lo hace en $L_1$.\\
}
	
	En el ejercicio 4 se probará que si $T$ y $S$ son tiempos de paro, entonces $T\wedge S$ también 
	es tiempo de paro. Con esto tenemos que si $T_1$ es tiempo de paro, entonces $T_1 \wedge n$ con 
	$n \in \N$ también es tiempo de paro. Definimos entonces 
	$$M_n = S_{T_1 \wedge n}.$$
	
	Veamos que los $M_n$ forman una martingala.
	
	\begin{itemize}
		\item 
			$M_n$ es adaptada a la filtración.
			\begin{align}
				M_n(w) = S_{T_1 \wedge n}(w) = 
				S_{T_1 \wedge n (w)}(w) = 
				\sum_{k=1}^{T_1 \wedge n (w)} X_k = 
				\sum_{k=1}^{n} (X_k \cdot \indic_{T_1 \geq k})(w).
			\end{align}
			
			De donde, podemos escribir:
			\begin{align}\label{problema1_3:descomposicion_de_M_n}
				M_n = \sum_{k=1}^{n} (X_k \cdot \indic_{T_1 \geq k}).
			\end{align}								 		
			
			Recordemos que $X_k$ es $\F_n$-medible para toda $k \leq n $. Por ser
			$T_1$ tiempo de paro, los conjuntos $A_k = \{T_1 = k\}$ y 
			$B_k = \{T_1 \leq k\}$	son $F_k$ medibles y por lo tanto 
			$A_k \cup B_k^c = \{ T_1 \geq k\}$ también lo es. De aquí que 
			$\indic_{T_1 \geq k}$ es $\F_k$-medible y por lo tanto también $\F_n$-medible
			para toda $n$ tal que $n \geq k$.\\
			  
			Entonces $M_n$ es suma y productos de funciones $\F_n$-medibles y por lo tanto
			$F_n$-medible. Que es lo que queríamos demostrar.\\
			
		\item
			$M_n \in L_1$\\
			
			De (\ref{problema1_3:descomposicion_de_M_n}) podemos ver que $M_n$ es 
			suma finita de variables acotadas. Por lo tanto $M_n \in L_1$.\\
			
		\item Ahora probaremos que	$\E(M_{n+1} | \F_{n}) = M_{n}$\\
			
			Primero:
			\begin{align}
				\E(M_{n+1} | \F_{n}) &= \E( S_{T_1 \wedge (n+1)} | \F_{n}) = 
				\E\bigg( \sum_{k=1}^{n+1} (X_k \cdot \indic_{T_1 \geq k})\bigg| \F_{n}\bigg) \\	 			
				& =\E\bigg( \sum_{k=1}^{n} (X_k \cdot \indic_{T_1 \geq k}) \bigg| \F_{n}\bigg) +
				\E\bigg((X_{n+1} \cdot \indic_{T_1 \geq n+1}) \bigg| \F_{n}\bigg) \\
				&\mbox{(Este paso es gracias a que $X_k$ y $\indic_{T_1 \geq k}$ son $\F_n$-medibles)}\\
				& = \sum_{k=1}^{n} (X_k \cdot \indic_{T_1 \geq k}) + 
				\E((X_{n+1} \cdot \indic_{T_1 \geq n+1}) | \F_{n}) \\
				& = S_{T_1 \wedge n} + \E((X_{n+1} \cdot \indic_{T_1 \geq n+1}) | \F_{n}) \\
				& = M_n + \E((X_{n+1} \cdot \indic_{T_1 \geq n+1}) | \F_{n})
			\end{align}
			
			Entonces, nos basta probar que $\E(X_{n+1} \cdot \indic_{T_1 \geq n+1} |
			 \F_{n}) = 0$ para terminar nuestra demostración.\\
			 
			Sean $A = \{T_1 = n\}$ y $B = \{ T_1 \leq n\}$. Por ser $T_1$ tiempo de paro,
			$A$ y $B$ son $\F_n$-medibles. Por lo tanto $B \setminus A$ también es $\F_n$-medible. 
			Notemos que $\{T_1 \geq n+1\} = (B \setminus A)^c$. Por lo tanto $\{T_1 \geq n+1\}$ es
			$\F_n$-medible. De donde  $\indic_{T_1 \geq n+1})$ es $\F_n$-medible.\\
			
			Con esto, ahora tenemos que:
			\begin{align}
				\E((X_{n+1} \cdot \indic_{T_1 \geq n+1}) | \F_{n}) &= \indic_{T_1 \geq n+1} \cdot \E(X_{n+1} | \F_{n}) \\
				&\mbox{(Este paso es gracias a que los $X_n$ son independientes)} \\
				&=\indic_{T_1 \geq n+1} \cdot \E(X_{n+1} ) \\
				&=\indic_{T_1 \geq n+1} \cdot 0 \\
				&= 0
			\end{align}	
			
			Como queríamos demostrar.
	\end{itemize}
	
	Ahora que tenemos que $(M_n)_{n \in \N}$ es martingala, confirmemos que converge casi seguramente.\\
	
	Notemos que $(T_1 \wedge n)_{n \rightarrow \infty} \rightarrow T_1$ c.s.\\
	
	De aquí que $(M_n)_{n \rightarrow \infty} = (S_{T_1 \wedge n})_{n \rightarrow \infty} = S_{T_1}$ c.s. \\				
	
	Veamos que la convergencia no ocurre en $L_1$. \\
				
	Dado que $T_1 \wedge n$ es un tiempo de paro acotado para toda $n \in \N$,
	podemos aplicar el Teorema de Muestreo Opcional de 	Doob. 
	El cual nos dice que $\E(M_n) = \E(S_{T_1 \wedge n}) = \E(S_0) = 0$.\\
	
	Por otro lado, por definición de $T_1$, $S_{T_1} = 1$ c.s.	De donde $\E(S_T) = 1$.\\
	
	\begin{align}
		\E(M_n) = 0 \not\rightarrow 1 = \E(S_{T_1}).
	\end{align}			
	
	Y con esto, queda demostrado que la convergencia no se da en $L_1$.\\
	