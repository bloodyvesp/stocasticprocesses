\RequirePackage{pgfornament,tkzexample,tikzrput}  


\documentclass[a5paper,oneside]{amsart}
\usepackage[scale={.8,.85}]{geometry}

\usepackage{mathrsfs}
\usepackage[utf8]{inputenc}

\usepackage{dsfont}
\usepackage{listings}



\newtheoremstyle{dotless}{}{}{\itshape}{}{\bfseries}{}{ }{}
\theoremstyle{dotless}
\renewenvironment{proof}{{\bfseries Demostración:}}



\newtheorem{teorema}{\;\;Teorema}[section]
\newtheorem{problema}[teorema]{\;\;Problema}
\newtheorem{definicion}[teorema]{\;\;Definición}
\newtheorem*{categoria}{Categorías:}
\newtheorem*{remark}{Remark}
\setcounter{secnumdepth}{3}
\setcounter{tocdepth}{3}



\input{definiciones/definitions.tex}
\newcommand{\N}{\mathbb{N}}
\newcommand{\R}{\mathbb{R}}
\newcommand{\E}{\mathbb{E}}
\newcommand{\indic}{\mathds{1}}
\newcommand{\F}{\mathscr{F}}
\newcommand{\textme}[1]{\;\text{#1}\;}
\newcommand{\nqed}{\center{\textbf{---------------------------------------------------------------------------------------}}}


%\renewcommand{\nqed}
{		
		\begin{tikzpicture}
			\node (A) at (0,0) {};
			\node (B) at (3,0) {};
			\node (C) at (6,0) {};
			\path (A.center) to [ornament=86,at=0] (B.center);
			\path (A.center) to [ornament=85,at=0.50] (B.center);
			\path (A.center) to [ornament=88,at=0.75] (C.center);
    		\path (B.center) to [ornament=85,at=1.50] (C.center);
			\path (B.center) to [ornament=86,at=2] (C.center);
		\end{tikzpicture}\\
}




\title[Problemas de Procesos I]{Problemas de Procesos Estocásticos I\\ Semestre 2013-II\\ Posgrado en Ciencias Matemáticas\\ Universidad Nacional Autónoma de México}

\usepackage[colorlinks,citecolor=blue,urlcolor=blue]{hyperref}

\begin{document}
	\maketitle
	\section{Tarea 1}
	
		\subsection{Problema 1.1}
		\input{tarea1/problema1_1/problema1_1.tex}
		\newpage
		
		\subsection{Problema 1.2}
		\begin{problema}
		Suponga que \(T\) es un tiempo de paro tal que para algún 
		\(N\in\mathbb{N}\) y \(\varepsilon>0\) se tiene que para toda \(n\in\mathbb{N}\):
		
        \begin{equation}\label{problema1_2:hipotesis_del_problema}
		\mathbb{P} (T \leq N + n | F_n) > \varepsilon \text{ casi seguramente}
		\end{equation}
		
        Al verificar la desomposici\'on
		
        \begin{equation}\label{problema1_2:sugerencia_del_problema}
			\mathbb{P} (T>kN)= \mathbb{P} (T>kN,T>(k-1)N)
		\end{equation}
		
        pruebe por inducci\'on que para cada \(k=1,2,\ldots\):
		
        \begin{align}
			\mw(T>kN)\leq (1-\varepsilon)^k.
		\end{align}
		
        Pruebe que \( \mathbb{E}(T)<\infty \). 
	\begin{categoria} Tiempos de paro.\end{categoria}
\end{problema}
\afterstatement
\begin{proof}
	Tenemos que: 
	
	\begin{align}
		\left(T>kN \Rightarrow T>(k-1)N\right) 	&\Rightarrow (T>kN) \subset (T>(k-1)N)          \\ 
                                                &\Rightarrow (T>kN) \cap (T>(k-1)N) = (T>kN)    \\ 
                                                &\Rightarrow \mw(T>kN, T>(k-1)N) = \mw(T>kN)	
	\end{align}\pn
		
	\textbf{Base de inducción.} $k=1$. Usando \eqref{problema1_2:hipotesis_del_problema}, con $n=0$ 
	
    \begin{align}
		\mw(T\leq 1N | \F_0) > \varepsilon\Rightarrow   \\
		\mw(T>N| \F_0) < 1 - \varepsilon
	\end{align}\pn
    
	Sustituyendo por la definición de probabilidad condicional tenemos:

    \begin{align}
        \E(\indic_{T>N} | \F_0)	&= \mw(T>N| \F_0) \\
                                &< 1 - \varepsilon
    \end{align}\pn

	Aplicando esperanza en ambos lados tenemos:
    
    \begin{align} 
        \mw(T>N) 	&= 	\E(\E(\indic_{T>N} | \F_0)) \\
                    &< 	\E(1 - \varepsilon)         \\
                    &= 1 - \varepsilon.
    \end{align}\pn
	
	\textbf{Hipótesis de induccion.} 
    Supongamos que $\mw(T>k_0N)\leq(1 - \varepsilon)^{k_0}$ para algún $k_0 \geq 1$.\pn
	
	\textbf{Paso inductivo.} 
	Utilizando \eqref{problema1_2:sugerencia_del_problema} tenemos que
    
    \begin{align}
        \mw(T>(k_0+1)N) = \mw(T>(k_0+1)N, T>k_0N) = \E(\indic_{T>(k_0+1)N} \cdot \indic_{T>k_0N}).
    \end{align}\pn
        
	Ahora, dado que $T>k_0N$ es un conjunto $\F_{k_0N}$-medible tenemos:
    
	\begin{align} 
		\E(\indic_{T>(k_0+1)N} \cdot \indic_{T>k_0N}) 	&=		\E\left(\E\left(\indic_{T>(k_0+1)N} \cdot \indic_{T>k_0N} | \F_{k_0N}\right)\right)                                     \\ 
														&=		\E\left(\indic_{T>k_0N} \E\left(\indic_{T>(k_0+1)N}|\F_{k_0N}\right)\right) \label{problema1_2:resultado_preliminar}
	\end{align}\pn
    
	Utilizando  $n=k_0N$ en \eqref{problema1_2:hipotesis_del_problema} tenemos
    
	\begin{align}
		\mw\left(T>k_0N+N|\F_{k_0N}\right) = \E\left(\indic_{T>(k_0+1)N}|\F_{k_0N}\right) < 1-\varepsilon
	\end{align}\pn
    
	Sustituyendo esto último en \eqref{problema1_2:resultado_preliminar} obtenemos:
    
    \begin{align}
             E\left(\indic_{T>k_0N} \E\left(\indic_{T>(k_0+1)N}|\F_{k_0N}\right)\right) 	&< 		E\left(\indic_{T>k_0N} (1-\varepsilon)\right)   \\
                                                                                            &=	 	(1-\varepsilon) E\left(\indic_{T>k_0N} \right)  \\
                                                                                            &=		(1-\varepsilon) \mw(T>k_0N)                     \\
                                                                                            &\leq   (1-\varepsilon)(1-\varepsilon)^{k_0}            \\
                                                                                            &= (1-\varepsilon)^{k_0 + 1}.
    \end{align}\pn
        
	Con lo que concluimos
    
    \begin{align}
        \mw(T>(k_0+1)N) &\leq (1-\varepsilon)^{k_0 + 1}.
    \end{align}\pn
        
	Terminando así la demostración por inducción.\pn
	
	
	Para la siguiente prueba notemos que si $X$ es una variable aleatoria y $r > s \in \mathbb{R}$ entonces
	$(X > s) \subset (X > r)$ y por lo tanto $\mw(X > s) \leq \mw(X > r)$.\pn
	
	En particular para nuestro tiempo de paro $T$, si dado $n \in \mathbb{N}$, $k \in \mathbb{N}$ es tal que 
	$kN \geq n < (k+1)N$, entonces $\mw(T > n) \leq \mw(T > kN)$.\pn

	Trabajando por bloques de tamaño $N$ tenemos que 
    
	\begin{align}
		\sum_{n = kN}^{(k+1)N -1} \mw(T > n) \leq N \mw(T > kN).
	\end{align}\pn		
	
	También recordemos que si $T$ es una variable aleatoria positiva con valores en los enteros entonces 
	
    \begin{align}
		\E(T) = \sum_{n=0}^{\infty} \mw(T > n).
	\end{align}\pn
	
	Sustituyendo nuestro penúltimo razonamiento en esta ultima fórmula tenemos:
	
	\begin{align}
		\E(T) 	&= 		\sum_{n=0}^{\infty} \mw(T > n)                              \\
				&= 		\sum_{k=0}^{\infty} \sum_{n = kN}^{(k+1)N -1} \mw(T > n)    \\
				&\leq 	N \sum_{k=0}^{\infty} \mw(T > kN)
	\end{align}\pn
		
	Sustituyendo nuetstro resultado de que $\mw(T>kN)<(1-\varepsilon)^{k}$ resulta que:
    
	\begin{align}
		\E(T)\leq N \sum_{k=0}^{\infty} \mw(T > kN) \leq N \sum_{k=0}^{\infty} (1-\varepsilon)^k.
	\end{align}\pn
	
	De lado derecho de esta última desigualdad tenemos una serie geométrica. Dado que $\varepsilon$ es mayor 
	que $0$ y entonces que $1-\varepsilon$ es menor que $1$, tenemos que es una serie geométrica que converge 
	en los reales y por lo tanto $\E(T)$ está acotada por un número real.
\end{proof}	
		\newpage
		
		\subsection{Problema 1.3}
		\begin{problema}
	\emph{Tomado de Mathematical Tripos, Part III, Paper 33, 2012, \url{http://www.maths.cam.ac.uk/postgrad/mathiii/pastpapers/}}

	Sean $\paren{X_i,i\in\na}$ variables aleatorias 
	independientes con $\proba{X_i=\pm 1}=1/2$. Sean $S_0=0$ y $S_n=\sum_{i=1}^n X_i$. 

	\begin{enumerate}
		\item[(i)] Sea $T_1=\min\set{n\geq 0:S_n=1}$. Explique por qu\'e $T_1$ es un 
		tiempo de paro y calcule su esperanza.
		
		\item[(ii)] Mediante el inciso anterior, construya una martingala que converge 
		casi seguramente pero no lo hace en $L_1$.
		
		\item[(iii)] Sea $M_n$ la martingala obtenida al detener a $-S$ en $T_1$. Utilice la solución al
		Problema de la Ruina para probar que $\mw(max_n M_n \geq M) = 1/(M+1)$ para todo $M \geq 1$. Concluya que
		$\E(max_m M_n) = \infty$ y que por lo tanto $\E(max_{m\leq n} M_m) \rightarrow \infty$ conforme 
		$n \rightarrow \infty$. Finalmente, deduzca que no puede haber una desigualdad de tipo Doob cuando $p=1$.
		
		\item[(iv)] Sea $T=\min\set{n\geq 2:S_n=S_{n-2} + 2}$ y $U=T-2$. ?`Son $T$ y $U$ 
		tiempos de paro? Justifique su respuesta.
		
		\item[inciso (v)] Para la variable $T$ que hemos definido, calcule $\esp{T}$. 
	\end{enumerate}

	\defin{Categor\'ias: } Tiempos de paro, problema de la ruina
\end{problema}
\begin{proof}
	\subsubsection{Inciso (i)}
	\emph
{
	Sea $T_1=\min\set{n\geq 0:S_n=1}$. Explique por qu\'e $T_1$ es un 
	tiempo de paro y calcule su esperanza.\\
}
\afterstatement
	Consideremos a la filtración $(\F_n)_{n\in\N}$ como la filtracion 
	generada por $X_1, X_2, \dots$.\par\null

	Es decir, $F_0 = \{\emptyset, \Omega\}$, $\F_n = \sigma(X_1, X_2, \dots, X_n)$\\

	Nótese que $S_0$ es medible bajo cualquier sigma álgebra por ser constante, en particular bajo
	$\F_0$.\par\null

	Basta demostrar que $(T_1 = n) \in \F_n$ para ver que $T_1$ es tiempo de paro. $T_1$ 
	representa el primer tiempo en que la suma es igual a $1$. Es decir, para cualquier 
	momento anterior, la suma no es $1$.\par\null

	Eso escrito en símbolos significa:

    \begin{align}
        (T_1 = n) = \bigcap_{i=0}^{n-1}(S_i \not= 1) \cup (S_n = 1).
    \end{align}\par\null

	Para $n=0$, $(S_0 = 1) = \omega \in \F_0$.\par\null

	Como $(S_i \not= 1) \in \F_j$ siempre que $i \leq j$. Para $n>0$, $(T_1 = n)$ es el resultado de 
	unir e intersectar conjuntos $\F_n$-medibles, lo cual resulta $\F_n$-medible.\par\null

	Para $m \in \N$. Definamos $T_m = min\{n \geq 0 : S_n = m\}$ 
	(Nótese que para el caso $m=1$, esta definición	coincide con la definición previa de $T_1$).\par\null
	
	Para $a,b \in \N$, podemos definir el tiempo de paro $T_{a,b} = T_{-a} \wedge T_b$, y 
	corresponde al 	tiempo de paro del problema de la ruina. Para este tiempo de paro ya conocemos 
	la esperanza y es
	
    \begin{align}
        \E(T_{a,b}) = ab.
    \end{align}\par\null
	
	Ahora, definamos la sucesion de variables aleatorias $T_{1,1}, T_{2,1}, T_{3,1}, \dots, T_{n,1}, 
	\dots$. Notemos que si $a>a' \in N$ entonces $T_{-a} > T_{-a'}$, pues $T_{-a}$ es la primera vez
	que se llega a $-a$, y para poder alcanzar $-a$ era necesario haber pasado por $-a'$.
	De aqui tenemos que si $a>a'$, entonces $T_{a,1} \geq T_{a',1}$. De donde nuestra suceción es 
	no decreciente.\par\null
	
	Por otro lado, que si $a>a' \in N$ entonces $T_{-a} > T_{-a'}$ implica que $T_{-n} n \in \N$ es 
	una suceción extrictamente creciente y por lo tanto 
	$\lim\limits_{n \rightarrow \infty} T_{-n} = \infty$, con esto tenemos que el límite de nuestra 
	suceción es 
	
    \begin{align}		
        \lim_{n\rightarrow\infty} T_{n,1}   &=  \lim_{n\rightarrow\infty} T_{-n} \wedge T_1 \\
                                            &=  \infty \wedge T_1                           \\
                                            &=  T_1
	\end{align}

	Tenemos todos los ingredientes para usar Teorema de convergencia monótona sobre nuestra suceción
	y la variable $T_1$. Nuestra sucecion es monótona y converge puntualmente a $T_1$. Utilizando
	dicho teorema obtenemos:
	
	\begin{align}
        \E(T_1)     &=  \E(\lim_{n\rightarrow\infty} T_{n,1})       \\
                    &=  \lim_{n\rightarrow\infty} \E(T_{n,1})       \\
                    &=  \lim_{n\rightarrow\infty} n\cdot 1          \\
                    &=  \infty.
	\end{align}
	
	Lo cual era intuitivo. Si $\E(T_1)$ fuese finito, diría que existe un número de volados donde
	uno puede apostar con mucha certeza que ganará un peso después de jugar "cerca" de esa cantidad
	de volados. Intuitivamente, esto vuelve injusto un juego de volados donde la moneda es
	justa.	
	\newpage
	
	\subsubsection{Inciso (ii)}
	\emph
{	
	Mediante el inciso anterior, construya una martingala que converge 
	casi seguramente pero no lo hace en $L_1$.\\
}
\afterstatement	
	En [\ref{problema1_4}] se probará que si $T$ y $S$ son tiempos de paro, entonces $T\wedge S$ también 
	es tiempo de paro. Con esto tenemos que si $T_1$ es tiempo de paro, entonces $T_1 \wedge n$ con 
	$n \in \N$ también es tiempo de paro. Definimos entonces 
	$$M_n = S_{T_1 \wedge n}.$$
	
	Veamos que los $M_n$ forman una martingala.
	
	\begin{itemize}
		\item 
			$M_n$ es adaptada a la filtración.
			\begin{align}
				M_n(w) = S_{T_1 \wedge n}(w) = 
				S_{T_1 \wedge n (w)}(w) = 
				\sum_{k=1}^{T_1 \wedge n (w)} X_k = 
				\sum_{k=1}^{n} (X_k \cdot \indic_{T_1 \geq k})(w).
			\end{align}
			
			De donde, podemos escribir:
			\begin{align}\label{problema1_3:descomposicion_de_M_n}
				M_n = \sum_{k=1}^{n} (X_k \cdot \indic_{T_1 \geq k}).
			\end{align}\pn							 		
			
			Recordemos que $X_k$ es $\F_n$-medible para toda $k \leq n $. Por ser
			$T_1$ tiempo de paro, los conjuntos $A_k = \{T_1 = k\}$ y 
			$B_k = \{T_1 \leq k\}$	son $F_k$ medibles y por lo tanto 
			$A_k \cup B_k^c = \{ T_1 \geq k\}$ también lo es. De aquí que 
			$\indic_{T_1 \geq k}$ es $\F_k$-medible y por lo tanto también $\F_n$-medible
			para toda $n$ tal que $n \geq k$.\pn
			  
			Entonces $M_n$ es suma y productos de funciones $\F_n$-medibles y por lo tanto
			$F_n$-medible. Que es lo que queríamos demostrar.\pn
			
		\item
			$M_n \in L_1$\pn
			
			De \eqref{problema1_3:descomposicion_de_M_n} podemos ver que $M_n$ es 
			suma finita de variables acotadas. Por lo tanto $M_n \in L_1$.\pn
			
		\item Ahora probaremos que	$\E(M_{n+1} | \F_{n}) = M_{n}$\pn
			
			Primero:
			\begin{align}
				\E(M_{n+1} | \F_{n})    &=  \E( S_{T_1 \wedge (n+1)} | \F_{n})                                                          \\
                                        &=  \E\bigg( \sum_{k=1}^{n+1} (X_k \cdot \indic_{T_1 \geq k})\bigg| \F_{n}\bigg)                \\	 			
                                        &=  \E\bigg( \sum_{k=1}^{n} (X_k \cdot \indic_{T_1 \geq k}) \bigg| \F_{n}\bigg) +
                                            \E\bigg((X_{n+1} \cdot \indic_{T_1 \geq n+1}) \bigg| \F_{n}\bigg)                           \\
                                        &\comment{Este paso es gracias a que $X_k$ y $\indic_{T_1 \geq k}$ son $\F_n$-medibles}  		\\
                                        &=  \sum_{k=1}^{n} (X_k \cdot \indic_{T_1 \geq k}) + 
                                            \E((X_{n+1} \cdot \indic_{T_1 \geq n+1}) | \F_{n})                                          \\
                                        &=  S_{T_1 \wedge n} + \E((X_{n+1} \cdot \indic_{T_1 \geq n+1}) | \F_{n})                       \\
                                        &=  M_n + \E((X_{n+1} \cdot \indic_{T_1 \geq n+1}) | \F_{n})
			\end{align}
			
			Entonces, nos basta probar que $\E(X_{n+1} \cdot \indic_{T_1 \geq n+1} |
			 \F_{n}) = 0$ para terminar nuestra demostración.\pn
			 
			Sean $A = \{T_1 = n\}$ y $B = \{ T_1 \leq n\}$. Por ser $T_1$ tiempo de paro,
			$A$ y $B$ son $\F_n$-medibles. Por lo tanto $B \setminus A$ también es $\F_n$-medible. 
			Notemos que $\{T_1 \geq n+1\} = (B \setminus A)^c$. Por lo tanto $\{T_1 \geq n+1\}$ es
			$\F_n$-medible. De donde  $\indic_{T_1 \geq n+1})$ es $\F_n$-medible.\pn
			
			Con esto, ahora tenemos que:
			\begin{align}
				\E((X_{n+1} \cdot \indic_{T_1 \geq n+1}) | \F_{n})  &=  \indic_{T_1 \geq n+1} \cdot \E(X_{n+1} | \F_{n})                        \\
                                                                    &\;\;\;\;\mbox{(Este paso es gracias a que los }                            \\
                                                                    &\;\;\;\;\;\mbox{$X_n$ son independientes)}                                 \\
                                                                    &=  \indic_{T_1 \geq n+1} \cdot \E(X_{n+1} )                                \\
                                                                    &=  \indic_{T_1 \geq n+1} \cdot 0                                           \\
                                                                    &=  0
			\end{align}\pn
			
			Como queríamos demostrar.
	\end{itemize}
	
	Ahora que tenemos que $(M_n)_{n \in \N}$ es martingala, confirmemos que converge casi seguramente.\pn
	
	Notemos que $(T_1 \wedge n)_{n \rightarrow \infty} \rightarrow T_1$ c.s.\pn
	
	De aquí que $(M_n)_{n \rightarrow \infty} = (S_{T_1 \wedge n})_{n \rightarrow \infty} = S_{T_1}$ c.s.\pn				
	
	Veamos que la convergencia no ocurre en $L_1$.\pn
				
	Dado que $T_1 \wedge n$ es un tiempo de paro acotado para toda $n \in \N$,
	podemos aplicar el Teorema de Muestreo Opcional de 	Doob. 
	El cual nos dice que $\E(M_n) = \E(S_{T_1 \wedge n}) = \E(S_0) = 0$.\pn
	
	Por otro lado, por definición de $T_1$, $S_{T_1} = 1$ c.s.	De donde $\E(S_{T_1}) = 1$.
	
	\begin{align}
		\E(M_n) = 0 \not\rightarrow 1 = \E(S_{T_1}).
	\end{align}\pn			
	
	Y con esto, queda demostrado que la convergencia no se da en $L_1$.	
	\newpage
	
	\subsubsection{Inciso (iii)}
	\emph{
	Sea $M_n$ la martingala obtenida al detener a $-S$ en $T_1$. Utilice la solución al
	Problema de la Ruina para probar que \\
	$\mw(max_n M_n \geq M) = 1/(M+1)$ para todo $M \geq 1$. Concluya que\\
	$\E(max_m M_n) = \infty$ y que por lo tanto $\E(max_{m\leq n} M_m) \rightarrow \infty$ conforme 
	$n \rightarrow \infty$. Finalmente, deduzca que no puede haber una desigualdad de tipo Doob cuando $p=1$.
}\pn
	
\afterstatement

	Definimos $M_n = -S_{T_1 \wedge n}$. Notemos que $M_n$ únicamente toma valores en $[-1, \infty]$.
	Para calcular $\mw(max_n M_n \geq M)$ notemos primero que:

	\begin{align}
		\mw(max_n M_n \geq M) = 1 - \mw(max_n M_n < M).
	\end{align}\pn
	
	$\max_n M_n < M$ significa que $M_n$ nunca alcanza el valor $M$.\pn
	 
	Intentando hacer analogía con el problema de la ruina, pensemos en dos concursantes,
	uno con $1$ peso y otro con $M$ pesos. Nunca alcanzar $M$ significa que nunca gana el que tiene $1$ peso.\pn
	
	Esta probabilidad ya la conocemos y es 
	
	\begin{align}
		\mw(max_n M_n < M) = \frac{M}{M + 1}
	\end{align}\pn
		
	Por lo tanto
	
	\begin{align}
		\mw(max_n M_n \geq M) 	&= 1 - \mw(max_n M_n < M) \\
								&= 1 - \frac{M}{M + 1}\\
								&= \frac{M+1}{M+1} - \frac{M}{M + 1}\\
								&= \frac{1}{M+1}
	\end{align}\pn
	
	Utilizando este resultado:

	\begin{align} \label{problema1_3:esperanza_del_maximo_de_M_n}
		\E(max_n M_n) 	&= - \mw(max_n M_n = -1) + \sum_{M=1}^{\infty} \mw(max_n M_n \geq M) \\
						&= - \mw(max_n M_n = -1) + \sum_{M=1}^{\infty} \frac{1}{M+1} \\ 
						&= - \mw(max_n M_n = -1) + \infty \\
						&= \infty
	\end{align}\pn						
	
	Ahora, tenemos que:
    
	\begin{align}
		\|\overline{M_{n}^{+}}\|_1  &=    \E{\overline{M_{n}^{+}}} \\
									&=    \E{\max_{m \leq n}M_m^+} \\
									&\geq \E{\max_{m \leq n}M_m}										
	\end{align}\pn
		
	Donde, el último término, tiende a infinito en base al resultado 
	\eqref{problema1_3:esperanza_del_maximo_de_M_n}.\pn

	Por otro lado:
    
	\begin{align}
		\|M_n^+\|_1=\|-S_{T_{1\wedge n}}^{+}\|_1  \longrightarrow  \|-S_{T_1}^+\|_1 = 0 < \infty
	\end{align}\pn
	
	Por lo tanto, no existe número $K$, tal que
	
    \begin{align}
		 \|\overline{M_n^+}\|_1 \leq  K \|M_n^+\|_1
	\end{align}\pn
	
	En otras palabras, no tenemos una desigualdad de tipo Doob para $p=1$.		
	\newpage
	
	\subsubsection{Inciso (iv)}
	\emph
{
	Sea $T=\min\set{n\geq 2:S_n=S_{n-2} + 2}$ y $U=T-2$. ?`Son $T$ y $U$ 
	tiempos de paro? Justifique su respuesta.
}\pn

\afterstatement

	Intuitivamente, $T$ significa, el primer tiempo tal que ganamos en dos volados consecutivos.
	También intuitivamente, esto debería ser un tiempo de paro.\pn
	
	Veamos que efectivamente así ocurre. Utilizando la siguiente prueba por inducción:\pn
	
	\textbf{Base de inducción:}\pn		
    
		\begin{align}
			\{T = 0\} 		&= \emptyset  				& 	\in \F_0 \\
			\{T = 1\} 		&= \emptyset  				& 	\in \F_1 \\
			\{T = 2\} 		&= \{ X_1 = 1, X_2 = 1\} 	&	\in \F_2
		\end{align}\pn					
	
	\textbf{Hipótesis de inducción:}\pn
	
		Supongamos que $\{T = n\} \in \F_n$ para cierto $n \geq 2$.\pn
		
	\textbf{Paso inductivo:}\pn
		
		\begin{align}
			\{T = n + 1 \} = \{ X_n = 1, X_{n+1} = 1\} \setminus \bigcup_{i=0}^{n} \{T = i\}.
		\end{align}\pn				
	
		Es claro que $\{ X_n = 1, X_{n+1} = 1\} \in \F_{n + 1}$ y que por hipótesis de inducción
		$\bigcup_{i=0}^{n} {T = i} \in \F_n \subset \F_{n + 1}$. Por lo tanto
		$\{T = n + 1 \} \in \F_{n+1}$ para toda $n \geq 2$ y con esto termina la demostración.\pn
		
	Ahora, intuitivamente $U$ significa el momento justo antes de ganar dos volados consecutivos.
	Esto, quedría decir que tenemos información sobre eventos que aún no ocurren. Así que intuitivamente
	esto no debería ser un tiempo de paro.\pn
	
	Efectivamente, si tomamos como ejemplo el conjunto: 

    \begin{align}
        \{ U = 1 \} = \{ T - 2 = 1\} = \{ T = 3\} = \{X_1 = -1, X_2 = 1, X_3 = 1\}
    \end{align}\pn
			
	Es fácil notar que es un conjunto que pertenece a $\F_3$, pero no a $\F_1$. Pues $\F_1$
	no contiene información alguna sobre $X_2$ y $X_3$. Así que el conjunto más pequeño de $\F_1$ 
	que contiene a $\{ U = 1 \}$ es $\{ X_1 = -1 \}$.
	\newpage
	
	\subsubsection{Inciso (v)}
	\emph
{
	Para la variable $T$ que hemos definido, calcule $\esp{T}$.\\
}
	Primero, platicaré de manera intuitiva cómo vamos a proceder para solucionar este problema.\\

	Imaginemos un juego de casino a base de un juego de volados con las siguientes reglas:\\
	\begin{itemize}
			\item En cada turno, cada jugador tiene que apostar todo el dinero que tiene.
			\item Si un jugador se queda sin dinero, tiene que abandonar el juego.
			\item Si cae ``sol" cada jugador recibe el doble de lo que había apostado en ese turno.  
	\end{itemize}
   
	\;El casino tiene dinero infinito y cada habitante cuenta con exactamente $1$ peso antes de
	iniciar el juego.\\
   
	Además, en cada nuevo turno entra exactamente un nuevo jugador al juego.\\
   
	Para nuestro problema, supongamos que $X_n = 1$ significa que en el $n$-ésimo turno, salió sol.
	Entonces, $T$ nos indica cuando es la primera vez que caen dos soles consecutivos.\\
   
	Sea $D_n$ la variable que indica cuánto dinero ha ganado el casino para el tiempo $n$.\\
   
	Observemos que en el momento que cae ``águila", todo jugador pierde todo su dinero y abandona el juego.
	Y que por cada jugador que pierde, el casino gana exáctamente $1$ peso (pues cada jugador en cada
	turno apuesta todo el dinero que posee, es decir el peso con el que empezó y todo lo que le habia
	ganado al casino).\\
   
	Entonces, al tiempo $T-2$, todo mundo había perdido. Es decir que al tiempo $T-2$ el casino ha ganado
	$T-2$ pesos.\\
   
	Luego, al tiempo $T-1$, ha caido un sol y hay exactamente un jugador al que el casino tuvo 
	que pagar $1$ peso.\\
   
	Al tiempo $T$, al jugador del turno pasado el casino tuvo que darle $2$ pesos y al jugador del nuevo
	turno tuvo que darle $1$ peso.\\
   
	Entonces, ya podemos decir cuanto dinero ha ganado el casino al tiempo $T$.
	\begin{align}\label{problema1_3:Dinero_al_tiempo_T}
		D_T = T-2 - 1 - 3 = T - 6. 
	\end{align}					   
   
	Notemos que además el juego es justo, en cada turno cada jugador tiene $1/2$ de probabilidad de
	ganar $2^t$ y $1/2$ de probabilidad de perder $2^t$. Es decir, la esperanza es $0$.\\
	
	$D_n$ es suma de este tipo de variables y por lo tanto su esperanza también será $0$.\\
   
	Esto, nos da la intuición de que $D_n$ es martingala, pero esa es la parte que demostraremos más adelante.\\
   
	Si logramos demostrar que $\E(D_T) = 0$. De (\ref{problema1_3:Dinero_al_tiempo_T}) concluimos
	\begin{align}
		0 = \E(T - 6) = \E(T) - 6
	\end{align}
	
	De donde $\E(T) = 6$.\\
	
	Ahora, para terminar con las formalidades, definamos bien a $D$ y comprobemos que es martingala y 
	que podemos utilizar el Teoremoa del muestreo opcional de Doob como lo hemos hecho.\\
	
	Sean entonces $(Y_n)_{n \in \N}$ variables aleatorias Bernulli de parámetro $1/2$ independientes.
	Y sea $Z_n^m$ la cantidad de dinero que el jugador $m$ ha dado al casino definidas como:
	
	\begin{itemize}
		\item 
			Si $n < m$ entonces $Z_n^m = 0$ . (El jugador $m$ no participa en el juego sino hasta el turno $m$).
		\item
			$Z_{n+1}^{m} = (Z_n^{m} - 1) \cdot 2(Y_{n+1}) + 1$. (Si $Y_{n + 1} = 1$ [El jugador gana el volado], entonces el casino
			pierde la cantidad apostada, que para el turno $n+1$ es $Z_n^n + 1$). Nótese que en cuanto un $Y_{n_0}$ se hace cero,
			$Z_{n_0}^{n}$ y todos los que le sigan son todos iguales a $1$ (Como el jugador deja el juego después de haber perdido
			un volado, deja su peso en el casino y entonces de ahí en adelante la cantidad que ha dado al casino es exactamente 1).
	\end{itemize}
	
	Veamos que para cada $m \in \N$, $(Z_n^m)_{n \in \N}$ forma una martingala con respecto a la filtración $(\G_n)_{n \in \N}$ 
	definida por los $Y_n$.\\
	
	Cada $Z_n^m$ es $\G_m$-medible por definición.\\
	
	Cada $Z_n^m$ es suma finita de variables acotadas. Por lo tanto cada una pertenece a $L_1$.\\
	
	Sólo nos falta verificar la propiedad de martingala.
	
	\begin{align}\label{problema1_3:Propiedad_de_martingala_para_el_dinero_perdido_por_un_apostador}
		\E(Z_{n+1}^{m} | \G_n) &= \E((Z_n^{m} - 1) \cdot 2(Y_{n+1}) + 1 | \G_n)\\
							   &= \E((Z_n^{m} -1) \cdot 2(Y_{n+1}) | \G_n) + \E( 1 | \G_n)\\
							   &\;\;\;\;\mbox{[Por ser $(Z_n^{m} - 1)$ una variable $\G_n$-medible.]}\\
							   &= (Z_n^{m} -1) \cdot \E(  2(Y_{n+1}) | \G_n) + \E( 1 | \G_n)\\
							   &= (Z_n^{m} -1) \cdot 2\E( Y_{n+1} | \G_n) + \E( 1 | \G_n)\\
							   &\;\;\;\;\mbox{[Por ser $Y_{n + 1}$ independiente $\G_n$.]}\\
							   &= (Z_n^{m} -1) \cdot 2\E( Y_{n+1}) + \E( 1 | \G_n)\\
							   &= (Z_n^{m} -1) \cdot 2 \frac{1}{2} + \E( 1 | \G_n)\\
							   &= (Z_n^{m} -1) \cdot + 1 \\
							   &= Z_n^m.
	\end{align}
	
	Ahora definamos a $D_n = \sum_{i=1}^n (Z_n^i)$. Que significa, La cantidad de dinero que el casino ha ganado al tiempo $n$. 
	Justo como lo habíamos dicho en la ``demostración intuitiva".\\
	
	Ver que $D$ es martingala es fácil. Cada $D_n$ es suma finita de variables finitas y por lo tanto pertenece a $L_1$.
	$D_n$ es suma de variables $\G_n$-medibles y por lo tanto también lo es. Y la propiedad de martingala se sigue directamente de
	(\ref{problema1_3:Propiedad_de_martingala_para_el_dinero_perdido_por_un_apostador}).\\
	
	Es cierto que podemos aplicar Doob sobre el tiempo $T \wedge n$ por ser acotado y de aquí que:
	$\E(D_{T \wedge n}) = \E(D_1) = 0$.\\
	
	Notemos que $D_{T \wedge n} \longrightarrow D_T \; c.s.$\\
	
	Nos gustaría poder decir lo mismo de sus esperanzas y para eso utilizaremos teroema de convergencia dominada.\\
	
	Es claro que al tiempo $T \wedge n$ el casino a lo más pudo haber ganado $T \wedge n$ pesos.
	De aquí que $D_{T \wedge n} \geq T \wedge n \geq T$.\\
	
	Además, por definición de $T$, para el tiempo $T$ el casino a lo más ha perdido $4$ pesos y es la primera vez que 
	pierde tanto. Así que $-4 \leq D_{T \wedge n}$.\\ 
	
	Entonces, nuestra martingala $D$ esta dominada por $\max(T, 4) < T + 4$. Bastaría demostrar que $\E(T+4) < \infty$ para poder
	utilizar el teorema de convergencia dominada.\\
	
	Notemos que 
	\begin{align}\label{problema1_3:Acotando_T}
		\mw(T > n) &= \frac{1}{2} \mw(T > n-1) + \frac{1}{4} \mw(T > n-2). 
	\end{align}
	
	Pues queremos garantizar que en los primeros $n$ turnos, no pierde dos veces consecutivas el casino.\\
	
	Si en el turno 1, gana el casino (de aquí el $\frac{1}{2}$), en el resto de los $n-1$ turnos tenemos que garantizar que 
	el casino no pierde dos veces consecutivas, como las variables están idénticamente distribuidas, esto es equivalente a que
	$T>n-1$. (De aqui el $\mw(T > n-1)$.\\
	
	Si el casino pierde en el turno 1, necesariamente tiene que ganar en el turno 2 (de aquí el $\frac{1}{4})$. Y la probabilidad
	de que no pierda dos veces consecutivas en los siguientes $n-2$ turnos es $\mw(T > n-2)$.\\					    
	
	De (\ref{problema1_3:Acotando_T}) podemos notar fácilmente que $\mw(T > 2) \leq \frac{3}{4}$\\
	
	También notemos que $\mw(T > n ) \geq \mw(T > n + 1)$ (El primer evento contiene al segundo).
	
	Ahora
	\begin{align}
		\mw(T > 2 (2)) &=     \frac{1}{2} \mw(T > 2+1) + \frac{1}{4} \mw(T > 2) \\
					   &\leq  \frac{1}{2} \mw(T > 2) + \frac{1}{4} \mw(T > 2) \\
					   &\leq  \frac{1}{2} \cdot \frac{3}{4} + \frac{1}{4} \cdot \frac{3}{4} = \bigg(\frac{3}{4}\bigg)^2.
	\end{align}
	
	De manera recursiva tenemos que
	\begin{align}
		\mw(T > 2(n+1)) &=     \frac{1}{2} \mw(T > 2(n+1) - 1) + \frac{1}{4} \mw(T > 2(n+1) - 2) \\
						&=     \frac{1}{2} \mw(T > 2(n) + 1) + \frac{1}{4} \mw(T > 2(n)) \\
					   &\leq  \frac{1}{2} \mw(T > 2n) + \frac{1}{4} \mw(T > 2n) \\
					   &\leq  \frac{1}{2} \cdot \bigg(\frac{3}{4}\bigg)^{n} + \frac{1}{4} \cdot \bigg(\frac{3}{4}\bigg)^{n} = 		
					   \bigg(\frac{3}{4}\bigg)^{n+1}.
	\end{align}
	
	Entonces,
	\begin{align}
		\E(T) 	&= 		\sum_{n=0}^\infty \mw(T > n) \\
				&\leq 	\sum_{n=0}^\infty 2\mw(T > 2n) \\
				&= 		2 \sum_{n=0}^\infty \mw(T > 2n) \\
				&\leq 	2 \sum_{n=0}^\infty \bigg(\frac{3}{4}\bigg)^{n} < \infty.
	\end{align}
	
	Por lo tanto $\E(T + 4) \leq \infty$. Y entonces nuestra martingala $D$ está dominada por una variable integrable y finalmente
	podemos aplicar teorema de convergencia dominada para concluir que:
	\begin{align}
		0 = \lim_{n \longrightarrow \infty} \E(D_{T \wedge n}) = \E(D_T).
	\end{align}
	
	Y con esto terminamos de demostrar todas las formalidades que nos hacían falta.	

\end{proof}
		\newpage
		
		\subsection{Problema 1.4}
		\begin{problema}[Extensiones del teorema de paro opcional]
	Sea ${M=\paren{M_n,n\in\na}}$ una (super)martingala respecto de una filtraci\'on ${\paren{\F_n,n\in\na}}$ y sean ${S}$ y ${T}$ tiempos de paro.
	
\begin{enumerate}
                \item[(i)] 
                	Pruebe que ${S\wedge T}$, ${S+T}$ y ${S\vee T}$ son tiempos de paro.
                
                \item[(ii)] 
                	\begin{esn}
                		\F_T=\set{A\in\F:A\cap\set{T\leq n}\in\F_n\text{ para toda } n}
                	\end{esn}
                	es una ${\sigma}$-\'algebra, a la que nos referimos como la ${\sigma}$-\'algebra 
                	detenida en ${\tau}$. Comente qu\'e puede fallar si ${T}$ no es tiempo de paro. 
                	Pruebe que ${T}$ es ${F_T}$-medible. 
                
                \item[(iii)] 
                	Pruebe que si ${T}$ es finito, entonces ${M_T}$ es ${\F_T}$-medible.
                
                \item[(iv)] 
                	Pruebe que si ${S\leq T\leq n}$ entonces ${\F_S\subset\F_T}$. Si adem\'as ${T}$ es acotado entonces ${X_S,X_T\in L_1}$ y 
                	\begin{esn}
	                	\espc{M_T}{\F_S}\leq M_S.
                	\end{esn}

                \item[(v)] 
                	Si ${X=\paren{X_n,n\in\na}}$ es un proceso estoc\'astico ${\paren{\F_n}}$-adaptado y tal que ${X_n\in L_1}$ y tal que 
                	para cualesquiera tiempos de paro acotados ${S}$ y ${T}$ se tiene que ${\esp{X_S}=\esp{X_T}}$ entonces ${X}$ es una 
                	martingala. Sugerencia: considere tiempos de paro de la forma ${n\indi{A}+(n+1)\indi{A^c}}$ con ${A\in\F_n}$.
\end{enumerate}

\defin{Categor\'ias: }Tiempos de paro, Muestreo opcional
\end{problema}

\begin{proof}
	\subsubsection{Inciso (i)} 
	\input{tarea1/problema1_4/inciso1.tex}
	
	\subsubsection{Inciso (ii)}
	\emph{
	\begin{align}
		\F_T=\set{A\in\F:A\cap\set{T\leq n}\in\F_n\text{ para toda } n}
	\end{align}
	es una $\sigma$-\'algebra, 
	a la que nos referimos como la $\sigma$-\'algebra detenida en $\tau$. Comente qu\'e puede fallar si $T$ no es tiempo de paro. 
	Pruebe que $T$ es $\F_T$-medible.\\		
}			
		
	Primero hay que demostrar que $\F_T$ es $\sigma$-algebra.\\
	
	\begin{itemize}
		\item $\Omega \in \F_T$. \\
		
			Notemos que 
			\begin{align}
				\Omega \cap \{T \leq n\} = \{T \leq n\} \in \F_n.
			\end{align}
			Donde la pertenencia a $\F_n$ es gracias a que $T$ es tiempo de paro. (Esta parte podria fallar si 
			$T$ no fuera tiempo de paro).\\
		
		\item $\F_T$ es cerrado bajo complementación.\\
		
			Sea $A \in \F_T$. Eso significa que para todo $n \in  \N$ ocurre que $B = A \cap \{ T \leq n \} \in \F_n$. 
			Por ser $\F_n$ una $\sigma$-algebra tenemos que el complemento de $B$ también debe estar en $\F_n$. 
			Escrito en símbolos:
			
			\begin{align}
				B^c 	&= (A   \cap \{ T \leq n \})^c \\
						&=  A^c \cup \{ T > n \} \in \F_n
			\end{align}
			
			Dado que $B^c$ y $\{ T \leq n \}$ se encuentran en $\F_n$, también \\
			$B^c \cap \{ T \leq n \} \cap \{ T \leq n \}$.\\
			
			Escribiendo esto último de otra manera:
			\begin{align}
				B^c \cap \{ T \leq n \} \cap \{ T \leq n \} 	&=		A^c \cup \{ T > n \} \cap \{ T \leq n \} \cap \{ T \leq n \} \\
																&= 		A^c \cap \{ T \leq n \} \cup \{ T > n \} \cap \{ T \leq n \} \\
																&= 		\bigg(A^c \cap \{ T \leq n \}\bigg) 
																			\bigcup 
																		\bigg(\{ T > n \} \cap \{ T \leq n \}\bigg) \\
																&=		\bigg(A^c \cap \{ T \leq n \}\bigg)	\bigcup \emptyset \\
																&=		A^c \cap \{ T \leq n \} \in \F_n
			\end{align}
			
			Y por lo tanto $A^c \in \F_T$.\\
			
		\item $\F_T$ es cerrado bajo uniones numerables.
		
			Sean $A_m \in \F_T$ para $m \in \N$.
			\begin{align}
				\bigg( \bigcup_{k=1}^\infty A_k \bigg) \cap \{ T \leq n \} 	&=	\bigcup_{k=1}^\infty \bigg( A_k \cap \{ T \leq n \} \bigg)
			\end{align}								
			
			La última parte, por definición de $\F_T$ es una unión numerable de elementos de $\F_n$ y por lo tanto, dicha unión
			tambien pertenece a $\F_n$. Y por último $\cup A_k \in \F_T$.
	\end{itemize}
	\null
	
	Ahora veamos que $T$ es $\F_T$-medible. Dado un $k \in \N$, ocurre que:
	
	\begin{align}
		\{T \leq k\} \cap \{T \leq n\} = \{T \leq min(k,n)\} \in \F_{min(k, n)} \subset \F_n \forall n \in \N.
	\end{align}

	Y por lo tanto $\{T \leq k\} \in \F_T$.
	
		
	\subsubsection{Inciso (iii)}
	\input{tarea1/problema1_4/inciso3.tex}
	
	\subsubsection{Inciso (iv)} 
	\emph{
	Pruebe que si ${S\leq T\leq n}$ entonces ${\F_S\subset\F_T}$. Si adem\'as ${T}$ es acotado entonces ${M_S,M_T \in L_1}$ y 
	\begin{align}
		\espc{M_T}{\F_S}\leq M_S.
	\end{align}	
}

	Primero probaremos que ${\F_S \subset \F_T}$. Sea $n \in \N$. Notemos que $\{ T \leq n \} \subset \{ S \leq n \}$ pues
	si ${\omega \in \{ T \leq n \}}$, entonces ${S(\omega) \leq T(\omega) \leq n }$ y por lo tanto ${\omega \in \{ S \leq n \}}$.\\

	Ahora sea ${A \in \F_S}$. Entonces

	\begin{align}
		A \cap \{ T \leq n \} 	&=		A  \cap \{ T \leq n \} \cap \{ S \leq n \} \\
								&=		(A  \cap \{ S \leq n \}) \cap \{ T \leq n \} \in \F_n
	\end{align}

	Como escogimos $n$ arbitrariamente, esto es cierto para toda $n \in \N$. Y por lo tanto $A \in \F_T$ y $\F_S \subset \F_T$.\\

	
	\subsubsection{Inciso (v)}
	\emph{
	Si ${X=\paren{X_n,n\in\na}}$ es un proceso estoc\'astico ${\paren{\F_n}}$-adaptado y tal que ${X_n\in L_1}$ y tal que 
	para cualesquiera tiempos de paro acotados ${S}$ y ${T}$ se tiene que ${\esp{X_S}=\esp{X_T}}$ entonces ${X}$ es una 
	martingala. Sugerencia: considere tiempos de paro de la forma ${n\indi{A}+(n+1)\indi{A^c}}$ con ${A\in\F_n}$.
}
\end{proof}
		
		\nqed
		
	\section{Tarea 2}
		\subsection{Problema 2.1}
		\begin{problema}
        Sea $S_n=X_1+\cdots+X_n$ una caminata aleatoria con saltos $X_i\in \{-1,0,1,\ldots\}$. 
        Sea $C_p$ una variable aleatoria geom\'etrica de par\'ametro $p$ independiente de $S$ y definimos 
        
        \begin{align}
            M_p=-\min_{n\leq C_p} S_n. 
        \end{align}\par\null
        
        El objetivo del ejercicio es determinar la distribuci\'on de $M_p$.\par\null

        (A las caminatas aleatorias como $S$ se les ha denominado Skip-free random walks Para aplicaciones de este tipo 
        de procesos, ver \cite{MR1978607}. Tambi\'en aparecen en el estudio de Procesos Galton-Watson. 
        Este ejercicio es el resultado b\'asico del estudio de sus extremos, denominado teor\'ia de fluctuaciones.)

    \begin{enumerate}
        \item[(i)]  [\ref{problema2_1:inciso1}]
                    Sea
                    
                    \begin{align}
                        g(\lambda)=E(e^{- \lambda X_1}).
                    \end{align} 
                    
                    Pruebe que $g(\lambda)\in (0,\infty)$ y que
                    
                    \begin{align}
                        M_n=e^{-\lambda S_n}g(\lambda)^{-n},n\geq 0
                    \end{align}
                    
                    es una martingala.
        
        \item[(ii)] [\ref{problema2_1:inciso2}]
                    Pruebe que $g$ es log-convexa al aplicar la desigualdad de H\"older. Pruebe que si $P(X_1=-1)>0$ 
                    (hip\'otesis que se utilizar\'a desde ahora) 
                    entonces $g(\lambda)\to\infty$ conforme $\lambda\to\infty$. Utilice esta informaci\'on para esbozar la gr\'afica de $g$. 
                    Defina $ f(s)=\inf \{ \lambda>0:g(\lambda)^{-1} < s\} $. Note que $1/g\circ f=Id$ en $(0,1)$. Pruebe que si $g(\lambda)>1$, 
                    la martingala $M$ es acotada hasta el tiempo de arribo de $S$ a $-k$ dado por 
                    
                    \begin{align}
                        T_k =\min \{n\in\na:S_n=-k\} 
                    \end{align}
                    
                    (donde se utiliza la convenci\'on $\inf\emptyset=\infty$ ). Aplique el teorema de muestreo opcional de Doob para mostrar que 
                    
                    \begin{align}
                        E(s^{T_k})=e^{-k f(s)}.
                    \end{align}
                    
                    Justifique MUY bien por qu\'e la f\'ormula es válida aún cuando $T_k$ puede tomar el valor $\infty$ 
                    y deduzca que de hecho $\p (T_k=\infty)=0$.
                        
        \item[(iii)] [\ref{problema2_1:inciso3}]
                    Argumente que
        
                    \begin{align}
                        P(M_p\geq n)=P(T_n\leq C_p)=E((1-p)^{T_n})
                    \end{align}
                    
                    para demostrar que $M_p$ tiene distribuci\'on geom\'etrica de par\'ametro $1-e^{-f(1-p)}$.
                    
        \item[(iv)] [\ref{problema2_1:inciso4}]
                    Tome el límite conforme $p\to 0$ para mostrar que la variable aleatoria 
                    \begin{align}
                        M=-\min_{n\geq 0}S_n
                    \end{align}
                    tiene una distribuci\'on geom\'etrica de par\'ametro $1-e^{-f(1)}$. Interprete esto cuando $f(1)=0$.
    \end{enumerate}

        \defin{Categor\'ias:} Caminatas aleatorias, muestreo opcional, fluctuaciones.
\end{problema}


\begin{proof}
    \subsection{Inciso (i)}     \label{problema2_1:inciso1}
    \emph{
	Sea
	\begin{align}
		g(\lambda)=E(e^{- \lambda X_1}).
	\end{align} 
	Pruebe que $g(\lambda)\in (0,\infty)$ y que
	\begin{align}
		M_n=e^{-\lambda S_n}g(\lambda)^{-n},n\geq 0
	\end{align}
	es una martingala.
}
    \newpage
    
    \subsection{Inciso (ii)}    \label{problema2_1:inciso2} 
    \emph{
    Pruebe que $g$ es $\log$-convexa al aplicar la desigualdad de H\"older. Pruebe que si $P(X_1=-1)>0$ (hip\'otesis que se utilizar\'a desde ahora) 
    entonces $g(\lambda)\to\infty$ conforme $\lambda\to\infty$. Utilice esta informaci\'on para esbozar la gr\'afica de $g$. 
    Defina $ f(s)=\inf \{ \lambda>0:g(\lambda)^{-1} < s\} $. Note que $1/g\circ f=Id$ en $(0,1)$. Pruebe que si $g(\lambda)>1$, 
    la martingala $M$ es acotada hasta el tiempo de arribo de $S$ a $-k$ dado por 
    \null
    \begin{align}
        T_k =\min \{n\in\na:S_n=-k\} 
    \end{align}
    \null
    (donde se utiliza la convenci\'on $\inf\emptyset=\infty$ ). Aplique el teorema de muestreo opcional de Doob para mostrar que 
    \null
    \begin{align}
        E(s^{T_k})=e^{-k f(s)}.
    \end{align}
    \null
    Justifique MUY bien por qu\'e la f\'ormula es válida aún cuando $T_k$ puede tomar el valor $\infty$ y deduzca que de hecho 
    $\p (T_k=\infty)=0$.
}
\afterstatement
    Primero probemos que $g$ es $\log$-convexa, es decir, que $\log \circ g$ es convexa.Sean $a < b$ en el dominio de $g$ 
    y sea $t \in [0, 1]$.\\
    
    Queremos demostrar que:
    
    \begin{align}
            \log\circ g ((1-t)a + (t)b) \leq (1-t)(\log\circ g (a)) + (t)(\log\circ g (b)).
    \end{align}
    
    Si $t=0$ tenemos:
    
    \begin{align}
        \log\circ g ((1-0)a + (0)b)     &= \log\circ g (a) \\
                                        &= (1-0)(\log\circ g (a)) + (0)(\log\circ g (b))
    \end{align}
    
    y por lo tanto no hay nada que demostrar. Análogamente ocurre con $t=1$.\\
    
    Entonces concentrémonos en el caso donde $t\in (0, 1)$.\\
    
    Recordemos que la desigualdad de Hölder dice que si $f \in L_p$ y $g \in L_q$ con 
    $p,q \in (1,\infty)$ y $\frac{1}{p} + \frac{1}{q} = 1$. Entonces
                
    \begin{align}
                \E(| fg |) \leq (\E(\abs{f}^p))^{\frac{1}{p}} (\E(\abs{g}^q))^\frac{1}{q}.
    \end{align}
    
    Sean entonces $p = \frac{1}{1-t}$ y $q = \frac{1}{t}$. Tenemos que\\
    
    \begin{align}
        \frac{1}{p} + \frac{1}{q}   &= \frac{1}{\frac{1}{1-t}} + \frac{1}{\frac{1}{t}}  \\
                                    &= t + 1 - t                                        \\ 
                                    &= 1.
    \end{align}
    
    Veamos que $e^{-a(1-t) X_1}$ pertenece a $L_p$. \\
    
    Como $e^x > 0$ para todo $x \in \R$. $\abs{e^x} = e^x$. Entonces
    
    \begin{align}
         \bigg(\E\bigg(\abs{e^{-a(1-t) X_1}}^p\bigg)\bigg)^{\frac{1}{p}} 
                &=  \bigg(\E\bigg(\abs{e^{-a(1-t) X_1}}^{\frac{1}{1-t}}\bigg)\bigg)^{{1-t}} \\
                &=  \bigg(\E\bigg(\abs{e^{-aX_1}}\bigg)\bigg)^{{1-t}}                       \\
                &=  \bigg(\E\bigg(e^{-aX_1}\bigg)\bigg)^{{1-t}}                             \\
                &=  ( g(a))^{{1-t}}                                                         \\
                &<   \infty
    \end{align}
    
    Donde la última desigualdad es gracias a que $a$ fue tomado en el dominio de $g$ y a lo demostrado en 
    [\ref{problema2_1:inciso1}]. Con esto hemos demostrado que $e^{-a(1-t) X_1}$ pertenece a $L_p$.\\
    
    De manera análoga se puede demostrar que $e^{-b(t) X_1}$ pertenece a $L_q$.\\
    
    Ahora que tenemos todas las hipótesis para la desigualdad de Hölder, basta aplicarla.\\
        
    \begin{align}
        \log\circ g ((1-t)a + (t)b)     &=       \log\bigg( \E\bigg(e^{-a(1-t) + b(t) X_1}\bigg) \bigg)                              \\
                                        &=       \log\bigg( \E\bigg(e^{-a(1-t) X_1} \cdot e^{-b(t) X_1}\bigg) \bigg)                 \\
                                        &=       \log\bigg( \E\bigg(\abs{e^{-a(1-t) X_1} \cdot e^{-b(t) X_1}}\bigg) \bigg)           \\
                                        &\leq    \log\bigg(
                                                            \E
                                                                \bigg(
                                                                    \abs{e^{-a(1-t) X_1}}^p
                                                                \bigg)^{\frac{1}{p}} 
                                                        \cdot 
                                                            \E
                                                                \bigg(
                                                                    \abs{e^{-b(t) X_1}}^q
                                                                \bigg)^\frac{1}{q}
                                                    \bigg)                                                                          \\
                                        &\;\;\;\;\mbox{(Esta desigualdad es gracias a la desigualdad de }                           \\
                                        &\;\;\;\;\mbox{ Hölder y a que $\log$ es una función creciente)}                            \\
                                        &=      \log\bigg( 
                                                            \E
                                                                \bigg(
                                                                    e^{-a(1-t) X_1 p}
                                                                \bigg)^{\frac{1}{p}} 
                                                        \cdot     
                                                            \E
                                                                \bigg(
                                                                    e^{-b(t) X_1 q}
                                                                \bigg)^\frac{1}{q}
                                                     \bigg)                                                                         \\
                                        &=      \log\bigg( 
                                                            \E
                                                                \bigg(
                                                                    e^{-a(1-t) X_1 \frac{1}{1-t}}
                                                                \bigg)^{\frac{1}{\frac{1}{1-t}}} 
                                                        \cdot    
                                                            \E
                                                                \bigg(
                                                                    e^{-b(t) X_1 \frac{1}{t}}
                                                                \bigg)^\frac{1}{\frac{1}{t}}
                                                     \bigg)                                                                         \\
                                        &=      \log\bigg( 
                                                        \E
                                                            \bigg(
                                                                e^{-a X_1}
                                                            \bigg)^{1-t} 
                                                        \cdot                                                             
                                                        \E
                                                            \bigg(
                                                                e^{-b X_1}
                                                            \bigg)^{t}
                                                     \bigg)                                                                         \\
                                        &=      \log\bigg( 
                                                        \E
                                                            \bigg(
                                                                e^{-a X_1}
                                                            \bigg)^{1-t} 
                                                \bigg)
                                                        + 
                                                \log\bigg(
                                                        \E
                                                            \bigg(
                                                                e^{-b X_1}
                                                            \bigg)^{t}
                                                \bigg)                                                                              \\
                                         &=      \log\bigg( 
                                                        g(a)^{1-t} 
                                                \bigg)
                                                        + 
                                                \log\bigg(
                                                        g(b)^{t}
                                                \bigg)                                                                              \\
                                         &=      (1-t)\log(g(a))+(t)\log(g(b))                                                      \\
                                         &=      (1-t)(\log \circ g (a))+(t)(\log \circ g (b))                                        
    \end{align}
    
    Que es lo que necesitabamos mostrar para probar que $g$ es $\log$-convexa.\\
    
    Ahora supongamos que $\mw(X_1 = -1) > 0$.\\
    
    Para ver que $g$ tiende a infinito conforme $\lambda$ crece, descompongamos a $g(\lambda)$ de la siguiente manera.
    
    \begin{align}
        g(\lambda)      &=      \E(e^{-\lambda X_1})                \\
                        &=      \sum_{-1 \leq i} e^{-\lambda i} \mw( X_1 = i)
    \end{align}
    
    De donde obtenemos que $g(\lambda) \geq e^{\lambda} \mw( X_1 = -1)$. Dado que $e^{\lambda}$ tiende a infinito comforme
    $\lambda$ crece, tenemos que $g(\lambda)$ también lo hace.\\
    
    Ahora notemos que $g$ es convexa también. $\log \circ g$ es convexa por la primera parte de este inciso y 
    $e^x$ es convexa y creciente porque su primera y segunda derivada siempre son mayor que cero. 
    Dado esto, notemos que $g = e^{\log(g)}$ y entonces por ser $g$ una composición de una función convexa con una
    convexa creciente tenemos que $g$ es convexa.\\
    
    Uno de los resultados de los cursos de cálculo de la licencuatura es que una función convexa con dominio abierto, es continua.
    $g$ está definida sobre todo $\R$, que es abierto, y por lo tanto $g$ es continua.\\
    
    Ahora sea $s \in (0,1)$. Y sea $\lambda_0 = f(s)$. Por definición, $\lambda_0 = \inf\{ \lambda > 0 : g(\lambda)^{-1} < s\}$. Por definición de ínfimo,
    para cualquier $n \in \N$, existe $\lambda_n > 0$ tal que 
    
    \begin{align}
        \lambda_0 \leq \lambda_n \leq \lambda_0 + \frac{1}{n}. \label{problema2_1:sucesion_convergente_a_lambda_0}
    \end{align}
     
    y que
    
    \begin{align}
        g(\lambda_n)^{-1} < s. \label{problema2_1:sucesion_dominada_por_s}
    \end{align}
    
    Por \eqref{problema2_1:sucesion_convergente_a_lambda_0} sabemos que $\lambda_n \rightarrow \lambda_0$.
    De \eqref{problema2_1:sucesion_dominada_por_s} obtenemos $\frac{1}{s} < g(\lambda_n)$. Por continuidad de $g$
    tenemos entonces que $\frac{1}{s} \leq g(\lambda_0)$.
    \newpage
        
    \subsection{Inciso (iii)}    \label{problema2_1:inciso3}
    \emph
{
    Argumente que
}

\begin{align}
    \mw(M_p\geq n)  &=  \mw(T_n\leq C_p)    \\
                    &=\E((1-p)^{T_n})
\end{align}\pn

\emph
{
    para demostrar que $M_p$ tiene distribuci\'on geom\'etrica de par\'ametro $1-e^{-f(1-p)}$.
}

\afterstatement\pn

    Recordemos que
    
    \begin{align}
        M_p     &=  -\min_{n\leq C_p} S_n. 
    \end{align}\pn
    
    Donde $C_p$ es una variable aleatoria con distribución geométrica de parámetro $p$.\pn
    
    Entonces 
    
    \begin{align}
        \{ M_p \geq n\}     &=  \{ -\min_{k\leq C_p} S_k \geq n \}          \\
                            &=  \{ \min_{k \leq C_p} S_k \leq -n \}         \\
                            &=  \{ \min\{ k \in \N : S_k = -n \}\leq C_p \}   \\
                            &=  \{ T_n \leq C_p \}.
    \end{align}\pn
    
    De donde tenemos $\mw(M_p\geq n)=\mw(T_n\leq C_p)$.\pn
    
    Ahora descompongamos $\mw(T_n \leq C_p)$.
    
    \begin{align}
        \mw(T_n \leq C_p)   &=  \sum_{i \in \N} \mw(T_n \leq C_p | T_n = i) \mw(T_n = i)        \\
                            &=  \sum_{i \in \N} \mw(i \leq C_p) \mw(T_n = i)                    \\
                            &=  \sum_{i \in \N} \sum_{j = i}^{\infty} \mw(C_p = j) \mw(T_n = i) \\
                            &=  \sum_{i \in \N} \sum_{j = i}^{\infty} (1-p)^{k-1} p\mw(T_n = i) \\
                            &=  \sum_{i \in \N} p\mw(T_n = i) \sum_{j = i}^{\infty} (1-p)^{k-1} \\                            
                            &=  \sum_{i \in \N} p\mw(T_n = i) \frac{(1-p)^{i-1}}{p}             \\                            
                            &=  \sum_{i \in \N} \mw(T_n = i) (1-p)^{i-1}                        \\                     
                            &=  \E((1-p)^{T_n}).                                                            
    \end{align}\pn
    
    Ahora calculemos $\mw(M_p = n)$.\pn
    
    \begin{align}
        \mw(M_p = n)    &=  \mw(M_p \geq n) - \mw(M_p \geq n + 1)       \\
                        &=  \mw(T_n \leq Cp)- \mw(T_{n+1} \leq n + 1)   \\
                        &=  \E((1-p)^{T_n}) - \E((1-p)^{T_{n+1}})       \\
                        &=  e^{-nf(1-p)}    - e^{-(n+1)f(1-p)}          \\
                        &=  e^{-nf(1-p)} (1- e^{-f(1-p)})               \\
                        &=  (e^{-f(1-p)})^n (1- e^{-f(1-p)}).           
    \end{align}\pn
    
    De donde tenemos que $M_n$ tiene la distribución geométrica de parámetro $(1- e^{-f(1-p)})$.
    \newpage
    
    \subsection{Inciso (iv)}    \label{problema2_1:inciso4}
    \emph{
    Tome el límite conforme $p\to 0$ para mostrar que la variable aleatoria 
    \begin{align}
        M=-\min_{n\geq 0}S_n
    \end{align}
    tiene una distribuci\'on geom\'etrica de par\'ametro $1-e^{-f(1)}$. Interprete esto cuando $f(1)=0$.
}

Cuando hablamos de $C_p$, sólo nos interesa su distribución.\pn

Así que para ejemplificar cómo se comporta una sucesión de geométricas con sus parámetros tendien a $0$,
mientras conservemos a $C_p$ con distribución geométrica, somos libres de elegir de que manera se comportan.\pn

Definamos $C_p : [0,1] \rightarrow \N$ (donde en $[0,1]$ usamos la medida de Lebesgue) de la siguiente manera:

\begin{align}
    C_{p}(x) = 
        \begin{cases}
            1       & \mbox{if } x \in [0, p)                                                                               \\
            2       & \mbox{if } x \in [p, p + (1-p)p)                                                                      \\
            \vdots  &                                                                                                       \\
            n       & \mbox{if } x \in \bigg[\sum_{i = 1}^{n - 1} (1 - p)^{i - 1}p , \sum_{i = 1}^n (1-p)^{i-1}p \bigg)     \\
            \vdots  &                                               
        \end{cases}
\end{align}

Claramente, $C_p$ definida de esta manera tiene distribución de una geométrica de parámetro $p$ (Porque no le dejamos de otra).\pn

Ahora podemos ver con más claridad que 

\begin{align}
    \mw(C_p > n) = 1 - \sum_{i = 1}^n (1-p)^{i-1}p
\end{align}

Analicemos la derivada del término $(1-p)^{i}p$ ($i$ fija) con respecto a $p$. Su derivada es $(1-p)^i - p i (1-p)^{i-1}$.
Evaluando en $0$ tenemos \\
$(1-0)^i - 0 i (1-0)^{i-1} = 1 > 0$. Es decir que cerca de $0$, cada uno de los términos de este tipo, es 
estrictamente creciente.\pn

Es decir, que si $p$ se va acercando hacia 0, cada uno de estos términos, decrece.\pn

A continuación una gráfica de cómo se comporta $1 - \sum_{i = 1}^{100} (1-p)^{i-1}p$ conforme variamos $p$.
(Se incluye el script de R que se implementó para realizar esta gráfica, junto con algunas otras gráficas para
distintos valores de $n$).\pn

\begin{center}
    \includegraphics[width=6cm]{tarea2/problema2_1/graficas_inciso2_1_4/probabilidadDeQueC_pSupere100.png}
\end{center}\pn

Con todo esto dicho, podemos garantizar que $C_p$ diverge a $\infty$ en distribución. Pues para todo rango $[m, n]$ con $m<n,\in \N$
$\mw(C_p \in [m, n]) \rightarrow 0$. \pn

Entonces 
\begin{align}
    - \min_{n \leq C_p} \rightarrow - \min_{n \leq \infty} S_n = \min_{n \geq 0} S_n = M
\end{align}\pn

Ahora recordemos que $g$ era una función continua y que $g>0$. Por lo tanto $1/g$ es una función continua,
con inverso derecho $f$ en $(0, 1)$. Es decir

Por lo tanto
\begin{align}
    \mw(M = n)  &=  \lim_{p \rightarrow 0} \mw(M_p = n)                         \\
                &=  \lim_{p \rightarrow 0} e^{-n f(1-p)}(1-e^{- f(1-p)})        \\
                &=   e^{-n f(1)}(1-e^{- f(1)}) \label{problema2_1:distribucion_de_M}
\end{align}

Lo cual corresponde a la distribución de una geométrica de parámetro $1-e^{- f(1)}$.\pn

El caso donde $f(1^-) = 0$, implica que $\lim_{p\rightarrow0} (1-e^{- f(1-p)}) = 0$. El cual es complétamente
análogo al caso anterior donde considerábamos $p \rightarrow 0$ en geométricas de parámetro $p$.\pn

Donde, habíamos dejado claro que conforme el parámetro disminuía hacia 0, las distribuciones de las geométricas divergía 
a la de $\infty$ y que por lo tanto la probabilidad de que nuestras geométricas tomaran cierto 
valor $n$ disminuía hacia $0$ conforme el parámetro tendía a $0$.\pn

Incluso para este caso, la ecuación \eqref{problema2_1:distribucion_de_M} es válida según nuestro análisis, pues

\begin{align}
\mw(M = n)  &=  \lim_{p \rightarrow 0} \mw(M_p = n)                         \\
                &=  \lim_{p \rightarrow 0} e^{-n f(1-p)}(1-e^{- f(1-p)})    \\
                &=  e^{-n 0}(1-e^{- 0})                                     \\
                &=  1(1-1)                                                  \\
                &=  0.
\end{align}

Y esto, para toda $n \in \N$, justo como nuestro análisis de las distribuciones geométricas
nos dijo.
\end{proof}
		\newpage

		\subsection{Problema 2.2}
		\begin{problema}
    \begin{enumerate}
        \item[(i)] 
            Instale \href{www.octave.org}{Octave} en su computadora
        \item[(ii)] 
            \'Echele un ojo a la documentaci\'on
        \item[(iii)] 
   
            Ejecute el siguiente c\'odigo linea por linea: 
            \texttt{
                    \lstinputlisting[caption=]{tarea2/problema2_2/polya1.R}
                    }
        \item[(iv)] 
            Lea las secciones sobre 
            \href{http://www.gnu.org/software/octave/doc/interpreter/Simple-Examples.html#Simple-Examples}{simple examples}, 
            \href{http://www.gnu.org/software/octave/doc/interpreter/Ranges.html#Ranges}{ranges}, 
            \href{http://www.gnu.org/software/octave/doc/interpreter/Random-Number-Generation.html#Random-Number-Generation}{random number generation} 
            y 
            \href{http://www.gnu.org/software/octave/doc/interpreter/Comparison-Ops.html#Comparison-Ops}{comparison operators} 
            y escriba su interpretaci\'on de lo que hace el c\'odigo anterior. Nota: est\'a relacionado con uno de los ejemplos del curso.
        \item[(v)] 
            Vuelva a correr el c\'odigo varias veces y escriba sus impresiones sobre lo que est\'a sucediendo.
    \end{enumerate}
\end{problema}

\afterstatement\par\null

En el código de arriba, se implementó una ejecución del proceso de las urnas de Poyla.\par\null

El vector \texttt{x} representa la proporción de las bolas ``verdes'' en la urna conforme el tiempo avanza. Que
al tiempo inicial se tenga \texttt{x=[1/2]} quiere decir que al inicio había tantas bolas ``verdes'' como ``rojas''.\par\null

Para octave los booleanos pueden operarse numéricamente. \texttt{true} es equivalente a \texttt{1} y
\texttt{false} es equivalente a \texttt{0}. Entonces, la parte que dice \\
\texttt{+(u(i)$<$x(i))} significa que se está sumando
uno o cero dependiendo de si la condición se satisface. 
Lo cual quiere decir que la constante de bolas que se agregan a la urna es $1$.\par\null

El vector \texttt{u} representa el resultado de sacar una bola. Si \texttt{u(i)$<$x(i)} significa que en el turno \texttt{i}, 
se obtuvo una bola ``verde''.\par\null

Notemos que en la implementación de arriba, \texttt{x(2) = 3/8} ó \texttt{x(2) = 5/8}. Lo que quiere decir que para 
el turno \texttt{2} existen al menos \texttt{8} bolas. Aún más, podemos asegurar que para el turno \texttt{2} existe 
un número par de bolas. Es decir que en el turno 1 el número de bolas era impar, pero la proporción de bolas ``verdes'' 
con respecto al total era \texttt{1/2}. Esto revela que existe un error de desfase en la implementación.\par\null 

Imagino, que lo que se intentó hacer fue una urna con una bola ``verde'' y una bola ``roja'' inicialmente. En este caso, lo que 
debería decir en el ciclo \texttt{for} es lo siguiente:\par\null

\begin{verbatim}
for i = 1:600
    x(i+1) = (2+i-1)/(2+i) x(i) + (u(i)<x(i))/(2+i);
endfor
\end{verbatim}\par\null

Con una chequeo rápido uno puede comprobar que en esta versión modificada de la implementación, \texttt{x(2) = 1/3} ó \texttt{x(2) = 2/3}. 
Como al inicio la proporción era \texttt{1/2}, esto significa que al inicio existían una bola ``verde'' y una ``roja'' exactamente.\par\null

A continuación una gráfica obtenida de ejcutar el código.

\begin{center}
    \includegraphics[width=8cm]{tarea2/problema2_2/poyla.PNG}
\end{center}
\begin{center}
    Gráfica de una ejecución de urnas de Poyla \par
    $1$ bola verde inicial, $1$ bola roja inicial y constante $1$. $1500$ iteraciones.
\end{center}\par\null

Se puede apreciar que conforme avanza el tiempo, la proporción de bolas ``verdes'' se estabiliza. Justo como se demostró en clase para
martingalas positivas.
		\newpage


		\subsection{Problema 2.3}
		
\begin{problema}[Ejercicios sueltos sobre martingalas]
	\begin{enumerate}
		\item 
			Sea $\paren{X_n,n\geq 0}$ una sucesi\'on $\paren{\F_n}$-adaptada. Pruebe que
			
			\begin{esn}
				\sum_{k=1}^n X_k-\espc{X_k}{\F_{k-1}}, \quad n\geq 0
			\end{esn}
			
			es una $\paren{\F_n}$-martingala.
		
		\item
			Descomposici\'on de Doob para submartingalas: Sea \(X=\paren{X_n}_{n\in\na}\) una submartingala. 
			Pruebe que $X$ se puede descomponer de manera \'unica como $X=M+A$ donde $M$ es una martingala y $A$ 
			es un proceso previsible con $A_0=0$. Sugerencia: Asuma que ya tiene la descomposici\'on y calcule 
			esperanza condicional de $X_{n+1}$ dada $X_n$. 
		
		\item 
			Sea \(S_n=\xi_1+\cdots+\xi_n\) donde las variables \(\xi\) son independientes y \(\xi_i\) tiene 
			media cero y varianza finita \(\sigma_i^2\). Pruebe que si \(\sum_i \sigma_i^2<\infty\) entonces 
			\(S_n\) converge casi seguramente y en \(L_2\) conforme \(n\to\infty\). Construya un ejemplo de 
			variables aleatorias \(\xi_i\) tales que la serie \(\sum_i \xi_i\) sea casi seguramente absolutamente 
			divergente y casi seguramente condicionalmente convergente (considere ejemplos simples!). 
			Explique heur\'isticamente por qu\'e cree que suceda esto.
			%Ser\'a que \sum_i\abs{x_i}=\infty casi seguramente si \sum_i\abs\esp{\xi_i}=\infty? 
		
		\item 
			Sean \(X\) y \(Y\) dos martingalas (respecto de la misma filtraci\'on) y tales que \(\esp{X_i},\esp{Y_i}<\infty\) 
			para toda \(i\). Pruebe la siguiente f\'ormula de integraci\'on por partes: 
			
			\begin{esn}
				\esp{X_nY_n}-\esp{X_0Y_0}=\sum_{i=1}^n \esp{\paren{X_i-X_{i-1}}\paren{Y_i-Y_{i-1}}}. 
			\end{esn}
		
		\item Desigualdad de Azema-Hoeffding, tomado de \cite[E14.2, p.237]{MR1155402}
			\begin{enumerate}
				\item Muestre que si \(Y\) es una variable aleatoria con valores en \([-c,c]\) y media cero entonces, para \(\theta\in\re\)
						$$\esp{e^{\theta Y}}\leq\imf{\cosh}{\theta c}\leq \imf{\exp}{\frac{1}{2}\theta^2c^2}. $$
				\item Pruebe que si \(M\) es una martingala nula en cero tal que para algunas constantes \(\paren{c_n,n\in\na}\) se tiene que
						$$\abs{M_n-M_{n-1}}\leq c_n\quad\forall n $$
						entonces, para \(x>0\)
						$$
						\proba{\max_{k\leq n} M_k\geq x}\leq \imf{\exp}{\frac{x^2}{2\sum_{k=1}^n c_k^2}}.
						$$
			\end{enumerate}
	\end{enumerate}
\end{problema}
		\newpage


		\subsection{Problema 2.4}
		\begin{problema}
	Sea $S_n=\sum_{i=1}^n X_i$ donde $X_1,X_2,\ldots$ son iid. Sea
	
	\begin{esn}
		\imf{\phi}{\lambda}=\esp{e^{\lambda S_n}}\in (0,\infty].
	\end{esn}

	\begin{enumerate}
		\item[(i)]
			Pruebe que si existen $\lambda_1<0<\lambda_2$ tales que $\imf{\phi}{\lambda_i}<\infty$ entonces $\imf{\phi}{\lambda}<\infty$ para toda $\lambda\in [\lambda_1,\lambda_2]$. Sugerencia: escriba $\lambda=a\lambda_1+(1-a)\lambda_2$ para alg\'un $a\in [0,1]$ y aplique la desigualdad de H\"older. A partir de ahora se asume la premisa de este inciso.
		\item[(ii)] 
			Pruebe que $\esp{\abs{S_n}^k}<\infty$ para toda $k\geq 0$. 
		\item[(iii)] 
			Sea $M^\lambda_t=e^{\lambda S_t}/\imf{\phi}{\lambda}$. Argumente que si $M^n$ es el proceso dado por
			
			\begin{esn}
				M^n_t=\left.\frac{\partial^n}{\partial \lambda^n}\right|_{\lambda=0}M^\lambda_t,
			\end{esn}
			
			entonces $M^n$ es una martingala para toda $n$. 
		\item[(iv)] 
			Calcule las primeras $4$ martingalas resultantes si $\proba{X_i=\pm 1}=1/2$. Util\'icelas para calcular el valor de $\esp{T^2}$ donde

			\begin{esn}
				T=\min\set{n\geq 0: S_n\in\set{-a,b}}
			\end{esn}y $a,b>0$. 
	\end{enumerate}

	\defin{Categor\'ias:} Caminatas aleatorias, muestreo opcional, ejemplos de martingalas. 
\end{problema}
		
		\nqed
		
	\section{Tarea 3}

		\center{\includegraphics[width=10cm]{tarea3/poylaBeta.PNG}}
\center{
    Gráfica del histagrama de los radios \"finales\" de muchas iteraciones del proceso \\
    de urnas de Poyla.
}
\\

		\input{tarea3/ejercicio3_3.tex}
		
		\nqed
		
	\bibliography{GenBib}
	\bibliographystyle{amsalpha}
\end{document}