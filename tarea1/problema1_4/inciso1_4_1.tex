\emph{
	Pruebe que $ S \wedge T $, $ S + T $ y $ S \vee T$ son tiempos de paro.\\
}
\afterstatement
	\begin{itemize}
		\item 
			Comprobemos que:
			\begin{align}
				\{ S \wedge T \leq n \} = \{ T \leq n \} \cup \{ S \leq n\}.
			\end{align}
			
			Sea $\omega \in \{ S \wedge T \leq n \}$. Entonces $T(\omega) \leq n$ ó $S(\omega) \leq n$.
			Por lo tanto:
			\begin{align}
					\{ S \wedge T \leq n \} \subset \{ T \leq n \} \cup \{ S \leq n\}.							
			\end{align}
			
			Por otro lado, si $\omega \in \{ T \leq n \} \cup \{ S \leq n\}$ entonces $T(\omega) \leq n$ ó $S(\omega) \leq n$.
			En particular, el mínimo tendrá que ser menor que $n$ y por lo tanto:
			\begin{align}
					\{ T \leq n \} \cup \{ S \leq n\} \subset \{ S \wedge T \leq n \}.						
			\end{align}
			
			Por último, $\{ T \leq n \} \in \F_n$ y $\{ S \leq n \} \in \F_n$. Por lo tanto	\\	
			$\{ T \leq n \} \cup \{ S \leq n\} = \{ S \wedge T \leq n \} \in \F_n$ y con esto demostramos que 
			$ S \wedge T$ es tiempo de paro.\par\null
			
		\item
			Comprobemos que:
			\begin{align}
				\{ S \vee T \leq n \} = \{ T \leq n \} \cap \{ S \leq n\}.
			\end{align}
			
			Sea $\omega \in \{ S \vee T \leq n \}$. Entonces $T(\omega) \leq n$ y $S(\omega) \leq n$.
			Por lo tanto:
			\begin{align}
					\{ S \vee T \leq n \} \subset \{ T \leq n \} \cap \{ S \leq n\}.							
			\end{align}
			
			Por otro lado, si $\omega \in \{ T \leq n \} \cap \{ S \leq n\}$ entonces $T(\omega) \leq n$ y $S(\omega) \leq n$.
			En particular, el máximo tendrá que ser menor que $n$ y por lo tanto:
			\begin{align}
					\{ T \leq n \} \cap \{ S \leq n\} \subset \{ S \vee T \leq n \}.						
			\end{align}
			
			Por último, $\{ T \leq n \} \in \F_n$ y $\{ S \leq n \} \in \F_n$. Por lo tanto	\\	
			$\{ T \leq n \} \cap \{ S \leq n\} = \{ S \vee T \leq n \} \in \F_n$ y con esto demostramos que 
			$ S \vee T$ es tiempo de paro.\par\null
			
		\item
			Comprobemos que:
			\begin{align}
				\{ S + T = n \} = \bigcup_{i = 1}^{n-1} \bigg( \{S = n - i\} \cap \{T = i\} \bigg).
			\end{align}
			
			Si $\omega \in \{ S + T = n \}$, entonces $S(\omega) + T(\omega) = n$, como $S$ y $T$ son positivas,
			entonces $1 \leq S(\omega) \leq n-1$. Entonces basta elegir $i = S(\omega)$ para afirmar que 
			$\omega \in \{S = n - i\} \cap \{T = i\}$ y por lo tanto
            
			\begin{align}
                \{ S + T = n \} \subset \bigcup_{i = 1}^{n-1} \bigg( \{S = n - i\} \cap \{T = i\} \bigg).
			\end{align}\par\null
			
			Por otro lado, si $\omega \in \bigcup_{i = 1}^{n-1} \bigg( \{S = n - i\} \cap \{T = i\} \bigg)$
			significa que existe un $i$ tal que $1\leq i \leq n-1$ y $\omega \in \{S = n - i\} \cap \{T = i\}$.
			Y por lo tanto: $S(\omega) = n-i$ y $T(\omega) = i$. De donde $(T + S)(\omega) = n$.\par\null
			
			De aquí que
            
			\begin{align}
				\bigcup_{i = 1}^{n-1} \bigg( \{S = n - i\} \cap \{T = i\} \bigg) \subset \{ S + T = n \}. 
			\end{align}\par\null
			
			Aohra, para cada $i$ tal que $1 \leq i \leq n-1$, $\{S = n - i\} \cap \{T = i\} \in \F_n$.
			Por lo tanto $\{ S + T = n\} \in \F_n$ y con esto queda demostrado que $ S + T $ es tiempo de paro.
	\end{itemize}
	