\begin{problema}
\mbox{}
\begin{enumerate}
\item Ejecute y explique la funci\'on del siguiente c\'odigo en Octave. Comente qu\'e teoremas del curso (y del curso de probabilidad) son importantes para interpretar la figura.
\lstinputlisting[caption=]{tarea3/problema3_4/polya2.R}
\item Ejecute y explique la funci\'on del siguiente c\'odigo en Octave. Incluya una gr\'afica en la que la longitud de la variable k sea mayor a 1000. (Puede modificar el programa...) En la gr\'afica observara un esbozo de la trayectoria de un proceso de ramificaci\'on continuo (en una escala distinta...).
\lstinputlisting[caption=]{tarea3/problema3_4/binaryGW.R}
\end{enumerate}
\end{problema}
