\begin{problema}
%Simulaci\'on de un proceso Poisson puntual... subordinador...
    Sea $\Xi$ una medida de Poisson aleatoria en $(0,\infty)\times (0,\infty)$ cuya 
    medida de intensidad $\nu$ est\'a dada por $\imf{\nu}{ds,dx}=\indi{0<x} \indi{s<t}C/x^{1+\alpha}\, ds\,dx$. 
    
    \begin{enumerate}
        \item[(i)]		[\ref{problema5_2:inciso1}] 
			Determine los valores de $\alpha$ para los cuales $\int 1 \wedge x\,\imf{\nu}{dx}<\infty$. 
    \end{enumerate}
    
    Nos restringimos ahora a valores de $\alpha$ para los cuales la integral anterior sea finita. 
	Sean $\imf{f_t}{s,x}=\indi{s\leq t}x$ y $X_t=\Xi f_t$. 
    
    \begin{enumerate}[resume]
        \item[(ii)]		[\ref{problema5_2:inciso2}] 
			Determine los valores  de $\alpha$ para los cuales $X_t < \infty$ para toda $t \geq 0$ casi seguramente.
    \end{enumerate}

    Nos restringiremos a dichos valores de $\alpha$. 
    
    \begin{enumerate}[resume]
        \item[(iii)]	[\ref{problema5_2:inciso3}] 
			Calcule $\esp{e^{-\lambda X_t}}$ y pruebe que $X_{t}$ tiene la misma distribuci\'on que $t^{1/\alpha}X_1$. 
        \item[(iv)]		[\ref{problema5_2:inciso4}] 
			Diga por qu\'e el siguiente c\'odigo en Octave simula la trayectoria aproximada del proceso $X$ en el intervalo $[0,1]$.
			\texttt{
				\lstinputlisting{tarea5/problema5_2/SuborEst.m}
			}
    \end{enumerate}
\end{problema}

\begin{proof}
	\subsection{Inciso (i)}	\label{problema5_2:inciso1}
	\emph{
	Determine los valores de $\alpha$ para los cuales \pn
	$\int 1\wedge x\,\imf{\nu}{dx}<\infty$.
}

\afterstatement\pn

Separemos la integral de manera conveniente:

\begin{align}
   \int 1\wedge x\, \nu(dx) &= \int (1 \wedge x) \indi{0<x} C/x^{1+\alpha} \,dx                                                 \\
                            &= \int_0^\infty (1 \wedge x) C/x^{1+\alpha} \,dx                                                   \\
                            &= \int_0^1 (1 \wedge x) C/x^{1+\alpha} \,dx + \int_1^\infty (1 \wedge x) C/x^{1+\alpha} \,dx       \\
                            &= \int_0^1 x C/x^{1+\alpha} \,dx + \int_1^\infty C/x^{1+\alpha} \,dx                               \label{problema5_2:descomposicion_de_integral}
\end{align}

Para que la integral que nos interesa sea finita es necesario y suficiente que los dos sumandos de la última 
parte de la ecuación lo sean. \pn

Veamos qué ocurre en el primer termino si $\alpha = 1$.

\begin{align}
    \int_0^1 x C/x^{1+\alpha} \,dx  &=  \int_0^1 x C/x^{2} \,dx                                                 \\
                                    &=  \int_0^1 C/x \,dx                                                       \\
                                    &=  \lim_{t \rightarrow 0^+}\int_t^1 C/x \,dx                               \\
                                    &=  C \lim_{t \rightarrow 0^+} \int_t^1 1/x \,dx                            \\
                                    &=  C \lim_{t \rightarrow 0^+} [\log(x)]_t^1                                \\
                                    &=  C \lim_{t \rightarrow 0^+} \log(1) - \log(t)                            \\
                                    &=  C \lim_{t \rightarrow 0^+} - \log(t)                                    \\
                                    &=  C \cdot \infty.                                                                
\end{align}\pn

Por lo tanto $\alpha \not= 1$. Veamos ahora qué ocurre con el primer sumando de \eqref{problema5_2:descomposicion_de_integral}
suponiendo que $\alpha \not= 1$.

\begin{align}
    \int_0^1 x C/x^{1+\alpha} \,dx  &=  \int_0^1 C/x^{\alpha}   \,dx                                                                                    \\
                                    &=  C \int_0^1 x^{-\alpha} \,dx                                                                                     \\
                                    &=  C \left[ \frac{x^{(1-\alpha)}}{(1-\alpha)}\right]_0^1                                                           \\
                                    &=  C \left( \frac{1^{(1-\alpha)}}{(1-\alpha)} - \lim_{t \rightarrow 0}\frac{t^{(1-\alpha)}}{(1-\alpha)}\right)     \\
\end{align}\pn

El límte $\lim_{t \rightarrow 0}\frac{t^{(1-\alpha)}}{(1-\alpha)}$ exíste únicamente si $0 < 1-\alpha$, es decir, si $\alpha < 1$. Entonces basta con 
que $\alpha < 1$ para que el primer sumando de \eqref{problema5_2:descomposicion_de_integral} sea finito.\pn

Ahora, si $\alpha = 0$ y analizamos el segundo sumando de \eqref{problema5_2:descomposicion_de_integral} vemos que

\begin{align}
    \int_1^\infty C/x^{1+\alpha} \,dx   &=  C \int_1^\infty 1/x \,dx                                            \\ 
                                        &=  C \lim_{t \rightarrow \infty} \int_1^t 1/x \,dx                     \\ 
                                        &=  C \lim_{t \rightarrow \infty} \left[ \log(x) \right]_1^t            \\ 
                                        &=  C \lim_{t \rightarrow \infty} \log(t) - \log(1)                     \\ 
                                        &=  C \lim_{t \rightarrow \infty} \log(t)                               \\ 
                                        &=  C \cdot \infty.                                                       
\end{align}

Por lo cual necesitamos $\alpha \not= 0$. Supongamos entonces $\alpha \not= 0$ y analicemos el segundo sumando de \eqref{problema5_2:descomposicion_de_integral}:

\begin{align}
    \int_1^\infty C/x^{1+\alpha} \,dx   &=  C \int_1^\infty 1/x^{1+\alpha} \,dx                                                                     \\
                                        &=  C \int_1^\infty x^{-1-\alpha} \,dx                                                                      \\
                                        &=  C \lim_{t \rightarrow \infty} \int_1^t x^{-1-\alpha} \,dx                                               \\
                                        &=  C \lim_{t \rightarrow \infty} \left[ \frac{x^{-\alpha}}{-\alpha} \right]_1^t                            \\
                                        &=  C \left( \lim_{t \rightarrow \infty} \frac{t^{-\alpha}}{-\alpha} - \frac{1}{-\alpha} \right)            \\
\end{align}

Donde el límite $\lim_{t \rightarrow \infty} \frac{t^{-\alpha}}{-\alpha}$ existe únicamente si $-\alpha < 0$, es decir, si $0 < \alpha$.\pn

Entonces podemos concluir que los valores de $\alpha$ para los que \eqref{problema5_2:descomposicion_de_integral} es finito están en $(0, 1)$. 
	\newpage
	
	\subsection{Inciso (ii)}	\label{problema5_2:inciso2}
	\emph{
	Determine los valores  de $\alpha$ para los cuales $X_t < \infty$ para toda $t \geq 0$ casi seguramente.
}

\afterstatement\pn
	\newpage
	
	\subsection{Inciso (iii)}	\label{problema5_2:inciso3}
	\emph{
	Calcule $\esp{e^{-\lambda X_t}}$ y pruebe que $X_{t}$ tiene la misma distribuci\'on que $t^{1/\alpha}X_1$.
}

\afterstatement\pn

La primera parte de la proposición $4.6$ de la versión de las notas que se adjuntó en este documento (ver [\ref{notas}]) dice que
si $f$ es medible y no negativa entonces la integral de $f$ con respecto de $\Xi$ es una variable aleatoria y 

\begin{align}
    \E\left( e^{-\Xi f} \right)   &=  e^{-\int (1-e^{-f}) \,d\nu}.
\end{align}\pn

Usando esto mismo pero con $\Xi \lambda f_t = \lambda X_t$, tenemos

\begin{align}
    \E\left( e^{- \lambda X_t} \right)   &=  e^{-\int (1-e^{-\lambda f_t}) \,d\nu}. \label{problema5_2:resultado4_6}
\end{align}\pn.

Analicemos qué ocurre con $\int (1-e^{-\lambda f_t}) \,d\nu$.

\begin{align}
    \int (1-e^{-\lambda f_t}) \,d\nu    &=  \int \int (1-e^{-\lambda f_t})\frac{C}{x^{1+\alpha}}\,ds\,dx                                        \\
                                        &=  \int \int (1-e^{-\lambda \indic_{\{s \leq t\}} x})\frac{C}{x^{1+\alpha}}\,ds\,dx                    \\
                                        &=  \int_0^\infty \int_0^t (1-e^{-\lambda  x})\frac{C}{x^{1+\alpha}}\,ds\,dx                            \\
                                        &=  \int_0^\infty t (1-e^{-\lambda  x})\frac{C}{x^{1+\alpha}}\,dx                                       \\
                                        &=  tC \int_0^\infty  (1-e^{-\lambda  x})\frac{1}{x^{1+\alpha}}\,dx                                     \\
    \intertext{
    Ahora reescribimos a $(1-e^{-\lambda  x})$ como $\int_0^x \lambda e^{-\lambda y} \,dy$ y entonces la ecuación 
    anterior se transforma en}
                                        &=  tC \int_0^\infty  \left(\int_0^x  \lambda e^{-\lambda y} \,dy \right)\frac{1}{x^{1+\alpha}}\,dx     \\
                                        &=  tC \int_0^\infty  \frac{\int_0^x  \lambda e^{-\lambda y}}{x^{1+\alpha}}\,dy\,dx                     \\
                                        &=  tC \int_0^\infty  \int_0^x\frac{ \lambda e^{-\lambda y}}{x^{1+\alpha}} \lambda \,dy\,dx                      \\
    \intertext{
    La función que se encuentra en la integral es positiva, por lo tanto podemos aplicar el Teorema de Tonelli 
    para intercambiar las integrales de la siguiente manera:}
                                        &=  tC \int_0^\infty  \int_y^\infty\frac{ \lambda e^{-\lambda y}}{x^{1+\alpha}}\,dx\,dy                 \\
    \intertext{
    Recordando que $\alpha \in (0,1)$, tenemos que 
    $\int_y^\infty\frac{ 1 }{x^{1+\alpha}}\,dx = \lim_{x \rightarrow \infty} \frac{x^{-\alpha}}{-\alpha} - \frac{y^{-\alpha}}{-\alpha} = 
    - \frac{y^{-\alpha}}{-\alpha} = \frac{y^{-\alpha}}{\alpha}$. Sustituyendo esto tenemos}
                                        &=  tC \int_0^\infty  \frac{y^{-\alpha}}{\alpha} \lambda e^{-\lambda y}\,dy                             \\
                                        &=  \frac{tC}{\alpha} \int_0^\infty  y^{-\alpha}  e^{-\lambda y} \lambda\,dy                            \\
                                        &=  \frac{tC \lambda^{\alpha}}{\alpha} \int_0^\infty  (\lambda y)^{-\alpha}  e^{-\lambda y}\,dy         \\
    \intertext{
    Recordemos que $\int_0^\infty  (\lambda y)^{-\alpha}  e^{-\lambda y}\,dy$ se puede escribir 
    como $\Gamma(1-\alpha)$ y entonces nuestra ecuación se puede escribir como}
                                        &=  \frac{tC \lambda^{\alpha}}{\alpha} \Gamma(1-\alpha).                                                 \\
\end{align}

Ahora, regresando a \eqref{problema5_2:resultado4_6}, tenemos que 

\begin{align}
    \E\left( e^{- \lambda X_t} \right)  &=  e^{-\int (1-e^{-\lambda f_t}) \,d\nu}                               \\
                                        &=  e^{-\frac{tC \lambda^{\alpha}}{\alpha} \Gamma(1-\alpha)}            \\
\end{align}\pn

Ahora de la ecuación anterior, remplacemos $t$ por $1$ y entonces obtenemos

\begin{align}
    \E\left( e^{- \lambda X_1} \right)   &= e^{-\frac{C \lambda^{\alpha}}{\alpha} \Gamma(1-\alpha)}
\end{align}

De esto último, remplacemos ahora $\lambda$ por $\lambda t^{1/\alpha}$

\begin{align}
    \E\left( e^{- \lambda t^{1/\alpha} X_1} \right)   &= e^{-\frac{C (\lambda t^{1/\alpha})^{\alpha}}{\alpha} \Gamma(1-\alpha)}    \\
                                                      &= e^{-\frac{C \lambda^\alpha t}{\alpha} \Gamma(1-\alpha)}    \\
\end{align}

De donde podemos observar fácilmente que
\begin{align}
    \E\left( e^{- \lambda X_t} \right)  &=  \E\left( e^{- \lambda t^{1/\alpha} X_1} \right).
\end{align}

Lo que acabamos de escribir no son mas que las funciones generadoras de momentos de $X_t$ y $t^{1/\alpha} X_1$ evaluadas en $-\lambda$. Lo que 
acabamos de demostrar es que sus funciones generadoras de momentos son idénticas para cada $\lambda$ y por lo tanto, sus distribuciones son iguales.
	\newpage
	
	\subsection{Inciso (iv)}	\label{problema5_2:inciso4}
	\emph{
	Diga por qu\'e el siguiente c\'odigo en Octave simula la trayectoria aproximada del proceso $X$ en el intervalo $[0,1]$.
	\texttt{
		\lstinputlisting{tarea5/problema5_2/SuborEst.m}
	}
}

\afterstatement\pn
\end{proof}