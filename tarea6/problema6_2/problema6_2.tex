\begin{problema}
	El objetivo de este problema es construir, a partir de movimientos brownianos en $[0,1]$, al movimiento browniano en $[0,\infty)$.
	\begin{enumerate}
		\item 
			Pruebe que existe un espacio de probabilidad $\ofp$ en el que existe 
			una sucesi\'on $B^1,B^2,\ldots$ de movimientos brownianos en $[0,1]$ 
			independientes. (Sugerencia: utilice la construcci\'on del movimiento 
			browniano de L\'evy  para que la soluci\'on sea corta.)
			
		\item 
			Defina a $B_t=B^1_1+\cdots+B^{\floor{t}}_1+B^{\ceil{t}}_{t-\floor{t}}$ 
			para $t\geq 0$. Pruebe que $B$ es un movimiento browniano. 
		
		%\item Pruebe que $\paren{B_t}^2-t$ no tiene incrementos independientes. Sugerencia: 
		%En el ejercicio anterior identific\'o la distribuci\'on de $\paren{B_t}^2$; calcule 
		%la transformada de Laplace conjunta de dos incrementos.
	\end{enumerate}
\end{problema}

\begin{proof}
    \subsection{Inciso (i)} \label{problema5_6:inciso1}
    \emph{
    Pruebe que existe un espacio de probabilidad $\ofp$ en el que existe 
    una sucesi\'on $B^1,B^2,\ldots$ de movimientos brownianos en $[0,1]$ 
    independientes. (Sugerencia: utilice la construcci\'on del movimiento 
    browniano de L\'evy  para que la soluci\'on sea corta.)
}
\afterstatement\pn

La idea principal es utilizar que $\aleph_0 \times \aleph_0 = \aleph_0$ y aprovechar
la construcción que se dió en las notas [véase \ref{notas}].\pn

Se sabe que existe un espacio de probabilidad $(\Omega, \F, \P)$ donde están
definidas $\xi_{i,n}$ ($0 \leq i \leq 2^n$) con distribuciónes normales de media 0 y 
varianza 1 e independientes [ver \ref{notas}].\pn

Entonces, podemos para cada $m \in \N$ dar una sucesión de variables $\xi_{i,n}^m$
($0 \leq i \ 2^n$) con distribuciones normales de media 0 y varianza 1 tales que
son independientes entre sí y de las otras $\Xi_{i,n}^{m'}$.\pn

Entonces, para cada $m$ podemos construir un movimiento browniano $B^m$ en $[0, 1]$ utilizando
las $\xi_{i,n}^m$. Dichos movimientos serán independientes entre sí por la selección
de las $(\xi_{i,n}^m)_{m \in \N}$.
    \newpage

    \subsection{Inciso (ii)} \label{problema5_6:inciso2}
    \emph{
    Defina a $B_t=B^1_1+\cdots+B^{\floor{t}}_1+B^{\ceil{t}}_{t-\floor{t}}$ 
    para $t\geq 0$. Pruebe que $B$ es un movimiento browniano. 
}
\afterstatement\pn 
\end{proof}