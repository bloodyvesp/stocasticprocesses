\begin{problema}
    (Tomado del examen general de conocimientos del \'area de Probabilidad del Posgrado 
    en Ciencias Matem\'aticas, UNAM, 
    \href{http://www.posgradomatematicas.unam.mx/contenidoEstatico/archivo/files/pdf/Examenes_Generales/Probabilidad/Probabilidad2011-2.pdf}{Agosto 2011})

    Sea $N$ un proceso de Poisson homog\'eneo de par\'ametro $\lambda$. Sea $E=\paren{-1,1}$ y $X_0$ una 
    variable aleatoria con valores en $E$ independiente de $N$. Se define el proceso
    \begin{esn}
        X_t=X_0 \times \paren{-1}^{N_t}, \quad t\geq 0.
    \end{esn}

    \begin{enumerate}
        \item[(i)]      [\ref{problema5_7:inciso1}]
            Explique por qu\'e $X$ es una cadena de Markov a tiempo continuo con valores en $E$.\pn 
        \item[(ii)]     [\ref{problema5_7:inciso1}]
            Calcule sus probabilidades de transici\'on y su matriz infinitesimal.\pn
        \item[(iii)]    [\ref{problema5_7:inciso1}]
            ?`Existe una distribuci\'on estacionaria para esta cadena? En caso afirmativo ?`Cu\'al es?\pn
    \end{enumerate}
\end{problema}

\begin{proof}
    \subsection{Inciso (i)} \label{problema5_7:inciso1}
    \emph{
    Explique por qu\'e $X$ es una cadena de Markov a tiempo continuo con valores en $E$.\pn 
}
\afterstatement\pn
    \newpage

    \subsection{Inciso (ii)} \label{problema5_7:inciso2}
    \emph{
    Calcule sus probabilidades de transici\'on y su matriz infinitesimal.\pn
}

\afterstatement\pn

Recordemos que los tiempos de cambio están dados por un proceso de Poisson
de parámetro $\lambda$ y que únicamente tenemos $2$ estados. Es decir que
para pasar de un estado al otro, la tasa de cambio es precisamente $\lambda$ y
que la tasa de cambio de un estado a sí mismo es $0$. Dicho esto, 
tenemos que la matriz infinitesimal tiene que ser

\begin{align}
        Q   &= 
            \begin{pmatrix}
                    -\lambda    &   \lambda     \\
                    \lambda     &   -\lambda    \\
            \end{pmatrix}
\end{align}\pn

Esto no le deja más opciones a la matriz de la cadena de Markov asociada más que ser:
\begin{align}
        P   &=
            \begin{pmatrix}
                0   &   1\\
                1   &   0\\
            \end{pmatrix}
\end{align}\pn

Para poder calcular $P_t$, escribimos a $Q$ como conjugada de una matriz diagonal:
\begin{align}
       Q    &=
            \frac{1}{2}
            \begin{pmatrix}
                1 & 1       \\
                1 & -1
            \end{pmatrix}           
            \begin{pmatrix}
                0 & 0       \\
                0 & -2\lambda
            \end{pmatrix}
            \begin{pmatrix}
                1 & 1   \\
                1 & -1
            \end{pmatrix}
\end{align}\pn

Utilizando de nuevo que $P_t = e^{tQ}$ y que si $Q = A D A^{-1}$ entonces $e^{tQ} = Ae^{tD}A^{-1}$, 
obtenemos que las matrices de transición son

\begin{align}
    P_t     &=
            \frac{1}{2}
            \begin{pmatrix}
                1 & 1               \\
                1 & -1
            \end{pmatrix}           
            \begin{pmatrix}
                1 & 0               \\
                0 & e^{-2\lambda}
            \end{pmatrix}
            \begin{pmatrix}
                1 & 1               \\
                1 & -1
            \end{pmatrix}           \\
            &=
            \frac{1}{2}
            \begin{pmatrix}
                1 & e^{-2\lambda}   \\
                1 & -e^{-2\lambda}
            \end{pmatrix}
            \begin{pmatrix}
                1 & 1               \\
                1 & -1
            \end{pmatrix}           \\
            &=
            \frac{1}{2}
            \begin{pmatrix}
                1+e^{-2\lambda} & 1-e^{-2\lambda}       \\
                1-e^{-2\lambda} & 1+e^{-2\lambda}
            \end{pmatrix}
\end{align}
    \newpage

    \subsection{Inciso (iii)} \label{problema5_7:inciso3}
    \emph{
    ?`Existe una distribuci\'on estacionaria para esta cadena? En caso afirmativo ?`Cu\'al es?\pn
}
\afterstatement\pn
\end{proof}